\section{Discussion}


\subsection{Mean values and rms eviations of $X$ and $P$ in a state $\ket{\varphi_n}$}

Neither $X$ nor $P$ comutes with $H$, and the eigenstates $\ket{\varphi_n}$ of $H$ are not eigenstates of $X$ or $P$. Consequently, if the harmonics oscillator is in stationary state $\ket{\varphi_n}$,
a measurement of the observable $X$ or $P$ can, a priori, yield any result.


We will compute the mean values of $X,P$ in such stationary state and also theirs rms deviation in order to set the uncertainty relation. 
We will use equations \eqref{eq:XPintermsofaa}, which show that neither $X$ nor $P$ has diagonal matrix elements:
\begin{align}
    \braket{\varphi_n|X|\varphi_n}=\braket{\varphi_n|P|\varphi_n}=0.
\end{align}
To obtain the rms deviations, we must calculate the mean value of $X^2$ and $P^2$. First, we note that 
\begin{align*}
    X^2&=\frac{\hbar}{2m\omega}(a^\dagger+a)(a^\dagger+a)=\frac{\hbar}{2m\omega}(a^{\dagger 2}+aa^\dagger+a^\dagger a+a^2)\\
    P^2&=-\frac{m\hbar\omega}{2}(a^\dagger-a)(a^\dagger-a)=-\frac{m\hbar\omega}{2}(a^{\dagger2}-aa^\dagger-a^\dagger a+a^2)
\end{align*}
The terms $a^2$ and $a^{\dagger2}$ do not contribute to the diagonal matrix elements, since $a^2\ket{\varphi_n}$ is proportional to $\ket{\varphi_{n-2}}$ and 
$a^{\dagger2}\ket{\varphi_n}$ to $\ket{\varphi_{n+2}}$; both are orthogonal to $\ket{\varphi_n}$. The rest of the terms yields:
\begin{align*}
    \braket{\psi_n|(a^\dagger a+aa^\dagger)|\varphi_n}=\braket{\varphi_n|(2a^\dagger a+1)|\varphi_n}=2n+1.
\end{align*}
Therefore, we have:
\begin{align}
    (\Delta X)^2&=\braket{\varphi_n|X|\varphi_n}-\braket{\varphi_n|X^2|\varphi_n}=\braket{\varphi_n|X^2|\varphi_n}=\left(n+\frac{1}{2}\right)\frac{\hbar}{m\omega}=\sigma^2\left(x+\frac{1}{2}\right).\\
    (\Delta P)^2&=\braket{\varphi_n|P|\varphi_n}-\braket{\varphi_n|P^2|\varphi_n}=\braket{\varphi_n|P^2|\varphi_n}=\left(n+\frac{1}{2}\right)m\hbar\omega=\frac{\hbar^2}{\sigma^2}\left(x+\frac{1}{2}\right).
\end{align}
The product is therefore 
\begin{align}
    \text{Uncertainty relation}\qquad\highlight{\Delta X\Delta P=\left(n+\frac{1}{2}\right)\hbar}.
\end{align}
We see that the lower bound is attained for $n=0$, that is, for the ground state.

\subsection{Properties of the ground state}
In classical mechanics, the lowest energy of the harmonic oscilaltor is obtained when the particle is at rest. In QM, the moninum energy sate is $\ket{\varphi_0}$, whose energy 
is not zero, and the associates wave function has a certain spatial extension, characterized by the rms deviation $\Delta X=\sqrt{\hbar/2m\omega}$.
The ground state correesponds to a compromie in which the sum of the kinetic and potential energy is as small as possible (uncertainty limitation).

\begin{emphasizer}
    The QHO possesses the peculiarity that due to the form of $V(x)$, the $\Delta X\Delta P$ attains its lower value at the ground state $\ket{\varphi_0}$. This is related to 
    the fact that the wave function of the ground state is Gaussian.
\end{emphasizer}

%%
\subsection{Time evolution of the mean values}
Consider thw state at $t=0$ 
\begin{align*}
    \ket{\psi(0)}=\sum_{n=0}^\infty c_n(0)\ket{\varphi_n}.
\end{align*}
Its state $\ket{\psi(t)}$ at $t$ can be obtained by using the evolution operator for conservative systems:
\begin{align}
    \ket{\psi(t)}=\sum_{n=0}^\infty c_n(0)e^{-iE_nt/\hbar}\ket{\varphi_n}=\sum_{n=0}^\infty c_n(0)e^{-i(n+1/2)\omega t}\ket{\varphi_n}.
\end{align}
The mean value of any physical quantity $A$ is 
\begin{align*}
    \braket{\varphi(t)|A|\varphi(t)}=\sum_{m,n=0}^\infty c_m^*(0)c_n(0)A_{mn}e^{i(m-n)\omega t},\quad\text{with}\quad A_{mn}=\braket{\varphi_m|A|\varphi_n}.
\end{align*}
The time evolution of the mean values involves only the frequency $\omega/2\pi$ and its various harmonics, which constitutes the Bohr frequencies of the harmonics oscillator.

If we consider $X$ and $P$, we know that the only non-zero elements $X_{mn}$ and $P_{mn}$ are those for which $m=n\pm1$. Consequently, the mean values of $X$ and $P$ include only terms in $e^{\pm\omega t}$.
Moreover, the form of the harmonics oscillator potential implies that for all $|ket{\varphi_n}$ the mean values of $X$ and $P$ rigorously satisfy the classical equations of motion. Using Ehrenfest theorem:
\begin{align}
    \begin{array}{l}
    \dfrac{d}{dt}\braket{X}=\dfrac{1}{i\hbar}\braket{[X,H]}=\dfrac{\braket{P}}{m}\\    
    \dfrac{d}{dt}\braket{P}=\dfrac{1}{i\hbar}\braket{[P,H]}=-m\omega^2\braket{X}
    \end{array}
    \stackrel{\int dt}{\longrightarrow}
    \begin{array}{l}
        \braket{X}(t)=\braket{X}(0)\cos\omega t+\dfrac{1}{m\omega}\braket{P}(0)\sin\omega t\\
        \braket{P}(t)=\braket{P}(0)\cos\omega t+m\omega\braket{X}(0)\sin\omega t
    \end{array}
\end{align}
\begin{itemize}[itemsep=0pt,topsep=0pt]
    \item In a stationary state $\ket{\varphi_n}$, the behavior of the harmonic oscillator is totally different from that predicted by classical mechanics. 
    The mean values of all the observables are constant over time.
\end{itemize}

