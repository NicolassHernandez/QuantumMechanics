\section{The isotropic three-dimensional harmonic oscillator}
The main idea is the same, but here there is an extension of the problem.

The spinless particle of mass $m$ is subjected to a central force 
\begin{align}
    \bm{F}=-k\bR.
\end{align}
This force is derived from the potential energy:
\begin{align}
    V(\bR)=\frac{1}{2}k\bR^2=\frac{1}{2}m\omega^2\bR^2,\quad\omega=\sqrt{\frac{k}{m}}.
\end{align}
The classical Hamiltonian is therefore:
\begin{align}
    \text{Classical Hamiltonian}\qquad\highlight{H(\bR,\bp)=\frac{\bp^2}{2m}+\frac{1}{2}m\omega^2\bR^2}.
\end{align}
Using the quantization rules, we get the Hamiltonian operator:
\begin{align}
    \text{Hamiltonian operator}\qquad\highlight{H=\frac{\BP^2}{2m}+\frac{1}{2}m\omega^2\BR^2}.
\end{align}
Since the Hamiltonian is time-independent, we shall solve its eigenequation $H\ket{\psi}=E\ket{\psi}$, where $\ket{\psi}\in\E_{\bR}$ the state space 
of the particle in three-dimensional space.

\begin{emphasizer}[Isotropic QHO]
    Due to $V(\bR)$ only depends on the distance $r=|\bR|$ of the particle from the origin (invariant to rotations), 
    this harmonic oscillator is said to be \bfemph{isotropic}. 
    \begin{align}
        V(\bR)=\frac{m}{2}(\omega^2_xx^2+\omega^2_yy^2+\omega^2_zz^2).
    \end{align}
    It is defined by a single frequency $\omega=\omega_x=\omega_y=\omega_z$.
\end{emphasizer}
%%
\subsection{Separation of variables in cartesian coordinates}
We assume the state space is a separable function so that:
\begin{align}
    \E_{\bR}=\E_x\otimes\E_y\otimes\E_z.
\end{align}
The expression for the Hamiltonian is therefore
\begin{align}
    H=\frac{1}{2m}(P_x^2+P_y^2+P_z^2)+\frac{1}{2}m\omega(Z^2+Y^2+Z^2)=H_z+H_y+H_z,\quad\text{with}\quad H_k=\frac{P_k^2}{2m}+\frac{1}{2}m\omega^2R_k^2.
\end{align}
Each operator acts in its own state space $\E_k$. $H_x,H_y,H_z$ commute, and also with the sum $H$. Consequently, the eigenequation can be solved by 
seeking the eigenvectors of $H$ that are also eigenvectors of $H_x,H_y,H_z$:
\begin{align*}
    \begin{array}{l}H_x\ket{\varphi_{n_x}}=\left(n_x+\frac{1}{2}\right)\hbar\omega\ket{\varphi_{n_x}}\;\quad\ket{\varphi_{n_x}}\in\E_x\\
    H_y\ket{\varphi_{n_y}}=\left(n_y+\frac{1}{2}\right)\hbar\omega\ket{\varphi_{n_y}}\;\quad\ket{\varphi_{n_y}}\in\E_y\\
    H_z\ket{\varphi_{n_z}}=\left(n_z+\frac{1}{2}\right)\hbar\omega\ket{\varphi_{n_z}}\;\quad\ket{\varphi_{n_z}}\in\E_z\end{array},\quad n_x,n_y,n_z\in\mathbb{N}^+_0.
\end{align*}
The eigenstates common to $H,H_x,H_y,H_z$ are of the form:
\begin{align}
    \ket{\psi_{n_x,n_y,n_z}}=\ket{\varphi_{n_z}}\ket{\varphi_{n_y}}\ket{\varphi_{n_z}}.
\end{align}
According to the above Hamiltonian and the eigenequations, we have 
\begin{align}
    H\ket{\psi_{n_x,n_y,n_z}}=E_{n_x,n_y,n_z}\ket{\psi_{n_x,n_y,n_z}}=\left(n_x+n_y+n_z+\frac{3}{2}\right)\hbar\omega\ket{\psi_{n_x,n_y,n_z}}.
\end{align}
The eigenvectors of $H$ are seen to be the \textbf{tensor product} os the eigenvectors of $H_x,H_y,H_z$ while the eigenvalues of $H$ to be the \textbf{sum} of 
eigenvalues of these operators. The energy levels $E_n$ of the isotropic 3D QHO are of the form:
\begin{align}
    \text{Energy levels}\qquad\highlight{E_n=\left(n+\frac{3}{2}\right)\hbar\omega,\quad n=n_x+n_y+n_z\in\mathbb{N}_0^+}.
\end{align}
The $a$ operators are defined analogolously:
\begin{align}
    \text{Operator $a$ and $a^\dagger$}\qquad\highlight{
        \begin{array}{l}a_j=\dfrac{1}{\sqrt{2}}\left(\dfrac{R_j}{\sigma_j}+\dfrac{iP_j\sigma_j}{\hbar}\right)\\
            a^\dagger_j=\dfrac{1}{\sqrt{2}}\left(\dfrac{R_j}{\sigma_j}-\dfrac{iP_j\sigma_j}{\hbar}\right)
        \end{array},\quad\text{with}\quad[a_i,a^\dagger_j]=\delta_{ij}}.
\end{align}
The action of $a_x$ and $a_x^\dagger$ on the state $\ket{\psi_{n_x,n_y,n_z}}$ is:
\begin{align}
    a_x\ket{\psi_{n_x,n_y,n_z}}&=(a_x\ket{\varphi_{n_x}})\ket{\varphi_{n_y}}\ket{\varphi_{n_z}}=\sqrt{n_x}\ket{\varphi_{n_x-1}}\ket{\varphi_{n_y}}\ket{\varphi_{n_z}}=\sqrt{n_x}\ket{\varphi_{n_x-1,n_y,n_z}}\\
    a_x^\dagger\ket{\psi_{n_x,n_y,n_z}}&=(a_x^\dagger\ket{\varphi_{n_x}})\ket{\varphi_{n_y}}\ket{\varphi_{n_z}}=\sqrt{n_x+1}\ket{\varphi_{n_x+1}}\ket{\varphi_{n_y}}\ket{\varphi_{n_z}}=\sqrt{n_x+1}\ket{\varphi_{n_x+1,n_y,n_z}}.
\end{align}
And for the other dimensions is analogous. We also know that 
\begin{align*}
    \ket{\varphi_{n_x}}=\frac{1}{\sqrt{n_x!}}(a_x^\dagger)^{n_x}\ket{\varphi_0},\quad a_x\in\E_x:\quad a_x\ket{\varphi_0}=0.
\end{align*}
In $\E_y$ and $\E_z$ we have something similar. Consequently, we can write 
\begin{align}
    \text{Excited state}\qquad\highlight{\ket{\psi_{n_x,n_y,n_z}}=\frac{1}{\sqrt{n_x!n_y!n_z!}}(a_x^\dagger)^{n_x}(a_y^\dagger)^{n_y}(a_z^\dagger)^{n_z}\ket{\psi_{0,0,0}}},
\end{align}
where $\ket{\psi_{0,0,0}}$ is the tensor product of the ground states of each dimension:
\begin{align*}
    a_x\ket{\psi_{0,0,0}}=a_y\ket{\psi_{0,0,0}}=a_z\ket{\psi_{0,0,0}}=0.
\end{align*}
Finally, the associated wave functions is of the form 
\begin{align}
    \braket{\bR|\psi_{n_x,n_y,n_z}}=\bra{z}\bra{y}\braket{x|0}\ket{0}\ket{0}=\varphi_{n_x}(x)\varphi_{n_y}(y)\varphi_{n_z}(z),
\end{align}
where $\varphi_{n_x},\varphi_{n_y},\varphi_{n_z}$ are stationary wave functions of the one-dimensional harmonics oscillator.
For instance,
\begin{align}
    \braket{\bR|\psi_{0,0,0}}=\left(\frac{1}{\pi\sigma^2}\right)^{3/4}e^{-\frac{1}{2\sigma^2}(x^2+y^2+z^2)},\quad\omega=\omega_x=\omega_y=\omega_z.
\end{align}
%
\subsection{Degeneracy of the energy levels}
We have that $\{H_x,H_y,H_z\}$ constitutes a CSCO in $\E_{\bR}$ (Others CSCOs are $\{H_x,H_y,H\},\{X,P_y,H_z\}$) so that there exists a unique ket $\ket{\psi_{n_x,n_y,n_z}}$ corresponding to a given set of eigenvalues for $H_x,H_y,H_z$.
However, $H$ alone does not form a CSCO since the energy levels $E_n$ are \textbf{degenerate}. Choosing an eigenvalue of $H$, $E_n=(n+3/2)\hbar\omega$, all the kets $\{\ket{\psi_{n_x,n_y,n_z}}\}$ basis that 
satisfy 
\begin{align*}
    n_x+n_y+n_z=n
\end{align*}
are eigenvectors of $H$ with eigenvalue $E_n$.
The degree of degeneracy $g_n$ of $E_n$ is equal to the number of different sets $\{n_x,n_y,n_z\}$ satisfying the above equation. It is equal then to 
\begin{align}
    \text{Degree of degeneracy of $E_n$}\qquad\highlight{g_n=\frac{(n+1)(n+2)}{2}}.
\end{align}
Therefore, only the ground state $E_n=3\hbar\omega/2$ is non-degenerate.

\begin{example}{Measurements of energies}
    \begin{enumerate}[itemsep=0pt,topsep=0pt,label=\alph*)]
        \item Measurement with $H$ and result if $\hbar\omega(1+3/2)$, then $n_x+n_y+n_z=1$.
        \item Measurement with $H_x$ and result is $n_x=0$, then $n_y+n_z=1$.
        \item Measurement with $H_z$ and result is $n_z=1$, then $n_y=0$.
    \end{enumerate}
    Therefore, the state is 
    \begin{align*}
        \ket{0,0,1}=\ket{n_x=0,n_y=0,n_z=1}.
    \end{align*}
\end{example}





