\section{The density operator}

To determine the state of a system at a given time, it suffices to perform on the system a set of measurements corresponding to a CSCO.
However, in practice, the state of the system is often not perfectly determined.
How can we incorporate into the formalism the incomplete information we possess about the state of the system, so that our predictions make 
maximum use of this partial information? We will then introduce the \bfemph{density operator}.

%%
\subsection{Concept of a statistical mixture of states}
When one has incomplete information about a system, one typically appeals to the concept of probability. This incomplete information is 
presented in the following way:
\begin{emphasizer}
    The state of this system may be either the state $\ket{\psi_1}$ with probability $p_1$ or $\ket{\psi_2}$ with probability $p_2$. Obviously,
    \begin{align*}
        \sum_kp_k=1.
    \end{align*}
\end{emphasizer}
We say then we are dealing with a \bfemph{statistica mixture} of states $\ket{\psi_1},\ket{\psi_2},\cdots$ with probabilities $p_1,p_2,\cdots$.

\begin{itemize}[itemsep=0pt,topsep=0pt]
    \item The various states are not necessarily orthogonal. However, they can always be chosen normalized.
    \item Probabilities intervene at two different levels: a) initial information about the system, b) postulates concerning the meassurement nature.
    \item It is impossible, in general, to describe a statistical mixture by an average state vector which would be a superposition of the states $\ket{\psi_k}$.
\end{itemize}

%%
\subsection{The pure case}
The density operator is an \textbf{average operator} which permits a simple description of the statistical mixture of states. 
We will first consider the case where the state of the system is perfectly known, that is, a \bfemph{pure state}. Characterizing the system by its 
state vector if completely equivalent to characterizing it by acertain operator acting in the state space.
%%
\subsubsection{Description by a state vector}
Let be a system whise state vector is 
\begin{align*}
    \ket{\psi(t)}=\sum_nc_n(t)\ket{u_n},\quad\text{with}\quad\sum_n|c_n(t)|^2=1.
\end{align*}
If $A$ is an observable with $A_{np}=\braket{u_n|A|u_p}$, then the mean value of $A$ is
\begin{align*}
    \braket{A}(t)=\braket{\psi(t)|A|\psi(t)}=\sum_{n,p}c_n^*(t)c_p(t)A_{np}.
\end{align*}
Finally, the evolution of $\ket{\psi(t)}$ is 
\begin{align*}
    i\hbar\partial_t\ket{\psi(t)}=H(t)\ket{\psi(t)}.
\end{align*}
%%
\subsubsection{Description by a density operator}
We introduce the density operator $\rho(t)$ as 
\begin{align}
    \text{Density operator}\qquad\highlight{\rho(t)=\ket{\psi(t)}\bra{\psi(t)}}.
    \label{eq:densityoperator}
\end{align}
The density operator is represented in $\{\ket{u_n}\}$ basys by a matrix called te \bfemph{density matrix} whose elements are:
\begin{align*}
    \rho_{pn}(t)=\braket{u_p|\rho(t)|u_n}=c_n^*(t)c_p(t).
\end{align*}
The specification of $\rho(t)$ suffices to characterize the quantum state of the system. 

First, we have the following normalization condition
\begin{align*}
    \text{Normalization condition}\qquad\sum_n|c_n(t)|^2=\sum_n\rho_{nn}(t)=\tr{\rho(t)}=1.
\end{align*}
Secondly, the mean value of $A$ is 
\begin{align*}
    \text{Mean value of $A$}\qquad\braket{A}(t)=\sum_{n,p}\braket{u_p|\rho(t)|u_n}\braket{u_n|A|u_p}=\sum_p\braket{u_p|\rho(t)|u_p}=\tr{\rho(t)A}.
\end{align*}
Finally, the time evolution of the operator can be deduced from the Schrodinger equation above:
\begin{align}
    \text{Time evolution of $\rho(t)$}\qquad \partial_t\rho(t)&=(\partial_t\ket{\psi(t)})\bra{\psi(t)}+\ket{\psi(t)}(\partial_t\bra{\psi(t)})\notag\\
    &=\frac{1}{i\hbar}H(t)\ket{\psi(t)}\bra{\psi(t)}-\frac{1}{i\hbar}\ket{\psi(t)}\bra{\psi(t)}\notag\\
    \partial_t\rho(t)&=\frac{1}{i\hbar}[H(t),\rho(t)].\label{eq:timeevolution_purecase}
\end{align}
The probabilities $P(a_n)$ are then given by  
\begin{align*}
    P(a_n)=\tr{P_n\rho(t)},\quad P_n=\text{Eigensubspace of $a_n$}.
\end{align*}

%%
\subsubsection{Properties of the density operator in a pure case}
In a pure case, a system can be described just as well by a density operator as by a state vector. However, the density operator presents a certain 
number of advantages. Using this operator eliminates the drawbacks related to the existence of an arbitrary global phase factor for the state vector.
Also, by looking the above formulas we see that the expression are linear with respect to $\rho(t)$.
Furtheremore, we have 
\begin{align}
    \rho^\dagger(t)=\rho(t),\quad \underbrace{\rho^2(t)=\rho(t),\quad\tr{\rho^2(t)}=1}_{\text{Only for pure case}}.
\end{align}

%%
\subsection{A statistical mixture of states (non-pure case)}
\subsubsection{Definition of the density operator}
Lets consider a system for which the various probabilities are arbitrary, on the condition that they satisfy the relations:
\begin{align*}
\left\{\begin{array}{l}
    0\leq p_1,p_2,\cdots,p_k,\cdots\leq1\\
    \sum_kp_k=1
\end{array}\right.
\end{align*}
How does one calculate hte probability $P(a_n)$ that a measurement of the observable $A$ will yield the result $a_n$?
Let $P_k(a_n)=\braket{\psi_k|P_n|\psi_k}$ be the probability of finding $a_n$ if the state vector were $\ket{\psi_k}$. 
To obtain the desired probability $P(a_n)$, one must weight $P_k(a_n)$ by $p_k$ and then sum over $k$:
\begin{align}
    P(a_n)=\sum_kp_kP_k(a_n)=\sum_kp_k\tr{\rho_kP_n}=\tr{\sum_kp_k\rho_kP_n}=\tr{\rho P_n},\quad\rho=\sum_kp_k\rho_k.
\end{align}
We see that the linearity of the formulas which use the density operator enables us to express all physical predictions in terms of $\rho$.

The same density operator can be interpreted as several different statistical mixtures of pure states. This situation is sometimes described as the \bfemph{multiple preparations} of the 
same density operator.
%%
\subsubsection{General properties of the density operator}
Since the coefficients $p_k$ are real, $\rho$ is obviously a Hermitian operator.
The trace of $\rho$ is
\begin{align*}
    \tr{\rho}=\sum_kp_k\tr{\rho_k}\stackrel{(a)}{=}\sum_kp_k1=1.
\end{align*}
In $(a)$ we saw that the trace of $\rho_k$ (trace of pure states) is always 1.
We can also generalize the formula of the mean value to statistical mixture:
\begin{align}
    \braket{A}=\sum_na_nP(a_n)=\tr{\rho\sum_na_nP_n}=\tr{\rho A}.
\end{align}
Now let us calculate the time evolution of the density operator. We will assume that, unlike the state of the system, its Hamiltoninan $H(t)$ is well known.
If the system at the initial time $t_0$ has the probability $p_k$ og being the state $\ket{\psi_k}$, then, at a subsequent time $t$, it has the same probability $p_k$ of 
being in the state $\ket{\psi_k(t)}$ given by
\begin{align*}
    \left\{\begin{array}{l}
        i\hbar\partial_t\ket{\psi_k(t)}=H(t)\ket{\psi_k(t)}\\
        \ket{\psi_k(t_0)}=\ket{\psi_k}
    \end{array}\right.
\end{align*}
The density operator at the instant $t$ will then be 
\begin{align}
    \rho(t)=\sum_kp_k\rho_k(t),\quad\text{with}\quad \rho_k(t)=\ket{\psi_k(t)}\bra{\psi_k(t)}.
\end{align}
According to the pure case, $\rho_k(t)$ obeys the evolution equation \eqref{eq:timeevolution_purecase}. Thus,
\begin{align}
\text{Time evolution of $\rho(t)$}\qquad\highlight{i\hbar\partial_t\rho(t)=[H(t),\rho(t)]}.
\end{align}
So, we could generalize most of theequations except to the one pointed out previously. Since $\rho$ is no longer a projector (as in the pure case), we have, in general:
\begin{align*}
    \rho^2(t)\neq\rho(t).
\end{align*}
and, consequently,
\begin{align*}
    \tr{\rho^2}\leq1.
\end{align*}
Finally, we see from a previous equation that, for any ket $\ket{u}$, we have 
\begin{align*}
    \braket{u|\rho|u}=\sum_kp_k\braket{u|\rho_k|u}=\sum_kp_k|\braket{u|\psi_k}|^2\Longrightarrow \braket{u|\rho|u}\geq0.
\end{align*}
Consequently, $\rho$ is a positive operator.
%%
\subsubsection{Populations; coherences}
What is the physical meaning of the matrix element $\rho_{np}$ in the $\{\ket{u_n}\}$ basis?
We analyze first the diagional elements $\rho_{nn}$:
\begin{align*}
    \rho_{nn}=\sum_kp_k[\rho_k]_{nn}=\sum_kp_k|c_n^{(k)}|^2,\quad\text{with}\quad |c_n^{(k)}|^2\geq0.
\end{align*}
$\rho_{nn}$ represents the average probability of finding the system in the state $\ket{u_n}$. Thats why $\rho_{nn}$ is called the population of the state $\ket{u_n}$.

A similar calculation can be carried out for non-diagonal elements $\rho_{np}$:
\begin{align*}
    \rho_{np}=\sum_kp_kc_n^{(k)}c_p^{(k)*}.
\end{align*}
We see that $c_n^{(k)}c_p^{(k)*}$ is a cros term. It reflects the \textbf{interference effects} between the states $\ket{u_n}$ and $\ket{u_p}$ which can appear when the 
state $\ket{\psi_k}$ is a coherent linear superposition of these states. $\rho_{np}$ is the average of these cross terms, taken over all possible states of the statistical mixture.
We can see that $\rho_{nn}$ is the sym of real positive numbers, while $\rho_{np}$ is the sum of complex numbers.

If $\rho_{np}\neq0$, means that a certain coherence subsists between these states (interference effects). This is why non-diagonal elements of $\rho$ are often called \bfemph{coherences}.
\begin{itemize}[itemsep=0pt,topsep=0pt]
    \item The distinction between populations and coherences obviously depends on the basis $\{\ket{u_n}\}$ chosen in the state space. Since $\rho$ is Hermitian, it is always 
    possible to find an orthonormal basis $\{\ket{\chi_l}\}$ where $\rho$ is diagonal and can be written as 
    \begin{align*}
        \rho=\sum_l\pi_l\ket{\chi_l}\bra{\chi_l}.
    \end{align*}
    Since $\rho$ is positive and $\tr{\rho}=1$, we have 
    \begin{align*}
        \left\{\begin{array}{l}
            0\leq\pi_l\leq1\\
            \sum_l\pi_l=1
        \end{array}\right.
    \end{align*}
    $\rho$ can thus be considered to describe a statistical mixture of the states $\ket{\chi_l}$ with the probabilities $\pi_l$ (no coherence between the states $\ket{\chi_l}$).
    \item If the kets $\ket{u_n}$ are eigenvectors of the Hamiltoninan $H$ (assumed time-independent), the populations are constant, and the coherences oscillates at the Bohr frequencies of the system.
    \item $\rho$ can have coherences only between states whose populations are not zero ($\rho_{nn}\rho_{pp}\geq|\rho_{np}|^2$).
\end{itemize}

%%
\subsection{Separate description of part of a physical system. Concept of a partial trace}
Consider two different systems (1) and (2) and the global system (1)+(2), whose state space is the tensor product:
\begin{align}
    \E=\E(1)\otimes\E(2).
\end{align}
If $\{u_n(1)\}$ is a basis in $\E(1)$ and $\{v_p(2)\}$ is a basis in $\E(2)$, the ket $\ket{u_n(1)}\ket{v_p(2)}$ form a basis in $\E$.

We shall construct from $\rho$ an operator $\rho(1)$ ($\rho(2)$) acting only in $\E(1)$ ($\E(2)$). This operation will be called \bfemph{partial trace}
with respect to (1) ((2)).

We introduce the operator $\rho(1)$, whose matrix elements are
\begin{align*}
    \braket{u_n(1)|\rho(1)|u_n'(1)}=\sum_p(\bra{u_n(1)}\bra{u_n(2)})\rho(\ket{u_n'(1)}\ket{v_p(2)}).
\end{align*}
$\rho(1)$ is obtained from $\rho$ by perfoming a partial trace on (2): $\rho(1)=\text{Tr}_2\rho$. Similarly, the operator $\rho(2)=\text{Tr}_1\rho$ has matrix elements:
\begin{align*}
    \braket{v_p(2)|\rho(2)|v_p'(2)}=\sum_n(\bra{u_n(1)}\bra{v_p(2)})\rho(\ket{u_n(1)}\ket{v_p'(2)}).
\end{align*}
We now that the total trace of $\rho$ is:
\begin{align*}
    \tr{\rho}=\sum_n\sum_p\left(\bra{u_n(1)}\bra{v_p(2)}\right)\rho\left(\ket{u_n(1)}\ket{v_p(2)}\right).
\end{align*}
For the partial traces, the indices $n$ and $n'$ (or $p$ and $p'$) are not required to be equal and the summation is performed only over $p$ (or $n$). We have, moreover:
\begin{align*}
    \tr{\rho}=\text{Tr}_1[\text{Tr}_2(\rho)]=\text{Tr}_2[\text{Tr}_1(\rho)].
\end{align*}
$\rho(1)$ and $\rho(2)$ are, like $\rho$, operators whose trace is unitary, Hermitian, and satisfy all the general properties of a density operator.

Now, let $A(1)$ be an observable acting on $\E(1)$, and $\tilde{A}(1)=A(1)\otimes\mathds{1}(2)$ its extension in $\E$.
We have that,
\begin{align*}
    \braket{\tilde{A}(1)}=&\tr{\rho\tilde{A}(1)}\\
    =&\sum_{n,p}\sum_{n',p'}\left(\bra{u_n(1)}\bra{v_p(2)}\right)\rho\left(\ket{u_n'(1)}\ket{v_p'(2)}\right)\cdot\left(\bra{u_n'(1)}\bra{v_p'(2)}\right)A(1)\otimes\mathds{1}(2)\left(\ket{u_n(1)}\ket{v_p(2)}\right)\\
    =&\sum_{n,p,n',p'}\left(\bra{u_n(1)}\bra{v_p(2)}\right)\rho\left(\ket{u_n'(1)}\ket{v_p'(2)}\right)\cdot\braket{u_n'(1)|A(1)|u_n(1)}\underbrace{\braket{v_p'(2)|v_p(2)}}_{\delta_{pp'}}\\
    =&\sum_{n,n'}\underbrace{\left[\sum_p\braket{u_n(1)v_p(2)|\rho|u_n'(1)v_p(2)}\right]}_{\rho(1)}\braket{u_n'(1)|A(1)|u_n(1)}\\
    =&\sum_n\braket{u_n(1)|\rho(1)A(1)|u_n(1)}\\
    \braket{\tilde{A}(1)}=&\tr{\rho(1)A(1)}.
\end{align*}
The partial trace $\rho(1)$ enables us to calculate all the mean values $\braket{\tilde{A}(1)}$ as if the system (1) were and had $\rho(1)$ for a density operator.
We see that $\rho(1)$ also enables us to obtain the probabilities of all the results of measurements bearing on system (1) alone.


\begin{itemize}[itemsep=0pt,topsep=0pt]
    \item Sometimes, it is impossible to assign a state vector to system (1) or (2) when the state of the global system (1)+(2) is not a product state. However, one can always, thanks to the partial trace operation, assign a density operator to subssystem (1) and (2).
    \item The traces $\tr{\rho^2(1)}$ and $\tr{\rho^2(2)}$ are not generally equal to 1.
    \item If the global system is in product state $\ket{\psi}=\ket{\varphi(1)}\ket{\chi(2)}$, we can verify that the corresponding density operator is
    \begin{align}
        \rho=\sigma(1)\otimes\tau(2),\quad\text{where}\quad\begin{array}{l}
            \sigma(1)=\ket{\varphi(1)}\bra{\varphi(1)}\\
            \tau(2)=\ket{\chi(2)}\bra{\chi(2)}
        \end{array}.
    \end{align}  
    The partial trace operation then yields:
    \begin{align}
        \text{Tr}_2[\sigma(1)\otimes\tau(2)]=\sigma(1),\quad\text{and}\quad\text{Tr}_1[\sigma(1)\otimes\tau(2)]=\tau(2).
    \end{align}
    \item If $\rho$ cannot be factored out as in the previous case, there is therefore a certain correlation between systems (1) and (2), which is no longer contained in the operator 
    $\rho=\rho(1)\otimes\rho(2)$.
    \item If the evolution of the global system is governed by \eqref{eq:timeevolution_purecase}, it is in general impossible to find a Hamiltonian operator relating to system (1) alone that would enable us to write an analogous equation for $\rho(1)$.
    The evolution of $\rho(1)$ is much more difficult to describe.
\end{itemize}