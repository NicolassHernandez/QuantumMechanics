\section{Evolution operator}

The transformation of $\ket{\psi(t_0)}$ into $\ket{\psi(t)}$ is linear. Therefore, there exists a linear operator $U(t,t_0)$ such that 
\begin{align}
    \text{Evolution operator}\qquad\highlight{\ket{\psi(t)}=U(t,t_0)\ket{\psi{(t_0)}}},
    \label{eq:evolutionoperator}
\end{align}
where $U(t,t_0)$ is the \bfemph{evolution operator}\index{Evolution operator} of the system.
%
\subsection{General properties}
From \eqref{eq:evolutionoperator} we know that 
\begin{align}
    U(t_0,t_0)=\mathds{1}.
    \label{eq:property1evolutionoperator}
\end{align}
If we substitute the linear operator into the Shcrodinger equation, we obtain:
\begin{align}
    i\hbar\frac{\partial}{\partial t}U(t,t_0)\ket{\psi(t_0)}=H(t)U(t,t_0)\ket{\psi(t_0)}\Longrightarrow i\hbar\frac{\partial}{\partial t}U(t,t_0)=H(t)U(t,t_0).
    \label{eq:differentialequationevolutionoperator}
\end{align}
This a first-order differential equation completely defined $U(t,t_0)$. Equations \eqref{eq:property1evolutionoperator} and \eqref{eq:differentialequationevolutionoperator} 
can be condenses into a single integral form:
\begin{align}
    U(t,t_0)=\mathds{1}=\int_{t_0}^tH(t')U(t',t_0)\;dt.
    \label{eq:integralequationevolutionoperator}
\end{align}

Lets know take three instants $t'',t'',t$ so that $t''<t'<t$, then
\begin{align*}
    \left.\begin{array}{l}
        \ket{\psi(t)}=U(t,t')\ket{\psi(t')}\\
        \ket{\psi(t')}=U(t',t'')\ket{\psi(t'')}
    \end{array}\right\}\Longrightarrow \ket{\psi(t)}=U(t,t')U(t',t'')\ket{\psi(t'')}=U(t,t'')\ket{\psi(t'')}.
\end{align*}
From last expression, we have:
\begin{align}
    U(t,t')U(t',t'')=U(t,t'')
\end{align}
If we set $t=t''$ and interchange the roles of $t$ and $t'$ we have
\begin{align}
    \mathds{1}=U(t',t)U(t,t')\Longrightarrow U(t',t)=U^{-1}(t,t').
\end{align}

On the other hand, the evolution operator between two instants separated by $dt$ is :
\begin{align*}
    d\ket{\psi(t)}=\ket{\psi(t+dt)}-\ket{\psi(t)}=-\frac{i}{\hbar}H(t)|\ket{\psi(t)}\;dt.
\end{align*}
From this we have 
\begin{align*}
    \ket{\psi(t+dt)}=\left[\mathds{1}-\frac{i}{\hbar}H(t)\;dt\right]\ket{\psi(t)}=U(t+dt,t)\ket{\psi(t)}.
\end{align*}
That is, we have the \bfemph{infinitesimal evolution operator}:
\begin{align}
    \text{Infinitesimal evolution operator}\qquad \highlight{U(t+dt,t)=\mathds{1}-\frac{i}{\hbar}H(t)\;dt}.
    \label{eq:infinitesimalevolutionoperator}
\end{align}
Since $H(t)$ is Hermitian, $U(t+dt,t)$ is unitary. It is not surprising that the evolution operator conserves the norm of 
vectors on which it acts. We saw preivously that the norm of the state vector does not change over time.

%%
\subsection{Case of conservative systems}
When the operator $H$ does not depend on time, equation \eqref{eq:differentialequationevolutionoperator} can be integrated easily:
\begin{align}
    U(t,t_0)=e^{-iH(t-t_0)/\hbar}.
\end{align}

Applying this operator on a state vector $\ket{\varphi_{n,\tau}}$ yields:
\begin{align}
    U(t,t_0)\ket{\varphi_{n,\tau}}=e^{-iH(t-t_0)/\hbar}\ket{\varphi_{n,\tau}}=e^{-iE_n(t-t_0)/\hbar}\ket{\varphi_{n,\tau}}.
\end{align}



