\section{Shcrodinger and Heisenberg pictures}

The evolution of the system is entirely contained in that of the state vector $\ket{\psi(t)}$ (written here $\ket{\psi_S(t)})$, and 
is obtained from the Schrodinger equation. This is why this approach is called \emph{Shcrodinger picture}.

We know that all the predictions of quantum mecanics are expressed i terms of scalar products of a bra an a key or matrix elements of operators.
These quatnties are invatiant when the same unitary transformation is performed on the kets and on the operators. This transformation can be chosen so 
as to make the transform of a ket $\ket{\psi_S(t)}$ a time-indenepdent ket.
The transformfs of the observable above then dpeend on time. We thus obtain the \emph{Heisenber picture}.

We will assign the subscript $S$ for Schrodinger and $H$ for Heisenberg pictures.

The state vector $\ket{\psi_S(t)}$ at $t$ is expresseed in terms of $\ket{\psi_S(t_0)}$ by the relation:
\begin{align*}
    \ket{\psi_S(t)}=U(t,t_0)\ket{\psi_S(t_0)}.
\end{align*}
Since this operator is unitary, it is sufficient to perform the unitary transformation associated with the operator $U^\dagger(t,t_0)$ to obtain
a constant transformed vector:
\begin{align*}
    \ket{\psi_H}=U^\dagger(t,t_0)\ket{\psi_S(t)}=U^\dagger(t,t_0)U(t,t_0)\ket{\psi_S(t_0)}=\ket{\psi_S(t_0)}.
\end{align*}
In the Heisenberg picture, the state vector, which is constant, is therefore equal to $\ket{\psi_S(t)}$ at time $t_0$.

The transform $A_H(t)$ of an operator $A_S(t)$ is given by
\begin{align*}
    \highlight{A_H(t)=U^\dagger(t,t_0)A_S(t)U(t,t_0)}.
\end{align*}
$A_H(t)$ generally depends on time, even if $A_S$ does not. Nevertheless, there exists an interesting special case in which, if $A_S$ is 
time-independent, the same is true of $A_H$: the case in whic the system is conservative and $[A_S,H_S]=0$. In this case, we have 
\begin{align*}
    U(t,t_0)=e^{-iH_S(t-t_0)/\hbar}.
\end{align*}
If $[A_S,H_S]=0$, then $[A_S,U(t,t_0)]=0$ so that 
\begin{align*}
    A_H(t)=U^\dagger(t,t_0)U(t,t_0)A_S=A_S.
\end{align*} 
The operators $A_S$ and $A_H$ are simply equal in this case; they indeed correspond to a constant of the motion.

When $A_S(t)$ is arbitrary, the evolution of $A_H(t)$ is:

\begin{align*}
    \frac{d}{dt}A_H(t)=&-\frac{1}{i\hbar}U^\dagger(t,t_0)H_S(t)A_S(t)U(t,t_0)+U^\dagger(t,t_0)\frac{dA_S(t)}{dt}U(t,t_0)+\frac{1}{i\hbar}U^\dagger(t,t_0)A_S(t)H_S(t)U(t,t_0)\\
    =&-\frac{1}{i\hbar}U^\dagger(t,t_0)H_S(t)\textcolor{red}{U(t,t_0)U^\dagger(t,t_0)}A_S(t)U(t,t_0)+U^\dagger(t,t_0)\frac{dA_S(t)}{dt}U(t,t_0)\\
    &+\frac{1}{i\hbar}U^\dagger(t,t_0)A_S(t)\textcolor{red}{U(t,t_0)U^\dagger(t,t_0)}H_S(t)U(t,t_0).
\end{align*}
Then, using the previous definitions:
\begin{align}
    \highlight{i\hbar\frac{d}{dt}A_H(t)=[A_H(t),H_H(t)]+i\hbar\left(\frac{d}{dt}A_S(t)\right)_H}.
\end{align}

\begin{itemize}[itemsep=0pt,topsep=0pt]
    \item The above equation is more general than the evolution of the operator found in the chapter. Also, the evolution of the mean value:
        \begin{align*}
            \braket{A}(t)=\braket{\psi_S(t)|A_S(t)|\psi_S(t)}
        \end{align*}    
        can be calculated, since in $\braket{A}(t)=\braket{\phi_H|A_H(t)|\psi_H}$ only $A_H$ depends on time ans therefore is obtained directly by differentiation.
    \item When the system is composed of a particle of mass $m$ uner a potential. Then we have (1D):
    \begin{align*}
        H_S(t)=\frac{P_S^2}{2m}+V(X_S,t),\quad\text{and therefore}\quad H_H(t)=\frac{P_H^2}{2m}+V(X_H,t).
    \end{align*}
    Using the fact that $[X_H,P_H]=[X_S,P_S]=i\hbar$, we obtain that:
    \begin{align}
        \text{Generalization of Ehrenfest theorem}\qquad\highlight{\begin{array}{l}
            \displaystyle\frac{d}{dt}X_H(t)=\frac{1}{m}P_H(t)\\
            \displaystyle\frac{d}{dt}P_H(t)=-\frac{\partial V}{\partial X}(X_H,t)
        \end{array}}.
    \end{align}
    \begin{emphasizer}
        An advantage of Heisenberg picture is that it leads to equations formally similar to those of classical mechanics.
    \end{emphasizer}
\end{itemize}


