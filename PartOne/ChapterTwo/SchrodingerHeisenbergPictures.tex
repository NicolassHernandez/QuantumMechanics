\section{Shcrodinger and Heisenberg pictures}
%
\subsection{Time-dependent reference frames}
The state of a system can be evaluated in various \bfemph{reference frames} that evolve in time. 

The state $\ket{\psi(t)}$ in one reference frame evolves in time according to the Schrodinger equation
\begin{align*}
    i\hbar\partial_t\ket{\psi(t)}=H(t)\ket{\psi(t)}.
\end{align*}
A second reference frame that may evolve in time relative to the first one if we assume the existence of a time-dependent unitary operator that operates over 
the first frame provifing an effective state in the second frame: 
\begin{align}
    \text{Effective state of the second frame}\qquad\highlight{\ket{\psi_E(t)}=F(t)\ket{\psi(t)},\quad F(t_0)=\mathds{1}}.
\end{align}
The effective state obeys the \bfemph{effective Schrodinger equation} obtained by inserting $\ket{\psi(t)}=F^\dagger(t)\ket{\psi_E(t)}$ in the Schrodinger equation of the first frame:
\begin{align}
    \text{Effective Schrodinger equation and Hamiltonian}\qquad\highlight{\begin{array}{l}
        i\hbar\partial_t\ket{\psi_E(t)}=H_E(t)\ket{\psi_E(t)}\\\\
        H_E(t)=F(t)H(t)F^\dagger(t)-i\hbar F(t)\left(\partial_tF^\dagger(t)\right)
    \end{array}}
\end{align}
Frame transformations are generally used to simplify calculations and time dependence of the Schrodinger equation.
%%
\subsection{Schrodinger, Heisenberg, and interaction pictures}
They are different frames of reference, and are distinguished by the specfic time-dependent unitary transformations.
\begin{itemize}[itemsep=0pt,topsep=0pt]
    \item\textbf{Schrodinger picture} State vectors $\ket{\psi_S(t)}$ evolve in time under the action of the Hamiltonian $\ket{\psi_S(t)}=U(t,t_0)\ket{\psi_S(t)}$. Position and momentum operators have no time dependence.
    \item\textbf{Heisenberg picture} Defined by the adjoint of the evolution operator of the S picture, so that its application on the Schrodinger-picture state vector $\ket{\psi_H}=U^\dagger(t,t_0)\ket{\psi_S(t)}=\ket{\psi_S(t_0)}$ 
    vanishes the time dependence. On the other hand, operators that have no time dependence in the S picture may now depend on time.
    \item\textbf{Interaction picture} Used when the S picture Hamiltonian is time dependent. In this picture, operators and state vectors generally evolve in time.
\end{itemize}
%%
\subsection{Schrodinger picture}
Position and momentum operators have no explicit tiem dependence in this picture. However, other operators with time dependence may be constructed.

The expectation value of an operator $A_S(t)$ wo;; generally have time dependence that results from both the time dependence of $\ket{\psi_S(t)}$ and from the operator itself:
\begin{align}
    \braket{A_S(t)}(t)=\braket{\psi_S(t)|A_S(t)|\psi_S(t)}\longrightarrow\highlight{\frac{d}{dt}\braket{A_S}=\frac{1}{i\hbar}\braket{[A_S,H_S]}+\braket{\frac{\partial A_S}{\partial t}}}.
\end{align}

Ehrenfest's equations are obtained by replacing $A_S(t)$ with the position and momentum operator $\bm{R}=(X,Y,Z)$ and $\bm{P}=(P_x,P_y,P_z)$, and noting that $\partial_t\bm{R}=\partial_t\bm{P}=0$:
\begin{align}
    \text{Ehrenfest's equations }\qquad\highlight{\begin{array}{l}
        \dfrac{d}{dt}\braket{\bm{R}}=\dfrac{1}{m}\braket{\bm{P}}\\\\
        \dfrac{d}{dt}\braket{\bm{P}}=-\braket{\nabla V(\bm{R})}
    \end{array},\quad\braket{V(\bm{R})}=\bigr[\braket{\partial_XV(\bm{R})},\braket{\partial_YV(\bm{R})},\braket{\partial_ZV(\bm{R})}\bigr]}.
\end{align}
%%
\subsection{Heisenberg picture}
This picture vanishes the time dependence of S picture state vector by applying the adjoint of the time evolution oeprator $U^\dagger(t,t_0)$, which defines a unitary transformation.

An arbitrary operator $A_S(t)$ in the S picture is transformed to the H picture as 
\begin{align*}
    \text{Heisenberg operator}\qquad\highlight{A_H(t)=U^\dagger(t,t_0)A_S(t)U(t,t_0)}.
\end{align*}
The time-dependent expectation value of $A_H(t)$ in the H picture is equivalent to that of $A_S(t)$, which must be the same in any picture.
The evolution of the operator $A_H(t)$ is then given by:
\begin{align*}
    \frac{d}{dt}A_H(t)=&-\frac{1}{i\hbar}U^\dagger(t,t_0)H_S(t)A_S(t)U(t,t_0)+U^\dagger(t,t_0)\frac{dA_S(t)}{dt}U(t,t_0)+\frac{1}{i\hbar}U^\dagger(t,t_0)A_S(t)H_S(t)U(t,t_0)\\
    =&-\frac{1}{i\hbar}U^\dagger(t,t_0)H_S(t)\textcolor{red}{U(t,t_0)U^\dagger(t,t_0)}A_S(t)U(t,t_0)+U^\dagger(t,t_0)\frac{dA_S(t)}{dt}U(t,t_0)\\
    &+\frac{1}{i\hbar}U^\dagger(t,t_0)A_S(t)\textcolor{red}{U(t,t_0)U^\dagger(t,t_0)}H_S(t)U(t,t_0)\\
    i\hbar\partial_tA_H(t)&=[A_H(t),H_H(t)]+i\hbar U^\dagger(t,t_0)(\partial_tA_S(t))U(t,t_0).
\end{align*}

In the table, $H_S$ and $\ket{\psi_S(t)}$ are the S picture Hamiltonian and state vector.
\begin{table}[h!]
    \centering
    \renewcommand{\arraystretch}{1.5}{
    \begin{tabular}{ll}
        \multicolumn{2}{c}{Heisenberg picture quantities and dynamics}\\
        \hline
        $\ket{\psi_H}$&$\equiv U^\dagger(t,t_0)\ket{\psi_S(t)}=\ket{\psi_S(t_0)}$\\
        $A_H(t)$&$\equiv U^\dagger(t,t_0)A_S(t)U(t,t_0)$\\
        $H_H$&$=H_S,\text{ for time-independent $H_S$}$\\
        $H_H(t)$&$=H_S(t),\text{ for $[H_S(t),H_S(t')]=0$}$\\
        $\braket{A_H(t)}(t)$&$=\braket{\psi_H|A_H(t)|\psi_H}$\\
        &$=\braket{\psi_S(t_0)|U^\dagger(t,t_0)A_S(t)U(t,t_0)|\psi_S(t_0)}$\\
        &$=\braket{\psi_S|A_S(t)|\psi_S(t)}=\braket{A_S(t)}(t)$\\
        $i\hbar\partial_tA_H(t)$&$=[A_H(t),H_H(t)]+i\hbar U^\dagger(t,t_0)\left(\partial_tA_S(t)\right)U(t,t_0)$
    \end{tabular}}
\end{table}
\begin{emphasizer}
    The effective Hamiltonian of the H picture is $H_E=0$. Therefore, the effect Schrodinger equation $i\hbar\partial_t\ket{\psi_H}=0$ is solved by $\ket{\psi_H}=\ket{\psi_S(t_0)}$.
    In the H picture, only operators evolve in time following the ODE in the last line of the table.
\end{emphasizer}
\begin{emphasizer}
    An advantage of Heisenberg picture is that it leads to equations formally similar to those of classical mechanics.
\end{emphasizer}
%%
\subsection{Interaction picture}
Obtained with a unitary transformation of state vectors and operators of the S picture.
This picture removes some of the time dependence of the S picture state vectors, while also altering the time dependence of operators. 
The interaction picture is typically used with a time-dependent S picture Hamiltonian 
\begin{align}
    \text{Time-dependent S picture Hamiltonian}\qquad\highlight{H_S(t)=H_0+W(t)},
\end{align}
where the eigenstates of $H_0$ are knwon, and $W(t)$ is a time-dependent \bfemph{perturbation} which induces time-dependent dynamics and \bfemph{transitions}
between the eigenstates of $H_0$. To transform into the interaction picture, an evolutino operator $U_0(t,t_0)=e^{-iH_0(t-t_0)/\hbar}$ is associated with $H_0$.
The transformed state vector and arbitrary operator expressed in the interaction picture are 
\begin{align}
    \text{Transformed state vector and operator}\qquad\highlight{\begin{array}{l}
        \ket{\psi_I(t)}=U_0^\dagger(t,t_0)\ket{\psi_S(t)}\\
        A_I(t)=U^\dagger_0(t,t_0)A_S(t)U_0(t,t_0)
    \end{array}}
\end{align}
The effective Schrodinger equation in the interaction picture is 
\begin{align}
    i\hbar\partial_t\ket{\psi_I(t)}=H_E(t)\ket{\psi_I(t)},\quad\text{where}\quad H_E(t)=U^\dagger_0(t,t_0)W(t)U(t,t_0)
\end{align}
is the effective Hamiltonian. If $W(t)=0$, then the interaction picture reduces to the H picture: $H_E(t)=0$ and $\ket{\psi_I(t)}=\ket{\psi_S(t_0)}$.
