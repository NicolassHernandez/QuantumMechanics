\section{The physical interpretation of the postulates}

\subsection{Quantization rules are consistent with probabilistic interpretation}
Lets consider a one-dimensional problem. If the particle is in the normalized state $\ket{\psi}$, the probability that a 
measurement of its position will yield a result included between $x$ and $x+dx$ is equal to (equation \eqref{eq:probabilityfunction}):
\begin{align*}
    dP(x)=|\braket{x|\psi}|^2\;dx.
\end{align*}
Now, to the eigenvector $\ket{p}$ of the observable $P$ corresponds the plane wave:
\begin{align}
    \braket{x|p}=(2\pi\hbar)^{-1/2}e^{\frac{ipx}{\hbar}}.
\end{align}
and we have seen that de Broglie relations associate with this wave a well-defined momentum which is precisely $p$.
In ddition, the probability of finding, for a particle in $\ket{\psi}$, a momnetum between $p$ and $p+dp$ is:
\begin{align}
    dP(p)=|\braket{p|\psi}|^2\;dp=|\tilde{\psi}(p)|^2\;dp.
\end{align}
%%
\subsection{The measurement process}
There is the question of the "fundamental" perturbation involved in the observation of quantum system.
The origin of these problems lies in hte fact that the system under study is treated independently from the measurement device, although hteir interaction s 
essential to the observation process. One should actually consider the system and the measurement device together as a whole. This raise 
delicate questions concerning the details of the measurement process.

The nondeterministic formulation of the fourth and fifth postulates is related to the problems that we have mentiones. OFr example, the abrup change from one stat evector 
to another de to the measurement corrsponds to the fundamental perturbation of which we have spoken.
We shall consider here only ideal measurements: the perturbation they provoke is due only to the quantum mechanical aspect of th emeasurement.
Of course, real devices always present imperfections that affect the measurement and the system.
%%
\subsection{Mean value of an observable in a given state}
The predictions deduced from the fourth postulate are expressed in terms of probabilities. To verigy them, it would be necessary to perform a large nuvmber of measurements
under identical ocnditions. This means measuring the same quantity in a large number of systems which are all in the same quantum state.
If these predictions are correct, the proportion of $N$ identical experiments resulting in a given event will approach, as $N\to\infty$, the 
theorically predicted probability $P$ of this event. In practice, of course, $N$ is finite, and statisticla techniques must be used to interpret the results.

The \bfemph{mean value of an observable}\index{Mean value of an observable} $A$ in the state $\ket{\psi}$, which we shall denote by 
$\braket{A}_\psi$, or $\braket{A}$, is defined as the average of the results obtained when a large number $N$ of measurements of this observable are performed
on systems which are all in the state $\ket{\psi}$. When $\ket{\psi}$ is given, we can compute the probabilities of findins all the possible results, 
and therefore, $\braket{A}_\psi$ is known.

If $\ket{\psi}$ is normalized, $\braket{A}$ is given by 
\begin{align}
    \braket{A}_\psi=\braket{\psi|A|\psi}
\end{align}
Assuming discrete spectrum, out of $N$ measurements of $\mathcal{A}$, the eigenvalue $a_n$ will be obtained $N(a_n)$ times, with 
\begin{align}
    \lim_{N\to\infty}\frac{N(a_n)}{N}=P(a_n),\quad\text{and}\quad\sum_nN(a_n)=N.
\end{align}
In the limit, we can approximate therefore the mean value of the results as 
\begin{align}
    \braket{A}_\psi=\lim_{N\to\infty}\frac{1}{N}\sum_na_nN(a_n)=\sum_na_nP(a_n).
\end{align}
The last expression is then treated:
\begin{align*}
    \braket{A}_\psi&=\sum_na_nP(a_n)=\sum_na_n\braket{\psi|P_n|\psi}=\sum_na_n\sum_{i=1}^{g_n}\braket{\psi|u_n^i}\braket{u_n^i|\psi}=\sum_n\sum_{i=1}^{g_n}\braket{\psi|a_n|u_n^i}\braket{u_n^i|\psi}\\
    &=\sum_n\sum_{i=1}^{g_n}\braket{\psi|A|u_n^i}\braket{u_n^i|\psi}=\bra{\psi}A\left[\sum_n\sum_{i=1}^{g_n}\ket{u_n^i}\bra{u_n^i}\right]\ket{\psi}=\braket{\psi|A\mathds{1}|\psi}=\braket{\psi|A|\psi}.
\end{align*}
In the continuous case, we have something similar:
\begin{align}
    \lim_{N\to\infty}\frac{dN(\alpha)}{N}=dP(\alpha).
\end{align}
In the limit, we can approximate the mean value of the results as 
\begin{align}
    \braket{A}_\psi=\lim_{N\to\infty}\frac{1}{N}\int\alpha\;dN(\alpha)=\int\alpha\;dP(\alpha).
\end{align}
The last expression is then treated:
\begin{align*}
    \braket{A}_\psi&=\int\alpha\;dP(\alpha)=\int\alpha\braket{\psi|v_\alpha}\braket{v_\alpha|\psi}\;d\alpha=\int\braket{\psi|A|v_\alpha}\braket{v_\alpha|\psi}\;d\alpha\\
    &=\bra{\psi}A\left[\int d\alpha\;\ket{v_\alpha}\bra{v_\alpha}\right]\ket{\psi}=\braket{\psi|A\mathds{1}|\psi}=\braket{\psi|A|\psi}.
\end{align*}
\begin{itemize}[itemsep=0pt,topsep=0pt]
    \item If the ket $\ket{\psi}$ is not normalized, then we use 
    \begin{align}
        \text{Mean value of $A$}\qquad\highlight{\braket{A}_\psi=\frac{\braket{\psi|A|\psi}}{\braket{\psi|\psi}}}.
    \end{align}
    \item In practice, one often places oneself in a particular representation to compute $\braket{A}_\psi$.
    \begin{align*}
        \braket{X}_\psi=\braket{\psi|X|\psi}&=\int d^3r\;\braket{\psi|\bR}\braket{\bR|X|\psi}=\int d^3r\;\psi^*(\bR)x\psi(\bR).\\
        \braket{P_x}_\psi=\braket{\psi|P_x|\psi}&=\int d^3r\;\tilde{\psi}^*(\bp)p_x\tilde{\psi}(\bp),\quad\text{or}\\
        \braket{P_x}_\psi=\braket{\psi|P_x|\psi}&=\int d^3r\;\braket{\psi|\bR}\braket{\bR|P_x|\psi}=\int d^3\;\psi^*(\bR)\left[\frac{\hbar}{i}\frac{\partial}{\partial x}\psi(\bR)\right].
    \end{align*}
\end{itemize}


%%
\subsection{The root mean square deviation}
$\braket{A}$ indicates the order of magnitude of the values of the observables $A$ when the system is in the state $\ket{\psi}$.
However, this mean values does not give any idea of the dispersion of the results we expect when measuring $A$.

We therefore define the \bfemph{root mean square deviation} $\Delta A$ as 
\begin{align}
    \text{RMS deviation}\qquad\highlight{\Delta A=\sqrt{\braket{(A-\braket{A})^2}}=\sqrt{\braket{A^2}-\braket{A}^2}}.
\end{align}
If this definition is applied to the observable $\BR$ and $\BP$, we can shown, using commutation realtions, that for any state $\ket{\psi}$,
we have 
\begin{align}
    \text{Heisenberg relations}\qquad\begin{array}{l}
    \Delta X\cdot\Delta P_x\geq\frac{\hbar}{2}\\
    \Delta Y\cdot\Delta P_y\geq\frac{\hbar}{2}\\
    \Delta Z\cdot\Delta P_z\geq\frac{\hbar}{2}
    \end{array}.
\end{align}

%%
\subsection{Compatibility of observables}
\subsubsection{Compatibility and commutation rules}
Let be two commutate observable $A$ and $B$ $[A,B]=0$, and assume discrete spectrum.
There exists a basis of the state space composed of eigenkets commont to $A$ and $B$, which we denote $\ket{a_n,b_p,i}$:
\begin{align*}
    A\ket{a_n,b_p,i}&=a_n\ket{a_n,b_p,i}\\
    B\ket{a_n,b_p,i}&=b_p\ket{a_n,b_p,i}.
\end{align*}
For any $a_n$ and $b_p$, there exists at least one state $\ket{a_n,b_p,i}$ for which a measurement of $A$ will always give $a_n$ and a 
measurement of $B$ will always give $b_p$. These observables which can be simultaneously determined are said to be \bfemph{compatible}\index{Compatible operators}.

The initial state of a system $\ket{\psi}$ can always be written as
\begin{align*}
    \ket{\psi}=\sum_{n,p,i}c_{n,p,i}\ket{a_n,b_p,i}.
\end{align*}
Assume we measure $A$ and then immediately we measure $B$. First, the probability of having $a_n$ is 
\begin{align}
    P(a_n)=\sum_{p,i}|c_{n,p,i}|^2.
\end{align}
When we then measure $B$, the system is no long in the state $\ket{\psi}$ but, if we found $a_n$ in the state $\ket{\psi'_n}$ we have 
\begin{align*}
    \ket{\psi_n'}=\frac{1}{\sqrt{\sum_{p,i}|c_{n,p,i}|^2}}\sum_{p,i}c_{n,p,i}\ket{a_n,b_p,i}.
\end{align*}
The probability of obtaining $b_p$ when it is known that the first measurement was $a_n$ is then
\begin{align}
    P_{a_n}(b_p)=\frac{1}{\sum_{p,i}|c_{n,p,i}|^2}\sum_i|c_{n,p,i}|^2.
\end{align}
The probability $P(a_n,b_p)$ of obtaining $a_n$ in the first measureement and $b_p$ in the second is then a composite event, we must first find $a_n$ and then find $b_p$.
Therefore,
\begin{align}
    P(a_n,b_p)=P(a_n)P_{a_n}(b_p)=\sum_i|c_{n,p,i}|^2.
\end{align}
The state of the system becomes imediately after the second measuremet 
\begin{align}
    \ket{\psi''_{n,p}}=\frac{1}{\sqrt{\sum_{i}|c_{n,p,i}|^2}}\sum_ic_{n,p,i}\ket{a_n,b_p,i}.
\end{align}
\begin{emphasizer}
    $\ket{\psi''_{n,p}}$ is an eigenvector common to $A$ and $B$ with the eigenvalues $a_n$ and $b_p$, respectively.
\end{emphasizer}
If we do the same in opposite order, that is, measuring $B$ and then $A$ we have 
\begin{align}
    P(b_p,a_n)=\sum_i|c_{n,p,i}|^2,\quad\text{and}\quad\ket{\psi''_{p,n}}=\frac{1}{\sqrt{\sum_i|c_{n,p,i}|^2}}\sum_ic_{n,p,i}\ket{a_n,b_p,i}.
\end{align}

\begin{emphasizer}[Consequence of compatible observables]
    When two observables are compatible, the physical predictions ae the \textbf{same}, whatever the order of performing the two measurements.
    The probability and the state after the last measurements are for both cases:
    \begin{align}
        P(a_n,b_p)=P(b_p,a_n)&=\sum_i|c_{n,p,i}|^2=\sum_i|\braket{a_n,b_p,i|\psi}|^2,\quad\text{and}\\
        \ket{\psi''_{n,p}}=\ket{\psi''_{p,n}}&=\frac{1}{\sqrt{\sum_i|c_{n,p,i}|^2}}\sum_ic_{n,p,i}\ket{a_n,b_p,i}.
    \end{align}
    When two observables $A$ and $B$ are compatibles, the measurement of $B$ does not cause any loss of information previously obtained from the measurement 
    of $A$, and viceversa.
\end{emphasizer}
New measurement of $A$ or $B$ will yields the same values again without fail.

\subsubsection{Preparation of a state}