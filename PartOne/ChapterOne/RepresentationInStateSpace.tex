\section{Representations in state space}

Choosing a representation means choosing and orthonormal (discrete or continuous) basis in the state space $\E$.
Vectors and operators are then represented in this basis by \emph{numbers}: components for the vectors, matrix elements for the operators.

We now translate all properties such as orthonormaliation relation and closure relation to the Dirac notation.
%%
\subsection{Relations characteristic of an orthonormal basis}
\subsubsection{Orthonormalization relation}
A set of kets, discrete $\{\ket{u_i}\}$ or continuous $\{\ket{w_\alpha}\}$ is said to be orthonotmal if they satisfy the following equation:
\begin{align}
    \text{Orthonormalization relation}\qquad\highlight{\begin{array}{l}
        \langle u_i|u_j\rangle=\delta_{ij}\\
        \langle w_\alpha|w_{\alpha'}\rangle=\delta(\alpha-\alpha')
    \end{array}}.
    \label{eq:orthonormalizationrelationdirac}
\end{align}
As can be seen, for a continuous set $\langle w_\alpha|w_\alpha\rangle$ \textbf{does not exists}: the $\ket{w_\alpha}$ have an infinite norm and 
therefore not belong to $\E$. Nevertheless, the vectors of $\E$ can be expanded on the $\ket{w_\alpha}$. It is useful then to accept $\ket{w_\alpha}$ as 
\emph{generalized kets}.
%%
\subsubsection{Closure relation}
A discrete or continuous set constitutes a basis if every ket $\ket{\psi}\in\E$ has a \textbf{unique} expansion on $\ket{u_i}$ or $\ket{w_\alpha}$:
\begin{align}
    \text{Closure relation}\qquad\highlight{\begin{array}{l}
        \ket{\psi}=\displaystyle\sum_ic_i\ket{u_i}\\
        \ket{\psi}=\displaystyle\int d\alpha\;c(\alpha)\ket{w_\alpha}
    \end{array}},
    \label{eq:closurerelationdirac}
\end{align}
whose components are obtained multiplying $\bra{u_j}$ ($\bra{\omega_{\alpha'}}$) in the closure relation and using equation \eqref{eq:orthonormalizationrelationdirac}:
\begin{align}
    \text{Coefficient expansion}\qquad\highlight{\begin{array}{l}
        \langle u_j|\psi\rangle=c_j\\
        \langle w_{\alpha'}|\psi\rangle=c(\alpha')
    \end{array}}.
    \label{eq:coefficientexpansiondirac}
\end{align}

We can reexpress the expansion employing the coefficient expansion equations:
\begin{align*}
    \ket{\psi}&=\sum_ic_i\ket{u_i}=\sum_i\langle u_i|\psi\rangle\ket{u_i}=\sum_i\ket{u_i}\langle u_i|\psi\rangle=\left[\sum_i\ket{u_i}\bra{u_i}\right]\ket{\psi}=P_{\{u_i\}}\ket{\psi},\\
    \ket{\psi}&=\int d\alpha\;c(\alpha)\ket{w_\alpha}=\int d\alpha\;\langle w_\alpha|\psi\rangle\ket{w_\alpha}=\int d\alpha\;\ket{w_\alpha}\langle w_\alpha|\psi\rangle=\left[\int d\alpha\;\ket{w_\alpha}\bra{w_\alpha}\right]\ket{\psi}=P_{\{w_\alpha\}}\ket{\psi}.\\
\end{align*}

We then have the projector onto a discrete and continuous basis:
\begin{align}
    \text{Projectors}\qquad\highlight{\begin{array}{l}
        P_{\{u_i\}}=\displaystyle\sum\limits_i\ket{u_i}\bra{u_i}=\mathds{1}\\
        P_{\{w_\alpha\}}=\displaystyle\int d\alpha\;\ket{w_\alpha}\bra{w_\alpha}=\mathds{1}
    \end{array}}\qquad\text{Closure relation},
\end{align}
where $\mathds{1}$ denotes the identiy operator in $\E$. These relations express the fact that $\{\ket{u_i}\}$ and $\{\ket{w_\alpha}\}$ constitute bases.

\begin{table}[h!]
    \centering
    \caption{Fundamental formulas for calculation in the $\{\ket{u_i}\}$ and $\{\ket{w_\alpha}\}$ representations.}
    \renewcommand{\arraystretch}{1.5}
    \begin{tabular}{l|l}
        $\{\ket{u_i}\}$ representation&$\{\ket{w_\alpha}\}$ representation\\
        \hline
        $\langle u_i|u_j\rangle=\delta_{ij}$&$\langle w_\alpha|w_{\alpha'}\rangle=\delta(\alpha-\alpha')$\\
        $P_{\{u_i\}}=\displaystyle\sum_i\ket{u_i}\bra{u_i}=\mathds{1}$&$P_{\{w_\alpha\}}=\displaystyle\int d\alpha\;\ket{w_\alpha}\bra{w_\alpha}=\mathds{1}$
    \end{tabular}
\end{table}
%%
\subsection{Representation of kets and bras}
In the $\{\ket{u_i}\}$ basis, the ket $\ket{\psi}$ is represented by a set of its components $x_i\langle u_i|psi\rangle$. These numbers are 
arranged vertically to form a column matrix. On the other hand, for continuous basis $\{\ket{w_\alpha}\}$, the ket $\ket{\psi}$ is represented by a continuous 
\textbf{infinity} of numbers $c(\alpha)=\langle w_\alpha|\psi\rangle$: a function of $\alpha$. We then draw a vertical axis with the values of $c(\alpha)$:
\begin{align*}
    \ket{\psi}_{\{\ket{u_i}\}}&=\mathds{1}\ket{\psi}=P_{\{\ket{u_i}\}}\ket{\psi}=\sum_i\ket{u_i}\braket{u_i|\psi}=\sum_ic_i\ket{u_i},\\
    \ket{\psi}_{\{\ket{w_\alpha}\}}&=\mathds{1}\ket{\psi}=P_{\{\ket{w_\alpha}\}}\ket{\psi}=\int d\alpha\;\ket{w_\alpha}\braket{w_\alpha|\psi}=\int d\alpha\;c(\alpha)\ket{\psi}.
\end{align*}
Then,
\begin{align}
    \ket{\psi}_{\{\ket{u_i}\}}=\begin{bmatrix}
        \braket{u_1|\psi}\\
        \braket{u_2|\psi}\\
        \vdots\\
        \braket{u_i|\psi}\\\vdots
    \end{bmatrix},\quad\text{and}\quad 
    \ket{\psi}_{\{\ket{w_\alpha}\}}=
    \begin{array}{ll}
    \alpha\downarrow&\begin{bmatrix}
        \vdots\\\vdots\\\braket{w_\alpha|\psi}\\\vdots\\\vdots
    \end{bmatrix}
    \end{array}.
\end{align}


Something similar happens to the respective bras:
\begin{align*}
    \bra{\varphi}_{\{\ket{u_i}\}}&=\bra{\varphi}\mathds{1}=\bra{\varphi}P_{\{\ket{u_i}\}}=\sum_i\braket{\varphi|u_i}\bra{u_i}=\sum_ib^*_i\bra{u_i},\\
    \bra{\varphi}_{\{\ket{w_\alpha}\}}&=\bra{\varphi}\mathds{1}=\bra{\varphi}P_{\{\ket{w_\alpha}\}}=\int d\alpha\;\braket{\varphi|w_\alpha}\bra{w_\alpha}=\int d\alpha\;b^*(\alpha)\bra{\varphi}.
\end{align*}
We can see that the components of the bra are the complex conjugates of the components $b_i=\braket{u_i|\varphi}$ and $b(\alpha)=\braket{w_\alpha|\varphi}$ of the ket $\ket{\varphi}$ associated with $\bra{\varphi}$.

Let us then arrange them horizontally, to form a row matrix:
\begin{align}
    \bra{\varphi}_{\{\ket{u_i}\}}=&\begin{bmatrix}
        \braket{\varphi|u_1}&\braket{\varphi|u_2}&
        \cdots&\braket{\varphi|u_i}&\cdots
    \end{bmatrix},\quad\text{and}\\
    \bra{\varphi}_{\{\ket{w_\alpha}\}}=&
    \begin{array}{c}
    \alpha\\\rightarrow\\
    \begin{bmatrix}
        \cdots&\cdots&\braket{\varphi|w_\alpha}&\cdots&\cdots
    \end{bmatrix}
    \end{array}.
\end{align}

The scalar product is then given by a \textbf{matrix multiplication}:
\begin{align}
    \braket{\varphi|\psi}&=\braket{\varphi|\mathds{1}|\psi}=\braket{\varphi|P_{\{u_i\}}|\psi}=\sum_i\braket{\varphi|u_i}\braket{u_i|\varphi}=\sum_ib_i^*c_i\\
    \braket{\varphi|\psi}&=\braket{\varphi|\mathds{1}|\psi}=\braket{\varphi|P_{\{w_\alpha\}}|\psi}=\int d\alpha\;\braket{\varphi|w_\alpha}\braket{w_\alpha|\psi}=\int d\alpha\;b^*(\alpha)c(\alpha).
\end{align}

%%
\subsection{Representation of operators}
\subsubsection{Representation of $A$ by a square matrix}
Given a linear operator $A$, we can in $\{\ket{u_i}\}$ or $\{\ket{w_\alpha}\}$ basis, associate with it a series of numbers defined by
\begin{align}
    A_{ij}=\braket{u_i|A|u_j},\quad\text{or}\quad A(\alpha,\alpha')=\braket{w_\alpha|A|w_{\alpha'}}.
\end{align}
They are arranged into a square matrix, as 
\begin{align}
    \begin{bmatrix}
        A_{11}&A_{12}&\cdots&A_{1j}&\cdots\\
        A_{21}&A_{22}&\cdots&A_{2j}&\cdots\\
        \vdots&\vdots&&\vdots&\\
        A_{i1}&A_{i2}&\cdots&A_{ij}&\cdots\\
        \vdots&\vdots&&\vdots&
    \end{bmatrix},\quad\text{or}\quad
    \begin{array}{cc}
        &\alpha'\\
        &\longrightarrow\\
        \alpha\downarrow&\begin{bmatrix}
            &&\vdots&\\
            &&\vdots&\\
            \cdots&\cdots&A(\alpha,\alpha')&\cdots\\
            &&\vdots&
        \end{bmatrix}
    \end{array}.
\end{align}

For the case of the matrix representing the operator $AB$ in the $\{\ket{u_i}\}$ basis, we have:
\begin{align*}
    \braket{u_i|AB|u_j}=\braket{u_i|A\mathds{1}B|u_j}=\braket{u_i|AP_{\{u_k\}}B|u_j}=\sum_k\braket{u_i|A|u_k}\braket{u_k|B|u_j}=\sum_kA_{ik}B_{kj}.
\end{align*}

%%
\subsubsection{Representation of the ket $\ket{\psi'}=A\ket{\psi}$}
Knowing the componetns of $\ket{\psi}$ and the matrix elements of $A$ in a given representation, how can we calculate the components of $\ket{\psi'}=S\ket{\psi}$ in the 
same representation?

We know that in the $\{\ket{u_i}\}$ and $\{\ket{w_\alpha}\}$ basis, we have
\begin{align*}
    c_i'=\braket{u_i|\psi'}=\braket{u_i|A|\psi},\quad\text{and}\quad c'(\alpha)=\braket{w_\alpha|\psi'}.
\end{align*}
Inserting the closure relation between $A$ and $\ket{\psi}$:
\begin{align*}
    c_i'&=\braket{u_1|A\mathds{1}|\psi}=\braket{u_i|AP_{\{u_j\}}|\psi}=\sum_j\braket{u_i|A|u_j}\braket{u_j|\psi}=\sum_jA_{ij}c_j\\
   c'(\alpha)&=\braket{w_\alpha|A\mathds{1}|\psi}=\int d\alpha'\;\braket{w_\alpha|A|w_{\alpha'}}\braket{w_{\alpha'}|\psi}=\int d\alpha'\;A(\alpha,\alpha')c(\alpha').
\end{align*}
We see that the column matrix representing $\ket{\psi'}$ is equal to the matrix multiplication of the column matrix $\ket{\psi}$ and the square matrix $A$.
\subsubsection{Expression for the number $\braket{\varphi|A|\psi}$}
On the other hand, we can derive an expression for $\braket{\varphi|A|\psi}$ for both basis:
\begin{align*}
    \braket{\varphi|A|\psi}=&\braket{\varphi|P_{\{u_i\}}AP_{\{u_j\}}|\psi}=\sum_{i,j}\braket{\varphi|u_i}\braket{u_i|A|u_j}\braket{u_j|\psi}=\sum_{i,j}b_i^*A_{ij}c_j,\\
    &\braket{\varphi|P_{\{w_\alpha\}}AP_{\{w_{\alpha'}\}}|\psi}=\iint d\alpha\;d\alpha'\;\braket{\varphi|w_\alpha}\braket{w_\alpha|A|w_{\alpha'}}\braket{w_{\alpha'}|\psi}=\iint d\alpha\;d\alpha'\;b^*(\alpha)A(\alpha,\alpha')c(\alpha').
\end{align*}

Thus, the term $\braket{\varphi|A|\psi}$ which is a number, can be computed as a matrix multiplication of the column vector $\ket{\psi}$ by the matrix $A$, and then by the row vector $\bra{\varphi}$.
%%
\subsubsection{Matrix representation of $A^\dagger$}
The adjoint of $A$ can also be written as:
\begin{align}
    \begin{array}{l}
        (A^\dagger)_{ij}=\braket{u_i|A^\dagger|u_j}=\braket{u_j|A|u_i}^*=A^*_{ji}\\
        A^\dagger(\alpha,\alpha')=\braket{w_\alpha|A^\dagger|w_{\alpha'}}=\braket{w_{\alpha'}|A|w_\alpha}^*=A^*(\alpha',\alpha).
    \end{array}
    \label{eq:adjointoperator}
\end{align}
The matrices representing $A$ and $A^\dagger$ are then Hermitian conjugates of each other.

If $A$ is Hermitian, then $A_{ij}=A^*_{ji}$ and $A(\alpha,\alpha')=A^*(\alpha',\alpha)$. A hermitian operator is therefore represented by a Hermitian matrix.
For $i=j$, $\alpha=\alpha'$ we see that $A_{ii}=A_{ii}^*$, $A(\alpha,\alpha)=A^*(\alpha,\alpha)$. Thus,
\begin{emphasizer}
    The diagonal elements of a Hermitian matrix are therefore always real numbers.
\end{emphasizer}
%%
\subsection{Change of representation}

We can representate a ket $\ket{\psi}$ in different bases, similar to representing a point in space by cartesian and spherical coordinates. 
The relation of these two representation allow us to turn from one to the other easily.

If we assume an old discrete basis $\{\ket{u_i}\}$ to a new one $\{\ket{t_k}\}$, the change of basis is defined by specifying the components $S_{ik}=\braket{u_i|t_k}$ of each 
kets of the new basis in terms of the kets of the old one. The adjoint is: $(S^\dagger)_{ki}=(S_{ik})^*=\braket{t_k|u_i}$. This transformation matrix is \bfemph{unitary}\index{Unitary}: $S^\dagger S=SS^\dagger=I$.


We will derive the transformation for a ket, a bra, and the matrix elements of an operator using the closure relation of each basis as explained in the derivation. 
\begin{align*}
    \braket{t_k|\psi}&=\braket{t_k|\mathds{1}|\psi}=\braket{t_k|P_{\{u_i\}}|\psi}=\sum_i\braket{t_k|u_i}\braket{u_i|\psi}=\sum_iS^\dagger_{ki}\braket{u_i|\psi}\\
    \braket{u_i|\psi}&=\braket{u_i|\mathds{1}|\psi}=\braket{u_i|P_{\{t_k\}}|\psi}=\sum_k\braket{u_i|t_k}\braket{t_k|\psi}=\sum_iS_{ik}\braket{t_k|\psi}\\
    \braket{\psi|t_k}&=\braket{\psi|\mathds{1}|t_k}=\braket{\psi|P_{\{u_i\}}|t_k}=\sum_i\braket{\psi|u_i}\braket{u_i|t_k}=\sum_i\braket{\psi|u_i}S_{ik}\\
    \braket{\psi|u_i}&=\braket{\psi|\mathds{1}|u_i}=\braket{\psi|P_{\{t_k\}}|u_i}=\sum_k\braket{\psi|t_k}\braket{t_k|u_i}=\sum_k\braket{\psi|t_k}S^\dagger_{ki}\\
        A_{kl}&=\braket{t_k|A|t_l}=\braket{t_k|\mathds{1}A\mathds{1}|t_l}=\braket{t_k|P_{\{u_i\}}AP_{\{u_j\}}|t_l}=\sum_{i,j}\braket{t_k|u_i}\braket{u_i|A|u_j}\braket{u_j|t_l}=\sum_{i,j}S^\dagger_{ki}A_{ij}S_{jl}\\
    A_{ij}&=\braket{u_i|A|u_j}=\braket{u_i|\mathds{1}A\mathds{1}|u_j}=\braket{u_i|P_{\{t_k\}}AP_{\{t_l\}}|u_j}=\sum_{k,l}\braket{u_i|t_k}\braket{t_k|A|t_l}\braket{t_l|u_j}=\sum_{k,l}S_{ik}A_{kl}S^\dagger_{lj}.
\end{align*}

The final results are shown in table 
\begin{table}[h!]
    \centering
    \renewcommand{\arraystretch}{1.5}
    \caption{Transformation of a ket, bra, and matrix elements from one basis to another.}
    \begin{tabular}{l|l}
        Transformation&Expression\\
        \hline
        Ket components $\{u_i\}\longrightarrow\{t_k\}$ representation&$\braket{t_k|\psi}=\displaystyle\sum_iS^\dagger_{ki}\braket{u_i|\psi}$\\
        Ket components $\{t_k\}\longrightarrow\{u_i\}$ representation&$\braket{u_i|\psi}=\displaystyle\sum_iS_{ik}\braket{t_k|\psi}$\\
        Bra components $\{u_i\}\longrightarrow\{t_k\}$ representation&$\braket{\psi|t_k}=\displaystyle\sum_i\braket{\psi|u_i}S_{ik}$\\
        Bra components $\{t_k\}\longrightarrow\{u_i\}$ representation&$\braket{\psi|u_i}=\displaystyle\sum_k\braket{\psi|t_k}S^\dagger_{ki}$\\
        Matrix elements $\{u_{i,j}\}\longrightarrow\{t_{k,l}\}$ representation&$A_{kl}=\displaystyle\sum_{i,j}S^\dagger_{ki}A_{ij}S_{jl}$\\
        Matrix elements $\{t_{k,l}\}\longrightarrow\{u_{i,j}\}$ representation&$A_{ij}=\displaystyle\sum_{k,l}S_{ik}A_{kl}S^\dagger_{lj}$
    \end{tabular}
\end{table}
