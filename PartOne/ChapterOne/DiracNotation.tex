\section{Dirac notation}
Each quantum state of a particle will be characterized by a \bfemph{state vector}\index{State vector},  belonging to an abstract space $\E_{\bR}$, called the
\bfemph{state space}\index{State space} of the particle. The fact that the space $\F$ is a subspace of $L^2$ means that $E_{\bR}$ is a subspace 
of a Hilbert space.

The introduction of these quantities permits a generalization of the formalism. In fact,
there exist physical systems whose quantum description cannot be given by a wavefunction.

\begin{definition}[State vector]
    The quantum state of any physical system is characterized by a state vector, belonging to a space $\E$ which is the state 
    space of the system. 
    The state space is the set of all of the possible states in which the system can exist.
\end{definition}
%%
\subsection{Ket and Bra vectors}


%% 
\subsection{Linear operators}

%% 
\subsection{Hermitian conjugation (adjoint)}