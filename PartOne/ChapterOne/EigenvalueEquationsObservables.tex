\section{Eigenvalue equations. Observables}

\subsection{Eigenket and eigenbra equations}

$\ket{\psi}$ is said to be an \bfemph{eigenvector}\index{Eigenvector} (or eigenket) of the linear operator $A$ if
\begin{align}
    \text{Eigenket equation of $A$}\qquad\highlight{A\ket{\psi}=\lambda\ket{\psi},\quad\lambda\in\mathbb{C}}.
    \label{eq:eigenketequation}
\end{align}
This eigenketequation possesses solutions only when $\lambda$ takes on certain values, called \bfemph{eigenvalues}\index{Eigenvalues} of $A$.
The set of the eigenvalues is called \bfemph{spectrum}\index{Spectrum} of $A$.
\begin{emphasizer}[Collinear of an eigenvector is also an eigenvector]
    If $\ket{\psi}$ is an eigenvector of $A$ with eigenvalue $\lambda$, then $\alpha\ket{\psi},\;\alpha\in\mathbb{C}$ is also an eigenvector of $A$.
\end{emphasizer}

The eigenvalue $\lambda$ is called \emph{non-degenerate} (or simple) when its corresponding eigenvector is \textbf{unique} to within a constant factor (collinear).
On the other hand, if there exists at least two linearly independent eigenkets with the \textbf{same} eigenvalue, the eigenvalue is said to be \emph{degenerate}.
Its \emph{degree of degeneracy} $g$ is then the number of linearly independent eigenvectors $\ket{\psi^i},\;i=\{1,2,\cdots,g\}$ associated with it.

The set of eigenkets associated with a degenerate eigenvalue constitutes a \emph{g-dimensional vector space} called \bfemph{eigensubspace} of $\lambda$.

Taking the adjoint of the eigeketnequation yields its corresponding form to eigenbraequation
\begin{align}
    \text{Eigenbra equation of $A^\dagger$}\qquad\highlight{\bra{\psi}A^\dagger=\lambda^*\bra{\psi}}.
    \label{eq:eigenbraequation}
\end{align}
If $\ket{\psi}$ is an eigenket of $A$ with $\lambda$, it can also be said that $\bra{\psi}$ is an eigenbra of $A^\dagger$ with $\lambda^*$.
%%
\subsubsection{Finding the eigenvalues and eigenvector in an operator}
Assuming the state space is of finite dimension $N$, granting the generalization to an infinite-dimensional state space.

Choosing $\{\ket{u_i}\}$, lets us project the vector \eqref{eq:eigenketequation} onto the various orthonormal basis vectors $\ket{u_i}$:
\begin{align}
    \braket{u_i|A|\psi}=\lambda\braket{u_i|\psi}.
\end{align}
Inserting the closure relation between $A$ and $\ket{\psi}$:
\begin{align}
    \braket{u_i|A\mathds{1}|\psi}=\sum_j\underbrace{\braket{u_i|A|u_j}}_{A_{ij}}\underbrace{\braket{u_j|\psi}}_{c_j}=\lambda\underbrace{\braket{u_i|\psi}}_{c_i}
    \longrightarrow\highlight{\sum_j[A_{ij}-\lambda\delta_{ij}]c_j=0}.
    \label{eq:findeigenvalues}
\end{align}
Equation \eqref{eq:findeigenvalues} is a system of equations with $N$ equations and $N$ unknowns $c_j$. It has non-trivial solution iff its characteristic equation is zero:
\begin{align}
    \text{Characteristic equation of the eigenket equation}\qquad\highlight{P(\lambda)=\det[A-\lambda I]=0}.
\end{align}
This expression enable us to determine the spectrum of $A$. The characteristic equation is \textbf{independent} of the representation chosen. Then,
\begin{emphasizer}
    The eigenvalues of an operator are the roots $\lambda$ of its $N$th order characteristic equation $P(\lambda)$.
\end{emphasizer}

%%% important part of eigenvalues degenerates
\subsubsection{Determination of eigenvectors}
Given a transformation $T(v)=Mv:\;V\in\mathbb{C^N}\longrightarrow W\in\mathbb{C}^N$, the theorem says:
\begin{align}
    \text{dim}(V)=\text{rank}(T)+\text{null}(T),
\end{align}
where
\begin{align*}
    \text{dim}(V)&=\text{Number of columns of V}\\
    \text{rank}(T)&=\text{Number of independent equations (non zero rows)}\\
    \text{null}(T)=\text{dim}[\text{ker}(T)]&=\text{Number of free variables, degree of freedom (dof)}.
\end{align*}
In our case, $T(v)=Mv=(A-\lambda I)v$ and $\text{dim}(V)=N$.


Based on the nature of the eigenvalue, we can have different eigenvalues but also repeated. Therefore, we define the 
following useful quantities:
\begin{itemize}[itemsep=0pt,topsep=0pt]
    \item\textbf{Algebraic multiplicity ($AM$)} Number of repetition of the eigenvalue (degree of degeneracy $g$).
    \item\textbf{Geometric multiplicity ($GM$)} Dimension of the subspace that the eigenvalues generate (how many linearly independent eigenvectors exist
    for that eigenvalue).
\end{itemize}

We then can have the following three cases:
\begin{itemize}
    %
    \item\textbf{$AM=GM=1$} Only one eigenvector corresponds to the eigenvalue (within a constant factor).
    At the moment of substituting an eigenvalue $\lambda_0$ into equation \eqref{eq:findeigenvalues} there will be $\text{rank}(M)=N-1$ 
    independent equations so one equation is redundant. When this happens, $\text{null}(M)=1$ free variable (or degree of freedom, dof) $c_1$ is available which can
    be defined arbitrarily and from which all other variables can be expressed.

    If we fix $c_1$, then all $c_j$ are proportional to it:
    \begin{align}
        c_j=\alpha^0_jc_1\quad(\alpha^0_1=1).
    \end{align}
    the $N-1$ coefficients $\alpha^0_j,\;j\neq1$ are determined from the matrix elements $A_{ij}$ and $\lambda_0$. The eigenvectors associated with $\lambda_0$
    differ only by the value chosen for $c_1$. They are therefore all given by
    \begin{align}
        \ket{\psi_0(c_1)}=\sum_j\alpha^0_jc_1\ket{u_j}=c_1\ket{\psi_0},\quad\text{with}\quad\ket{\psi_0}=\sum_j\alpha^0_j\ket{u_j}.
    \end{align}
    \begin{emphasizer}
        When $\lambda_0$ is simple, only one eigenvector corresponds to it.
    \end{emphasizer}
    \begin{example}{Simple eigenvalues}
        In the matrix 
        \begin{align*}
            A=\begin{bmatrix}
                1&1&0\\0&2&1\\0&0&3
            \end{bmatrix}
        \end{align*}
        the eigenvalues are $\lambda\in\{1,2,3\}$. Lets make $\lambda_0=1$ and replace it into the eigenvalue problem:
        \begin{align*}
            (A-\lambda I)v=(A-I)v=\begin{bmatrix}
                0&1&0\\0&1&1\\0&0&2
            \end{bmatrix}=0
        \end{align*}
        We see that there is no value in the first column, meaning that $x_1$ is free whereas $x_2=x_3=0$. Therefore, the eigenvector is $v_1=(1\;0\;0)$.
    \end{example}
    %
    \item\textbf{$AM=GM>1$} When evaluating $\lambda_0$, the system will have $\text{rank}(M)=N-p$ independent equations ($1<p<q$). To the eigenvalue 
    $\lambda_0$ there corresponds an eigensubspace of dimension $\text{null}(M)=p$, and $\lambda_0$ is a p-fold degenerate eiganvalue. 
    
    Assuming that for $\lambda=\lambda_0$ is composed of $N-2$ linearly independent equations. These equations enable us to calculate the coefficients
    $c_j$ in terms of any of them, for example $c_1$ and $c_2$: 
    \begin{align*}
        c_j=\beta_j^0c_1+\gamma_j^0c_2.
    \end{align*} 
    Al the eigenvectors associated with $\lambda_0$ are then of the form
    \begin{align}
        \ket{\psi_0(c_1,c_2)}=c_1\ket{\psi_0^1}+c_2\ket{\psi_0^2},\quad\text{with}\quad \ket{\psi_0^1}=\sum_j\beta_j^0\ket{u_j},\quad\ket{\psi_0^2}=\sum_j\gamma_j^0\ket{u_j}.
    \end{align}
    The vectors $\ket{\psi_0(c_1,c_2)}$ do indeed constitute a two-dimensional vector space, this beeing characteristic of a two-fold degenerate eigenvalue.
    %
    \item\textbf{$AM>GM>1$} In this case, the subspace is less than the degree of degeneracy and therefore not all degenerate eigenvectors are linearly independent. This means that 
    there is not enough information to create a basis. However, techniques such as Jordan canonical form helps to create generalized eigenvector and to span the whole space.
\end{itemize}
\begin{emphasizer}
    When an operator is Hermitian, it can be shown that the degree of degeneracy $p$ of an eigenvalue $\lambda$ is always equal to the muliplicity of the corresponding root in 
    the characteristic equations.
    In a space of finite dimension $N$, a Hermitian operator always has $N$ linearly independent eigenvectors: this operator can therefore be diagonalized.
\end{emphasizer}

%%
\subsection{Observables}
\subsubsection{Properties of the eigenvalues and eigenvectors of a Hermitian operator}
\begin{enumerate}[itemsep=0pt,topsep=0pt,label=\roman*)]
    \item The eigenvalues of a Hermitian operator are real.
    \item Two eigenvectors of a Hermitian operator corresponding to two different eigenvalues are orthogonal.
\end{enumerate}
%%
\subsubsection{Definition of a observable}
Consider a Hermitian operator $A$ with discrete spectrum. The degree of degeneracy of the eigenvalue $a_n$ is denoted by $g_n$. We shall denote
by $\ket{\psi_n^i}$ $g_n$ linearly independent vectors chosen in the eigensubspace $\E_n$ of $a_n$:
\begin{align}
    A\ket{\psi_n^i}=a_n\ket{\psi_n^i},\quad i=1,2,\cdots,g_n.
\end{align}

Every vector of $\E_n$ is orthogonal to every vector of another subspace $\E_{n'}$: $\braket{\psi_n^i|\psi_{n'}^j}=0,\quad n\neq n'$.

Inside the subspace $\E_n$, the $\ket{\psi_n^i}$ can always be chosen orthonormal, such that
\begin{align}
    \braket{\psi_n^i|\psi_n^j}=\delta_{ij}.
\end{align}
If such a choise is made, the result is an orthonormal system of eigenvectors of $A$: the $\ket{\psi_n^i}$ satisfying the relations:
\begin{align}
    \braket{\psi_n^i|\psi_{n'}^{i'}}=\delta_{nn'}\delta_{ii'}.
\end{align}

\begin{definition}[Observable]
    The Hermitian operator $A$ is an \bfemph{observable}\index{Observable} if its eigenvectors \textbf{form a basis} in the state space:
    \begin{align}
        \text{Closure relation of an observable}\qquad\highlight{\sum_{n=1}^\infty\sum_{i=1}^{g_n}\ket{\psi_n^i}\bra{\psi_n^i}=\mathds{1}}.
        \label{eq:closurerelationobservable}
    \end{align}
    The projector onto the subspace $\E_n$ is written as:
    \begin{align}
        P_n=\sum_{i=1}^{g_n}\ket{\psi_n^i}\bra{\psi_n^i}.
    \end{align}
    The observable $A$ is the ngiven by:
    \begin{align}
        \highlight{A=\sum_na_nP_n}.
    \end{align}
\end{definition}

Equation \eqref{eq:closurerelationobservable} can be generalized to include cases of continuous spectrum using the previous table of the first section.

If $A$ has a mixed spectrum, then it is an observable if this system form a basis, that is, if
\begin{align}
    \sum_{n}\sum_{i=1}^{g_n}\ket{\psi_n^i}\bra{\psi_n^i}+\int_{\nu_1}^{\nu_2}d\nu\;\ket{\psi_\nu}\bra{\psi_\nu}=\mathds{1}.
\end{align}

\begin{emphasizer}[The projector $P_\psi$ is an observable]
    The projector $P_\psi=\ket{\psi}\bra{\psi}$ is an observable. We know that it is Hermitian, and that its eigenvalues are 1 and 0, the first one is simple and the other 
    infinitely degenerate. It can be shown that any ket $\ket{\psi}$ can be expanded on these eigenkets, therefore $P_\psi$ is an observable.
\end{emphasizer}
%%
\subsection{Sets of commuting observables}

\subsubsection{Important theorems}
\begin{definition}[Theorem I]
    If two operators $A$ and $B$ commute, and if $\ket{\psi}$ is an eigenvector of $A$, $B\ket{\psi}$ is aso an eigenvectr of $A$, with the same eigenvalue.

    Another form:

    If two operators $A$ and $B$ comute, every eigensubspace of $A$ is globaly invariant under the action of $B$ ($B\ket{\psi}$ belongs to the eigensubspace $\E_a$ of $A$, corresponding to the 
    eigenvalue $a$).
\end{definition} 

\begin{definition}[Theorem II (consequence of theorem I)]
    If two observables $A$ and $B$ commute, and if $\ket{\psi_1}$ and $\ket{\psi_2}$ are two eigenvectors of $A$ with different eigenvalues, the matrix element 
    $\braket{\psi_1|B|\psi_2}$ is zero. 
\end{definition}

\begin{definition}[Theorem III]
    If two observables $A$ and $B$ commute, one can construct an orthonormal basis of the state spce with eigenvectors common to $A$ and $B$.
\end{definition}

Lets prove the theorem III.
Consider two commuting observables $A$ and $B$, with discrete spectrum. Since $A$ is observable, there exists at least one orthonormal system of eigenvectors $\ket{u_n^i}$ which forms a 
basis in the state space:
\begin{align}
    A\ket{u_n^i}=a_n\ket{u_n^i},\quad\begin{array}{l}
        n=1,2,\cdots\\
        i=1,2,\cdots,g_n
    \end{array}
\end{align} 

We also have $\braket{u_n^i|u_{n'}^{i'}}=\delta_{nn'}\delta_{ii'}$. What does the matrix look like which represent $B$ in the $\{\ket{u_n^i}\}$ basis?
We know that the matrix elements $\braket{u_n^i|B|u_{n'}^{i'}}$ are zero when $n\neq n'$ (theorem II). Let us arrange the basis vectors $\ket{u_n^i}$ in the order:
\begin{align*}
    \ket{u_1^1},\ket{u_1^2},\cdots,\ket{u_1^{g_1}};\;\ket{u_2^1},\cdots,\ket{u_2^{g_2}};\;\ket{u_3^1},\cdots
\end{align*}
Whe then obtain for $B$ a block-diagonal matrix of the form:

\begin{align}
    \left[
        \begin{array}{ccc|ccc|c|cc}
        &\E_1&&\E_2&&&\E_3&\cdots&\\
        \E_1&\ddots&\ddots&&0&&0&0&\\
        &\ddots&\ddots&&&&&&\\
        \hline
        \E_2&&&\ddots&\ddots&\ddots&&&\\
        &0&&\ddots&\ddots&\ddots&0&0&\\
        &&&\ddots&\ddots&\ddots&&&\\
        \hline
        \E_3&0&&&0&&\ddots&0&\\
        \hline
        \vdots&0&&&0&&0&\ddots&\ddots\\
        &&&&&&&\ddots&\ddots
    \end{array}\right]
\end{align}

Then the degeneracy of the eigenvalue is 1, then the block reduces to a $1\times1$ matrix. In the column associated with $\ket{u_n}$ all the other matrix 
elements are zero, this expresses the fact that $\ket{u_n}$ is an eigenvector common to $A$ and $B$.
When $a_n$ is a $g_n$-degenerate eigenvalue of $A$, the block which represents $B$ in $\E_n$ is not, in general, diagonal: the $\ket{u_n^i}$ are not, in general, eigenvector of $B$.
The action of $A$ in the $g_n$ eigenvectors $\ket{u_n^i}$ reduces to $a_n\ket{u_n^i}$, the matrix representing the restriction of $A$ wo within $\E_n$ is equal to $a_nI_{g_n\times g_n}$.
The matrix representing the operator $A$ in $\E_n$ is always diagonal and equal to $a_nI_{g_n\times g_n}$. 

We use this property to obtain a basis of $\E_n$ composed of vectors that are also eigenvectors of $B$. The matrix representing $B$ in $\E_n$ when the basis is chosen is
\begin{align}
    \{\ket{u_n^i}\;,\quad i=1,2,\cdots,g_n\},
\end{align}
has for its elements:
\begin{align}
    \beta_{ij}^{(n)}=\braket{u_n^i|B|u_n^j}.
\end{align}
Thi matri is Hermitian, since $B$ is a Hermitian operator. It is therefore diagonizable: one can find a new basis $\{\ket{v_n^i};\;i=1,2,\cdots,g_n\}$ in which $B$ is 
represented by a diagonal matrix:
\begin{align}
    \braket{v_n^i|B|v_n^j}=\beta^{n}_i\delta_{ij}.
\end{align}
This means that the new basis vectors in $\E_n$ are eigenvectors of $B$:
\begin{align}
    B\ket{v_n^i}=\beta{i^{(n)}}\ket{v_n^i}.
\end{align}
These vectors are automatically eigenvectors of $A$ with an eigenvalue $a_n$ since they belong to $\E_n$.

\begin{emphasizer}
Eigenvectors of $A$ associated with degenerate eigenvalues are not necessarily eigenvectors of $B$. It is always possible to choose, in every eigensubspace of $A$, a basis 
of eigenvectors common to $A$ and $B$.    
\end{emphasizer}
If we perform this operation in all the subsespaces $\E_n$, we obtain a bsis of $\E$, formed by eigenvectors common to $A$ and $B$.


We shall denote by $\ket{u_{n,p}^i}$ the eigenvectors common to $A$ an $B$:
\begin{align}
    A\ket{u_{n,p}^i}=a_n\ket{u_{n,p}^i},\quad\text{and}\quad B\ket{u_{n,p}^i}=b_p\ket{u_{n,p}^i}.
\end{align} 
The index $i$ will be used to distinguish between different basis vectors which correspond to the same eigenvalues $a_n$ and $b_p$.

%%
\subsubsection{Complete sets of commting observables (C.S.C.O.)}
Consider an observable $A$ and a basis $\E$ composed of eigenvectors $\ket{u_n^i}$ of $A$.

If none of the eigenvalues of $A$ is degenerate, the various basis vectors of $\E$ can be labelled by the eigenvalue $a_n$ (index $i$ is not necessary). Thefroe,
specifying the eigenvalue determines in a unique way the corresponding eigenvector. In other words, there exists only one basis of $\E$ formed by 
the eigenvectors of $A$. It is said that the observable $A$ constitutes, by itself, a C.S.C.O.

On the other hand, if at least one eigenvalue of $A$ is degenerate, specifying $a_n$ is no longer always sufficient to characterize a basis vector:
the basis of eigenvectors of $A$ is not unique. One can choose any basis inside each of the degenerate eigensubspaces $\E_n$.
We can choose another observable $B$ which commute with $A$ to construct an orthonormal basis of eigenvectors common to $A$ and $B$.
$A$ and $B$ form a C.S.C.O. if this basis is unique, that is, if to each of the possible pairs ofeigevalues $\{a_n,b_p\}$ there corresponds only one basis vector.
For $A$ and $B$ to constitute a C.S.C.O., it is necessary and sufficient that, inside each of tese subspaces, all the $g_n$ eigenvalues of $B$ be distinct.
We can add indifinitely observables until we reach the C.S.C.O.

\begin{emphasizer}
    A set of observables $A,B,C,\cdots$ is called a complete set of commuting observables if:
    \begin{enumerate}[itemsep=0pt,topsep=0pt,label=(\roman*)]
        \item all the observables commute by pairs.
        \item specifying the eigenvalue of all the operators determines a unique common eigenvector. The ket then is denoted as $\ket{a_n,b_p,c_r,\cdots}$.
    \end{enumerate}
    This means that they are C.S.C.O. if there exists a unique orthonormal basis of common eigenvectors. 
\end{emphasizer}
Idenfitication of CSCOs is necessary in order to construct physically meaningful bases for $\E$. Knowing the CSCOs that are available tells the experimenter the possible sets 
of measurements that can be made to achieve this goal. 

We list some CSCOs for specific problems. (table 31 anerson)