\section{Two important examples of representation and observables}
%%
\subsection{The $\{\bR\}$ and $\{\bp\}$ representations}

Recall the following bases of $\F$. They are not composed of functions belonging to $\F$:
\begin{align}
    \xi_{\bR_0}(\bR)=\delta(\bR-\bR_0)\quad\text{and}\quad v_{\bp_0}(\bR)=(2\pi\hbar)^{-3/2}e^{\frac{i}{\hbar}\bp_0\cdot\bR}
\end{align}
Every sufficiently regular equare-integrable function can be expanded in one or the other of these bases.
The ket associted with $\xi_{\bR_0}(\bR)$ and $v_{\bp_0}(\bR)$ will be denoted as:
\begin{align}
    \xi_{\bR_0}(\bR)\Longleftrightarrow\ket{\bR_0}\quad\text{and}\quad v_{\bp_0}(\bR)\Longleftrightarrow\ket{\bp_0}.
\end{align}
Using these bases $\{\xi_{\bR_0}(\bR)\}$ and $\{v_{\bp_0}(\bR)\}$ of $\F$ we thus define in $\E_{\bR}$ two representation:
the $\{\ket{\bR_0}\}$ and the $\{\ket{\bp_0}\}$ representations.
%
\subsubsection{Orthonormalization and closure relations}
If we calculate the scalar product of two kets, we have
\begin{align*}
    \braket{\bR_0|\bR_0'}=\int d^3r\;\xi^*_{\bR_0}(\bR)\xi_{\bR'_0}(\bR)=\delta(\bR_0-\bR_0')\quad\text{and}\quad\braket{\bp_0|\bp_0'}=\int d^3r\;v^*_{\bp_0}(\bR)
    v_{\bp'_0}(\bR)=\delta(\bp_0-\bp_0').
\end{align*}
Thus, te baseses are therefore orthonormal in the extended sense. The fact that the set of the $\ket{\bR_0}$ or that of $\ket{\bp_0}$ constitutes a basis in $\E_{\bR}$
can be expressed by a closure relation in $\E_{\bR}$:
\begin{align}
    \text{Orthonormality relation}&\qquad\text{Closure relation}\\
    \highlight{\begin{array}{l}
        \braket{\bR_0|\bR_0'}=\delta(\bR_0-\bR_0')\\
        \braket{\bp_0,\bp_0'}=\delta(\bp_0-\bp_0')
    \end{array}}&\qquad\highlight{\begin{array}{l}
        \displaystyle\int d^3r_0\;\ket{\bR_0}\bra{\bR_0}=\mathds{1}\\
        \displaystyle\int d^3p_0\;\ket{\bp_0}\bra{\bp_0}=\mathds{1}
    \end{array}}
\end{align}
%
\subsubsection{Components of a ket}
Consider a ket $\ket{\psi}$ corresponding to $\psi(\bR)$. We can expand it in each representation using the closure relation:
\begin{align}
    \ket{\psi}&=\int d^3r_0\;\ket{\bR_0}\braket{\bR_0|\psi},\quad\text{where}\quad\braket{\bR_0|\psi}=\psi(\bR_0)=\int d^3r\;\xi^*_{\bR_0}(\bR)\psi(\bR).\\
    \ket{\psi}&=\int d^3p_0\;\ket{\bp_0}\braket{\bp_0|\psi},\quad\text{where}\quad\braket{\bp_0|\psi}=\tilde{\psi}(\bp_0)=\int d^3r\;v^*_{\bp_0}(\bR)\psi(\bR).
\end{align}
We see that $\tilde{\psi}(\bp_0)$ is the Fourier transform of $\psi(\bR_0)$. Each value corresponds to the components of $\ket{\psi}$ on the 
basis vector of the respective representation.

%
Now, we redefine the above bases to just $\ket{\bR}$ and $\ket{\bp}$:
\begin{align}
    \highlight{
    \begin{array}{ll}
        \braket{\bR|\psi}=\psi(\bR)&\braket{\bp|\psi}=\tilde{\psi}(\bp)\\
        \braket{\bR|\bR'}=\delta(\bR-\bR')&\braket{\bp,\bp'}=\delta(\bp-\bp')\\
        \displaystyle\int d^3r\ket{\bR}\bra{\bR}=\mathds{1}&\displaystyle\int d^3p\;\ket{\bp}\bra{\bp}=\mathds{1}
    \end{array}}
    \label{eq:propertiesoperatorsRandP}
\end{align}
%
\subsubsection{Changing from $\{\ket{\bR}\}$ to $\{\ket{\bp}\}$ representation}
Changing from one basis to the other brinds in the numbers:
\begin{align}
    \braket{\bR|\bp}=\braket{\bp|\bR}^*=\int d^3r'\;\braket{\bR|\bR'}\braket{\bR'|\bp}=\int d^3r'\;\delta(\bR-\bR')(2\pi\hbar)^{-3/2}e^{\frac{i}{\hbar}\bp\cdot\bR'}=(2\pi\hbar)^{-3/2}e^{\frac{i}{\hbar}\bp\cdot\bR}.
    \label{eq:transformationmatrix}
\end{align}
A given ket $\ket{\psi}$ is represented by $\braket{\bR|\psi}=\psi(\bR)$ in the $\{\ket{\bR}\}$ representation and by $\braket{\bp|\psi}=\tilde{\psi}(\bp)$ in the 
$\{\ket{\bp}\}$ representation.

Therefore,
\begin{align}
    \psi(\bR)&=\braket{\bR|\psi}=\int d^3p\;\braket{\bR|\bp}\braket{\bp|\psi}=(2\pi\hbar)^{-3/2}\int d^3p\;e^{\frac{i}{\hbar}\bp\cdot\bR}\tilde{\psi}(\bp)\\
    \tilde{\psi}(\bp)&=\braket{\bp|\psi}=\int d^3r\;\braket{\bp|\bR}\braket{\bR|\psi}=(2\pi\hbar)^{-3/2}\int d^3r\;e^{-\frac{i}{\hbar}\bp\cdot\bR}\psi(\bR)\\
    A(\bp,\bp')&=(2\pi\hbar)^{-3}\int d^3r\;\int d^3r'\;e^{\frac{i}{\hbar}(\bp\cdot\bR-\bp'\cdot\bR')}A(\bR',\bR)
\end{align}
%%
\subsection{The R and P operators}
We define the $X$, $Y$, $Z$ operators whsoe action, in the $\{\ket{\bR}\}$ representation, is given by:
\begin{align}
    \highlight{
    \begin{array}{l}
        \braket{\bR|X|\psi}=x\braket{\bR|\psi}\\
        \braket{\bR|Y|\psi}=y\braket{\bR|\psi}\\
        \braket{\bR|Z|\psi}=z\braket{\bR|\psi}
    \end{array}}
\end{align}
X, Y, and Z will be considered to be the components of a vector operator $\BR$.
Similarly, we define the vector operator $\BP$ by its components $P_x$, $P_y$, $P_z$, whose action, in the $\{\ket{\bp}\}$ representation is given by:
\begin{align}
    \highlight{
    \begin{array}{l}
        \braket{\bp|P_x|\psi}=p_x\braket{\bp|\psi}\\
        \braket{\bp|P_y|\psi}=p_y\braket{\bp|\psi}\\
        \braket{\bp|P_z|\psi}=p_z\braket{\bp|\psi}
    \end{array}}
\end{align}

How $\BP$ operator acts in the $\{\ket{\bR}\}$ representation?
We use the closure relation to obtain:
\begin{align*}
    \braket{\bR|P_x|\psi}&=\int d^3p\;\braket{\bR|\bp}\braket{\bp|P_x|\psi}\\
    &\stackrel{(a)}{=}(2\pi\hbar)^{-3/2}\int d^3p\;e^{-\frac{i}{\hbar}\bp\cdot\bR}p_x\tilde{\psi}(\bp)\\
    \braket{\bR|P_x|\psi}&\stackrel{(b)}{=}\frac{\hbar}{i}\frac{\partial}{\partial x}\psi(\bR).
\end{align*}
In $(a)$ we have used the equation \eqref{eq:transformationmatrix} while in $(b)$ we have used the property of the derivative of a Fourier transform.
Generally, the result is:
\begin{align}
    \highlight{\braket{\bR|\BP|\psi}=\frac{\hbar}{i}\nabla\braket{\bR|\psi}=\frac{\hbar}{i}\nabla\psi(\bR)}.
\end{align}
In the $\{\ket{bR}\}$ representation, the $\BP$ operator coincides with the differential operator $(\hbar/i)\nabla$ applied to the wave functions.

If we compute the commutator say, $[X,P_x]$, we have
\begin{align*}
    \braket{\bR|[X,P_x]|\psi}&=\braket{\bR|(XP_x-P_xX)|\psi}\\
    &=\braket{\bR|XP_x|\psi}-\braket{\bR|P_xX|\psi}\\
    &=\int d^3r'\;\braket{\bR|X|\bR'}\braket{\bR'|P_x|\psi}-\int d^3r'\;\braket{\bR|P_x|\bR'}\braket{\bR'|X|\psi}\\
    &=\int d^3r'\;\left[x'\delta(\bR-\bR')\right]\frac{\hbar}{i}\frac{\partial}{\partial x}\braket{\bR'|\psi}-\int d^3r'\;\left[\frac{\hbar}{i}\frac{\partial}{\partial x}\delta(\bR-\bR')\right]x'\braket{\bR'|\psi}\\
    &=\frac{\hbar}{i}x\frac{\partial}{\partial x}\braket{\bR|\psi}-\frac{\hbar}{i}\frac{\partial}{\partial x}(x\braket{\bR|\psi})\\
    &=\frac{\hbar}{i}x\frac{\partial}{\partial x}\braket{\bR|\psi}-\frac{\hbar}{i}\braket{\bR|\psi}-\frac{\hbar}{i}x\frac{\partial}{\partial x}\braket{\bR|\psi}\\
    \braket{\bR|[X,P_x]|\psi}&=i\hbar\braket{\bR|\psi}.
\end{align*}
Thus, one finds $[X,P_x]=i\hbar$. In the same way, we find all the other commutators between the components of $\BR$ and $\BP$:

\begin{align}
    \text{Canonical commutation relations}\qquad\highlight{[R_i,R_j]=0,\quad[P_i,P_j]=0,\quad[R_i,P_j]=i\hbar\delta_{ij},\quad i,j=1,2,3}.
\end{align}


\begin{emphasizer}[R and P are Hermitian]
    All the components of $\BR$ and $\BP$ are Hermitian operators.
\end{emphasizer}
For example,
\begin{align*}
    \braket{\varphi|X|\psi}=\int d^3r\;\varphi^*(\bR)x\psi(\bR)=\left[\int d^3r\;\varphi(\bR)x\psi^*(\bR)\right]^*=\braket{\psi|X|\varphi}^*.
\end{align*}
%
\subsubsection{Eigenvectors of R and P}
Consider the action of $X$ on the ket $\ket{\bR_0}$:
\begin{align*}
    \braket{\bR|X|\bR_0}=x\braket{\bR|\bR_0}=x\delta(\bR-\bR_0)=x_0\delta(\bR-\bR_0)=x_0\braket{\bR|\bR_0}.
\end{align*}
The components in $\{\ket{\bR}\}$ representation of the ket $X\ket{\bR_0}$ are equal to those of the ket $\bR_0$ multiplied by $x_0$:
\begin{align}
    X\ket{\bR_0}=x_0\ket{\bR_0}.
\end{align}
Omitting the index zero, and doing the same for the other components of $\BR$ and $\BP$ in their respective representations yield:
\begin{align}
    \highlight{\begin{array}{l}
        X\ket{\bR}=x\ket{\bR}\\
        Y\ket{\bR}=y\ket{\bR}\\
        Z\ket{\bR}=z\ket{\bR}
    \end{array}},\quad\text{and}\quad\highlight{\begin{array}{l}
        P_x\ket{\bp}=p_x\ket{\bp}\\
        P_y\ket{\bp}=p_y\ket{\bp}\\
        P_z\ket{\bp}=p_z\ket{\bp}
    \end{array}}
\end{align}
%
\subsubsection{R and P are observables}
We have already demonstrated the closure relation for each representation $\{\ket{\bR}\}$ and $\{\ket{\bp}\}$ in equation \eqref{eq:propertiesoperatorsRandP}. Therefore,
$\BR$ and $\BP$ are observables. In a three-dimensional space, is necessary to specify the eigenvalues $x_o,y_o,z_o$ as they uniquely determines the corresponding eigevector $\ket{\bR_0}$.

\begin{emphasizer}
    The set of the three operators $X,Y,Z$ and the set of the three operators $P_x,P_y,P_z$  constitute a CSCO in $\E_{\bR}$.
\end{emphasizer}
Recall that an operator must have eigenvectors that span the whole state vector, so missing one coordinate will degenerate it and therefore is no longer 
uniquely determined. One can also mix X with P as $\{X,P_y,P_z\}$ to create CSCOs.