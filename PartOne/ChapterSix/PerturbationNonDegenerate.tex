\section{Perturbation of a non-degenerate case}
Let be a non-denegerate eigenvalue $E_n^0$ of $H_0$, which is associated with the unique eigenvector $\ket{\varphi_n}$. We want to determine the 
modifications in this unperturbed energy in the corresponding statinary state produces by $W$.

For the eigenvalues of $H(\lambda)$ that approaches $E_n^0$ when $\lambda\to0$ we ave 
\begin{align}
    \varepsilon_0=E_n^0\Longrightarrow \ket{0}=\ket{\varphi_n}.
    \label{eq:spdt_nondegenerate_statement}
\end{align}
Thus, when $\lambda\to0$ we again find the unperturbed state $\ket{\varphi_n}$ with the same phase. We call $E_n(\lambda)$ the eigenvalue of $H(\lambda)$ which when 
$\lambda\to0$, approaches the eigenvalue $E_n^0$ of $H_0$, with eigenvector $\ket{\psi_n(\lambda)}$. We now compute the expansion of $E_n(\lambda)$ and 
$\ket{\psi_n(\lambda)}$ in powers of $\lambda$.
%
\subsection{First-order corrections}
We will determine the eigenvalue $\varepsilon_1$ and the eigenvector $\ket{1}$.
%
\subsubsection{Energy correction}
Projecting the first-order equation from \eqref{eq:spdt_statement} onto $\ket{\varphi_n}$ yields
\begin{align}
    \braket{\varphi_n|(H_0-\varepsilon_0)|1}+\braket{\varphi_n|(W-\varepsilon_1)|0}=\cancelto{0}{\braket{\varphi_n|(\varepsilon_0-\varepsilon_0)|1}}+\braket{\varphi_n|(W-\varepsilon_1)|0}=0.
\end{align}
Considering equation \eqref{eq:spdt_nondegenerate_statement}, we have that:
\begin{align}
    \varepsilon_1=\braket{\varphi_n|W|0}=\braket{\varphi_n|W|\varphi_n}.
\end{align}
The eigenvalue $E_n(\lambda)$ of $H$ which corresponds to $E_n^0$ can be written, to first order in the perturbation as:
\begin{align}
    \text{First-order eigenvalue of $H$}\qquad\highlight{E_n(\lambda)=E_n^0+\braket{\varphi_n|W|\varphi_n}+O(\lambda^2)}.
    \label{eq:spdt_firstorder_eigenvalue}
\end{align}
The first-order correction to a non-degenerate energy $E_n^0$ is simply equal to the averae value of the perturbation term $W$ in the unperturbed state $\ket{\varphi_n}$.
%
\subsubsection{Eigenvector correction}
To exhaust all the information of the first-order perturbation equation \eqref{eq:spdt_statement}, we must project it onto all the vectors of the $\{\ket{\varphi_p^i}\}$ basis other than 
$\ket{\varphi_n}$. Using \eqref{eq:spdt_nondegenerate_statement}:
\begin{align*}
    \braket{\varphi_p^i|(H_0-E_n^0)|1}+\braket{\varphi_p^i|(W-\cancelto{0\scriptsize\text{ Orthogonality}}{\varepsilon_1})|\varphi_n}&=0,\quad p\neq n\\
    (E_p^0-E_n^0)\braket{\varphi_p^i|1}+\braket{\varphi_p^i|W|\varphi_n}&=
\end{align*}
THe index $i$ is for any possible degeneration in the energies $E_p^0$ other than $E_n^0$. 
This expression gives the coefficients of the desired expansion of the vector $\ket{1}$ on all the unperturbed basis states, except $\ket{\varphi_n}$:
\begin{align}
    \braket{\varphi_p^i}=\frac{1}{E_n^0-E_p^0}\braket{\varphi_p^i|W|\varphi_n},\quad p\neq n.
\end{align}
The las coefficient which we lack, $\braket{\varphi_n|1}=\braket{0|1}$, is zero by the second condition of \eqref{eq:spdt_statement}. We therefore know the 
vector $\ket{1}$ since we know its expansion on the $\{\ket{\varphi_p^i}\}$ basis:
\begin{align}
    \ket{1}=\sum_{p\neq n}\sum_i\frac{\braket{\varphi_p^i|W|\varphi_n}}{E_n^0-E_p^0}\ket{\varphi_p^i}.
\end{align}
Thus, to first order in the perturbation, the eigenvector $\ket{\varphi_n(\lambda)}$ of $H$ corresponding to the unperturbed state $\ket{\varphi_n}$ can be written as:
\begin{align}
    \text{First-order eigenvector of $H$}\qquad\highlight{\ket{\psi_n(\lambda)}=\ket{\varphi_n}+\sum_{p\neq n}\sum_i\frac{\braket{\varphi_p^i|W|\varphi_n}}{E_n^0-E_p^0}+O(\lambda^2)}.
    \label{eq:spdt_firstorder_eigenvector}
\end{align}
It is a linear combination of all unperturbed states other than $\ket{\varphi_n}$: the perturbation $W$ is said to produce a mixing of the state $\ket{\varphi_n}$ with the other iegenstates 
of $H_0$. The contribution of a given state $\ket{\varphi_p^i}$ is zero if the perturbation has no matrix element between $\ket{\varphi_n}$ and $\ket{\varphi_p^i}$.

\begin{emphasizer}[First-order correction constraint]
    The first order correction of the state vector is small only if the non-diagonal matrix elements of $W$ are much smaller than the corresponding unperturbed energy differences.
\end{emphasizer}
%
\subsection{Second-order corrections}
The process is analogous to the first-order, with the use of the second equation in \eqref{eq:spdt_statement}.
%
\subsubsection{Energy correction}
We project second equation of \eqref{eq:spdt_statement} onto the vector $\ket{\varphi_n}$, using the conditions:
\begin{align*}
    \cancelto{0\scriptsize\text{ Orthogonality}}{\braket{\varphi_n|(H_0-E_n^0)|2}}+\braket{\varphi_n|(W-\cancelto{0\scriptsize\text{ Orhtogonality}}{\varepsilon_1})|1}-\varepsilon_2\braket{\varphi_n|\varphi_n}=0\Longrightarrow\varepsilon_2=\braket{\varphi_n|W|1}.
\end{align*}
Using the expression for $\ket{1}$ from the first-order section, we have that:
\begin{align}
    \varepsilon_2=\sum_{p\neq n}\sum_i\frac{|\braket{\varphi_p^i|W|\varphi_n}|^2}{E_n^0-E_p^0}.
\end{align}
This enables us to write the energy $E_n(\lambda)$ to second order in the perturbation in the form:
\begin{align}
    \text{Second-order eigenvalue of $H$}\qquad\highlight{E_n(\lambda)=E_n^0+\braket{\varphi_n|W|\varphi_n}+\sum_{p\neq n}\sum_i\frac{|\braket{\varphi_p^i|W|\varphi_n}|^2}{E_n^0-E_p^0}+O(\lambda^3)}.
    \label{eq:spdt_secondorder_eigenvalue}
\end{align}
To second-order, the closer the state $\ket{\varphi_p^i}$ to $\ket{\varphi_n}$, and th estronger the coupling $|\braket{\varphi_p^i|W|\varphi_n}|$, the more 
these two levels repel each other.
%
\subsubsection{Eigenvector correction}
By projecting the second equation \eqref{eq:spdt_statement} onto the set of basis vectors $\ket{\varphi_p^i}$ different from $\ket{\varphi_n}$, and by using conditions we can obtain the 
expression for the ket $\ket{2}$ and therefore the eigenvector to second order. It is not present in CT nor FG.

In general, we retain one more term in the energy expansion than in that of the eigenvector. This is because when projecting the second-order equation onto $\ket{\varphi_n}$, one makes
the first term go to zero, which gives $\varepsilon_q$ in terms of the corrections of order $q-1,q-2,\cdots$ of the eigenvector.
%
\subsubsection{Upper limit of $\varepsilon_2$}
If we limit the energy expansion to first order in $\lambda$, we can obtain an approximate idea of the error involver by evaluating the second-order term.
We denote by $\Delta E=$ the absolute value of the difference between the energy $E_n^0$ of the level being studied and that of the closest level. We have:
\begin{align}
    \Delta E\leq|E_n^0-E_p^0|,\quad\forall n.
\end{align}
This gives us an upper limit for the absolute value of $\varepsilon_2$:
\begin{align*}
    |\varepsilon_2|\leq&\frac{1}{\Delta E}\sum_{p\neq n}\sum_i|\braket{\varphi_p^i|W|\varphi_n}|^2\\
    =&\frac{1}{\Delta E}\sum_{p\neq n}\sum_i\braket{\varphi_n|W|\varphi_p^i}\braket{\varphi_p^i|W|\varphi_n}\\
    =&\frac{1}{\Delta E}\bra{\varphi_n}W\left[\sum_{p\neq n}\sum_i\ket{\varphi_p^i}\bra{\varphi_p^i}\right]W\ket{\varphi_n}\\
    \stackrel{(a)}{=}&\frac{1}{\Delta E}\bra{\varphi_n}W\bigr[1-\ket{\varphi_n}\bra{\varphi_n}\bigr]W\ket{\varphi_n}\\
    =&\frac{1}{\Delta E}\left[\braket{\varphi_n|W^2|\varphi_n}-(\braket{\varphi_n|W|\varphi_n})^2\right]
\end{align*}
In $(a)$, we have used the fact that inside the brakets, we have the projector onto the basis $\{\ket{\varphi_p^i}\}$ minus a single elements $\ket{\varphi_n}$.

Finally, we multiply both sides by $\lambda^2$ to obtain an upper limit for the second-order term in the expansin of $E_n(\lambda)$, in the form:
\begin{align}
    \text{Upper limit for second-order term in $E_n(\lambda)$}\qquad\highlight{|\lambda^2\varepsilon_2|\leq\frac{1}{\Delta E}(\Delta W)^2}.
\end{align}
This indicates the order of magnitude of the error on the energy resulting from taking only the first-order correction into account.