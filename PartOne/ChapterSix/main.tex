\chapter{Stationary perturbation theory}
\startcontents[chapters]
\printcontents[chapters]{}{1}{}
{\vspace{2em}\color{black}\titlerule[2pt]}
\vspace*{\fill}

%% introduction talk
In real problems, the Hamiltonian are not that easy to solve analytically. In general, the equation is too 
complicated for us to be able to find its solutions in an analytic form.

There exists, however, \bfemph{approximation methods} that enable us to obtain analytically approximate solutions 
of the basic eigenequation in certain cases. This technique is known as \bfemph{stationary perturbation theory}\index{Stationary perturbation theory}.

In studying a phenomenon or a physical system, one begins by isolating the principal effects that are responsible for the main 
features of this phenomenon or this system. Once understood, one tries to explain the finer details by taking into account less 
important effects that were neglected in the first approximation. It is in treating these secondary effects that one 
commonly uses perturbation theory.

There is also another approximation method called the \bfemph{variational method}.
\newpage
\columnratio{.7}
\section{Description of the method}
%
\subsection{Statement of the problem}
Perturbation theory is applicable when the Hamiltonian $H$ of the system being studied can be put in the form 
\begin{align}
    \text{Hamiltonian of the system}\qquad\highlight{H(\lambda)=H_0+\underbrace{\lambda\hat{W}}_{W},\quad\text{with}\quad\lambda\ll1},
    \label{eq:spdt_hamiltonian}
\end{align}
where the spectrum of $H_0$ is known, and where $W$ is \textbf{much smaller} than $H_0$. This means that the matrix elements of $W$ are much 
smaller than those of $H_0$. 
The operator $H_0$ is time-independent and is knwon as \bfemph{unperturbed Hamiltonian}, while $W$ is called the \bfemph{perturbation}. When $W$ is time-independent, we 
say we are dealing a \emph{stationary perturbation}.

The problem then is to find the modifications produced in the enery levels of the system and its stationary states by the 
influence of $W$. Perturbation theory consists of expanding the eigenpairs of $H$ in power of $\lambda$ with a finite number of terms.

We assume the spectrum of $H_0$ is discrete, where eiganvalue $E_p^0$ and eigenstate $\ket{\varphi_p^i}$. We therefore have:
\begin{align}
    \text{Spectrum of $H_0$}\qquad\highlight{H_0\ket{\varphi_p^i}=E_p^0\ket{\varphi_p^i},\quad\text{with}\quad\begin{array}{l}
        \braket{\varphi_p^i|\varphi_p^i}=\delta_{pp'}\delta_{ii'}\\
        \displaystyle\sum_p\sum_i\ket{\varphi_p^i}\bra{\varphi_p^i}=1
    \end{array}}.
\end{align}

Figure \ref{fig:variationofe} represents possible forms of the variations of the eigenvalues $E(\lambda)$ of \eqref{eq:spdt_hamiltonian} with respect to $\lambda$.
For a given value of $\lambda$, these vectors form a basis of the state space of $H(\lambda)$. The operator may have several degeneracies, as in $E_3^0$ and $E_4^0$.
The perturbation is able to remove some degeneracies as for $E_3^0$.
\begin{figure}[h!]
    \centering
    \includegraphics[width=.3\columnwidth]{PartOne/ChapterSix/spdt_variationofe.png}
    \caption{Variation of $E(\lambda)$ with respect to $\lambda$. For $\lambda=0$, we obtain the spectrum of $H_0$. We see the two-fold of $E_3^0$ and $E_4^0$ where 
    perturbation removes the degeneracy of $E_3^0$ but not that of $E_4^0$. Additional two-fold degeneracy appears at $\lambda=\lambda_1$.}
    \label{fig:variationfofe}
\end{figure}
%
\subsection{Approximate solution of $H(\lambda)$}
Lets define the eigenequation for the Hermitian operator $H(\lambda)$:
\begin{align}
    \text{Eigenequation of $H(\lambda)$}\qquad\highlight{H(\lambda)\ket{\psi(\lambda)}=E(\lambda)\ket{\psi(\lambda)}}.
    \label{eq:spdt_eigenequation}
\end{align}
We assume that $E(\lambda)$ and $\ket{\psi(\lambda)}$ can be expanded in powers of $\lambda$ in the form:
\begin{align}
    E(\lambda)=\sum_{q=1}^\infty\lambda^q\varepsilon_q,\quad\text{and}\quad\ket{\psi(\lambda)}=\sum_{q=1}^\infty\lambda^q\ket{q}.
\end{align}
Substituting the above along with \eqref{eq:spdt_hamiltonian} in the eigenequation \eqref{eq:spdt_eigenequation} yields 
\begin{align*}
    (H_0+\lambda W)\left[\sum_{q=1}^q\lambda^q\ket{q}\right]=\left[\sum_{q'=1}^{q'}\lambda^{q'}\varepsilon_q\right]\left[\sum_{q=1}^\infty\lambda^q\ket{q}\right].
\end{align*}
Equating the terms of the same power $q$:
\begin{align}
    \text{0th-order}:&\qquad H_0\ket{0}=\varepsilon_0\ket{0}\\
    \text{1st-order}:&\qquad (H_0-\varepsilon_0)\ket{1}=(W-\varepsilon_1)\ket{0}=0\\
    \text{2nd-order}:&\qquad (H_0-\varepsilon_0)\ket{2}+(W-\varepsilon_1)\ket{1}-\varepsilon_2\ket{0}=0\\
    \vdots&\notag\\
    \text{qth-order}:&\quad (H_0-\varepsilon_0)\ket{q}+(W-\varepsilon_1)\ket{q-1}-\varepsilon_2\ket{q-2}-\cdots-\varepsilon_q\ket{0}=0
\end{align}

We can choose the norm of $\ket{\psi(\lambda)}$ and its phase: we require it to be normalized, we choose its phase such that the scalar product 
$\braket{0|\psi(\lambda)}$ is real. To 0th-order, this implies that the vector $\ket{0}$ must be normalized:
\begin{align}
    \braket{0|0}=1.
\end{align}
Its phase remain arbitrary, we will choose it after. To 1st order, the square of the nomro of $\ket{\psi(\lambda)}$ is written 
\begin{align}
    \braket{\psi(\lambda)|\psi(\lambda)}&=[\bra{0}+\lambda\bra{1}][\ket{0}+\lambda\ket{1}]+O(\lambda^2)\\
    &=\braket{0|0}+\lambda[\braket{1|0}+\braket{0|1}]+O(\lambda^2)\\
    &=1+\lambda[\braket{1|0}+\braket{0|1}]+O(\lambda^2)=1.
\end{align}
The choice of phase indicates that the scalar product $\braket{0|1}$ is real since $\lambda$ is real. Therefore,
\begin{align}
    \braket{0|1}=\braket{1|0}=0.
\end{align}
Anologously, 
\begin{align}
    \braket{0|2}=\braket{2|0}=-\frac{1}{2}\braket{1|1},
\end{align}
and in general:
\begin{align}
    \braket{0|q}=\braket{q|0}=-\frac{1}{2}[\braket{q-1|1}+\braket{q-2|2}+\cdots+\braket{2|q-2}+\braket{1|q-1}].
\end{align}

When we confine to second order in $\lambda$, the perturbation equations and the conditions, from above,
\begin{align}
    \text{Second-order in $\lambda$}\qquad\highlight{\begin{array}{ll}
        \text{Equations}&\text{Conditions}\\
        H_0\ket{0}=\varepsilon_0\ket{0}&\braket{0|0}=1\\
        (H_0-\varepsilon_0)\ket{1}=(W-\varepsilon_1)\ket{0}=0&\braket{0|1}=\braket{1|0}=0\\
        (H_0-\varepsilon_0)\ket{2}+(W-\varepsilon_1)\ket{1}-\varepsilon_2\ket{0}=0&\braket{0|2}=\braket{2|0}=-\frac{1}{2}\braket{1|1}
    \end{array}}.
    \label{eq:spdt_statement}
\end{align}

From them, we now that $\ket{0}$ is an eigenvector of $H_0$ with eigenvalue $\varepsilon_0$, with the particular value $E_n^0$.
Consider the set of eigenstates of $H(\lambda)$ corresponding to the various eigenvalues $E(\lambda)$ that approach $E_n^0$ when $\lambda\to0$. 
They span a vector subspace whose dimension clearly cannot vary discontinuously when $\lambda$ varies in the vecinity of zero. This dimension is consequently equal 
to the degeneracy $g_n$ of $E_n^0$. We now consider the case of non-degenerae, and degenerate levels of $H_0$.
\section{Perturbation of a non-degenerate case}
Let be a non-denegerate eigenvalue $E_n^0$ of $H_0$, which is associated with the unique eigenvector $\ket{\varphi_n}$. We want to determine the 
modifications in this unperturbed energy in the corresponding statinary state produces by $W$.

For the eigenvalues of $H(\lambda)$ that approaches $E_n^0$ when $\lambda\to0$ we ave 
\begin{align}
    \varepsilon_0=E_n^0\Longrightarrow \ket{0}=\ket{\varphi_n}.
    \label{eq:spdt_nondegenerate_statement}
\end{align}
Thus, when $\lambda\to0$ we again find the unperturbed state $\ket{\varphi_n}$ with the same phase. We call $E_n(\lambda)$ the eigenvalue of $H(\lambda)$ which when 
$\lambda\to0$, approaches the eigenvalue $E_n^0$ of $H_0$, with eigenvector $\ket{\psi_n(\lambda)}$. We now compute the expansion of $E_n(\lambda)$ and 
$\ket{\psi_n(\lambda)}$ in powers of $\lambda$.
%
\subsection{First-order corrections}
We will determine the eigenvalue $\varepsilon_1$ and the eigenvector $\ket{1}$.
%
\subsubsection{Energy correction}
Projecting the first-order equation from \eqref{eq:spdt_statement} onto $\ket{\varphi_n}$ yields
\begin{align}
    \braket{\varphi_n|(H_0-\varepsilon_0)|1}+\braket{\varphi_n|(W-\varepsilon_1)|0}=\cancelto{0}{\braket{\varphi_n|(\varepsilon_0-\varepsilon_0)|1}}+\braket{\varphi_n|(W-\varepsilon_1)|0}=0.
\end{align}
Considering equation \eqref{eq:spdt_nondegenerate_statement}, we have that:
\begin{align}
    \varepsilon_1=\braket{\varphi_n|W|0}=\braket{\varphi_n|W|\varphi_n}.
\end{align}
The eigenvalue $E_n(\lambda)$ of $H$ which corresponds to $E_n^0$ can be written, to first order in the perturbation as:
\begin{align}
    \text{First-order eigenvalue of $H$}\qquad\highlight{E_n(\lambda)=E_n^0+\braket{\varphi_n|W|\varphi_n}+O(\lambda^2)}.
    \label{eq:spdt_firstorder_eigenvalue}
\end{align}
The first-order correction to a non-degenerate energy $E_n^0$ is simply equal to the averae value of the perturbation term $W$ in the unperturbed state $\ket{\varphi_n}$.
%
\subsubsection{Eigenvector correction}
To exhaust all the information of the first-order perturbation equation \eqref{eq:spdt_statement}, we must project it onto all the vectors of the $\{\ket{\varphi_p^i}\}$ basis other than 
$\ket{\varphi_n}$. Using \eqref{eq:spdt_nondegenerate_statement}:
\begin{align*}
    \braket{\varphi_p^i|(H_0-E_n^0)|1}+\braket{\varphi_p^i|(W-\cancelto{0\scriptsize\text{ Orthogonality}}{\varepsilon_1})|\varphi_n}&=0,\quad p\neq n\\
    (E_p^0-E_n^0)\braket{\varphi_p^i|1}+\braket{\varphi_p^i|W|\varphi_n}&=
\end{align*}
THe index $i$ is for any possible degeneration in the energies $E_p^0$ other than $E_n^0$. 
This expression gives the coefficients of the desired expansion of the vector $\ket{1}$ on all the unperturbed basis states, except $\ket{\varphi_n}$:
\begin{align}
    \braket{\varphi_p^i}=\frac{1}{E_n^0-E_p^0}\braket{\varphi_p^i|W|\varphi_n},\quad p\neq n.
\end{align}
The las coefficient which we lack, $\braket{\varphi_n|1}=\braket{0|1}$, is zero by the second condition of \eqref{eq:spdt_statement}. We therefore know the 
vector $\ket{1}$ since we know its expansion on the $\{\ket{\varphi_p^i}\}$ basis:
\begin{align}
    \ket{1}=\sum_{p\neq n}\sum_i\frac{\braket{\varphi_p^i|W|\varphi_n}}{E_n^0-E_p^0}\ket{\varphi_p^i}.
\end{align}
Thus, to first order in the perturbation, the eigenvector $\ket{\varphi_n(\lambda)}$ of $H$ corresponding to the unperturbed state $\ket{\varphi_n}$ can be written as:
\begin{align}
    \text{First-order eigenvector of $H$}\qquad\highlight{\ket{\psi_n(\lambda)}=\ket{\varphi_n}+\sum_{p\neq n}\sum_i\frac{\braket{\varphi_p^i|W|\varphi_n}}{E_n^0-E_p^0}+O(\lambda^2)}.
    \label{eq:spdt_firstorder_eigenvector}
\end{align}
It is a linear combination of all unperturbed states other than $\ket{\varphi_n}$: the perturbation $W$ is said to produce a mixing of the state $\ket{\varphi_n}$ with the other iegenstates 
of $H_0$. The contribution of a given state $\ket{\varphi_p^i}$ is zero if the perturbation has no matrix element between $\ket{\varphi_n}$ and $\ket{\varphi_p^i}$.

\begin{emphasizer}[First-order correction constraint]
    The first order correction of the state vector is small only if the non-diagonal matrix elements of $W$ are much smaller than the corresponding unperturbed energy differences.
\end{emphasizer}
%
\subsection{Second-order corrections}
The process is analogous to the first-order, with the use of the second equation in \eqref{eq:spdt_statement}.
%
\subsubsection{Energy correction}
We project second equation of \eqref{eq:spdt_statement} onto the vector $\ket{\varphi_n}$, using the conditions:
\begin{align*}
    \cancelto{0\scriptsize\text{ Orthogonality}}{\braket{\varphi_n|(H_0-E_n^0)|2}}+\braket{\varphi_n|(W-\cancelto{0\scriptsize\text{ Orhtogonality}}{\varepsilon_1})|1}-\varepsilon_2\braket{\varphi_n|\varphi_n}=0\Longrightarrow\varepsilon_2=\braket{\varphi_n|W|1}.
\end{align*}
Using the expression for $\ket{1}$ from the first-order section, we have that:
\begin{align}
    \varepsilon_2=\sum_{p\neq n}\sum_i\frac{|\braket{\varphi_p^i|W|\varphi_n}|^2}{E_n^0-E_p^0}.
\end{align}
This enables us to write the energy $E_n(\lambda)$ to second order in the perturbation in the form:
\begin{align}
    \text{Second-order eigenvalue of $H$}\qquad\highlight{E_n(\lambda)=E_n^0+\braket{\varphi_n|W|\varphi_n}+\sum_{p\neq n}\sum_i\frac{|\braket{\varphi_p^i|W|\varphi_n}|^2}{E_n^0-E_p^0}+O(\lambda^3)}.
    \label{eq:spdt_secondorder_eigenvalue}
\end{align}
To second-order, the closer the state $\ket{\varphi_p^i}$ to $\ket{\varphi_n}$, and th estronger the coupling $|\braket{\varphi_p^i|W|\varphi_n}|$, the more 
these two levels repel each other.
%
\subsubsection{Eigenvector correction}
By projecting the second equation \eqref{eq:spdt_statement} onto the set of basis vectors $\ket{\varphi_p^i}$ different from $\ket{\varphi_n}$, and by using conditions we can obtain the 
expression for the ket $\ket{2}$ and therefore the eigenvector to second order. It is not present in CT nor FG.

In general, we retain one more term in the energy expansion than in that of the eigenvector. This is because when projecting the second-order equation onto $\ket{\varphi_n}$, one makes
the first term go to zero, which gives $\varepsilon_q$ in terms of the corrections of order $q-1,q-2,\cdots$ of the eigenvector.
%
\subsubsection{Upper limit of $\varepsilon_2$}
If we limit the energy expansion to first order in $\lambda$, we can obtain an approximate idea of the error involver by evaluating the second-order term.
We denote by $\Delta E=$ the absolute value of the difference between the energy $E_n^0$ of the level being studied and that of the closest level. We have:
\begin{align}
    \Delta E\leq|E_n^0-E_p^0|,\quad\forall n.
\end{align}
This gives us an upper limit for the absolute value of $\varepsilon_2$:
\begin{align*}
    |\varepsilon_2|\leq&\frac{1}{\Delta E}\sum_{p\neq n}\sum_i|\braket{\varphi_p^i|W|\varphi_n}|^2\\
    =&\frac{1}{\Delta E}\sum_{p\neq n}\sum_i\braket{\varphi_n|W|\varphi_p^i}\braket{\varphi_p^i|W|\varphi_n}\\
    =&\frac{1}{\Delta E}\bra{\varphi_n}W\left[\sum_{p\neq n}\sum_i\ket{\varphi_p^i}\bra{\varphi_p^i}\right]W\ket{\varphi_n}\\
    \stackrel{(a)}{=}&\frac{1}{\Delta E}\bra{\varphi_n}W\bigr[1-\ket{\varphi_n}\bra{\varphi_n}\bigr]W\ket{\varphi_n}\\
    =&\frac{1}{\Delta E}\left[\braket{\varphi_n|W^2|\varphi_n}-(\braket{\varphi_n|W|\varphi_n})^2\right]
\end{align*}
In $(a)$, we have used the fact that inside the brakets, we have the projector onto the basis $\{\ket{\varphi_p^i}\}$ minus a single elements $\ket{\varphi_n}$.

Finally, we multiply both sides by $\lambda^2$ to obtain an upper limit for the second-order term in the expansin of $E_n(\lambda)$, in the form:
\begin{align}
    \text{Upper limit for second-order term in $E_n(\lambda)$}\qquad\highlight{|\lambda^2\varepsilon_2|\leq\frac{1}{\Delta E}(\Delta W)^2}.
\end{align}
This indicates the order of magnitude of the error on the energy resulting from taking only the first-order correction into account.
\section{Perturbation of a degenerate level}

% bibliography
\newpage
%\section*{Bibliography}
%
\begin{refsection}
    \nocite{*}
    %
    \subsection*{Mathematics}
    \printbibliography[heading=none,title={}, keyword=bib_mathematicalformalism]
    %

    %
\end{refsection}


