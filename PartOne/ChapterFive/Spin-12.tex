\section{Spin 1/2 particle: quantization of the angular momentum}
We will study a fundamental experiment that revealed the quantization of a simple physical quantity, the angular momentum.
We will see that the component along Oz of the AM can take only certain values ($\pm\hbar/2$), that is, is quantized.

A silver atom in the ground state is said to be a \bfemph{spin 1/2 particle}. We shall alo study the exolution of a spin 1/2 particle in a uniform magneti field (Larmor precession).

\subsection{Experimental demonstration}
We are going to describe and analyze the Stern-Gerlach experiment, which demonstrated the quantization of the componentes of the angular momentum.
%
\subsubsection{The Stern-Gerlach apparatus}
The experiment consists of studying the deflection of a beam of neutral paramgnetic atoms (in this case silver atoms).
The leave a furnace $E$ through a small opening and propagate in a straigth line in the high vacuum existing inside the apparatus.
Then, the atomic beam traverses the electromagnet $A$ and thus being deflected before reaching the plate $P$.

This B-field has a plane of symetry yOz that contains the initial direction Oy of the atomic beam.
The B-field has no components along Oy, and its larges component is along Oz; it varies strongly with z.
Since the B-field has a conserved flux $\nabla\cdot\bB=0$, it must also have a component along Ox which varies with the distance x from the plane of symmetry.
\begin{figure}[h!]
    \centering
    \begin{subfigure}{.4\columnwidth}
        \centering
        \includegraphics[width=\columnwidth]{PartOne/ChapterFive/stern1.png}
        \caption{Stern-gerlach apparatus}
    \end{subfigure}
    \hfill
    \begin{subfigure}{.3\columnwidth}
        \centering
        \includegraphics[width=.8\columnwidth]{PartOne/ChapterFive/stern2.png}
        \caption{B-field distribution}
    \end{subfigure}
    \hfill
    \begin{subfigure}{.2\columnwidth}
        \centering
        \includegraphics[width=.5\columnwidth]{PartOne/ChapterFive/sternresults.png}
        \caption{Experimental results}
    \end{subfigure}
    \caption{The Stern-Gerlach experiment.}
\end{figure}
%%
\subsubsection{Classical calculations of the deflection}
The neutral silver atoms posses a permanent magnetic moment $\bm{\mu}$ (they are paramgnetic atoms); the resulting forces are derived from the potential energy:
\begin{align}
    W_B=-\bm{\mu}\cdot\bB.
\end{align}
For a given atomic level, the magnetic moment $\bm{\mu}$ and the angular momentum $\bJ$ are proportional:
\begin{align}
    \bm{\mu}=\gamma\bJ,
\end{align} 
where $\gamma$ is the \bfemph{gyromagnetic ratio} of the level. Before the atoms traverse the electromagnet, the magnetic moments of the silver are oriented randomly (isotropically).

The resultat foces exerted on te atom is:
\begin{align}
    \bm{F}=\nabla(\bm{\mu}\cdot\bB).
\end{align}
The angular momentum theorem can be written:
\begin{align}
    \partial_t\bL=\bm{\Gamma}=\gamma\bL\times\bB.
\end{align}
The atom thus behaves like a gyroscope: $\partial_t\bL$ is perpendicualr to $\bL$, and the angular momentum turns about the magnetic field, the angle $\theta$ between $|bL$ and $\bB$ remaining constant.
\begin{figure}[h!]
    \centering
    \includegraphics[width=.2\columnwidth]{PartOne/ChapterFive/larmorprecession.png}
    \caption{The effect of a uniform B-field $\bB$ is to cause $\bm{\mu}$ to turn about $\bB$ with a 
    constant angular velocity (Larmor precession).}
\end{figure}

The rotational angular velocity is equal to the product of the gyromagnetic ratio $\gamma$ and the modulus of the B-field.

To calculate the force $\bm{F}$, we can, to a very good aproximation, neglect in thte potential energy $W$ the terms proportional to $\mu_x$ and $\mu_y$ and take $\mu_z$ to be constant.
This is becuase the frequency of oscillation sue to rotation of $\bm{\mu}$ is so great that only the time-average values of $\mu_x$ and $\mu_y$ can play a role in $W$, and there are both zero.

Consequently, it is as if the atom were submitted to the sole force:
\begin{align}
    \bm{F}'=\nabla(\mu_zB_z)=\mu_z\nabla B_z.
\end{align}
The force on the atom is therefore parallel to Oz and proportional to $\mu_z$. Consequently, because this force produced the deflection HN, measuring HN is 
equivalent to measuring $\mu_z$ and $L_z$.
%%
\subsubsection{Results and conclusions}
The results of the experiment (performed in 1922) are in complete contradiction with the preceding predictions.
We do not observe a single spot centered at H, but two spots centered at $N_1$ and $N_2$.
The predictions of classical mechanics are therefore shown to be invalidated by the experiment.

It is possible, in order to describe the motion of the silver atoms, to construct wave packets whose width $\Delta z$ and momentum dispersion $\Delta p$ are negligible. They must satisfy the 
Heisenberg relation:
\begin{align*}
    \Delta z\cdot\Delta p_z\geq\hbar.
\end{align*} 
Numerically, the mass $M$ of a silver atom is $1.8\times10^{-25}\;kg$. $\Delta z$ and the velocity uncertainty $\Delta v_z=\Delta p_z/M$ must be such that:
\begin{align}
    \Delta z\cdot\Delta v_z\geq\frac{\hbar}{M}\approx10^{-9}.
\end{align}

Is is then easy to find uncertainties $\Delta z$ and $\Delta v_z$ which, while satisfying above, are negligible on the scale of the experiment being considered.
Is is possible to reason in terms of quasi-pointlike wave packets moving along classical trajectories. Therefore, it is correct to claim that 
measurement of the deflection HN constitutes a measurement of $\mu_z$ or $L_z$.

\begin{emphasizer}[Conclusion of the experiment]
    If we measure the component $L_z$ of the intrinsic angular momentum of a silver atom in its ground state, we can find only one or the other of two values 
    corresponding to the deflections $HN_1$ and $HN_2$. $L_z$ is a \textbf{quantized} physical quantity whose discrete spectrum includes only two eigenvalues ($\pm\hbar/2$).
\end{emphasizer}

%%
\subsection{Theorical description}
We are now going to show how QM describes the degrees of freedom of a silver atom, that is, of a spin 1/2 particle.
The idea is to give precise examples of kets and observables, to show how physical predictions can be extracted from them and how to distinguish clearly between
the various stages of an experiment (preparation, evolution, measurement).

We must therefore define the state space and the observables corresponding to the components of $\bL$: $L_x,L_y,L_z$, or more generally,
$L_{\bm{u}}=L\cdot\bm{u}$, where $\bm{u}$ is an arbitrary unit vector.
%
\subsubsection{The observale $S_z$ and the spin state space}
With $L_z$ we must associate an observable $S_z$ which has two eigenvalues $\pm\hbar/2$.
We assume that these two eigenvalues are not degenrerate, ans we denot by $\ket{\pm}$ the respective orthonormal eigenvectors:
\begin{align}
    \text{Eigenequation of $S_z$}\qquad \highlight{S_z\ket{\pm}=\pm\frac{\hbar}{2}\ket{\pm},\quad\text{with}\quad\begin{array}{l}
        \braket{+|+}=\braket{-|-}=1\\
        \braket{\pm|\mp}=0
    \end{array}}
    \label{eq:sz}
\end{align}
$S_z$ alone forms a CSCO, and the spin state space is the two-dimensional space $\E_s$ spanned by the eigenvectors $\ket{\pm}$. This is mathematically expressed 
by the closure relation:
\begin{align}
    \text{Closure relation of $\E_s$}\qquad\ket{+}\bra{+}+\ket{-}\bra{-}=\mathds{1}.
\end{align}
The very most general vector in $\E_s$ is a linear superposition os these eigenvectors:
\begin{align}
    \ket{\psi}=\alpha\ket{+}+\beta\ket{-},\quad\text{with}\quad|\alpha|^2+|\beta|^2=1.
    \label{eq:generalpsispin}
\end{align}
In the $\{\ket{\pm}\}$ basis, the matrix representing $S_z$ is diagonal and is written as
\begin{align}
    S_z=\frac{\hbar}{2}\begin{bmatrix}
        1&0\\0&-1
    \end{bmatrix}.
\end{align}
%%
\subsubsection{The other spin observables}
With the $L_x$ and $L_y$ components of $\bL$ will be associates the observables $S_x$ and $S_y$. The three components of the angular momentum
do not commute with each other but satisfy well-defined commutation relations.  The matrices representing $S_x$ and $S_y$ in the basis 
of the eigenvectors $\ket{\pm}$ of $S_z$ are the following:
\begin{align}
    S_x=\frac{\hbar}{2}\begin{bmatrix}
        0&1\\1&0
    \end{bmatrix},\quad S_y=\frac{\hbar}{2}\begin{bmatrix}
        0&-i\\i&0
    \end{bmatrix}.
\end{align}
We see that they can be expressed in terms of the \bfemph{Pauli matrices} $\sigma_x,\sigma_y,\sigma_z$.

As fot the $L_{\bm{u}}$ component of $\bL$ along the unit vector $\bm{u}$, charactrerized by the angles $(\theta,\phi)$, it is written
\begin{align}
    L_{\bm{u}}=\bL\cdot\bm{u}=L_x\sin\theta\cos\phi+L_y\sin\theta\sin\phi+L_z\cos\theta.
\end{align}
Using the previous definitions of $S_x,S_y,S_z$, we easily find the matrix that represents the correspondin observable $S_{\bm{u}}=\bS\cdot\bm{u}$ in the $\{\ket{\pm}\}$ basis:
\begin{align}
    S_u=S_x\sin\theta\cos\phi+S_y\sin\theta\sin\phi+S_z\cos\theta=\frac{\hbar}{2}\begin{bmatrix}
        \cos\theta&\sin\theta e^{-i\phi}\\\sin\theta e^{i\phi}&-\cos\theta
    \end{bmatrix}.
\end{align}
We now get the eigenvaleus of each observable defined. The $S_x,S_y,S_u$ operators have the \textbf{same eigenvalues}, $+\hbar/2$ and $-\hbar/2$, as $S_z$.
This is equivalent to rotate the Stern-Gerlach device, as all directions of space have the same properties, it must be invariant under rotations.
The measurements of $L_x,L_y,L_u$ can therefore yield only one of two results: $+\hbar/2$ or $-\hbar/2$.

As for the eigenvectors of $S_x,S_y,S_u$, we shall denote them rspectively by $\ket{\pm}_x,\ket{\pm}_y,\ket{\pm}_u$. The expansion of the eigenvectors of $S_u$ in the 
$\{\ket{\pm}_z\}$ basis is:
\begin{align}
    \text{Eigenstates of $S_u$ in the $\{\ket{\pm}_z\}$ basis}\qquad\highlight{\begin{array}{l}
        \displaystyle\ket{+}_u=\cos\frac{\theta}{2}e^{-i\phi/2}\ket{+}_z+\sin\frac{\theta}{2}e^{i\phi/2}\ket{-}_z\\
        \displaystyle\ket{-}_u=-\sin\frac{\theta}{2}e^{-i\phi/2}\ket{+}_z+\cos\frac{\theta}{2}e^{i\phi/2}\ket{-}_z
    \end{array}}
    \label{eq:eigenstatesSu}
\end{align}
We see that $S_x$ is obtained when $(\theta,\phi)=(\pi/2,0)$ while for $S_y$ through $(\theta,\phi)=(\pi/2,\pi)$.
