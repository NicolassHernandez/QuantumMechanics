\section{Spin 1/2 particle: quantization of the angular momentum}
\subsection{Experimental demonstration}
We are going to describe and analyze the Stern-Gerlach experiment, which demonstrated the quantization of the componentes of the angular momentum.
%
\subsubsection{The Stern-Gerlach apparatus}
The experiment consists of studying the deflection of a beam of neutral paramgnetic atoms (in this case silver atoms).
The leave a furnace $E$ through a small opening and propagate in a straigth line in the high vacuum existing inside the apparatus.
Then, the atomic beam traverses the electromagnet $A$ and thus being deflected before reaching the plate $P$.

This B-field has a plane of symetry yOz that contains the initial direction Oy of the atomic beam.
The B-field has no components along Oy, and its larges component is along Oz; it varies strongly with z.
Since the B-field has a conserved flux $\nabla\cdot\bB=0$, it must also have a component along Ox which varies with the distance x from the plane of symmetry.
\begin{figure}[h!]
    \centering
    \begin{subfigure}{.45\columnwidth}
        \centering
        \includegraphics[width=\columnwidth]{PartOne/ChapterFive/stern1.png}
        \caption{Stern-gerlach apparatus}
    \end{subfigure}
    \hfill
    \begin{subfigure}{.45\columnwidth}
        \centering
        \includegraphics[width=.8\columnwidth]{PartOne/ChapterFive/stern2.png}
        \caption{B-field distribution}
    \end{subfigure}
    \caption{}
\end{figure}
%%
\subsection{Classical calculations of the deflection}
The neutral silver atoms posses a permanent magnetic moment $\bm{\mu}$ (they are paramgnetic atoms); the resulting forces are derived from the potential energy:
\begin{align}
    W_B=-\bm{\mu}\cdot\bB.
\end{align}
For a given atomic level, the magnetic moment $\bm{\mu}$ and the angular momentum $\bJ$ are proportional:
\begin{align}
    \bm{\mu}=\gamma\bJ,
\end{align} 
where $\gamma$ is the \bfemph{gyromagnetic ratio} of the level.
