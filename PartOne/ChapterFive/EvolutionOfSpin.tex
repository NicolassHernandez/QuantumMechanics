\section{Illustration of the postules in the case of a spin 1/2}
We are now to apply the postulaes of QM to a certain number of experiments on silver atoms.
%%
\subsection{Actual preparation of the various spin states}
In order to make predictions about the results of a measurement, we must know the state of the system immediately 
before the measurement.
%
\subsubsection{Preparation of the states $\ket{\pm}$}
Let assume we have the hole at $N_1$. The atoms which are deflected downward contnue to condense about $N_2$, while some of those which are 
deflected upwards pass through $N_1$ of the plate.
\begin{figure}[h!]
    \centering
    \includegraphics[width=.7\columnwidth]{PartOne/ChapterFive/preparationsystem.png}
    \caption{The atoms that pass through the hole made are all in the spin state $\ket{+}$. The Stern-Gerlach is then acting like a polarizer.}
\end{figure}

Each of the atoms of the beam which propagates to the right of the plate is a physical system on which we have just performed a measurement
of the observable $S_z$, the result being $+\hbar/2$: the atom is in the state $\ket{+}$.

The device of the figure thus produces a beam of atoms which are all in the spin state $\ket{+}$, acting like a polarizer.
%
\subsubsection{Preparation of the states $\ket{\pm}_x,\;\ket{\pm}_y,\;\ket{\pm}_z$}
To prepare one of eigenstates of $S_x$ (which is a CSCO), we must simply select, after a measurement of $S_x$, the atoms for which this measurement 
has yieldes the correspoding eigenvalue. If we rotate the device through an angle of $+\pi/2$ about Oy, we obtain a beam of atoms 
whose spin state is $\ket{+}_x$.

\begin{emphasizer}[]
    By placing the Stern-Gerlach device so that the axis of the magnetic field is parallel to an arbitrary unit vector $\bm{u}$, and piiercing the plate (at $N_1$ or $N_2$), we 
    can prepare silver atoms in the spin state $\ket{+}_u$ or $\ket{-}_u$.
\end{emphasizer}

%
\subsubsection{Preparation of the most general state}
It is possible to prepare atoms whose spin state is described by the correpsondin ket $\ket{\psi}$?

Tere exists, for all $\ket{\psi}$, a unit vector $\bm{u}$ such that $\ket{\psi}$ is collinear with the ket $\ket{+}_{\bm{u}}$.
We choose two complex numbers $\alpha,\beta$ that satisfy normalization equation in\eqref{eq:generalpsispin}.

We find that there necessarily exists an angle $\theta$ such that 
\begin{align}
    \cos\frac{\theta}{2}=|\alpha|\land\sin\frac{\theta}{2}=|\beta|,\quad\theta\in[0,\pi].
\end{align}
Let us set 
\begin{align}
    \begin{array}{l}
        \varphi=\arg\beta-\arg\alpha\\
        \chi=\arg\beta+\arg\alpha
    \end{array}\Longrightarrow
    \begin{array}{l}
        \arg\beta=\frac{1}{2}\chi+\frac{1}{2}\varphi\\
        \arg\alpha=\frac{1}{2}\chi-\frac{1}{2}\varphi
    \end{array}
\end{align}
With this notation, the ket $\ket{\psi}$ is written as 
\begin{align}
    \ket{\psi}=e^{i\chi/2}\left[\cos\frac{\theta}{2}e^{-i\varphi/2}\ket{+}+\sin\frac{\theta}{2}e^{i\varphi/2}\ket{-}\right].
\end{align} 
Consequently, to prepare silver atoms in the state $\ket{\psi}$, it suffices to place the Stern-Gerlach apparatus (with its plate pierced at $N_1$) so that its axis is directed along the vector $\bm{u}$.
%%
\subsection{Spin measurements}
If we place two Stern-Gerlach magnets one after the other, we can verify experimentally the predictions of the postules. The 
first acts like ap olarizer, while the second is used to measure a specified component of the angular momentum $\bL$: the analyzer.
%
\subsubsection{First experiment}
Assuming the axes of the two devices parallel to Oz. The first one prepares the aomt in the state $\ket{+}$ and the second one measure $L_z$.
\begin{figure}[h!]
    \centering
    \includegraphics[width=.6\columnwidth]{PartOne/ChapterFive/stern_first.png}
    \caption{The result obtained is certain $+\hbar/2$.}
\end{figure}

Snce the state of the system under studiy is an eigenstat of $S_z$ which we want to measure, the measurement result is \textbf{certain}: we 
find, without fail, the corresponding eigenvalue $+\hbar/2$. This is indeed what is observed experimentally: all the atoms strike the second plate 
in the vecinit of $N_1$, none hittin near $N_2$.
%
\subsubsection{Second experiment}
Let us place the axis of th first deviec along the unit vector $\bm{u}$, with $\theta,\phi=\pi$. $\bm{u}$ is therefore contained in the xOz plane.
The axis of the second device rameins parallel to Oz. 
\begin{figure}[h!]
    \centering
    \includegraphics[width=.6\columnwidth]{PartOne/ChapterFive/stern_second.png}
    \caption{The possible results are $+\hbar/2$ with $\cos^2\theta$ of change and $-\hbar/2$ with $\sin^2\theta/2$.}
\end{figure}

According to \eqref{eq:eigenstatesSu}, the spins state of the atoms when they leave the polarizer is:
\begin{align}
    \ket{\psi}=-\cos\frac{\theta}{2}\ket{+}+\sin\frac{\theta}{2}\ket{-}.
\end{align}
In this case, we find that certain atoms condense at $N_1$, and others at $N_2$.
We can compute directly that the the probabilities of the eigenstates are, respectively $\cos^2\theta/2$ and $\sin^2\theta/2$.
We observe experimentally that the intensity of the spots corresponds to numbers of aatoms which are proportional, to their respective probabilities.

The mean value of the results which would be obtained in a large number of identical experiments is $\braket{S_z}=\frac{\hbar}{2}\cos\theta$, which corresponds ot the classical result.
%
\subsubsection{Third experiment}
Let us take the second experiment, but with the analyzer rotated until its axis is directed along Ox, so that it measures the $L_x$ componente of the AM.

We must expand the state $\ket{\psi}$ after the first device in terms of the eigenstates of $S_x$:
\begin{align}
    \begin{array}{l}
        _x\braket{+|\psi}=\frac{1}{\sqrt{2}}(\cos\frac{\theta}{2}+\sin\frac{\theta}{2})=\cos(\frac{\pi}{4}-\frac{\theta}{2})\\
        _x\braket{-|\psi}=\frac{1}{\sqrt{2}}(\cos\frac{\theta}{2}-\sin\frac{\theta}{2})=\sin(\frac{\pi}{4}-\frac{\theta}{2})
    \end{array}
\end{align}
The probability of find the eienvalue $+\hbar/2$ of $S_x$ s therefore $\cos^2(\frac{\pi}{4}-\frac{\theta}{2})$ and that of finding $-\hbar/2$, $\sin^2(\frac{\pi}{4}-\frac{\theta}{2})$.

It is possible to verify these predictions by measuring th inensity of the two spots on the plate situated at the exit of the second Stern-Gerlach device.
%
\subsubsection{Mean values}
If we calculate the mean value of the possible results in the second experiment, we obtain:
\begin{align}
    \braket{S_z}=\frac{1}{N}\left[\frac{\hbar}{2}N\cos^2\frac{\theta}{2}-\frac{\hbar}{2}N\sin^2\frac{\theta}{2}\right]=\frac{\hbar}{2}\cos\theta.
\end{align}
This is indeed the value of the matrix element $\braket{\psi|S_z|\psi}$. Similarly, the average of the measurement results in the third experiment is
\begin{align}
    \braket{S_x}=\frac{1}{N}\left[\frac{\hbar}{2}N\cos^2(\frac{\pi}{2}-\frac{\theta}{2})-\frac{\hbar}{2}N\sin^2(\frac{\pi}{4}-\frac{\theta}{2})\right]=\frac{\hbar}{2}\sin\theta.
\end{align}
The matrix element can be computed matricially:
\begin{align}
    \braket{\psi|S_x|\psi}=\frac{\hbar}{2}\begin{bmatrix}
        \cos\theta/2&\sin\theta/2
    \end{bmatrix}\begin{bmatrix}
        0&1\\1&0
    \end{bmatrix}\begin{bmatrix}
        \cos\theta/2\\\sin\theta/2
    \end{bmatrix}=\frac{\hbar}{2}\sin\theta.
\end{align}
The mean value of $L_x$ is indeed equal to the matrix element, in the state $\ket{\psi}$, of the associated observable $S_x$.

There is an equivalence with the motion of classical mechanics. Using matrix computation, the mean values of $S_x,S_y,S_z$ in the state $\ket{+}_u$ is:
\begin{align}
    _u\braket{+|S_x|+}_u=\frac{\hbar}{2}\sin\theta\cos\phi,\quad _u\braket{+|S_y|+}_u=\frac{\hbar}{2}\sin\theta\sin\phi,\quad _u\braket{+|S_z|+}_u=\frac{\hbar}{2}\cos\theta.
\end{align}
These mean values are equal to the components of a classical agular momentum of modulus $\hbar/2$ oriented along the vector $\bm{u}$ with angles $(\theta,\phi)$.
However, recall that the only possible results of a measurements are $\pm\hbar/2$, not the above results of the mean values.

In the quantum sense, the motion of the atom is a linear superposition of the possible results, so both $\pm\hbar/2$ are traveling.
%%
\subsection{Evolution of a spin 1/2 particle in a uniform magnetic field}

\subsubsection{The interaction Hamiltonian and the Schrodinger equation}
Consider a silver atom in a unifrom magnetic field $\bB_0$, and choose the Oz axis along $\bB_0$.
The classical potential energy of the magnetic moment $\bm{\mu}=\gamma\bJ$ of this atom is then:
\begin{align}
    W=-\bm{\mu}\cdot\bB_0=-\mu_zB_0=\underbrace{-\gamma B_0}_{\omega_0}J_z,
\end{align}
where $\omega_0$ is the \bfemph{Larmor frequency}.
Since we are quantizing onl the internal degrees of freedom of the particle, $J_z$ must be replaced by the operator $S_z$, and the clasical energy above becomes an operator:
it is the Hamiltonian $H$ which describes the evolution of the spin of the atom in the field $\bB_0$:
\begin{align}
    H=\omega_0S_z.
    \label{eq:Hamiltonianspin12}
\end{align}
Since $H$ is time-independent, we solve the respective eigenequation. We see that the eigenvectors of $H$ are tose of $S_z$:
\begin{align}
    H\ket{\pm}=\pm\frac{\hbar\omega_0}{2}\ket{\pm}=E_\pm\ket{\pm}.
\end{align}
There are theerefore two energy levels, $E_\pm$. Their separation $\hbar\omega_0$ is proportional to the B-field; they define a single Bohr frequency:
\begin{align}
    \nu_{+-}=\frac{1}{\hbar}(E_+-E_-)=\frac{\omega_0}{2\pi}.
\end{align}
\begin{itemize}[itemsep=0pt,topsep=0pt]
    \item If $\bB_0$ is parallel to the unit vector $\bm{u}$, the Hmailtonian \eqref{eq:Hamiltonianspin12} msut be replaced by its general form:
    \begin{align}
        \text{General form Hamiltonian}\qquad\highlight{H=\omega_0\bS\cdot\bm{u}}.
    \end{align}
    \item For silver atoms, $\gamma<0$; $\omega_0$ is therefore positive.
\end{itemize}
%%
\subsubsection{Larmor precession}
Consider the spin at $t=0$ in the state 
\begin{align}
    \ket{\psi(0)}=\cos\frac{\theta}{2}e^{-i\phi/2}\ket{+}+\sin\frac{\theta}{2}e^{i\phi/2}\ket{-}.
\end{align}
We saw that any state can be put in this form. To calculate the state at $t>0$, we apply the evolution operator:
\begin{align*}
    \ket{\psi(t)}=\cos\frac{\theta}{2}e^{-i\phi/2}e^{-iE_+t/\hbar}\ket{+}+\sin\frac{\theta}{2}e^{i\phi/2}e^{-iE_-t/\hbar}\ket{-}=\cos\frac{\theta}{2}e^{-\frac{i(\phi+\omega_0t)}{2}}\ket{+}+\sin\frac{\theta}{2}e^{\frac{i(\phi+\omega_0t)}{2}}\ket{-}.
\end{align*}
The presence of $\bB_0$ therefore introduces a phase shift between $\ket{+}$ and $\ket{-}$. The direction of $\bm{u}(t)$ along which the spin 
component if $+\hbar/2$ with certainty is defined by the polar angles:
\begin{align}
    \begin{array}{l}
        \theta(t)=\theta\\
        \phi(t)=\phi+\omega_0t
    \end{array}
\end{align}
The angle between $\bm{u}(t)$ and Oz therefore remains constant, but $\bm{u}(t)$ revolves about Oz at an angular velocity of $\omega_0$. THis effect
is called the \bfemph{Larmor precession}. 

It can verified from $\ket{\psi(t)}$ that the probabilities of obtaining $+\hbar/2$ or $-\hbar/2$ in a measurement of this observables are time-indendent. These probabilities are equal, respectively, to 
$\cos^2\theta/2$ and $\sin^2\theta/2$. The mean value of $S_z$ is also time-independent:
\begin{align}
    \braket{\psi(t)|S_z|\psi(t)}=\frac{\hbar}{2}\cos\theta.
\end{align}
Because $S_x$ and $S_y$ do not comute with $H$, we have that 
\begin{align}
    \braket{\psi(t)|S_x|\psi(t)}=\frac{\hbar}{2}\sin\theta\cos(\phi+\omega_0t),\quad\braket{\psi(t)|S_y|\psi(t)}=\frac{\hbar}{2}\sin\theta\sin(\phi+\omega_0t).
\end{align}
We again find the Bohr frequencies $\omega_0/2\pi$ of the system. Moreover, the mean values above behave like the components of a classical AM of modulus $\hbar/2$ undergoing Larmos precession.


