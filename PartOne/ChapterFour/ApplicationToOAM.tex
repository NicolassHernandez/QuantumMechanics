\section{Application to orbital angular momentum}
We return now to the orbital angular momentum $\bL$ of a spinless particle and see how the general theory developed applies to this particular case.

Using the $\{\ket{\bR}\}$ representation, we shall show that the eigenvalues of the operator $\bL^2$ are the numbers $l(l+1)\hbar^2$. Then, we shall indicate 
the eigenfunctions common to $\bL^2$ and $L_z$ and their principal properties.
%%
\subsection{Eigenvalues and eigenfunctions of $\bL^2$ and $L_z$}
%%
\subsubsection{Eigenequation in the $\{\ket{\bR}\}$ representation}
In the $\{\ket{\bR}\}$ representation, the observables $\BR$ and $\BP$ correspond respectively to multiplication by $\bR$ and to the differential operator $(\hbar/i)\nabla$. 
The three components can therefore be written as
\begin{align}
    L_x=\frac{\hbar}{i}(y\partial_z-z\partial_y),\quad L_y=\frac{\hbar}{i}(z\partial_x-x\partial_z),\quad L_z=\frac{\hbar}{i}(x\partial_y-y\partial_x).
\end{align}
It is more convenient to work in spherical coordinates, since the various AM operators act only on the angular variables $\theta$ and $\phi$, and not on $r$.

The conversion of a point M at $\bR=(x,y,z)$ in the spherical coorindates $\bR=(r,\theta,\phi)$ is therefore:
\begin{align}
    \text{Spherical coordinates}\qquad\begin{array}{l}
        x=r\sin\theta\cos\phi\\
        y=r\sin\theta\sin\phi\\
        z=r\cos\theta
    \end{array},\quad\text{with}\quad\begin{array}{l}
        r\geq0\\
        0\in[0,\pi]\\
        \phi\in[0,2\pi]\\
        d^3r=r^2\sin\theta dr d\theta d\phi
    \end{array}.
\end{align}
Using them in the AM operators yield
{\small
\begin{align}
    &\highlight{L_x=i\hbar\left(\sin\phi\partial_\theta+\frac{\cos\phi}{\tan\theta}\partial_\phi\right),\quad L_y=i\hbar\left(-\cos\phi\partial_\theta+\frac{\sin\phi}{\tan\theta}\partial_\phi\right),\quad L_z=\frac{\hbar}{i}\partial_\phi}\\
    &\highlight{\bL^2=-\hbar^2\left(\partial^2_\theta+\frac{1}{\tan\theta}\partial_\theta+\frac{1}{\sin^2\theta}\partial^2_\phi\right),\quad L_+=\hbar e^{i\phi}\left(\partial_\theta+i\cot\theta\partial_\phi\right),\quad L_-=\hbar e^{-i\phi}\left(-\partial_\theta+i\cot\theta\partial_\phi\right).}
    \label{eq:Loperatorspherical}
\end{align}}
In the $\{\ket{\bR}\}$ representation, the eigenfunctions associated with the eigenvalues $l(l+1)\hbar^2$ of $\bL^2$ and $m\hbar$ of $L_z$ are the solutions of the following partial differential equations:
\begin{align}
    \left.\begin{array}{r}
        \displaystyle-\left[\partial^2_\theta+\frac{1}{\tan\theta}\partial_\theta+\frac{1}{\sin^2\theta}\partial^2_\phi\right]\psi(r,\theta,\phi)=l(l+1)\psi(r,\theta,\phi)\\
        \displaystyle-i\partial_\phi\psi(r,\theta,\phi)=m\psi(r,\theta,\phi)
    \end{array}\right\rfloor
\end{align}
We already know that $l$ is integral or half-integral and that, for fixed $l$, $m$ can take only the values $-l,\cdots,l$.
We also notice from the equations that $r$ does not appear in any differential operator, se we can consider it to be a parameter and take into account only the $\theta-$ and $\phi-$dependence of $\psi$.

Thus, we denote by $Y_l^m(\theta,\phi)$ a common eigenfunction of $\bL^2$ and $L_z$ which corresponds to the eigenvalues $l(l+1)\hbar^2$ and $m\hbar$:
\begin{align}
    \begin{array}{l}
        \bL^2Y_l^m(\theta,\phi)=l(l+1)\hbar^2 Y_l^m(\theta,\phi)\\
        L_zY_l^m(\theta,\phi)=m\hbar Y_l^m(\theta,\phi)
    \end{array}
    \label{eq:equationYml}
\end{align}
These equations have only one solution for each pairt of allowed values of $l$ and $m$, both indices are sufficient.

\begin{itemize}[itemsep=0pt,topsep=0pt]
    \item Equation \eqref{eq:equationYml} gives the $\theta,\phi-$dependence of the eigenfuntions of $\bL^2$ and $L_z$, after which we construct the complete eigenfunction:
    \begin{align}
        \phi_{l,m}(r,\theta,\phi)=f(r)Y_l^m(\theta,\phi).
    \end{align}
    The fact that $f(r)$ is arbitrary shows that $\bL^2$ and $L_z$ do not form a CSCO in the space $\E_r$ of functions of $r$.
    \item In order to normalize $\phi_{l,m}$ it is convenient to normalize $Y_l^m$ and $f$ \textbf{separately}:
    {\small
    \begin{align}
        \int\limits_{r=0}^\infty\int\limits_{\phi=0}^{2\pi}\int\limits_{\theta=0}^\pi|\psi_{m,l}(r,\theta,\phi)|^2\;r^2\sin\theta drd\theta d\phi=\int\limits_{\phi=0}^{2\pi}\int\limits_{\theta=0}^\pi|Y_l^m(\theta,\phi)|^2\;\sin\theta d\theta d\phi=\int\limits_{r=0}^\infty r^2|f(t)|^2\;dr=1.
    \end{align}}
\end{itemize}
%%
\subsubsection{Values of $l$ and $m$}
\begin{itemize}[itemsep=0pt,topsep=0pt]
    \item\textbf{$l$ and $m$ must be integral}\\
    Using $L_z=\frac{\hbar}{i}\partial_\phi$ in \eqref{eq:equationYml} we have:
    \begin{align}
        \frac{\hbar}{i}\partial_\phi Y_l^m(\theta,\phi)=m\hbar Y_l^m(\theta,\phi)\longrightarrow Y_l^m(\theta,\phi)=F_l^m(\theta)e^{im\phi}.
        \label{eq:YmlFmlexp}
    \end{align}
    We can cover all space by letting $\phi$ vary in $[0,2\pi]$. We must have by continuity of the solution,
    \begin{align}
        Y_l^m(\theta,\phi=0)=Y_l^m(\theta,\phi=2\pi)\Longrightarrow e^{2im\pi}=1.
    \end{align}
    Therefore, in an orbital AM $l$ and $m$ must be integer.
    \item\textbf{All integral values $\geq0$ of $l$ can be found}\\
    Given an integral value of $l$, we know from the prevous section that:
    \begin{align*}
        L_+Y_l^l(\theta,\psi)=0.
    \end{align*}
    Using \eqref{eq:eqationYml} and \eqref{eq:YmlFmlexp} yields
    \begin{align*}
        \left[\partial_\theta-l\cot\theta\right]F_l^l(\theta)=0.
    \end{align*}
    This first-order ODE can be integrated noting that $\cos\theta\;d\theta=d(\sin\theta)/\sin\theta$. The solution is:
    \begin{align*}
        F_l^l(\theta)=c_l(\sin\theta)^l.
    \end{align*}
    Thus, for each positive or zero integral value of $l$, there exists a function $Y_l^l(\theta,\phi)$ which is unique:
    \begin{align}
        \text{Spherical harmonics}\qquad\highlight{Y_l^l=c_l(\sin\theta)^le^{il\phi}}.
        \label{eq:sphericalharmonic}
    \end{align}
    Through repeated action fo $L_-$, we construct $Y_l^{l-1},\cdots,Y_l^m,\cdots,Y_l^{-l}$. We see then there corresponds to the pair of eigenvalues $l(l+1)\hbar^2$ and $m\hbar$
    one and only one eigenfunction $Y_l^m(\theta,\phi)$, which can be computed from \eqref{eq:sphericalharmonic}.
\end{itemize}
%%
\subsubsection{Fundamental properties of the spherical harmonics}
We summarize some properties of spherical harmonics.
\begin{itemize}[itemsep=0pt,topsep=0pt]
    \item\textbf{Recurrence relations}\\
    \begin{align}
        \begin{array}{l}
        e^{i\phi}\left[\partial_\theta-m\cot\theta\right]Y_l^m(\theta,\phi)=\sqrt{l(l+1)-m(m+1)}Y_l^{m+1}(\theta,\phi)\\
        e^{i\phi}\left[-\partial_\theta-m\cot\theta\right]Y_l^m(\theta,\phi)=\sqrt{l(l+1)-m(m-1)}Y_l^{m-1}(\theta,\phi)
        \end{array}
    \end{align}
    \item\textbf{Orthonormalization and closure relations}\\
    Normalization imposed on the spherical harmonics yields
    \begin{align}
        \int_0^{2\pi}\int_0^\pi\sin\theta Y_{l'}^{m'*}(\theta,\phi)Y_l^m(\theta,\phi)\;\sin\theta d\theta d\phi=\delta_{l'l}\delta_{m'm}.
    \end{align}
    Furtheremore, any function $f(\theta,\phi)$ can be expanded in terms of the spherical hermonics:
    \begin{align}
        f(\theta,\phi)=\sum_{l=0}^\infty\sum_{m=-l}^{+l}c_{l,m}Y_l^m(\theta,\phi),\quad\text{with}\quad c_{l,m}=\int_0^{2\pi}\int_0^\pi f(\theta,\phi)Y_l^{m*}(\theta,\phi)\;d\theta d\phi.
    \end{align}
    The spherical harmonics therefore constitute an orthonormal basis in the space $\E_\Omega$, which means the surface of a sphere with fixed value of $r$.
    This fact in expressed by the closure relation:
    \begin{align}
        \sum_{l=0}^\infty\sum_{m=-l}^{l}Y_l^m(\theta,\phi)Y_l^{m*}(\theta',\phi')=\delta(\cos\theta-\cos\theta')\delta(\phi-\phi')=\frac{1}{\sin\theta}\delta(\theta-\theta')\delta(\phi-\phi').
    \end{align}
    \item\textbf{Parity and complex conjugation}\\
    A reflection through the coordinate origin is expressed as $\bR\longrightarrow-\bR$ in cartesian coordinates, and as 
    \begin{align}
        \text{Reflection}\qquad\begin{array}{l}
            r\longrightarrow r\\
            \theta\longrightarrow \pi-\theta\\
            \phi\longrightarrow \pi+\phi
        \end{array}
    \end{align}
    in spherical coordinates. It is simple to show that a reflection in $Y_l^m$ yields:
    \begin{align}
        \text{Reflection in spherical harmonics}\qquad\highlight{Y_l^m(\pi-\theta,\pi+\phi)=(-1)^lY_l^m(\theta,\phi)}.
    \end{align}
    Also, that a complex conjugation results in
    \begin{align}
        \text{Complex conjugatino of spherical harmonics}\qquad\highlight{[Y_l^m(\theta,\phi)]^*=(-1)^mY_l^{-m}(\theta,\phi).}
    \end{align}
\end{itemize}

%%
\subsubsection{Standard bases of the wave function space of a spinless particle}
We know that $\bL^2$ and $L_z$ do not constitute a CSCO in the wave function spcae of a spinless particle. We shall now indicate the form of the standard bases of this space.
By repeated application of $L_-$ on $\psi_{k,l,l}(\bR)$, we then construct the functions $\psi_{k,l,m}(\bR)$ which complete the standard basis for $m\neq l$. They satisfy the equations in \label{eq:sumeigenequationJ1} and \label{eq:sumeigenequationJ2}:

But we saw that all eigenfunctions common to $\bL^2$ and $L_z$ that correspond to given eigenvalues $l(l+1)\hbar^2$ and $m\hbar$ have the same angualr dependence in $Y_l^m(\theta,\phi)$, and only the radial dependence differs.
From the same above equations we deduce that the functions $\psi_{k,l,m}(\bR)$ of a standard basis of the wave function space of a spinless particle have the form:
\begin{align}
    \psi_{k,l,m}(\bR)=R_{k,l}(r)Y_l^m(\theta,\phi).
\end{align}
Notice the dependence indices for each term of the right-side. The orthonormalization relation for the radial function is:
\begin{align}
    \int_0^\infty R^*_{k,l}(r)R_{k',l}(r)\;r^2dr=\delta_{kk'}.
\end{align}
Notice also the orthogonality is to respect to the $k$ variable.
\begin{itemize}[itemsep=0pt,topsep=0pt]
    \item If we want the basis function $\psi_{k,l,m}(\bR)$ to be continuous, only the radial functions corresponding to $l=0$ can be non-zero at $r=0$.
\end{itemize}
%%
\subsection{Physical considerations}

\subsubsection{Study of a $\ket{k,l,m}$ state}



\subsubsection{Calculations of the physical predictions concerning measurements of $\bL^2$ and $L_z$}


