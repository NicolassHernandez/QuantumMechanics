\section{Introduction}
We know the important role of angular momentum in cassical mechanics; the totl angular momentum of an isolated physical system is a \bfemph{constant of motion}, even in some 
non-isolated systems. In other cases, when  point particle is moving in a central potential (radial dependence only), the force on the particle is always directed towards the origin of the potential.
Its moment with respect to this origin is zero, and the angular momentum theorem implies that:
\begin{align*}
    \partial_t\mathcal{L}=0.
\end{align*}
The motion of the particle is therefore limited to a fixed plane, and satisfies the law of constant areas (Kepler's second law). All these properties have their 
equivalences in quantum mechanics.


With an angular momentum $\mathcal{L}$ of a classical system is associated an observable $\bL$, actually a set of three observables $L_x,L_y,L_z$, which correspond to the three components of 
$\mathcal{L}$ in a Cartesia frame. We must also introduce typically quantum mechanical angular momenta, which have no classical equivalents.

We shall denote:
\begin{itemize}[itemsep=0pt,topsep=0pt]
    \item\bfemph{Orbital angular momentum $\bL$} is any angular momentum that has a classical equivalent.\\
    \item\bfemph{Spin angular momentum $\bS$} is any intrinsic angular momentum of an elementary particle.
    \item\bfemph{Total angular momentum $\bJ$} is the sum of $\bL$ and $\bS$. We also used to refer to any angular momentum without specifying the type.
\end{itemize}

We will establish the general quanum mechanical properties associated with all AM, whatever their nature. These properties follow from the commutation relations
satisfied by the three observables $J_x,J_y,J_z$, the compnents of an arbitrary AM $\bJ$. The origin of these commutation relations is the consequences of the quantization rules and the 
canonical relations.







