\documentclass[letterpaper,11pt,twoside]{article}
\usepackage{graphicx} % Required for inserting images
\usepackage[table,xcdraw,dvipsnames]{xcolor}
\usepackage{amsmath,amsfonts,amssymb,amsthm}
\usepackage{listings}
\usepackage{lipsum}
\usepackage{hyperref}
\usepackage{mathrsfs}

\usepackage{enumitem}

\usepackage{tikz}
\usepackage[siunitx, RPvoltages]{circuitikz}
\usetikzlibrary{3d}
\usepackage{comment}
\usepackage{caption,subcaption}
\usepackage{pgfplots}
\pgfplotsset{compat=newest} % or a newer version if available
\usepgfplotslibrary{groupplots}
\usetikzlibrary{pgfplots.groupplots}
\usetikzlibrary{shapes.geometric, arrows}
\tikzstyle{arrow} = [->,>=stealth,shorten >=2pt]
\newcommand{\ket}[1]{|#1\rangle}
\newcommand{\bra}[1]{\langle#1|}
\newcommand{\br}{\bm{r}}
\newcommand{\bR}{\bm{R}}
\newcommand{\bp}{\bm{p}}
\newcommand{\bP}{\bm{P}}
\newcommand{\braket}[1]{\langle#1\rangle}
\newcommand{\F}{\mathscr{F}}
\newcommand{\E}{\mathscr{E}}
\newcommand{\re}[1]{\text{Re}\left(#1\right)}
\newcommand{\im}[1]{\text{Im}\left(#1\right)}
\usepackage{dsfont}
\usepackage{cancel}
\usepackage{bm}
\usepackage{fancyhdr}
\usepackage[utf8x]{inputenc}
\usepackage[T1]{fontenc}
\usepackage[margin=0.8in,top=1in,bottom=1in]{geometry}
%%%%%
\begin{filecontents*}{refs.bib}
@book{bornwolf,
  author    = {Born, M. and Wolf, E.},
  title     = {Principles of Optics},
  publisher = {Pergamon Press},
  edition   = {7},
  year      = {1999}
}
@book{hecht,
  author    = {Hecht, E.},
  title     = {Optics},
  publisher = {Addison-Wesley},
  edition   = {5},
  year      = {2016}
}
\end{filecontents*}
%
\newcommand{\institution}{University of Arizona}
\newcommand{\autor}{Nicolás Hernández Alegría}
\newcommand{\course}{OPTI 570 Quantum Mechanics}
\newcommand{\assignment}{Assignment 8}
%
\title{\textbf{\assignment}\\\course\\{\Large\institution}}
\author{\autor}
\date{\today\\Total time: 8 hours}
%
\renewcommand{\sectionmark}[1]{\markright{#1}}
\fancypagestyle{mainstyle}{
    \fancyhf{} % Clear all header and footer fields
    \fancyfoot[C]{\thepage}
    \fancyhead[LE,RO]{\course} % Section name on odd pages
    \fancyhead[LO,RE]{\assignment}
    % Optional: Thin rules
    \renewcommand{\headrulewidth}{0pt} % Header rule
    \renewcommand{\footrulewidth}{0pt} % No footer rule
}
%
\begin{document}

\pagestyle{mainstyle}
\maketitle
%%
\section*{Problem I}
Given that $j=1$, we have the magnetic quantum number $m\in\{1,0,-1\}$ and $\{\ket{z_+},\ket{z_0},\ket{z_-}\}=\{\ket{1,1},\ket{1,0},\ket{1,-1}\}$.
The state vector can then be spanned as:
\begin{align*}
  \ket{\psi}=a\ket{z_+}+b\ket{z_0}+c\ket{z_-}\longrightarrow\ket{\psi}=\begin{bmatrix}
    a\\b\\c
  \end{bmatrix}.
\end{align*}
\begin{enumerate}[itemsep=0pt,topsep=0pt,label=\alph*)]
  \item 
\begin{enumerate}[itemsep=0pt,topsep=0pt,label=\roman*)]
  \item In the state vector method, we want to compute $\braket{\psi|J_x|\psi}$. Using the eigenequation for the ladder operators:
  \begin{align*}
    J_\pm=\ket{j,m}=\hbar\sqrt{j(j+1)-m(m\pm1)}\ket{j,m\pm1},
  \end{align*}
  we have for $j=1$
  \begin{align*}
    j_+\ket{1,0}=\hbar\sqrt{2}\ket{1,1},\quad J_+\ket{1,-1}=\hbar\sqrt{2}\ket{1,0},\quad J_-\ket{1,1}=\hbar\sqrt{2}\ket{1,0},\quad J_-\ket{1,0}=\hbar\sqrt{2}\ket{1,-1}.
  \end{align*}
  In matrix form, we have 
  \begin{align*}
    J_+=\hbar\sqrt{2}\begin{bmatrix}
      0&1&0\\0&0&1\\0&0&0
    \end{bmatrix},\quad J_-=\hbar\sqrt{2}\begin{bmatrix}
      0&0&0\\1&0&0\\0&1&0
    \end{bmatrix}\longrightarrow J_x=\frac{1}{2}(J_++J_-)=\frac{\hbar}{\sqrt{2}}\begin{bmatrix}
      0&1&0\\1&0&1\\0&1&0
    \end{bmatrix}.
  \end{align*}
  Its action on the state is a matrix-multiplication:
  \begin{align*}
    \braket{\psi|J_x|\psi}=\frac{\hbar}{\sqrt{2}}\begin{bmatrix}
      a^*&b^*&c^*
    \end{bmatrix}\begin{bmatrix}
      0&1&0\\1&0&1\\0&1&0
    \end{bmatrix}\begin{bmatrix}
      a\\b\\c
    \end{bmatrix}=\frac{\hbar}{\sqrt{2}}(a^*b+b^*a+b^*c+c^*b)\equiv\frac{2\hbar}{\sqrt{2}}[\re{a^*b}+\re{b^*c}].
  \end{align*}
  \item In the density operator method, the expectation we want to compute is 
  \begin{align*}
    \braket{J_x}=\text{Tr}(\rho J_x).
  \end{align*}
  In a pure state, we have that 
  \begin{align*}
    \ket{\psi}=a\ket{z_+}+b\ket{z_0}+c\ket{z_-},\quad\rho=\ket{\psi}\bra{\psi}.
  \end{align*}
  In matrix form, the density operator is 
  \begin{align*}
    \rho=\begin{bmatrix}
      |a|^2&ab^*&ac^*\\
      ba^*&|b|^2&bc^*\\
      ca^*&cb^*&|c|^2
    \end{bmatrix}.
  \end{align*}
  The trace then mean to sum over the diagonal of this matrix multiplied by $J_x$, that is, over the following elements:
  \begin{align*}
    (\rho J_x)_{11}&=\rho_{12}J_{21}+\rho_{13}J_{31}=ab^*\frac{\hbar}{\sqrt{2}}\\
    (\rho J_x)_{22}&=\rho_{21}J_{12}+\rho_{23}J_{32}=ba^*\frac{\hbar}{\sqrt{2}}+bc^*\frac{\hbar}{\sqrt{2}}\\
    (\rho J_x)_{33}&=\rho_{31}J_{13}+\rho_{32}J_{23}=cb^*\frac{\hbar}{\sqrt{2}}
  \end{align*}
  Then, the trace, and therefore the meanvalue is,
  \begin{align*}
    \braket{J_x}=(\rho J_x)_{11}+(\rho J_x)_{22}+(\rho J_x)_{33}=\frac{\hbar}{\sqrt{2}}[ab^*+ba^*+bc^*+cb^*]=\frac{2\hbar}{\sqrt{2}}[\re{a^*b}+\re{b^*c}].
  \end{align*}
\end{enumerate}
  \item For $j=1$, 
  \begin{align*}
    J^2=J_x^2+J_y^2+J_z^2=j(j+1)\hbar^2\mathds{1}=2\hbar^2\mathds{1}.
  \end{align*}
  Then, for any state we have that 
  \begin{align*}
    \braket{J^2}=\braket{\psi|2\hbar\mathds{1}|\psi}=2\hbar^2(|a|^2+|b|^2+|c|^2)=2\hbar^2.
  \end{align*}
  \item We know that J operator matrices for the spin-1 are:
  \begin{align*}
    J_x=\frac{\hbar}{\sqrt{2}}\begin{bmatrix}
      0&1&0\\1&0&1\\0&1&0
    \end{bmatrix},\quad J_y=\frac{\hbar}{\sqrt{2}}\begin{bmatrix}
      0&-i&0\\i&0&-i\\0&i&0
    \end{bmatrix},\quad J_z=\hbar\begin{bmatrix}
      1&0&0\\0&0&0\\0&0&-1
    \end{bmatrix}
  \end{align*}
  If we square them, we have:
  \begin{align*}
    J_x^2=\frac{\hbar^2}{2}\begin{bmatrix}
      1&0&1\\0&2&0\\1&0&1
    \end{bmatrix},\quad 
    J_y^2=\frac{\hbar^2}{2}\begin{bmatrix}
      1&0&-1\\0&2&0\\-1&0&1
    \end{bmatrix},\quad 
    J_z^2=\hbar^2\begin{bmatrix}
      1&0&0\\0&0&0\\0&0&1
    \end{bmatrix}
  \end{align*}
  The expectation $\braket{\psi|J^2_{i}|\psi}$ are then:
  \begin{align*}
    \braket{J_x^2}&=\frac{\hbar^2}{2}(|a|^2+2|b|^2+|c|^2+2\re{a^*c})\\
    \braket{J_y^2}&=\frac{\hbar^2}{2}(|a|^2+2|b|^2+|c|^2-2\re{a^*c})\\
    \braket{J_z^2}&=\hbar^2(|a|^2+|b|^2)\\
    \braket{J^2}&=\braket{J_x^2}+\braket{J_y^2}+\braket{J_z^2}=2\hbar(|a|^2+|b|^2+|c|^2)=\braket{J^2}
  \end{align*} 
  \item We have from our previous parts both expectation required, so it becomes an algebra problem
  \begin{align*}
    (\Delta J_x)^2&=\braket{J_x^2}-\braket{J_x}^2\\
    &=\frac{\hbar^2}{2}[|a|^2+2|b|^2+|c|^2+2\re{a^*c}-|a^*b+b^*a+b^*c+c^*b|^2]\\
    (\Delta J_x)^2&=\frac{\hbar^2}{2}[1+|b|^2+2\re{a^*c}-4(\re{a^*b}+\re{b^*c})^2]\quad(|a|^2+|b|^2+|c|^2=1)
  \end{align*}
  \item If $b=0$, then the normalization condition becomes $|a|^2+|c|^2=1$, and 
  \begin{align*}
    \braket{J_x}=0,\quad\braket{J_x^2}=\frac{\hbar^2}{2}[|a|^2+|c|^2+2\re{a^*c}]=\frac{\hbar^2}{2}[1+2\re{a^*c}].
  \end{align*}
  Thus, we need to minimize the alst result, which means that $\re{a^*c}$ must take its minimum value. If $a=|a|e^{i\alpha}$ and $c=|c|e^{i\gamma}$, then 
  we can express the real part as $|a||c|\cos(\alpha-\gamma)$. The minimum value of the cosine is when $\alpha-\gamma=\pi$. Assumind that both magnitudes are equal (simplest scenario), we therefore have $c=-a$ and :
  \begin{align*}
    a=\frac{e^{i\phi}}{\sqrt{2}},\quad b=0,\quad c=-\frac{e^{i\phi}}{\sqrt{2}}.
  \end{align*} 
  Which then gives 
  \begin{align*}
    \braket{J_x^2}=\frac{\hbar^2}{2}(1-1)=0\Longrightarrow(\Delta J_x)^2=0.
  \end{align*}
  \item We can just pick an eigenstate of $J_z$,
  \begin{align*}
    a=0,\quad b=e^{i\phi},\quad c=0.
  \end{align*}
  Thus,
  \begin{align*}
    \ket{\psi}=\ket{z_0}\quad (\Delta J_z)^2=0.
  \end{align*}
  We could also pick a non-zero value for $a$ or $c$.
  \item We use the generalized Heisenberg uncertainty relation to compute the required uncertainty product:
  \begin{align*}
    \Delta J_y\Delta J_z\geq\frac{1}{2}|\braket{[J_y,J_z]}|=\frac{1}{2}|\braket{i\hbar J_x}|=\frac{\hbar^2}{\sqrt{2}}|\braket{\re{a^*b}+\re{b^*c}}|.
  \end{align*}
  We see that the uncertainty relation of two perpendicular angular momentum components $J_y$ and $J_z$ dependds on how much angular momentum you have along the third component $x$.
  This relations is telling us that the more precisely the $J_x$ is known on average, the more uncertain are the perpendicular components.
\end{enumerate}


%%
\section*{Problem II}
\begin{enumerate}[itemsep=0pt,topsep=0pt,label=\alph*)]
  \item The expansion of $\ket{x_+}$ for $\lambda=+\hbar$ is 
  \begin{align*}
    \ket{x_+}=\frac{1}{2}\ket{z_+}+\frac{1}{\sqrt{2}}\ket{z_0}+\frac{1}{2}\ket{z_-}.
  \end{align*}
  If $\ket{\psi(0)}=\ket{z_0}$, then 
  \begin{align*}
    \braket{x_+|\psi(0)}=\braket{x_+|z_0}=\frac{1}{\sqrt{2}}.
  \end{align*}
  The probability to measure $J_x\hbar$ is therefore 
  \begin{align*}
    P_x(\hbar)=|\braket{x_+|z_0}|^2=\frac{1}{2}.
  \end{align*}
  \item Now for $\lambda=0$, the expansion is 
  \begin{align*}
    \ket{x_0}=\frac{1}{\sqrt{2}}(\ket{z_+}-\ket{z_-}).
  \end{align*}
  The inner product $\ket{x_0|\psi(0)}$ is now zero, so the probability of measuring $J_x=0$ is zero.
  \item For $\lambda=-\hbar$, 
  \begin{align*}
    \ket{x_-}=\frac{1}{2}\ket{z_+}-\frac{1}{\sqrt{2}}\ket{z_0}+\frac{1}{2}\ket{z_-}.
  \end{align*}
  The projection of $\ket{\psi(0)}$ onto $\ket{x_-}$ is $1/\sqrt{2}$, so that 
  \begin{align*}
    P_x(-\hbar)=|\braket{x_-|z_0}|^2=\frac{1}{2}.
  \end{align*}
  \item Since $\ket{\psi(0)}=\ket{z_0}$, it is evident that the probability of getting $\ket{z_+}$ is not possible. So,
  \begin{align*}
    P_z(\hbar)=|\braket{z_+|z_0}|^2=0.
  \end{align*}
  \item After the measurement, the state will evolve and live between the eigenstate of $J_x$ with certain probabilities:
  \begin{align*}
    \rho=\frac{1}{2}\ket{x_+}\bra{x_+}+\frac{1}{2}\ket{x_-}\bra{x_-}\quad(P_x(0)=0).
  \end{align*}
  This is a mixture of the eigenstate with the coefficient being the probability. If now we take a measurement with $J_z$, we 
  will have, for a given angular momentum $m$ 
  \begin{align*}
    P_z(m\hbar)=\text{Tr}(\ket{z_m}\bra{z_m}\rho)=\frac{1}{2}\left[|\braket{zm|x_+}|^2+|\braket{zm|x_-}|^2\right].
  \end{align*}
  In $\ket{x_+}$ and $\ket{x_-}$ we substitute their expression in terms of the eigenstate of $J_z$ and evaluate for $m\in\{\hbar,0,-\hbar\}$:
  \begin{align*}
    P_z(\hbar)=\frac{1}{2}\left(\frac{1}{4}+\frac{1}{4}\right)=\frac{1}{4},\quad P_z(0)=\frac{1}{2}\left(\frac{1}{2}+\frac{1}{2}\right)=\frac{1}{2},\quad P_z(-\hbar)=\frac{1}{2}\left(\frac{1}{4}+\frac{1}{4}\right)=\frac{1}{4}.
  \end{align*}
  \item If the measurement is uncertain, then we can express it as a mixed state. However, if we know the result then it collapses into a single eigenstate of $J_x$, that is, a pure state. We 
  can see for the statistical mixture that the trace of the density operator squared is less than 1, reinforcing this fact: $\text{Tr}(\rho^2)=1/2<1$.
  \item This is the same we have computed previouslty, its matrix form is 
  \begin{align*}
    \rho^{(x)}=\frac{1}{2}\ket{x_+}\bra{x_+}+\frac{1}{2}\ket{x_-}\bra{x_-}=\begin{bmatrix}
      1/2&0&0\\0&0&0\\0&0&1/2
    \end{bmatrix}.
  \end{align*}
  \item The transformation matrix is needed to pass from $x$ to $z$, so we use the above definition in terms of $x$ to express in terms of $z$ and the formulation given in the field guide.
  \begin{align*}
    \ket{x_+}&=\frac{1}{2}\ket{z_+}+\frac{1}{\sqrt{2}}\ket{z_0}+\frac{1}{2}\ket{z_-}\\
    \ket{x_0}&=\frac{1}{\sqrt{2}}\ket{z_+}-\frac{1}{\sqrt{2}}\ket{z_-}\\
    \ket{x_-}&=\frac{1}{2}\ket{z_+}-\frac{1}{\sqrt{2}}\ket{z_0}+\frac{1}{2}\ket{z_-}
  \end{align*}
  We concatenate them to construct the transformation matrix:
  \begin{align*}
    M=\begin{bmatrix}
      |&|&|\\
      \ket{x_+}&\ket{x_0}&\ket{x_-}\\
      |&|&|
    \end{bmatrix}=\begin{bmatrix}
      \frac{1}{2}&\frac{1}{\sqrt{2}}&\frac{1}{2}\\
      \frac{1}{\sqrt{2}}&0&-\frac{1}{\sqrt{2}}\\
      \frac{1}{2}&-\frac{1}{\sqrt{2}}&\frac{1}{2}
    \end{bmatrix}.
  \end{align*}
  We use it to transform the state vector and the density operator as:
  \begin{align*}
    \ket{\psi}_z=M\ket{\psi}_x,\quad\text{and}\quad\rho^{(z)}=M\rho^{(x)}M^\dagger.
  \end{align*}
  This is exactly the same process shown in the field guide, because these elements on $M$ correspond to the inner product $M_{ij}=\braket{z_i|x_j}$.
  \item The specific matrix $\rho^{(z)}$ is 
  \begin{align*}
    \rho^{(z)}=\begin{bmatrix}
      \frac{1}{4}&0&\frac{1}{4}\\0&\frac{1}{2}&0\\\frac{1}{4}&0&\frac{1}{4}.
    \end{bmatrix}
  \end{align*}
  Using it in the expectation values yields 
  \begin{align*}
    \braket{J_z}&=\text{Tr}(\rho^{(z)}J_z)=\hbar\text{Tr}\left(\begin{bmatrix}
      \frac{1}{4}&0&\frac{1}{4}\\0&\frac{1}{2}&0\\\frac{1}{4}&0&\frac{1}{4}.
    \end{bmatrix}\cdot\begin{bmatrix}
      1&0&1\\0&0&0\\0&0&-1
    \end{bmatrix}\right)=\hbar\left(\frac{1}{4}-\frac{1}{4}\right)=0\\
    \braket{J_x}&=\text{Tr}(\rho^{(z)}J_x)=\frac{\hbar}{\sqrt{2}}\text{Tr}\left(\begin{bmatrix}
      \frac{1}{4}&0&\frac{1}{4}\\0&\frac{1}{2}&0\\\frac{1}{4}&0&\frac{1}{4}.
    \end{bmatrix}\cdot\begin{bmatrix}
      0&1&0\\1&0&1\\0&1&0
    \end{bmatrix}\right)=0.
  \end{align*}
  \item They should match the answer from part e as the transformation matrix shouldnt change the meaning of them.
  \begin{align*}
    P_z(\hbar)&=\text{Tr}(\rho^{(z)}P_+)=\text{Tr}\left(\begin{bmatrix}
      \frac{1}{4}&0&\frac{1}{4}\\0&\frac{1}{2}&0\\\frac{1}{4}&0&\frac{1}{4}.
    \end{bmatrix}\cdot\begin{bmatrix}
      1&0&0\\0&0&0\\0&0&0
    \end{bmatrix}\right)=\text{Tr}\left(\begin{bmatrix}
      1/4&0&0\\0&0&0\\1/4&0&0
    \end{bmatrix}\right)=\frac{1}{4}\\
    P_z(0)&=\text{Tr}(\rho^{(z)}P_0)=\text{Tr}\left(\begin{bmatrix}
      \frac{1}{4}&0&\frac{1}{4}\\0&\frac{1}{2}&0\\\frac{1}{4}&0&\frac{1}{4}.
    \end{bmatrix}\cdot\begin{bmatrix}
      0&0&0\\0&1&0\\0&0&0
    \end{bmatrix}\right)=\text{Tr}\left(\begin{bmatrix}
      0&0&0\\0&1/2&0\\0&0&0
    \end{bmatrix}\right)=\frac{1}{2}\\
    P_z(-\hbar)&=\text{Tr}(\rho^{(z)}P_-)=\text{Tr}\left(\begin{bmatrix}
      \frac{1}{4}&0&\frac{1}{4}\\0&\frac{1}{2}&0\\\frac{1}{4}&0&\frac{1}{4}.
    \end{bmatrix}\cdot\begin{bmatrix}
      0&0&0\\0&0&0\\0&0&1
    \end{bmatrix}\right)=\text{Tr}\left(\begin{bmatrix}
      0&0&1/4\\0&0&0\\0&0&1/4
    \end{bmatrix}\right)=\frac{1}{4}
  \end{align*}
  We see that these probabilities coincide with part e). The $P_m$ are the projectors into the respective $m$ eigenstate.
\end{enumerate}

%%
\section*{Problem III}
\begin{enumerate}[itemsep=0pt,topsep=0pt,label=\alph*)]
  \item By looking the table, the function asked to integrate is:
  \begin{align*}
    F(\theta,\phi)=(Y_0^0)^*Y_1^0Y_1^1=-\frac{3}{4\pi\sqrt{8\pi}}\cos\theta\sin\theta e^{i\phi}.
  \end{align*}
  Its integration is:
  \begin{align*}
    \int_0^{2\pi}\int_0^\pi F(\theta,\phi)\;\sin\theta d\theta d\phi&=-\frac{3}{4\pi\sqrt{8\pi}}\int_0^{2\pi}\int_0^\pi\cos\theta\sin\theta e^{i\phi}\;\sin\theta d\theta d\phi\\
    &=\int_0^{\pi}\cos\theta\sin^2\theta\left[\underbrace{\int_0^{2\pi} e^{i\phi}\;d\phi}_{0}\right]d\theta\\
    \int_0^{2\pi}\int_0^\pi F(\theta,\phi)\;\sin\theta d\theta d\phi&=0.
  \end{align*}
  \item Integration of the function yields
  \begin{align*}
    \int_0^{2\pi}\int_0^\pi F(\theta,\phi)\;\sin\theta d\theta d\phi&=\frac{3}{8\pi\sqrt{4\pi}}\int_0^{2\pi}\int_0^\pi \sin^2\theta e^{i2\phi}\;\sin\theta d\theta d\phi\\
    &=\frac{3}{8\pi\sqrt{4\pi}}\int_0^\pi\sin^3\theta\left[\underbrace{\int_0^{2\pi}e^{i2\phi}\;d\phi}_{0}\right]\;d\theta\\
    \int_0^{2\pi}\int_0^\pi F(\theta,\phi)\;\sin\theta d\theta d\phi&=0.
  \end{align*}
  \item Here, we have 
  \begin{align*}
    \int_0^{2\pi}\int_0^\pi F(\theta,\phi)\;\sin\theta d\theta d\phi&=-\frac{3}{8\pi\sqrt{4\pi}}\int_0^{2\pi}\int_0^\pi\sin^2\theta\;\sin\theta\;d\theta d\phi\\
    &=-\frac{3}{4\sqrt{4\pi}}\int_0^{\pi}\sin^3\theta\;d\theta\\
    &=-\frac{3}{4\sqrt{4\pi}}\frac{4}{3}\\
    \int_0^{2\pi}\int_0^\pi F(\theta,\phi)\;\sin\theta d\theta d\phi&=-\frac{1}{\sqrt{4\pi}}.
  \end{align*}
  \item In this case, 
  \begin{align*}
    \int_0^{2\pi}\int_0^\pi F(\theta,\phi)\;\sin\theta d\theta d\phi&=\frac{3}{4\pi\sqrt{4\pi}}\int_0^{2\pi}\int_0^\pi \cos^2\theta\;\sin\theta d\theta d\phi\\
    &=\frac{3}{2\sqrt{4\pi}}\int_0^\pi\cos^2\theta\sin\theta\;d\theta\\
    &=\frac{3}{2\sqrt{4\pi}}\frac{2}{3}\\
    &=\frac{1}{\sqrt{4\pi}}.
  \end{align*}
  \item The integration is as follows
  \begin{align*}
    \int_0^{2\pi}\int_0^\pi F(\theta,\phi)\;\sin\theta d\theta d\phi&=\frac{3\sqrt{15}}{16\pi^{3/2}}\int_0^{2\pi}\int_0^\pi\sin^2\theta\cos^2\theta\;\sin\theta d\theta d\phi\\
    &=\frac{3\sqrt{15}}{8\sqrt{\pi}}\int_0^\pi\sin^3\theta\cos^2\theta\;d\theta\\
    &=\frac{3\sqrt{15}}{8\sqrt{\pi}}\frac{4}{15}\quad(\text{integration by parts, }u=\cos\theta,\;du=-\sin\theta d\theta)\\
    \int_0^{2\pi}\int_0^\pi F(\theta,\phi)\;\sin\theta d\theta d\phi&=\frac{\sqrt{15}}{10\sqrt{\pi}}.
  \end{align*}
  \item As each spherical harmonic is a separable function, we can treat them as the product of a $\theta$-dependent function and a $\phi$-dependent function:
  \begin{align*}
    F(\theta,\phi)=g(\theta)h(\phi),\quad\text{where}\quad h(\phi)=e^{-im_1\phi}e^{im_2\phi}e^{m_3\phi}.
  \end{align*}
  Because of the hint, we perform the integration only along $\phi$:
  \begin{align*}
    \int_0^{2\pi}h(\phi)\;d\phi=\int_0^{2\pi}e^{i(-m_1+m_2+m_3)\phi}\;d\phi=2\pi\delta_{-m_1+m_2+m_3,0}.
  \end{align*}
  The Kronecker function is non-zero only when the indicex match, that is,
  \begin{align*}
    m_1=m_2+m_3.
  \end{align*}
  \item If $l_2=1$ means that $m_2\in\{-1,0,+1\}$. In addition, using the $\Delta m_{13}$ we have 
  \begin{align*}
    m_1&=m_2+m_3\\
    m_1-m_3&=m_2\\
    \Delta m_{13}&=m_2.
  \end{align*}
  Therefore, 
  \begin{align*}
    \Delta m_{13}\in\{-1,0,+1\}.
  \end{align*}
\end{enumerate}
In this problem, all the integrals were raised and then if needed was used integral calculator to facilitate the result.
%%
\section*{Problem IV}
\begin{enumerate}[itemsep=0pt,topsep=0pt,label=\alph*)]
  \item We have a spin-1/2 system, where the spin operators are defined as $S_i=(\hbar/2)\sigma_i$. Measuring with $S_y$ and getting the eigenvalue $\hbar/2$ means that 
  \begin{align*}
    S_y\ket{y}=\frac{\hbar}{2}\ket{y}.
  \end{align*}
  The two eigenvectors from it allows to span the $\ket{\pm y}$ as follows:
  \begin{align*}
    \ket{\pm y}=\frac{1}{\sqrt{2}}(\ket{+}\pm i\ket{-}).
  \end{align*}
  After the measurement, we will have:
  \begin{align*}
    \ket{\psi(0)}=\ket{+y}=\frac{1}{\sqrt{2}}(\ket{+}+i\ket{-}).
  \end{align*}
  \item The Hamiltonian in this case is 
  \begin{align*}
    H(t)=\omega_0(t)S_z=\frac{\hbar}{2}\omega_0(t)\sigma_z,\quad0\leq t\leq T.
  \end{align*}
  Applying the evolution operator to $\ket{\psi(0)}$ allows to have the state at time $t$. Recall that the eigenvalues of $\sigma_z$ are $\pm1$.
  In addition, as the B-field behaves as a ramp function we have 
  \begin{align*}
    \omega_0(t)=\frac{\omega_0}{T}t,\quad0\leq t\leq T.  
  \end{align*}
  Another property is that the Hamiltonian is always proportional to $S_z$ at any arbitrary time, which is the same as 
  \begin{align*}
    [H(t),H(t')]=0,\quad\forall t,t'.
  \end{align*}
  This makes us express the time evolution as:
  \begin{align*}
    U(t,0)=e^{-\frac{i}{\hbar}\int_0^tH(t')\;dt'}=e^{-i\frac{i}{\hbar}\int_0^t\omega_0(t')S_z\;dt'}=e^{-\frac{i}{\hbar}S_z\frac{\omega_0}{2T}t^2}.
  \end{align*}
  Application of $U(t,0)$ to the initial state yields
  \begin{align*}
    \ket{\psi(t)}=\frac{1}{\sqrt{2}}\left(e^{-\frac{i}{2}\frac{\omega_0}{2T}t^2}\ket{+}+ie^{+\frac{i}{2}\frac{\omega_0}{2T}t^2}\ket{-}\right)=\frac{1}{\sqrt{2}}(e^{i\theta(t)}\ket{+}+ie^{-i\theta(t)}\ket{-}),\quad \theta(t)=-\frac{\omega_0}{4T}t^2.
  \end{align*}
  This is a rotation around $z$ by the angle $\theta(t)$ on the Bloch sphere.
  \item In the previous part we expressed the expansion of $\ket{\pm y}$, so that the probabilities are computed as the projection of $\ket{\psi(t)}$ onto them:
  \begin{align*}
    P_y(\hbar/2)&=|\braket{+y|\psi}|^2=\left|\frac{e^{i\theta}+e^{-i\theta}}{2}\right|^2=\cos^\theta\\
    P_y(-\hbar/2)&=|\braket{-y|\psi}|^2=\left|\frac{e^{i\theta}-e^{-i\theta}}{2}\right|^2=\sin^2\theta.
  \end{align*}
  At $t>T$, we have $\theta(t)=\omega_0T/4$ and:
  \begin{align*}
    P_y(\hbar/2)=\cos^2\frac{\omega_0T}{4},\quad P_y(-\hbar/2)=\sin^2\frac{\omega_0T}{4}.
  \end{align*} 
  To be certain of getting $\hbar/2$, the cosine term must be unity, meaning that we must have:
  \begin{align*}
    \cos^2\frac{\omega_0T}{4}&=1\\
    \cos\frac{\omega_0T}{4}&=1/\cos^{-1}(\cdot)\\
    \frac{\omega_0T}{4}&=\pi n,\quad n\in\mathbb{Z}\\
    \omega_0T&=4\pi n,\quad n\in\mathbb{Z}.
  \end{align*}
  Similarly, to be sure of getting $-\hbar/2$ we must have the sine term to be one, and therefore:
  \begin{align*}
    \sin^2\frac{\omega_0T}{4}&=1\\
    \sin\frac{\omega_0T}{4}&=1/\sin^{-1}(\cdot)\\
    \frac{\omega_0T}{4}&=\frac{\pi(2n+1)}{2},\quad n\in\mathbb{Z}\\
    \omega_0T&=2\pi(2n+1),\quad n\in\mathbb{Z}.
  \end{align*}
\end{enumerate}



%\nocite{*}
%\bibliographystyle{plain}   % or unsrt, alpha, apalike, etc.
%\bibliography{refs}

\end{document}
