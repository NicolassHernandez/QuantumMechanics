\documentclass[letterpaper,11pt,twoside]{article}
\usepackage{graphicx} % Required for inserting images
\usepackage[table,xcdraw,dvipsnames]{xcolor}
\usepackage{amsmath,amsfonts,amssymb,amsthm}
\usepackage{listings}
\usepackage{lipsum}
\usepackage{hyperref}
\usepackage{enumitem}

\usepackage{tikz}
\usepackage[siunitx, RPvoltages]{circuitikz}
\usetikzlibrary{3d}
\usepackage{comment}
\usepackage{caption,subcaption}
\usepackage{pgfplots}
\pgfplotsset{compat=newest} % or a newer version if available
\usepgfplotslibrary{groupplots}
\usetikzlibrary{pgfplots.groupplots}
\usetikzlibrary{shapes.geometric, arrows}
\tikzstyle{arrow} = [->,>=stealth,shorten >=2pt]

\usepackage{cancel}
\usepackage{bm}
\usepackage{fancyhdr}
\usepackage[utf8x]{inputenc}
\usepackage[T1]{fontenc}
\usepackage[margin=0.8in,top=1in,bottom=1in]{geometry}
%%%%%
\begin{filecontents*}{refs.bib}
@book{bornwolf,
  author    = {Born, M. and Wolf, E.},
  title     = {Principles of Optics},
  publisher = {Pergamon Press},
  edition   = {7},
  year      = {1999}
}
@book{hecht,
  author    = {Hecht, E.},
  title     = {Optics},
  publisher = {Addison-Wesley},
  edition   = {5},
  year      = {2016}
}
\end{filecontents*}
%
\newcommand{\institution}{University of Arizona}
\newcommand{\autor}{Nicolás Hernández Alegría}
\newcommand{\course}{OPTI 570 Quantum Mechanics}
\newcommand{\assignment}{Assignment 2}
%
\title{\textbf{\assignment}\\\course\\{\Large\institution}}
\author{\autor}
\date{\today\\Total time: 8 hours}
%
\renewcommand{\sectionmark}[1]{\markright{#1}}
\fancypagestyle{mainstyle}{
    \fancyhf{} % Clear all header and footer fields
    \fancyfoot[C]{\thepage}
    \fancyhead[LE,RO]{\course} % Section name on odd pages
    \fancyhead[LO,RE]{\assignment}
    % Optional: Thin rules
    \renewcommand{\headrulewidth}{0pt} % Header rule
    \renewcommand{\footrulewidth}{0pt} % No footer rule
}
%
\begin{document}

\pagestyle{mainstyle}
\maketitle
%%
\section{Exercise 1}%
A particle under a delta function potential has the following hamiltonian:
\begin{align*}
  H=-\frac{\hbar}{2m}\frac{d^2}{dx^2}-\alpha\delta(x),
\end{align*}
where $\delta(x)$ is the potential to be analyzed which is centered at $x=0$.
\begin{enumerate}[itemsep=0pt,topsep=0pt,label=(\alph*)]
  \item
  \begin{enumerate}[itemsep=0pt,topsep=0pt,label=(a.\arabic*)]
    \item The integration of the eigenvalue equation 
    \begin{align*}
      H\varphi(x)&=E\varphi(x)\\
      -\frac{\hbar^2}{2m}\frac{d^2}{dx^2}\varphi(x)-\alpha\delta(x)\varphi(x)&=E\varphi(x)
    \end{align*}
    in the range $x\in[-\epsilon,\epsilon]$ is 
    \begin{align*}
      \int_{-\epsilon}^\epsilon\left[-\frac{\hbar^2}{2m}\frac{d^2\varphi}{dx^2}(x)-\alpha\delta(x)\varphi(x)\right]\;dx=\int_{-\epsilon}^\epsilon E\varphi(x)\;dx,
    \end{align*}
    where in this infinitesimal region around $x=0$ different things will happen:
    \begin{itemize}
      \item By property of the delta function, we will have
      \begin{align*}
        \int_{-\infty}^\infty \alpha\delta(x)\varphi(x)\;dx=\alpha\varphi(0).
      \end{align*}
      \item The integration of the second derivative will become a substraction of two first derivatives:
      \begin{align*}
        \int_{-\epsilon}^\epsilon\left[-\frac{\hbar^2}{2m}\frac{d^2\varphi}{dx^2}(x)\right]\;dx=-\frac{\hbar^2}{2m}\left[\frac{d\varphi}{dx}(\epsilon)
        -\frac{d\varphi}{dx}(-\epsilon)\right].
      \end{align*}
      \item The right-hand term will be zero, as the function cannot change abruptly (to be physically realizable) in an infinitesimal range:
      \begin{align*}
        \int_{-\epsilon}^\epsilon E\varphi(x)\;dx=0.
      \end{align*}
    \end{itemize}
    Putting all in the eigenvalue equation yields
    \begin{align}
      -\frac{\hbar^2}{2m}\left[\frac{d\varphi}{dx}(\epsilon)
        -\frac{d\varphi}{dx}(-\epsilon)\right]-\alpha\varphi(0)=0,\quad x\in[-\epsilon,\epsilon].
        \label{eq:inteigenvalueequation}
    \end{align}
    \item Recall that continuity of a function $f(x)$ at $x=x_0$ needs that both lateral limits are equal as they approach to the point, 
    and that the value is equal to the function evaluated at that point: $f(x_0^-)=f(x_0^+)=f(x_0$).

    Taking the limit $\epsilon\to0$ in a rearranged version of the equation \eqref{eq:inteigenvalueequation} allow us to construct the difference of the 
    derivative used to prove continuity in the first derivate of $\varphi(x)$:
    \begin{align*}
      \lim_{\epsilon\to0}\left[\frac{d\varphi}{dx}(\epsilon)
        -\frac{d\varphi}{dx}(-\epsilon)\right]=-\frac{2m\alpha}{\hbar^2}\varphi(0).
    \end{align*}
    We can see that lateral derivatives are not equal, but rather there is a finite jump of $-\frac{2m\alpha}{\hbar^2}\varphi(0)$ at $x=0$. Therefore, we conclude that 
    $\varphi'(x)$ is not continue at that point.
  \end{enumerate}
  %%
  \item
    \begin{enumerate}[itemsep=0pt,topsep=0pt,label=(b.\arabic*)]
      \item Assuming that $E=\hbar\omega<0$, in order to obtain an expression for $\rho$ we replace the wavefunctions 
      \begin{align*}
        x<0:&\quad\varphi(x)=A_1e^{\rho x}+A_1'e^{-\rho x}\\
        x>0:&\quad\varphi(x)=A_2e^{\rho x}+A_2'e^{-\rho x}.
      \end{align*}
      into the eigenvalue problem. These solutions must be stable, therefore we evaluate them at $-\infty$ and $+\infty$ to see if any term blows up:
      \begin{align*}
        \varphi(-\infty)&=A_1e^{\rho(-\infty)}+\cancelto{\text{blows up}}{A_1'e^{-\rho(-\infty)}},\\
        \varphi(+\infty)&=\cancelto{\text{blows up}}{A_2e^{\rho(\infty)}}+A_2'e^{-\rho(\infty)}.
      \end{align*}
      Therefore, to keep with stable and normalizable functions we redefine the above:
      \begin{align*}
        \varphi(x)=\begin{cases}
          A_1e^{\rho x},&x<0\\
          A_2'e^{-\rho x},&x>0
        \end{cases}.
      \end{align*}
      Now, we can proceed. In order to obtain an expression for $\rho$, we will study each domain separately. 

      For $x<0$, the function has not passed through the delta potential yet so the hamiltonian reduces to the derivative term only.
      \begin{align*}
        H\varphi(x)&=E\varphi(x)\\
        -\frac{\hbar^2}{2m}\frac{d^2}{dx^2}\left[A_1e^{\rho x}\right]&=E\left[A_1e^{\rho x}\right]\\
        -\frac{\hbar^2}{2m}\frac{d}{dx}\left[\rho A_1e^{\rho x}\right]&=EA_1e^{\rho x}\\
        -\frac{\hbar^2}{2m}\rho^2\cancel{A_1e^{\rho x}}&=E\cancel{A_1e^{\rho x}}\\
        \rho&=\sqrt{-\frac{2mE}{\hbar^2}}
      \end{align*}

      For $x>0$, despite that the delta function is not active, it has imposed initial conditions on the function in the first derivative. 

      First, we know that the function must be continuous throught the whole domain. Matematically, that means that 
      \begin{align*}
        \varphi(0^-)=\varphi(0^+)=\varphi(0)\Longrightarrow A_1=A_2'=A.
      \end{align*}
      Then, again 
      \begin{align*}
        \varphi(x)=Ae^{-\rho|x|}=\begin{cases}
          Ae^{\rho x},&x<0\\
          Ae^{-\rho x},&x>0
        \end{cases}.
      \end{align*}

      Now, if we look at the equation of discontinuity we have derived in (a.2) and replace the function above we can solve for $\rho$.
      \begin{align*}
        \varphi'(0^+)-\varphi(0^+)&=-\frac{2m\alpha}{\hbar^2}\varphi(0)\\
        -\rho Ae^{-\rho(0)}-\rho Ae^{\rho(0)}&=-\frac{2m\alpha}{\hbar^2}\left[Ae^{-\rho(0)}\right]\\
        -2\rho&=-\frac{2m\alpha}{\hbar^2}\\
        \rho&=\frac{m\alpha}{\hbar^2}.
      \end{align*}
      Because the variable $\rho$ must be equal in either region, we conclude that 
      \begin{align*}
        \rho=\sqrt{-\frac{2mE}{\hbar^2}}=\frac{m\alpha}{\hbar^2}.
      \end{align*}

      The dimension of $\alpha$ can now be found easily. We know that $\rho$ must have dimensions of $[L]^{-1}$ in order to have a dimensionless quantity in the argument of the 
      exponential of the wave $\varphi(x)$. Knowing also that the mass have dimension of $[M]$ and $\hbar^2$ of $[M^2L^4T^{-2}]$, doing dimensional analysis with the last equation yields
      \begin{align*}
        [L^{-1}]=[M][M^2L^4T^{-2}]^{-1}[\alpha]\longrightarrow[\alpha]=[L]^{-1}[M]^{-1}[M^2L^4T^{-2}]=[ML^3T^{-2}].
      \end{align*}
      \item Above we have already limited the general equation to only terms that dont blow up when $x$ goes to $\pm\infty$, meaning that they are 
      square-interable over the space. We have also implicitly set continuity in $\varphi(x)$.

      The energy possible upon these constraints is obtained solving for $E$ in the previous equation derived.
      \begin{align*}
        \sqrt{-\frac{2mE}{\hbar^2}}=\frac{m\alpha}{\hbar^2}\longrightarrow E=-\frac{m\alpha^2}{2\hbar^2}.
      \end{align*}
      The wavefunction must be normalized in order to satisfy the normalization condition and be interpreted as a probability density function:
      \begin{align*}
        \int_{-\infty}^{\infty}|\varphi(x)|^2\;dx=1.
      \end{align*}
      For that, we will separate the domain in two regions, one for $x<0$ and other for $x>0$ and in each subdomain we will use the respective wavefunction.
      \begin{align*}
        \int_{-\infty}^0|Ae^{\rho x}|^2\;dx+\int_0^\infty|Ae^{-\rho x}|^2&=1\\
        \int_{-\infty}^0A^2e^{2\rho x}\;dx+\int_0^\infty A^2e^{-2\rho x}\;dx&=\\
        \frac{A^2}{2\rho}\left[e^{2\rho x}\right]\biggr|_{-\infty}^0-\frac{A^2}{2\rho}\left[e^{-2\rho x}\right]\biggr|_0^\infty&=\\
        \frac{A^2}{2\rho}\left[1-0\right]-\frac{A^2}{2\rho}\left[0-1\right]&=\\
        \frac{A^2}{\rho}&=1\\
        A&=\sqrt{\rho}\quad[L]^{-1/2}.
      \end{align*}
      Therefore, the normalized wavefunctions are:
      \begin{align*}
        \varphi(x)=\sqrt{\rho}e^{-\rho|x|}=\begin{cases}
          \sqrt{\rho}e^{\rho x},&x<0\\
          \sqrt{\rho}e^{-\rho x},&x>0
        \end{cases}.
      \end{align*}
    \end{enumerate}
  \item The plot of the wavefunction is illustrated below. We can see that is symmetric with the respect to the $y$ axis.
  \begin{figure}[htbp]
    \centering
    \begin{circuitikz}
      \def\rro{1}
      \def\hwhm{0.693/\rro}
      \draw[arrow](-3.5,0)--(3.5,0)node[right]{$x$};
      \draw[arrow](0,0)--(0,2)node[right]{$\varphi(x)$};
      \draw[very thick,NavyBlue,domain=-3:0,samples=100] plot(\x,{ sqrt(\rro)*exp( \rro*\x) });
      \draw[very thick,NavyBlue,domain=0:3,samples=100] plot(\x,{ sqrt(\rro)*exp( -\rro*\x) });
      \foreach \x in {-3,...,3}{\draw(\x,.1)--(\x,-.1)node[below]{\small$\x$};}
      \draw(.1,{sqrt(\rro)})--(-.1,{sqrt(\rro)})node[left]{$\varphi_{max}$};
      \draw[|-|](-\hwhm,-.7)--(\hwhm,-.7)node[midway,below]{$\Delta x$};
      \draw(2,1)node[]{$\rho=\rro$};
    \end{circuitikz}
    \caption{Normalized wavefunction $\varphi(x)$ with its respective $\Delta x$.}
  \end{figure}

  We are going to use the Full Width at Half Maximum to define $\Delta x$. It is defined as the diameter (full width) at which 
  half of the peak is reached. We will use the $x>0$ function and the value of $x$ obtained will be multiplied by two because of symmetry.
  \begin{align*}
    \varphi(x)&=\frac{\varphi_{max}}{2}\\
    \sqrt{\rho}e^{-\rho x}&=\frac{\sqrt{\rho}}{2}\biggr/\ln(\cdot)\\
    -\rho x&=-\ln2\\
    x&=\frac{\ln2}{\rho}\longrightarrow \Delta x=\frac{2\ln2}{\rho}.
  \end{align*}

  The order of magnitude of $\Delta x$ would be something proportional to the inverse of $\rho$ as the other terms are constant:
  \begin{align*}
    \Delta x\sim\frac{1}{\rho}.
  \end{align*}

  \item The Fourier transform applied to our wavefunction is:
  \begin{align*}
    \tilde{\varphi}(p)&=\frac{1}{\sqrt{2\pi\hbar}}\int_{-\infty}^\infty \varphi(x)\exp\left\{-i\frac{p}{\hbar}x\right\}\;dx\\
    &=\frac{\sqrt{\rho}}{\sqrt{2\pi\hbar}}\left[\int_{-\infty}^0\exp\left\{\rho x-i\frac{p}{\hbar}x\right\}\;dx+\int_0^\infty \exp\left\{-\rho x-i\frac{p}{\hbar}x\right\}\;dx\right]\\
    &=\sqrt{\frac{\rho}{2\pi\hbar}}\left[\int_{-\infty}^0\exp\left\{\left(\frac{\rho\hbar-ip}{\hbar}\right)x\right\}\;dx+\int_0^\infty \exp\left\{\left(-\frac{\rho\hbar+ip}{\hbar}\right)x\right\}\;dx\right]\\
    &=\sqrt{\frac{\rho}{2\pi\hbar}}\left[\frac{\hbar}{\rho\hbar-ip}\exp\{u\}\biggr|_{-\infty}^0-\frac{\hbar}{\rho\hbar+ip}\exp\left\{u\right\}\biggr|_0^{-\infty}\right]\\
    &=\sqrt{\frac{\rho}{2\pi\hbar}}\left[\frac{\hbar}{\rho\hbar-ip}+\frac{\hbar}{\rho\hbar+ip}\right]\\
    \tilde{\varphi}(p)&=\sqrt{\frac{\rho}{2\pi\hbar}}\frac{2\hbar^2\rho}{\rho^2\hbar^2+p^2}.
  \end{align*}

  We then define the FWHM as we did for $\varphi(x)$.
  \begin{align*}
    \tilde{\varphi}(p)&=\frac{\tilde{\varphi}_{max}}{2}\\
    \sqrt{\frac{\rho}{2\pi\hbar}}\frac{2\hbar^2\rho}{\rho^2\hbar^2+p^2}&=\sqrt{\frac{\rho}{2\pi\hbar}}\frac{1}{\rho}\\
    \frac{2\hbar^2\rho}{\rho^2\hbar^2+p^2}&=\frac{1}{\rho}\\
    2\hbar^2\rho&=\rho\hbar^2+\frac{p^2}{\rho}\\
    p&=\hbar\rho\longrightarrow\Delta p=2\hbar\rho.
  \end{align*}

  The figure of $\tilde{\varphi}(p)$ with $\Delta p$ is illustrated. Note that the constant $\hbar$ has been set to one only for visualization.
  \begin{figure}[htbp]
    \centering
    \begin{circuitikz}
      \def\rro{1}
      \def\hh{1}
      \def\hwhm{\hh*\rro}
      \draw[arrow](-3.5,0)--(3.5,0)node[right]{$p$};
      \draw[arrow](0,0)--(0,{(sqrt(\rro/(2*pi*\hh)))*(2/\rro)+1})node[right]{$\tilde{\varphi}(p)$};
      \draw[very thick,NavyBlue,domain=-3:3,samples=100] plot(\x,{ (sqrt(\rro/(2*pi*\hh)))*(2*\hh^2*\rro)/(\rro^2*\hh^2+\x*\x) });
      \foreach \x in {-3,...,3}{\draw(\x,.1)--(\x,-.1)node[below]{\small$\x$};}
      \draw(.1,{(sqrt(\rro/(2*pi*\hh)))*(2/\rro)})--(-.1,{(sqrt(\rro/(2*pi*\hh)))*(2/\rro)})node[above left]{$\tilde{\varphi}_{max}$};
      \draw[|-|](-\hwhm,-.7)--(\hwhm,-.7)node[midway,below,align=center]{$\Delta p$};
      \draw(2,1)node[align=center]{$\rho=\rro$\\$\hbar=\hh$};
    \end{circuitikz}
    \caption{Fourier transform $\tilde{\varphi}(p)$ with a normalized $\hbar=1$ \textbf{only} for visualization.}
  \end{figure}

  The product of the widths is then:
  \begin{align*}
  \Delta x\Delta p=\frac{2\ln2}{\rho}2\hbar\rho=4\ln2\;\hbar.
  \end{align*}

  The order of magnitude of $\Delta p$ and $\Delta x\Delta p$ would be:
  \begin{align*}
    \Delta p\sim \hbar\rho,\quad\text{and}\quad\Delta x\Delta p\sim\hbar.
  \end{align*}
\end{enumerate}



\section{Exercise 2}
\begin{enumerate}[itemsep=0pt,topsep=0pt,label=\alph*.]
  \item Compute the adjoint of the operator $U(m,n)$ is the same that invert the order of them (transpose) and conjugate them (complex conjugation).
  That means that a ket will become a bra and viceversa: 
  \begin{align*}
    U^\dagger(m,n)=(|\varphi_m\rangle\langle\varphi_n|)^\dagger=|\varphi_n\rangle\langle \varphi_m|.
  \end{align*}
  \item This is done by algebra of the operation and the definition above. Lets also
  recall that $H$ is hermitian so that its eigenvalues $E_n$ are real:
  \begin{align*}
    [H,U(m,n)]&=HU(m,n)-U(m,n)H=H|\varphi_m\rangle\langle\varphi_n|-|\varphi_m\rangle\langle\varphi_n|H.
  \end{align*}
  The first and last terms corresponds to the eigenvalue problem, where $E_n$ is the respective eigenvalue of the $H$ operator.
  \begin{align*}
    [H,U(m,n)]&=E_m\langle\varphi_n|-|\varphi_m\rangle E_n^*\stackrel{(a)}{=}E_n\langle\varphi_n|-E_m|\varphi_m\rangle,
  \end{align*}
  where $(a)$ is due that an hermitian operator has real eigenvalue and therefore $E_n=E_n^*$.
  \item Replacing each projector:
  \begin{align*}
    U(m,n)U^\dagger (p,q)=|\varphi_m\rangle\langle\varphi_n|\varphi_q\rangle\langle\varphi_p|=\begin{cases}
      0,\quad n\neq q\\
      |\varphi_m\rangle\langle\varphi_p|,\quad n=q
    \end{cases}
  \end{align*}
  When multiplying both operator, there will be created an inner product in the middle: $\langle\varphi_n|\varphi_q\rangle$, which 
  will return a scalar proportion to the "collinearity" of both state vectors. If we assume the basis is composed of linear independent elements $|\varphi_i\rangle$,
  then all the inner product for which $n\neq q$ will be zero, and only when both index are equal will return $1$, if the basis is properly normalized.
  We have put the cases of these two scenarios, but an even compact form would be through the use of the kronecker function $\delta_{ij}$, defined as:
  \begin{align*}
    \delta_{ij}=\begin{cases}
      0,&i\neq j\\
      1,&i=j
    \end{cases}.
  \end{align*}
  In consequence,
  \begin{align*}
    U(m,n)U^\dagger (p,q)=|\varphi_m\rangle\delta_{nq}\langle\varphi_p|=\delta_{nq}|\varphi_m\rangle\langle\varphi_p|=\delta_{nq}U(m,p).
  \end{align*}
  \item The trace over an operator $A$ with the discrete orthonormal basis $\{|\varphi_n\rangle\}\in\mathcal{E}$ is the sum of 
  the diagonal element of $A$:
  \begin{align*}
    \text{Tr}(A)=\sum_i\langle \varphi_i|A|\varphi_i\rangle.
  \end{align*}
  Applying it into $U(m,n)=|\varphi_m\rangle\langle\varphi_n|$ yields
  \begin{align*}
    \text{Tr}[U(m,n)]=\sum_i\langle \varphi_i|\varphi_m\rangle\langle\varphi_n|\varphi_i\rangle=\sum_i\langle\varphi_n|\varphi_i\rangle\langle \varphi_i|\varphi_m\rangle=\langle\varphi_n|\left(\sum_i|\varphi_i\rangle\langle\varphi_i|\right)|\varphi_m\rangle.
  \end{align*}
  The term in parenthesis is a property of closure relation for projectors, which is 
  \begin{align*}
    \sum_i|\varphi_i\rangle\langle\varphi_i|=\mathbb{I}.
  \end{align*}
  Therefore, we have 
  \begin{align*}
    \text{Tr}[U(m,n)]=\langle\varphi_n|\mathbb{I}|\varphi_m\rangle=\langle\varphi_n|\varphi_m\rangle=\delta_{nm}.
  \end{align*}
  \item If $A$ is an operator with matrix elements $A_{mn}=\langle\varphi_m|A|\varphi_n\rangle$, then
  \begin{align*}
    A&=\sum_{m,n}A_{mn}U(m,n)\\
    &=\sum_{m,n}\langle\varphi_m|A|\varphi_n\rangle |\varphi_m\rangle\langle\varphi_n|\\
    &=\sum_{m,n}|\varphi_m\rangle\langle\varphi_m|A|\varphi_n\rangle\langle\varphi_n|\\
    &=\left(\sum_m|\varphi_m\rangle\langle\varphi_m|\right)A\left(\sum_n|\varphi_n\rangle\langle\varphi_n|\right)\\
    &=\mathbb{I}A\mathbb{I}\\
    A&=A.
  \end{align*}
  Where we have used the closure relation twice
  \item The matrix element $A_{pq}=\langle\varphi_p|A|\varphi_q\rangle$ is used to compute the trace under the influence of $U^\dagger$:
  \begin{align*}
    A_{pq}&=\text{Tr}\{AU^\dagger(p,q)\}\\
    \langle\varphi_p|A|\varphi_q\rangle&=\sum_i \langle\varphi_i|A|\varphi_q\rangle\langle \varphi_p|\varphi_i\rangle\\
    &=\sum_i\langle\varphi_p|\varphi_i\rangle\langle\varphi_i|A|\varphi_q\rangle\\
    &=\langle\varphi_p|\left(\sum_i|\varphi_i\rangle\langle\varphi_i|\right)A|\varphi_q\rangle\\
    &=\langle\varphi_p|\mathbb{I}A|\varphi_q\rangle\\
    \langle\varphi_p|A|\varphi_q\rangle&=\langle\varphi_p|A|\varphi_q\rangle.
  \end{align*}
  Using again, the closure relation of projector operator.
\end{enumerate}




%\nocite{*}
%\bibliographystyle{plain}   % or unsrt, alpha, apalike, etc.
%\bibliography{refs}

\end{document}
