\documentclass[letterpaper,11pt,twoside]{article}
\usepackage{graphicx} % Required for inserting images
\usepackage[table,xcdraw,dvipsnames]{xcolor}
\usepackage{amsmath,amsfonts,amssymb,amsthm}
\usepackage{listings}
\usepackage{lipsum}
\usepackage{hyperref}
\usepackage{mathrsfs}

\usepackage{enumitem}

\usepackage{tikz}
\usepackage[siunitx, RPvoltages]{circuitikz}
\usetikzlibrary{3d}
\usepackage{comment}
\usepackage{caption,subcaption}
\usepackage{pgfplots}
\pgfplotsset{compat=newest} % or a newer version if available
\usepgfplotslibrary{groupplots}
\usetikzlibrary{pgfplots.groupplots}
\usetikzlibrary{shapes.geometric, arrows}
\tikzstyle{arrow} = [->,>=stealth,shorten >=2pt]
\newcommand{\ket}[1]{|#1\rangle}
\newcommand{\bra}[1]{\langle#1|}
\newcommand{\br}{\bm{r}}
\newcommand{\bR}{\bm{R}}
\newcommand{\bp}{\bm{p}}
\newcommand{\bP}{\bm{P}}
\newcommand{\braket}[1]{\langle#1\rangle}
\newcommand{\F}{\mathscr{F}}
\newcommand{\E}{\mathscr{E}}
\newcommand{\re}[1]{\text{Re}\left(#1\right)}
\newcommand{\im}[1]{\text{Im}\left(#1\right)}
\usepackage{dsfont}
\usepackage{cancel}
\usepackage{bm}
\usepackage{fancyhdr}
\usepackage[utf8x]{inputenc}
\usepackage[T1]{fontenc}
\usepackage[margin=0.8in,top=1in,bottom=1in]{geometry}
%%%%%
\begin{filecontents*}{refs.bib}
@book{bornwolf,
  author    = {Born, M. and Wolf, E.},
  title     = {Principles of Optics},
  publisher = {Pergamon Press},
  edition   = {7},
  year      = {1999}
}
@book{hecht,
  author    = {Hecht, E.},
  title     = {Optics},
  publisher = {Addison-Wesley},
  edition   = {5},
  year      = {2016}
}
\end{filecontents*}
%
\newcommand{\institution}{University of Arizona}
\newcommand{\autor}{Nicolás Hernández Alegría}
\newcommand{\course}{OPTI 570 Quantum Mechanics}
\newcommand{\assignment}{Assignment 7}
%
\title{\textbf{\assignment}\\\course\\{\Large\institution}}
\author{\autor}
\date{\today\\Total time: 15 hours}
%
\renewcommand{\sectionmark}[1]{\markright{#1}}
\fancypagestyle{mainstyle}{
    \fancyhf{} % Clear all header and footer fields
    \fancyfoot[C]{\thepage}
    \fancyhead[LE,RO]{\course} % Section name on odd pages
    \fancyhead[LO,RE]{\assignment}
    % Optional: Thin rules
    \renewcommand{\headrulewidth}{0pt} % Header rule
    \renewcommand{\footrulewidth}{0pt} % No footer rule
}
%
\begin{document}

\pagestyle{mainstyle}
\maketitle
%%
\section*{Problem I}
\begin{enumerate}[itemsep=0pt,topsep=0pt,label=\alph*)]
  \item The evolution operator would be of the form 
  \begin{align*}
    \hat{\mathbb{U}}_E(t)=e^{-i\hat{H}_1t/\hbar}=e^{-i\Omega(\hat{N}^2-1/2)t}.
  \end{align*}
  The checking is as follows:
  \begin{align*}
    \hat{\mathbb{U}}_E(\frac{2\pi}{\Omega})\ket{\varphi_n}&=e^{-i\Omega(n^2-1/2)\frac{2\pi}{\Omega}}\ket{\varphi_n}\\
    &=e^{-i2\pi(n^2-1/2)}\ket{\varphi_n}\\
    &=(e^{-2\pi})^{n^2}e^{i\pi}\ket{\varphi_n}\\
    \hat{\mathbb{U}}_E(\frac{2\pi}{\Omega})\ket{\varphi_n}&=-\ket{\varphi_n}.
  \end{align*}
  \item For $\tau=\pi/2\Omega$, the evolution is 
  \begin{align*}
    \hat{\mathbb{U}}_E(\tau)\ket{\varphi_n}&=e^{-i\Omega(n^2-1/2)\frac{\pi}{2\Omega}}\ket{\varphi_n}\\
    &=e^{-i\frac{\pi}{2}n^2}e^{i\frac{\pi}{4}}\ket{\varphi_n}\\
    &=(e^{-i\frac{\pi}{2}})^{n^2}e^{i\frac{\pi}{4}}\ket{\varphi_n}\\
    &=(-i)^{n^2}e^{i\frac{\pi}{4}}\ket{\varphi_n}\\
   \hat{\mathbb{U}}_E(\tau)\ket{\varphi_n}&=\begin{cases}
      e^{i\frac{\pi}{4}}=\frac{1}{\sqrt{2}}+\frac{i}{\sqrt{2}},&\text{$n$ even}\\
      e^{-i\frac{\pi}{4}}=\frac{1}{\sqrt{2}}-\frac{i}{\sqrt{2}},&\text{$n$ odd}
    \end{cases}\ket{\varphi_n}.
  \end{align*}
  \item We use the fact that in a coherent state, we can express it in terms of the energy eigenstates.
  \begin{align*}
    \ket{\alpha_0}=e^{-\frac{|\alpha_0|^2}{2}}\sum_{n=0}^\infty\frac{\alpha_0^n}{\sqrt{n!}}\ket{n}.
  \end{align*}
  We have found that 
  \begin{align*}
    \hat{\mathbb{U}}_E(\tau)=\begin{cases}
      e^{i\frac{\pi}{4}}=\frac{1}{\sqrt{2}}+\frac{i}{\sqrt{2}},&\text{$n$ even}\\
      e^{-i\frac{\pi}{4}}=\frac{1}{\sqrt{2}}-\frac{i}{\sqrt{2}},&\text{$n$ odd}
    \end{cases}.
  \end{align*}
  We then, must split the $\ket{\alpha_0}$ accordingly, in even and odd term so that the application of the evolution operator gives
  \begin{align*}
    \ket{\psi_E(\tau)}=e^{-\frac{|\alpha_0|^2}{2}}\left[e^{i\frac{\pi}{4}}S_{\text{even}}+e^{-i\frac{\pi}{4}}S_{\text{odd}}\right],
  \end{align*}
  where 
  \begin{align}
    S_{\text{even}}=\sum_{\text{$n$ even}}^\infty\frac{\alpha_0^n}{\sqrt{n!}}\ket{n},\quad\text{and}\quad S_{\text{odd}}=\sum_{\text{$n$ odd}}^\infty\frac{\alpha_0^n}{\sqrt{n!}}\ket{n}.
  \end{align}
  We then have that 
  \begin{align*}
    \left.\begin{array}{l}
      \ket{\alpha_0}=e^{-\frac{|\alpha_0|^2}{2}}(S_{\text{even}}+S_{\text{odd}})\\
      \ket{-\alpha_0}=e^{-\frac{|\alpha_0|^2}{2}}(S_{\text{even}}-S_{\text{odd}})      
    \end{array}\right\}\longrightarrow\begin{array}{l}
      S_{\text{even}}=\frac{1}{2}e^{\frac{|\alpha_0|^2}{2}}(\ket{\alpha_0}+\ket{-\alpha_0})\\
      S_{\text{odd}}=\frac{1}{2}e^{\frac{|\alpha_0|^2}{2}}(\ket{\alpha_0}-\ket{-\alpha_0})
    \end{array}.
  \end{align*}
  Substituting those in the evolution equation and rearranging:
  \begin{align*}
    \ket{\psi_E(\tau)}=\frac{1}{2}\left[(e^{i\frac{\pi}{4}}+e^{-i\frac{\pi}{4}})\ket{\alpha_0}+(e^{i\frac{\pi}{4}}-e^{-i\frac{\pi}{4}})\ket{-\alpha_0}\right]=\frac{1}{\sqrt{2}}[\ket{\alpha_0}+i\ket{-\alpha_0}],
  \end{align*}
  where 
  \begin{align*}
    \ket{\pm\alpha_0}=e^{-\frac{|\alpha_0|^2}{2}}\sum_{n=0}^\infty\frac{(\pm\alpha_0)^n}{\sqrt{n!}}\ket{n}.
  \end{align*}
  \item The transfrmation from the Interaction picture to the Schrodinger picture is 
  \begin{align*}
    \ket{\psi(\tau)}&=\hat{\mathbb{U}}_0(\tau)\ket{\psi_E(\tau)},\qquad\hat{\mathbb{U}}_0(\tau)=e^{-iH_0\tau/\hbar}\\
    &=e^{-i\omega\tau(\hat{N}+1/2)}\ket{\psi_E(\tau)}\\
    &=\frac{1}{\sqrt{2}}e^{-i\omega\tau(\hat{N}+1/2)}[\alpha_0+i\ket{-\alpha_0}]\ket{n}\\
    &=\frac{1}{\sqrt{2}}e^{-\frac{|\alpha_0|^2}{2}}\sum_{n=0}^\infty\frac{1}{\sqrt{n!}}\left[\alpha_0^ne^{-i\omega\tau(n+1/2)}+i(-\alpha_0)^ne^{-i\omega\tau(n+1/2)}\right]\ket{n}\\
    &=\frac{1}{\sqrt{2}}e^{-i\frac{\omega}{2}\tau}e^{-\frac{|\alpha_0|^2}{2}}\sum_{n=0}^\infty\frac{1}{\sqrt{n!}}\left[\left(\alpha_0e^{-i\omega\tau}\right)^n\ket{n}+i\left(-\alpha_0e^{-i\omega\tau}\right)\ket{n}\right]\\
    \ket{\psi(\tau)}&=\frac{1}{\sqrt{2}}e^{-i\frac{\omega}{2}\tau}\left[\ket{\alpha_0e^{-i\omega\tau}}+i\ket{-\alpha_0e^{-i\omega\tau}}\right].
  \end{align*}
  \item Evaluating with $\tau=\pi/2\omega$,
  \begin{align*}
    \ket{\psi(\frac{\pi}{2\omega})}&=\frac{1}{\sqrt{2}}e^{-i\frac{\omega}{2}\frac{\pi}{2\omega}}\left[\ket{\alpha_0e^{-i\omega\frac{\pi}{2\omega}}}+i\ket{-\alpha_0e^{-i\omega\frac{\pi}{2\omega}}}\right]\\
    &=\frac{1}{\sqrt{2}}e^{-i\frac{\pi}{4}}\left[\ket{\alpha_0e^{-i\frac{\pi}{2}}}+i\ket{-\alpha_0e^{-i\frac{\pi}{2}}}\right]\\
    \ket{\psi(\frac{\pi}{2\omega})}&=\frac{1}{\sqrt{2}}e^{-i\frac{\pi}{4}}\left[\ket{-i\alpha_0}+i\ket{i\alpha_0}\right].
  \end{align*}
  In addition, at $t=0$ we have 
  \begin{align*}
    \ket{\psi_E(0)}=\ket{\alpha_0}.
  \end{align*}
  Then,
  \begin{figure}[h!]
    \centering
    \begin{circuitikz}
      \draw[arrow](-2,0)--(2,0)node[below]{$\text{Re}({\alpha})$};
      \draw[arrow](0,-2)--(0,2)node[right]{$\text{Im}({\alpha})$};
      \draw[NavyBlue,very thick,-*](0,0)--(1,0)node[above]{$\ket{\alpha_0}$};
      \draw[OliveGreen,very thick,-*](0,0)--(0,1)node[right]{$\ket{i\alpha_0}$};
      \draw[OliveGreen,very thick,-*](0,0)--(0,-1)node[right]{$\ket{-i\alpha_0}$};
    \end{circuitikz}
  \end{figure}
\end{enumerate}
%%
\section*{Problem II}
The Hamiltonian in the whole range is:
\begin{align*}
  \hat{H}=\hat{H_0}+\hat{W}=\begin{cases}
    \frac{\hat{P}^2}{2m},&t<0\\\\
    \frac{\hat{P}^2}{2m}+\frac{1}{2}m\omega^2\hat{X}^2,&0\leq t<\tau\\\\
    \frac{\hat{P}^2}{2m},&t\geq0
  \end{cases},\qquad\hat{W}=\frac{1}{2}m\omega^2\hat{X}^2.
\end{align*}
The evolution operator is $U_0(t)=e^{-iH_0t/\hbar}=e^{-i\hat{P}^2t/2m\hbar}$.
The effective Hamiltonian in terms of the Schrodinger picture position and momentum operators is:
\begin{align*}
  H_E&=U^\dagger_0(t,0)H_1U_0(t,0)=e^{i\frac{\hat{P}^2t}{2m\hbar}}\left[\frac{1}{2}m\omega^2\hat{X}^2\right]e^{-i\hat{P}^2t/2m\hbar}=\frac{1}{2}m\omega^2e^{i\hat{P}^2t/2m\hbar}\hat{X}^2e^{-i\hat{P}^2t/2m\hbar}\\
  &\stackrel{(a)}{=}\frac{1}{2}m\omega^2\left[e^{i\hat{P}^2t/2m\hbar}\hat{X}e^{-i\hat{P}^2t/2m\hbar}\right]^2.
\end{align*}
In $(a)$, we used the property. We can see the term iinside the brackets as the product $ABC$ of operators, where we would like to switch the position of $\hat{X}$ with the right exponential, that why we use 
\begin{align*}
  ABC=A[B,C]+ACB.
\end{align*}
The commutator $[B,C]$ is
\begin{align*}
  [B,C]=[\hat{X},e^{-i\frac{\hat{P}^2t}{2m\hbar}}]=i\hbar\partial_{\hat{P}}e^{-i\frac{\hat{P}^2t}{2m\hbar}}=i\hbar\frac{-i2\hat{P}t}{2m\hbar}e^{-i\frac{\hat{P}^2t}{2m\hbar}}=\frac{\hat{P}t}{m}e^{-i\frac{\hat{P}^2t}{2m\hbar}}.
\end{align*}
Then, substituting this commutator in the above relation
\begin{align*}
  e^{i\frac{\hat{P}^2t}{2m\hbar}}\hat{X}e^{-i\frac{\hat{P}^2t}{2m\hbar}}=e^{i\frac{\hat{P}^2t}{2m\hbar}}\frac{\hat{P}t}{m}e^{-i\frac{\hat{P}^2t}{2m\hbar}}+e^{i\frac{\hat{P}^2t}{2m\hbar}}e^{-i\frac{\hat{P}^2t}{2m\hbar}}\hat{X}=\hat{X}+\frac{\hat{P}t}{m}.
\end{align*}
Finally,
\begin{align*}
  H_E=\frac{1}{2}m\omega^2\left[\hat{X}+\frac{\hat{P}t}{m}\right]^2.
\end{align*}
%%
\section*{Problem III}
The Hamiltonian is 
\begin{align*}
  H=\begin{cases}
    H_0,&t<0\\
    H_0+W(t),&0\leq t<\tau=\frac{4\pi}{\omega}\\
    H_0,&t>\tau
  \end{cases}.
\end{align*}
\begin{enumerate}[itemsep=0pt,topsep=0pt,label=\alph*)]
  \item sa
  \begin{align*}
    \ket{\psi_I(t)}=U_0^\dagger(\tau,0)\ket{\psi_S(t)}=e^{-i4\pi(\hat{N}+1/2)}\ket{\psi_S(t)}=e^{-i4\pi n}e^{-i2\pi}\ket{\psi_S(t)}=\ket{\psi_S(t)}.
  \end{align*}
  \item The effective Hamiltonian is:
  \begin{align*}
    H_E&=U_0^\dagger W(t)U_0\\
    &=\frac{i\hbar\Omega}{2}\left[e^{i\omega(\hat{N}+1/2)t}(\hat{a}^2e^{i2\omega t}-(\hat{a}^\dagger)^2e^{-i2\omega t})e^{-i\omega(\hat{N}+1/2)t}\right]\\
    &=\frac{i\hbar\Omega}{2}\left\{e^{i2\omega t}[e^{i\omega(\hat{N}+1/2)t}\hat{a}e^{-i\omega(\hat{N}+1/2)t}]^2-e^{-i2\omega t}[e^{i\omega(\hat{N}+1/2)t}\hat{a}^\dagger e^{-i\omega(\hat{N}+1/2)t}]^2\right\}\\
    H_E&=\frac{i\hbar\Omega}{2}\left\{\hat{a}^2-(\hat{a}^\dagger)^2\right\}.
  \end{align*}
  \item We use the expression of the $\hat{a}$ operators in terms of $\hat{X}$ and $\hat{P}$:
  \begin{align*}
    \hat{H}_E&=\frac{i\hbar\Omega}{2}\left\{\frac{1}{2}\left(\frac{\hat{X}}{\sigma}+i\frac{\hat{P}\sigma}{\hbar}\right)^2-\frac{1}{2}\left(\frac{\hat{X}}{\sigma}-i\frac{\hat{P}\sigma}{\hbar}\right)^2\right\}\\
    &=\frac{i\hbar\Omega}{4}\left\{\frac{\hat{X}^2}{\sigma^2}+\frac{i}{\hbar}(\hat{X}\hat{P}+\hat{P}\hat{X})-\frac{\hat{P}^2\sigma^2}{\hbar^2}-\left[\frac{\hat{X}^2}{\sigma^2}-\frac{i}{\hbar}(\hat{X}\hat{P}+\hat{P}\hat{X})-\frac{\hat{P}^2\sigma^2}{\hbar^2}\right]\right\}\\
    &=-\frac{\Omega}{2}(\hat{X}\hat{P}+\hat{P}\hat{X})\\
    \hat{H}_E&=-\frac{\Omega}{2}\{\hat{X},\hat{P}\},\quad\{\cdot\}=\text{anti-commutator}.
  \end{align*}
  \item We use the effective Hamiltonian of part b):
  \begin{align*}
    \hat{\mathbb{U}}_E(\tau)=e^{-\frac{i\tau}{\hbar}\frac{i\hbar\Omega}{2}(\hat{a}^2-(\hat{a}^\dagger)^2)}=e^{\frac{\Omega\tau}{2}(\hat{a}^2-(\hat{a}^\dagger)^2)}=e^{\frac{b}{2}(\hat{a}^2-(\hat{a}^\dagger)^2)}.
  \end{align*}
  \item In this case, we compare the definition the above operator with the new definition, and we look at the exponent of each function:
  \begin{align*}
    e^{-\hat{B}}=\hat{\mathbb{U}}_E(\tau)=e^{\frac{b}{2}(\hat{a}^2-(\hat{a}^\dagger)^2)}\Longrightarrow\hat{B}=-\frac{b}{2}(\hat{a}^2-(\hat{a}^\dagger)^2).
  \end{align*}
  As a verification, the adjoint is:
  \begin{align*}
    \hat{B}^\dagger=-\frac{b}{2}((\hat{a}^\dagger)^2-\hat{a}^2)=\frac{b}{2}(\hat{a}^2-(\hat{a}^\dagger)^2)=-\hat{B}.
  \end{align*}
  So $\hat{B}$ is anti-Hermitian.
  \item asgasg
  \begin{align*}
    [\hat{B},\hat{a}]&=-\frac{b}{2}\{[\hat{a}^2,\hat{a}]-[(\hat{a}^\dagger)^2,\hat{a}]\}=\frac{b}{2}[(\hat{a}^\dagger)^2,\hat{a}]=\frac{b}{2}\{\hat{a}^\dagger[\hat{a}^\dagger,\hat{a}]+[\hat{a}^\dagger,\hat{a}]\hat{a}^\dagger\}=-b\hat{a}^\dagger\\
    [\hat{B},\hat{a}^\dagger]&=-\frac{b}{2}\{[\hat{a}^2,\hat{a}^\dagger]-[(\hat{a}^\dagger)^2,\hat{a}^\dagger]\}=-\frac{b}{2}[\hat{a}^2,\hat{a}^\dagger]=-\frac{b}{2}\{\hat{a}[\hat{a},\hat{a}^\dagger]+[\hat{a},\hat{a}^\dagger]\hat{a}\}=-b\hat{a}.
  \end{align*}
  \item We now from the previous part the commutators, and also 
  \begin{align*}
    \hat{\mathbb{Q}}(b)=e^{-\hat{B}}=e^{\frac{b}{2}(\hat{a}^2-(\hat{a}^\dagger)^2)},\quad\hat{\mathbb{Q}}^\dagger(b)=e^{\hat{B}}=e^{-\frac{b}{2}(\hat{a}^2-(\hat{a}^\dagger)^2)},\quad\hat{\mathbb{Q}}(b)\hat{\mathbb{Q}}^\dagger(b)=\mathds{1}.
  \end{align*}
  To the application of the formula provided, we compute the following commutators:
  \begin{align*}
    [\hat{B},\hat{a}]=-b\hat{a}^\dagger,\quad[\hat{B},[\hat{B},\hat{a}]]=[\hat{B},-b\hat{a}^\dagger]=b^2\hat{a},\quad[\hat{B},[\hat{B},[\hat{B},\hat{a}]]]=-b^3\hat{a}^\dagger,\quad\cdots
  \end{align*}
  Then,
  \begin{align*}
    \hat{\mathbb{Q}}^\dagger(b)\hat{a}\hat{\mathbb{Q}}(b)&=\hat{a}+(-b\hat{a}^\dagger)+\frac{1}{2!}[\hat{B},[\hat{B},\hat{a}]]+\frac{1}{3!}[\hat{B},[\hat{B},[\hat{B},\hat{a}]]]+\cdots\\
    &=\hat{a}-b\hat{a}^\dagger+\frac{b^2}{2!}\hat{a}-\frac{b^3}{3!}\hat{a}^\dagger+\cdots\\
    &=\left(1+\frac{b^2}{2!}+\cdots\right)\hat{a}-\left(b-\frac{b^3}{3!}\right)\hat{a}^\dagger\\
    \hat{\mathbb{Q}}^\dagger(b)\hat{a}\hat{\mathbb{Q}}(b)&=\text{cosh}(b)\hat{a}-\text{sinh}(b)\hat{a}^\dagger,
  \end{align*}
  and,
  \begin{align*}
    [\hat{\mathbb{Q}}^\dagger(b)\hat{a}\hat{\mathbb{Q}}(b)]^\dagger=\hat{\mathbb{Q}}^\dagger(b)\hat{a}^\dagger\hat{\mathbb{Q}}(b)=\text{cosh}(b)\hat{a}^\dagger-\text{sinh}(b)\hat{a}.
  \end{align*}
  \item We use both result from above
  \begin{align*}
    \hat{\mathbb{Q}}^\dagger(b)\hat{X}\hat{\mathbb{Q}}(b)&=\frac{\sigma}{\sqrt{2}}\hat{\mathbb{Q}}^\dagger(b)(\hat{a}^\dagger+\hat{a})\hat{\mathbb{Q}}(b)=\frac{\sigma}{\sqrt{2}}\left[\hat{\mathbb{Q}}^\dagger(b)\hat{a}^\dagger\hat{\mathbb{Q}}(b)+\hat{\mathbb{Q}}^\dagger(b)\hat{a}\hat{\mathbb{Q}}(b)\right]\\
    &=\frac{\sigma}{\sqrt{2}}\left[\text{cosh}(b)\hat{a}^\dagger-\text{sinh}(b)\hat{a}+\text{cosh}(b)\hat{a}-\text{sinh}(b)\hat{a}^\dagger\right]\\
    &=\frac{\sigma}{\sqrt{2}}[(\hat{a}^\dagger+\hat{a})(\text{cosh}(b)-\text{sinh}(b))]\\
    \hat{\mathbb{Q}}^\dagger(b)\hat{X}\hat{\mathbb{Q}}(b)&=e^{-b}\hat{X}.
  \end{align*}
  For the momentum operator, we have similarly
  \begin{align*}
    \hat{\mathbb{Q}}^\dagger(b)\hat{P}\hat{\mathbb{Q}}(b)&=\frac{i\hbar}{\sqrt{2}\sigma}\hat{\mathbb{Q}}^\dagger(b)(\hat{a}^\dagger-\hat{a})\hat{\mathbb{Q}}(b)=\frac{i\hbar}{\sqrt{2}\sigma}\left[\hat{\mathbb{Q}}^\dagger(b)\hat{a}^\dagger\hat{\mathbb{Q}}(b)-\hat{\mathbb{Q}}^\dagger(b)\hat{a}\hat{\mathbb{Q}}(b)\right]\\
    &=\frac{i\hbar}{\sqrt{2}\sigma}\left[\text{cosh}(b)\hat{a}^\dagger-\text{sinh}(b)\hat{a}-\text{cosh}(b)\hat{a}+\text{sinh}(b)\hat{a}^\dagger\right]\\
    &=\frac{i\hbar}{\sqrt{2}\sigma}[(\hat{a}^\dagger+\hat{a})(\text{cosh}(b)+\text{sinh}(b))]\\
    \hat{\mathbb{Q}}^\dagger(b)\hat{P}\hat{\mathbb{Q}}(b)&=e^{b}\hat{P}.
  \end{align*}
  The remaining ones are computed easily with the formular used in past exercise:
  \begin{align*}
    \hat{\mathbb{Q}}^\dagger(b)\hat{X}^2\hat{\mathbb{Q}}(b)&=[\hat{\mathbb{Q}}^\dagger(b)\hat{X}\hat{\mathbb{Q}}(b)]^2=e^{-2b}\hat{X}^2\\
    \hat{\mathbb{Q}}^\dagger(b)\hat{P}^2\hat{\mathbb{Q}}(b)&=[\hat{\mathbb{Q}}^\dagger(b)\hat{P}\hat{\mathbb{Q}}(b)]^2=e^{2b}\hat{P}^2.
  \end{align*}
  \item The ground state of QHO is:
  \begin{align*}
    \braket{0|\hat{X}|0}=0,\quad\braket{0|\hat{P}|0}=0,\quad\braket{0|\hat{X}^2|0}=\frac{\sigma^2}{2},\quad\braket{0|\hat{P}|0}=\frac{\hbar^2}{2\sigma^2}.
  \end{align*}
  Using the above results, and the definition of the state $\ket{\varphi}=\hat{\mathbb{Q}}(b)\ket{0}$
  \begin{align*}
    \braket{\varphi|\hat{X}|\varphi}&=e^{-b}\braket{0|\hat{X}|0}=0\\
    \braket{\varphi|\hat{P}|\varphi}&=e^b\braket{0|\hat{P}|0}=0\\
    \braket{\varphi|\hat{X}^2|\varphi}&=e^{-2b}\braket{0|\hat{X}^2|0}=e^{-2b}\frac{\sigma^2}{2}\\
    \braket{\varphi|\hat{P}^2|\varphi}&=e^{2b}\braket{0|\hat{P}^2|0}=e^{2b}\frac{\hbar^2}{2\sigma^2}.
  \end{align*}
  \item The uncertainty is therefore 
  \begin{align*}
    \left.\begin{array}{l}
      \Delta\hat{X}=\sqrt{\braket{\hat{X}^2}}=e^{-b}\frac{\sigma}{\sqrt{2}}\\
      \Delta\hat{P}=\sqrt{\braket{\hat{P}^2}}=e^b\frac{\hbar}{\sqrt{2}\sigma}
    \end{array}\right\}\Delta\hat{X}\Delta\hat{P}=\frac{\hbar}{2}.
  \end{align*}
  \item The wavefunction in the $\{\ket{x}\}$ representation is $\varphi(x)=\braket{x|\varphi}$. We need to know the action of t
  \begin{align*}
    \hat{X}(\hat{\mathbb{Q}}(b)\ket{x})=\hat{\mathbb{Q}}(b)\hat{\mathbb{Q}}^\dagger(b)\hat{X}\hat{\mathbb{Q}}(b)\ket{x}=\hat{\mathbb{Q}}(b)e^{-b}\hat{X}\ket{x}=xe^{-b}(\hat{\mathbb{Q}}(b)\ket{x}).
  \end{align*}
  Therefore, 
  \begin{align*}
    \hat{X}(\hat{\mathbb{Q}}(b)\ket{x})=xe^{-b}(\hat{\mathbb{Q}}(b)\ket{x})\Longrightarrow\hat{\mathbb{Q}}(b)\ket{x}=C\ket{e^{-b}x}.
  \end{align*}
  The eigenstate $\ket{e^{-b}x}$ is proportional by the factor $C$. We find the coeficient $c$
  \begin{align*}
    \braket{x'|x}=\braket{x|\hat{\mathbb{Q}}^\dagger(b)\hat{\mathbb{Q}}(b)|x}=|c|^2\braket{e^{-b}x'|e^{-b}x}=|c|^2\delta[e^{-b}(x'-x)]=|c|^2e^b\delta(x'-x)=1.
  \end{align*}
  The coeffient is:
  \begin{align*}
    |c|^2e^b=1\longrightarrow c=e^{-b/2}.
  \end{align*}
  Because the expression for the ground state is a gaussian of the form:
  \begin{align*}
    \psi_0(x)=\left(\frac{1}{\pi\sigma^2}\right)^{1/4}e^{-\frac{x^2}{2\sigma^2}},
  \end{align*}
  we construct our function as:
  \begin{align*}
    \varphi(x)=\braket{x|\hat{\mathbb{Q}}(b)|0}=e^{-b/2}\braket{e^bx|0}=C\psi_0(e^bx)=e^{-b/2}\left(\frac{1}{\pi\sigma^2}\right)^{1/4}e^{b/2}e^{-\frac{e^{2b}x^2}{2\sigma^2}}=\left(\frac{1}{\pi\gamma^2}\right)^{1/4}e^{-\frac{x^2}{2\gamma^2}},
  \end{align*}
  with $\gamma=\sigma e^{-b}$.
  \item Knowing that 
  \begin{align*}
    \hat{H_0}=\frac{\hat{P}^2}{2m}+\frac{1}{2}m\omega^2\hat{X}^2,\quad \braket{\hat{X}^2}_\varphi=e^{-2b}\frac{\sigma^2}{2},\quad \braket{\hat{P}^2}_\varphi=e^{2b}\frac{\hbar^2}{2\sigma^2}.
  \end{align*}
  The mean value of the Hamiltonian can be expressed in terms of the mean values of the position and momentum operators:
  \begin{align*}
    \braket{H_0}_\varphi=\frac{1}{2m}e^{2b}\frac{\hbar^2}{2\sigma^2}+\frac{1}{2}m\omega^2e^{-2b}\frac{\sigma^2}{2}=\frac{\hbar\omega}{2}\text{cosh}(2b),
  \end{align*}
  where we substituted $\sigma=\sqrt{\hbar/m\omega}$ to simplify further the expression. For $b=0$, we see that $\braket{H_0}_\varphi=\hbar\omega/2$ the ground state energy level.
  \item The operator $\hat{\mathbb{Q}}(p)$ changes the uncertainty of the quadratures increasing one and reducing the other respectively so that the uncertainty product is maintained.
\end{enumerate}

%%
\section*{Problem IV}
\begin{enumerate}[itemsep=0pt,topsep=0pt,label=\alph*)]
  \item We plot the function $\text{sech}(x)$ to verify its parity. We can see that it is \textbf{even}.
  \begin{figure}[h!]
    \centering
    \begin{circuitikz}
      \draw[arrow](-3,0)--(3,0)node[below]{$x$};
      \draw[arrow](0,0)--(0,3)node[right]{$f(x)$};
      \draw[very thick,NavyBlue,domain=-2:2,samples=100] plot(\x,{ 1/cosh(\x) });
    \end{circuitikz}
  \end{figure}

  This fact will facilitate us when computing $\Delta X$, as we must integrate over $|\phi(x)|^2$ which therefore, is also even.
  We then have,
  \begin{align*}
    \braket{X}&=\int_{-\infty}^\infty x|\phi(x)|^2\;dx=\frac{1}{2\beta}\int_{-\infty}^\infty x\;\text{sech}(x/\beta)\;dx=0\\
    \braket{X^2}&=\int_{-\infty}^\infty x^2|\phi(x)|^2\;dx=\frac{1}{2\beta}\int_{-\infty}^\infty x^2\text{sech}(x/\beta)\;dx=\frac{\beta^2}{2}\int_{-\infty}^\infty u^2\text{sech}^2(u)\;du=\frac{\pi^2\beta^2}{12}.
  \end{align*}
  The $X$ uncertainty is 
  \begin{align*}
    \Delta X=\sqrt{\braket{X^2}-\braket{X}^2}=\frac{\pi\beta}{2\sqrt{3}}.
  \end{align*}
  Similarly, for the Fourier transform we have:
  \begin{align*}
    \braket{P}&=\int_{-\infty}^\infty p|\hat{\phi}(p)|^2\;dp=\frac{\pi\beta}{4\hbar}\int_{-\infty}^\infty p\;\text{sech}^2(\frac{\pi\beta p}{2\hbar})\;dp=0\\
    \braket{P^2}&=\int_{-\infty}^\infty p^2|\hat{\phi}(p)|^2\;dp=\frac{\pi\beta}{4\hbar}\int_{-\infty}^\infty p^2\text{sech}^2(\frac{\pi\beta p}{2\hbar})\;dp=\frac{2\hbar^2}{\pi^2\beta^2}\int_{-\infty}^\infty u^2\text{sech}^2(u)\;du=\frac{\hbar^2}{\beta^23}.
  \end{align*}
  Thus 
  \begin{align*}
    \Delta P=\sqrt{\braket{P^2}-\braket{P}^2}=\frac{\hbar}{\beta\sqrt{3}}.
  \end{align*}
  The uncertainty product is 
  \begin{align*}
    \Delta X\Delta P=\frac{\pi\beta}{2\sqrt{3}}\frac{\hbar}{\beta\sqrt{3}}=\frac{\hbar\pi}{6}.
  \end{align*}
  \item The evolution in $\pi/2\omega$ gives a well-known quantity, a scaled Fourier transform of the wavefunction.
  \begin{align*}
    \Phi(x,\frac{\pi}{2\omega})=U(\frac{\pi}{2\omega},0)\Phi(x,0)=e^{-i\pi/4}\sqrt{\frac{\hbar}{\sigma^2}}\mathcal{F}\{\Phi(x,0)\}\bigr|_{p=\hbar x/\sigma^2}
  \end{align*}
  We can see that the function to be computed its Fourier transform is spatially shifted by $x_0$ so we could directly use the respective property of Fourier transform of a shifter function:
  \begin{align*}
    \mathcal{F}\{\Phi(x,0)\}=\hat{\Phi}(p,0)\Longrightarrow \mathcal{F}\{\Phi(x-x_0,0)\}=e^{-ipx_0/\hbar}\hat{\Phi}(p,0).
  \end{align*}
  So,
  \begin{align*}
    \Phi(x,\frac{\pi}{2\omega})=-e^{-i\pi/4}\sqrt{\frac{\hbar}{\sigma^2}}\left[e^{-ipx_0/\hbar}\hat{\Phi}(p,0)\right]\biggr|_{p=\hbar x/\sigma^2}=-\sqrt{\frac{\pi\beta}{4\sigma^2}}e^{-i\pi/4}e^{-i\frac{xx_0}{\sigma^2}}\;\text{sech}(\frac{\pi\beta x}{2\sigma^2}).
  \end{align*}
  \item To maintain the width $\Delta X=\frac{\pi\beta}{2\sqrt{3}}$, we compute $\Delta X$ for $\Phi(0,\pi/2\omega)$ and equate it to the uncertainty at $t=0$:
  \begin{align*}
    \left.
    \begin{array}{l}
      \braket{X}=0\\
      \displaystyle\braket{X^2}=\frac{\pi\beta}{4\sigma^2}\int_{-\infty}^\infty x^2\text{sech}^2(\frac{\pi\beta x}{2\sigma^2})\;dx=\frac{\sigma^4}{3\beta^2}.
    \end{array}\right\}\Delta X=\sqrt{\braket{X^2}}=\frac{\sigma^2}{\sqrt{3}\beta}.
  \end{align*} 
  Equating it with the uncertainty of the wavefunction at $t=0$:
  \begin{align*}
    \frac{\pi\beta}{2\sqrt{3}}&=\frac{\sigma^2}{\sqrt{3}\beta}\longrightarrow\beta=\sqrt{\frac{2\sigma^2}{\pi}}.
  \end{align*}
\end{enumerate}

%%
\section*{Problem V}
  We handle the problem by approximating the potential given with its second-order Taylor expansion:
    \begin{align*}
    V(x)=-V_0-\frac{V_0}{2b^2}x^2+O(x^3)+\cdots=\frac{1}{2}m\omega^2x^2.
  \end{align*}
  The figure below represents the behavior of this approximation versus the real potential.
  \begin{figure}[h!]
    \centering
    \begin{circuitikz}
      \def\va{1}
      \def\bb{1}
      \draw[arrow](-2,0)--(2.5,0)node[below]{$x$};
      \draw[arrow](0,-2)--(0,2)node[right]{$V(x)$};
      \draw[very thick,NavyBlue,domain={-\bb}:{\bb},samples=500] plot(\x,{ -\va*sqrt( 1 - (\x/\bb)^2 ) });
      \draw[very thick,NavyBlue](-2,0)--({-\bb},0)({\bb},0)--(2,0);
      \draw[very thick,dashed,black,domain={-\bb-1}:{\bb+1},samples=500] plot(\x,{ -\va+(\va/(2*\bb^2))*\x*\x });
      \draw(2,1.5)node[]{\small$V(x)\approx-V_0[1+\frac{x^2}{2b}]$}(-.1,{-\va})--(.1,{-\va})node[below right]{$-V_0$};
    \end{circuitikz}
  \end{figure}
  
  Comparing the quadratic term of the expansion with the QHO yields the following frequency:
  \begin{align*}
    \omega=\sqrt{\frac{V_0}{mb^2}}.
  \end{align*}
  The energy levels in the QHO is:
  \begin{align*}
    E_n=\hbar\omega\left(n+\frac{1}{2}\right),\quad n=0,1,2,\cdots.
  \end{align*}
  So, in this case they will be shifted
  \begin{align*}
    E'_n=-V_0+E_n=-V_0+\hbar\omega\left(n+\frac{1}{2}\right),\quad n=0,1,2,\cdots.
  \end{align*}
  The ground and first excited state energy eigenvalues are:
  \begin{align*}
    E_0=-V_0+\frac{1}{2}\hbar\omega,\quad E_1=-V_0+\frac{3}{2}\hbar\omega.
  \end{align*}



%\nocite{*}
%\bibliographystyle{plain}   % or unsrt, alpha, apalike, etc.
%\bibliography{refs}

\end{document}
