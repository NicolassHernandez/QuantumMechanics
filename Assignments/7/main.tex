\documentclass[letterpaper,11pt,twoside]{article}
\usepackage{graphicx} % Required for inserting images
\usepackage[table,xcdraw,dvipsnames]{xcolor}
\usepackage{amsmath,amsfonts,amssymb,amsthm}
\usepackage{listings}
\usepackage{lipsum}
\usepackage{hyperref}
\usepackage{mathrsfs}

\usepackage{enumitem}

\usepackage{tikz}
\usepackage[siunitx, RPvoltages]{circuitikz}
\usetikzlibrary{3d}
\usepackage{comment}
\usepackage{caption,subcaption}
\usepackage{pgfplots}
\pgfplotsset{compat=newest} % or a newer version if available
\usepgfplotslibrary{groupplots}
\usetikzlibrary{pgfplots.groupplots}
\usetikzlibrary{shapes.geometric, arrows}
\tikzstyle{arrow} = [->,>=stealth,shorten >=2pt]
\newcommand{\ket}[1]{|#1\rangle}
\newcommand{\bra}[1]{\langle#1|}
\newcommand{\br}{\bm{r}}
\newcommand{\bR}{\bm{R}}
\newcommand{\bp}{\bm{p}}
\newcommand{\bP}{\bm{P}}
\newcommand{\braket}[1]{\langle#1\rangle}
\newcommand{\F}{\mathscr{F}}
\newcommand{\E}{\mathscr{E}}
\newcommand{\re}[1]{\text{Re}\left(#1\right)}
\newcommand{\im}[1]{\text{Im}\left(#1\right)}
\usepackage{dsfont}
\usepackage{cancel}
\usepackage{bm}
\usepackage{fancyhdr}
\usepackage[utf8x]{inputenc}
\usepackage[T1]{fontenc}
\usepackage[margin=0.8in,top=1in,bottom=1in]{geometry}
%%%%%
\begin{filecontents*}{refs.bib}
@book{bornwolf,
  author    = {Born, M. and Wolf, E.},
  title     = {Principles of Optics},
  publisher = {Pergamon Press},
  edition   = {7},
  year      = {1999}
}
@book{hecht,
  author    = {Hecht, E.},
  title     = {Optics},
  publisher = {Addison-Wesley},
  edition   = {5},
  year      = {2016}
}
\end{filecontents*}
%
\newcommand{\institution}{University of Arizona}
\newcommand{\autor}{Nicolás Hernández Alegría}
\newcommand{\course}{OPTI 570 Quantum Mechanics}
\newcommand{\assignment}{Assignment 7}
%
\title{\textbf{\assignment}\\\course\\{\Large\institution}}
\author{\autor}
\date{\today\\Total time: 15 hours}
%
\renewcommand{\sectionmark}[1]{\markright{#1}}
\fancypagestyle{mainstyle}{
    \fancyhf{} % Clear all header and footer fields
    \fancyfoot[C]{\thepage}
    \fancyhead[LE,RO]{\course} % Section name on odd pages
    \fancyhead[LO,RE]{\assignment}
    % Optional: Thin rules
    \renewcommand{\headrulewidth}{0pt} % Header rule
    \renewcommand{\footrulewidth}{0pt} % No footer rule
}
%
\begin{document}

\pagestyle{mainstyle}
\maketitle
%%
\section*{Problem I}
\begin{enumerate}[itemsep=0pt,topsep=0pt,label=\alph*)]
  \item The evolution operator would be of the form 
  \begin{align*}
    \hat{\mathbb{U}}_E(t)=e^{-i\hat{H}_1t/\hbar}=e^{-i\Omega(\hat{N}^2-1/2)t}.
  \end{align*}
  The checking is as follows:
  \begin{align*}
    \hat{\mathbb{U}}_E(\frac{2\pi}{\Omega})\ket{\varphi_n}&=e^{-i\Omega(n^2-1/2)\frac{2\pi}{\Omega}}\ket{\varphi_n}\\
    &=e^{-i2\pi(n^2-1/2)}\ket{\varphi_n}\\
    &=(e^{-2\pi})^{n^2}e^{i\pi}\ket{\varphi_n}\\
    \hat{\mathbb{U}}_E(\frac{2\pi}{\Omega})\ket{\varphi_n}&=-\ket{\varphi_n}.
  \end{align*}
  \item For $\tau=\pi/2\Omega$, the evolution is 
  \begin{align*}
    \hat{\mathbb{U}}_E(\tau)\ket{\varphi_n}&=e^{-i\Omega(n^2-1/2)\frac{\pi}{2\Omega}}\ket{\varphi_n}\\
    &=e^{-i\frac{\pi}{2}n^2}e^{i\frac{\pi}{4}}\ket{\varphi_n}\\
    &=(e^{-i\frac{\pi}{2}})^{n^2}e^{i\frac{\pi}{4}}\ket{\varphi_n}\\
    &=(-i)^{n^2}e^{i\frac{\pi}{4}}\ket{\varphi_n}\\
   \hat{\mathbb{U}}_E(\tau)\ket{\varphi_n}&=\begin{cases}
      e^{i\frac{\pi}{4}}=\frac{1}{\sqrt{2}}+\frac{i}{\sqrt{2}},&\text{$n$ even}\\
      e^{-i\frac{\pi}{4}}=\frac{1}{\sqrt{2}}-\frac{i}{\sqrt{2}},&\text{$n$ odd}
    \end{cases}\ket{\varphi_n}.
  \end{align*}
  \item We use the fact that in a coherent state, we can express it in terms of the energy eigenstates.
  \begin{align*}
    \ket{\alpha_0}=e^{-\frac{|\alpha_0|^2}{2}}\sum_{n=0}^\infty\frac{\alpha_0^n}{\sqrt{n!}}\ket{n}.
  \end{align*}
  We have found that 
  \begin{align*}
    \hat{\mathbb{U}}_E(\tau)=\begin{cases}
      e^{i\frac{\pi}{4}}=\frac{1}{\sqrt{2}}+\frac{i}{\sqrt{2}},&\text{$n$ even}\\
      e^{-i\frac{\pi}{4}}=\frac{1}{\sqrt{2}}-\frac{i}{\sqrt{2}},&\text{$n$ odd}
    \end{cases}.
  \end{align*}
  We then, must split the $\ket{\alpha_0}$ accordingly, in even and odd term so that the application of the evolution operator gives
  \begin{align*}
    \ket{\psi_E(\tau)}=e^{-\frac{|\alpha_0|^2}{2}}\left[e^{i\frac{\pi}{4}}S_{\text{even}}+e^{-i\frac{\pi}{4}}S_{\text{odd}}\right],
  \end{align*}
  where 
  \begin{align}
    S_{\text{even}}=\sum_{\text{$n$ even}}^\infty\frac{\alpha_0^n}{\sqrt{n!}}\ket{n},\quad\text{and}\quad S_{\text{odd}}=\sum_{\text{$n$ odd}}^\infty\frac{\alpha_0^n}{\sqrt{n!}}\ket{n}.
  \end{align}
  We then have that 
  \begin{align*}
    \left.\begin{array}{l}
      \ket{\alpha_0}=e^{-\frac{|\alpha_0|^2}{2}}(S_{\text{even}}+S_{\text{odd}})\\
      \ket{-\alpha_0}=e^{-\frac{|\alpha_0|^2}{2}}(S_{\text{even}}-S_{\text{odd}})      
    \end{array}\right\}\longrightarrow\begin{array}{l}
      S_{\text{even}}=\frac{1}{2}e^{\frac{|\alpha_0|^2}{2}}(\ket{\alpha_0}+\ket{-\alpha_0})\\
      S_{\text{odd}}=\frac{1}{2}e^{\frac{|\alpha_0|^2}{2}}(\ket{\alpha_0}-\ket{-\alpha_0})
    \end{array}.
  \end{align*}
  Substituting those in the evolution equation and rearranging:
  \begin{align*}
    \ket{\psi_E(\tau)}=\frac{1}{2}\left[(e^{i\frac{\pi}{4}}+e^{-i\frac{\pi}{4}})\ket{\alpha_0}+(e^{i\frac{\pi}{4}}-e^{-i\frac{\pi}{4}})\ket{-\alpha_0}\right]=\frac{1}{\sqrt{2}}[\alpha_0+i\ket{-\alpha_0}],
  \end{align*}
  where 
  \begin{align*}
    \ket{\pm\alpha_0}=e^{-\frac{|\alpha_0|^2}{2}}\sum_{n=0}^\infty\frac{(\pm\alpha_0)^n}{\sqrt{n!}}\ket{n}.
  \end{align*}
  \item The transfrmation from the Interaction picture to the Schrodinger picture is 
  \begin{align*}
    \ket{\psi(\tau)}&=\hat{\mathbb{U}}_0(\tau)\ket{\psi_E(\tau)},\qquad\hat{\mathbb{U}}_0(\tau)=e^{-iH_0\tau/\hbar}\\
    &=e^{-i\omega\tau(\hat{N}+1/2)}\ket{\psi_E(\tau)}\\
    &=\frac{1}{\sqrt{2}}e^{-i\omega\tau(\hat{N}+1/2)}[\alpha_0+i\ket{-\alpha_0}]\\
    &=\frac{1}{\sqrt{2}}e^{-\frac{|\alpha_0|^2}{2}}\sum_{n=0}^\infty\frac{1}{\sqrt{n!}}\left[\alpha_0^ne^{-i\omega\tau(n+1/2)}+i(-\alpha_0)^ne^{-i\omega\tau(n+1/2)}\right]\ket{n}\\
    &=\frac{1}{\sqrt{2}}e^{-i\frac{\omega}{2}\tau}e^{-\frac{|\alpha_0|^2}{2}}\sum_{n=0}^\infty\frac{1}{\sqrt{n!}}\left[\left(\alpha_0e^{-i\omega\tau}\right)^n\ket{n}+i\left(-\alpha_0e^{-i\omega\tau}\right)\ket{n}\right]\\
    \ket{\psi(\tau)}&=\frac{1}{\sqrt{2}}e^{-i\frac{\omega}{2}\tau}\left[\ket{\alpha_0e^{-i\omega\tau}}+i\ket{-\alpha_0e^{-i\omega\tau}}\right].
  \end{align*}
  \item asgagasga
\end{enumerate}
%%
\section*{Problem II}

%%
\section*{Problem III}
\begin{enumerate}[itemsep=0pt,topsep=0pt,label=\alph*)]
  \item sagasg
  \item asgasg
  \item asgag
  \item gasgas
  \item asgagasga
  \item asgasg
  \item asgag
  \item asgasg
  \item asgasgasgasg
  \item asgasgasgasg
  \item asgag
  \item asgasg
  \item asfas
\end{enumerate}

%%
\section*{Problem IV}
\begin{enumerate}[itemsep=0pt,topsep=0pt,label=\alph*)]
  \item We plot the function $\text{sech}(x)$ to verify its parity. We can see that it is \textbf{even}.
  \begin{figure}[h!]
    \centering
    \begin{circuitikz}
      \draw[arrow](-3,0)--(3,0)node[below]{$x$};
      \draw[arrow](0,0)--(0,3)node[right]{$f(x)$};
      \draw[very thick,NavyBlue,domain=-2:2,samples=100] plot(\x,{ 1/cosh(\x) });
    \end{circuitikz}
  \end{figure}

  This fact will facilitate us when computing $\Delta X$, as we must integrate over $|\phi(x)|^2$ which therefore, is also even.
  We then have,
  \begin{align*}
    \braket{X}&=\int_{-\infty}^\infty x|\phi(x)|^2\;dx=\frac{1}{2\beta}\int_{-\infty}^\infty x\;\text{sech}(x/\beta)\;dx=0\\
    \braket{X^2}&=\int_{-\infty}^\infty x^2|\phi(x)|^2\;dx=\frac{1}{2\beta}\int_{-\infty}^\infty x^2\text{sech}(x/\beta)\;dx=\frac{\beta^2}{2}\int_{-\infty}^\infty u^2\text{sech}^2(u)\;du=\frac{\pi^2\beta^2}{12}.
  \end{align*}
  The $X$ uncertainty is 
  \begin{align*}
    \Delta X=\sqrt{\braket{X^2}-\braket{X}^2}=\frac{\pi\beta}{2\sqrt{3}}.
  \end{align*}
  Similarly, for the Fourier transform we have:
  \begin{align*}
    \braket{P}&=\int_{-\infty}^\infty p|\bar{\phi}(p)|^2\;dp=\frac{\pi\beta}{4\hbar}\int_{-\infty}^\infty p\;\text{sech}^2(\frac{\pi\beta p}{2\hbar})\;dp=0\\
    \braket{P^2}&=\int_{-\infty}^\infty p^2|\bar{\phi}(p)|^2\;dp=\frac{\pi\beta}{4\hbar}\int_{-\infty}^\infty p^2\text{sech}^2(\frac{\pi\beta p}{2\hbar})\;dp=\frac{2\hbar^2}{\pi^2\beta^2}\int_{-\infty}^\infty u^2\text{sech}^2(u)\;du=\frac{\hbar^2}{\beta^23}.
  \end{align*}
  Thus 
  \begin{align*}
    \Delta P=\sqrt{\braket{P^2}-\braket{P}^2}=\frac{\hbar}{\beta\sqrt{3}}.
  \end{align*}
  The uncertainty product is 
  \begin{align*}
    \Delta X\Delta P=\frac{\pi\beta}{2\sqrt{3}}\frac{\hbar}{\beta\sqrt{3}}=\frac{\hbar\pi}{6}.
  \end{align*}
  \item The evolution in $\pi/2\omega$ gives a well-known quantity, a scaled Fourier transform of the wavefunction.
  \begin{align*}
    \Phi(x,\frac{\pi}{2\omega})=U(\frac{\pi}{2\omega},0)\Phi(x,0)=e^{-i\pi/4}\sqrt{\frac{\hbar}{\sigma^2}}\mathcal{F}\{\Phi(x,0)\}\bigr|_{p=\hbar x/\sigma^2}
  \end{align*}
  We can see that the function to be computed its Fourier transform is spatially shifted by $x_0$ so we could directly use the respective property of Fourier transform of a shifter function:
  \begin{align*}
    \mathcal{F}\{\Phi(x,0)\}=\bar{\Phi}(p,0)\Longrightarrow \mathcal{F}\{\Phi(x-x_0,0)\}=e^{-ipx_0/\hbar}\bar{\Phi}(p,0).
  \end{align*}
  So,
  \begin{align*}
    \Phi(x,\frac{\pi}{2\omega})=-e^{-i\pi/4}\sqrt{\frac{\hbar}{\sigma^2}}\left[e^{-ipx_0/\hbar}\bar{\Phi}(p,0)\right]\biggr|_{p=\hbar x/\sigma^2}=-\sqrt{\frac{\pi\beta}{4\sigma^2}}e^{-i\pi/4}e^{-i\frac{xx_0}{\sigma^2}}\;\text{sech}(\frac{\pi\beta x}{2\sigma^2}).
  \end{align*}
  \item To maintain the width $\Delta X=\frac{\pi\beta}{2\sqrt{3}}$, we compute $\Delta X$ for $\Phi(0,\pi/2\omega)$ and equate it to the uncertainty at $t=0$:
  \begin{align*}
    \left.
    \begin{array}{l}
      \braket{X}=0\\
      \displaystyle\braket{X^2}=\frac{\pi\beta}{4\sigma^2}\int_{-\infty}^\infty x^2\text{sech}^2(\frac{\pi\beta x}{2\sigma^2})\;dx=\frac{\sigma^4}{3\beta^2}.
    \end{array}\right\}\Delta X=\sqrt{\braket{X^2}}=\frac{\sigma^2}{\sqrt{3}\beta}.
  \end{align*} 
  Equating it with the uncertainty of the wavefunction at $t=0$:
  \begin{align*}
    \frac{\pi\beta}{2\sqrt{3}}&=\frac{\sigma^2}{\sqrt{3}\beta}\longrightarrow\beta=\sqrt{\frac{2\sigma^2}{\pi}}.
  \end{align*}
\end{enumerate}

%%
\section*{Problem V}





%\nocite{*}
%\bibliographystyle{plain}   % or unsrt, alpha, apalike, etc.
%\bibliography{refs}

\end{document}
