\documentclass[letterpaper,11pt,twoside]{article}
\usepackage{graphicx} % Required for inserting images
\usepackage[table,xcdraw,dvipsnames]{xcolor}
\usepackage{amsmath,amsfonts,amssymb,amsthm}
\usepackage{listings}
\usepackage{lipsum}
\usepackage{hyperref}
\usepackage{enumitem}

\usepackage{tikz}
\usepackage[siunitx, RPvoltages]{circuitikz}
\usetikzlibrary{3d}
\usepackage{comment}
\usepackage{caption,subcaption}
\usepackage{pgfplots}
\pgfplotsset{compat=newest} % or a newer version if available
\usepgfplotslibrary{groupplots}
\usetikzlibrary{pgfplots.groupplots}
\usetikzlibrary{shapes.geometric, arrows}
\tikzstyle{arrow} = [->,>=stealth,shorten >=2pt]
\newcommand{\ket}[1]{|#1\rangle}
\newcommand{\bra}[1]{\langle#1|}
\newcommand{\br}{\bm{r}}
\newcommand{\bR}{\bm{R}}
\newcommand{\bp}{\bm{p}}
\newcommand{\bP}{\bm{P}}
\newcommand{\braket}[1]{\langle#1\rangle}
\newcommand{\F}{\mathscr{F}}
\newcommand{\E}{\mathscr{E}}
\usepackage{dsfont}
\usepackage{cancel}
\usepackage{bm}
\usepackage{fancyhdr}
\usepackage[utf8x]{inputenc}
\usepackage[T1]{fontenc}
\usepackage[margin=0.8in,top=1in,bottom=1in]{geometry}
%%%%%
\begin{filecontents*}{refs.bib}
@book{bornwolf,
  author    = {Born, M. and Wolf, E.},
  title     = {Principles of Optics},
  publisher = {Pergamon Press},
  edition   = {7},
  year      = {1999}
}
@book{hecht,
  author    = {Hecht, E.},
  title     = {Optics},
  publisher = {Addison-Wesley},
  edition   = {5},
  year      = {2016}
}
\end{filecontents*}
%
\newcommand{\institution}{University of Arizona}
\newcommand{\autor}{Nicolás Hernández Alegría}
\newcommand{\course}{OPTI 570 Quantum Mechanics}
\newcommand{\assignment}{Assignment 3}
%
\title{\textbf{\assignment}\\\course\\{\Large\institution}}
\author{\autor}
\date{\today\\Total time: 8 hours}
%
\renewcommand{\sectionmark}[1]{\markright{#1}}
\fancypagestyle{mainstyle}{
    \fancyhf{} % Clear all header and footer fields
    \fancyfoot[C]{\thepage}
    \fancyhead[LE,RO]{\course} % Section name on odd pages
    \fancyhead[LO,RE]{\assignment}
    % Optional: Thin rules
    \renewcommand{\headrulewidth}{0pt} % Header rule
    \renewcommand{\footrulewidth}{0pt} % No footer rule
}
%
\begin{document}

\pagestyle{mainstyle}
\maketitle
%%
\section*{Part I}
\subsection*{Problem 2}
\begin{enumerate}[itemsep=0pt,topsep=0pt,label=\alph*.]
  \item The operator $\sigma_y$ is Hermitian as 
  \begin{align*}
    \sigma_y^\dagger=\left(\begin{bmatrix}
      0&-i\\i&0
    \end{bmatrix}^*\right)^T=\begin{bmatrix}
    0&i\\-i&0
  \end{bmatrix}^T=\begin{bmatrix}
    0&-i\\i&0
  \end{bmatrix}=\sigma_y.
  \end{align*}
  Its eigenvalues are then, the roots of the characteristic polynomial
  \begin{align*}
    P(\lambda)=\det(\sigma_y-\lambda I)=\lambda^2-1=0,
  \end{align*}
  from which we have 
  \begin{align}
    \lambda\in\{-1,1\}\in\mathbb{R}.
  \end{align}
  The eigenvalues are obtained evaluating each eigenvalue in the eigenvalue problem $(\sigma_y-\lambda)\bm{v}=\bm{0}$.
  We only list the final results as they were calculated in the assignment 1:
  \begin{align}
    \bm{v}\in\{\bm{v}_1,\bm{v}_2\}=\left\{\frac{1}{\sqrt{2}}\begin{bmatrix}
    1\\-i
    \end{bmatrix},\frac{1}{\sqrt{2}}\begin{bmatrix}
    1\\i
    \end{bmatrix}\right\}=\left\{\frac{1}{\sqrt{2}}(\ket{1}-i\ket{2}),\frac{1}{\sqrt{2}}(\ket{1}+i\ket{2})\right\}.
  \end{align}
  \item The projectors is 
  \begin{align*}
    P=\sum_{i=1}^2\ket{\bm{v}_i}\bra{\bm{v}_i}=\ket{\bm{v}_1}\bra{\bm{v}_1}+\ket{\bm{v}_2}\bra{\bm{v}_2}.
  \end{align*}
  It consists of the sum of two outer products, which we already know:
  \begin{align*}
    \ket{\bm{v}_1}\bra{\bm{v}_1}&=\frac{1}{\sqrt{2}}\begin{bmatrix}
    1\\-i
    \end{bmatrix}\cdot\frac{1}{\sqrt{2}}\begin{bmatrix}
    1&i
    \end{bmatrix}=\frac{1}{2}\begin{bmatrix}
      1&i\\-i&1
    \end{bmatrix}\\
    \ket{\bm{v}_2}\bra{\bm{v}_2}&=\frac{1}{\sqrt{2}}\begin{bmatrix}
    1\\i
    \end{bmatrix}\cdot\frac{1}{\sqrt{2}}\begin{bmatrix}
    1&-i
    \end{bmatrix}=\frac{1}{2}\begin{bmatrix}
      1&-i\\i&1
    \end{bmatrix}
  \end{align*}
  The orthonormality relation states that that multiplication of two terms will produces in a zero matrix when they project onto 
  different eigenvectors:
  \begin{align*}
    \ket{\bm{v}_1}\braket{\bm{v}_1|\bm{v}_2}\bra{\bm{v}_2}=\frac{1}{2}\begin{bmatrix}
      1&i\\-i&1
    \end{bmatrix}\cdot\frac{1}{2}\begin{bmatrix}
      1&-i\\i&1
    \end{bmatrix}=\begin{bmatrix}
      1-1&-i+i\\-i+i&-1+1
    \end{bmatrix}=\begin{bmatrix}
      0&0\\0&0
    \end{bmatrix}=\bm{0},
  \end{align*}
  which would also be stated by simply computing 
  \begin{align*}
    \braket{\bm{v}_1|\bm{v}_2}=\frac{1}{\sqrt{2}}\begin{bmatrix}
    1&i
    \end{bmatrix}\cdot\frac{1}{\sqrt{2}}\begin{bmatrix}
    1\\i
    \end{bmatrix}=\frac{1}{2}\begin{bmatrix}
      1-1
    \end{bmatrix}=0.
  \end{align*}

  The closure relation must sum the identity matrix:
  \begin{align*}
    \ket{\bm{v}_1}\bra{\bm{v}_1}+\ket{\bm{v}_2}\bra{\bm{v}_2}=\frac{1}{2}\begin{bmatrix}
      1&i\\-i&1
    \end{bmatrix}+\frac{1}{2}\begin{bmatrix}
      1&-i\\i&1
    \end{bmatrix}=\begin{bmatrix}
      1&0\\0&1
    \end{bmatrix}=\mathds{1}.
  \end{align*}
  \item Omitted
\end{enumerate}
%%
\subsection*{Problem 3}
\begin{enumerate}[itemsep=0pt,topsep=0pt,label=\alph*.]
  \item To verify wether they are normalized, we must compute the norm in each ket and see if it is one. First, we represent each of them into the $\{\ket{u_1},\ket{u_2},\ket{u_3}\}$ basis:
  \begin{align*}
    \ket{\psi_0}=\begin{bmatrix}
      \cfrac{1}{\sqrt{2}}\\\cfrac{i}{2}\\\cfrac{1}{2}
    \end{bmatrix},\quad\text{and}\quad
    \ket{\psi_1}=\begin{bmatrix}
      \cfrac{1}{\sqrt{3}}\\0\\\cfrac{i}{\sqrt{3}}
    \end{bmatrix}.
  \end{align*} 
  The, the norm is:
  \begin{align*}
    \braket{\psi_0|\psi_0}&=\begin{bmatrix}
      \cfrac{1}{\sqrt{2}}&\cfrac{-i}{2}&\cfrac{1}{2}
    \end{bmatrix}\cdot\begin{bmatrix}
      \cfrac{1}{\sqrt{2}}\\\cfrac{i}{2}\\\cfrac{1}{2}
    \end{bmatrix}=\frac{1}{2}+\frac{1}{4}+\frac{1}{4}=1\\
    \braket{\psi_1|\psi_1}&=\begin{bmatrix}
      \cfrac{1}{\sqrt{3}}&0&\cfrac{-i}{\sqrt{3}}
    \end{bmatrix}\cdot\begin{bmatrix}
      \cfrac{1}{\sqrt{3}}\\0\\\cfrac{i}{\sqrt{3}}
    \end{bmatrix}=\frac{1}{3}+0+\frac{1}{3}=\frac{2}{3}\neq1.
  \end{align*}
  By looking the results, we san say that $\ket{\psi_0}$ is normalized by $\ket{\psi_1}$ does not. If we want to normalize it we must divide the ket by the value we have obtained:
  \begin{align*}
    \ket{\psi_1'}=\frac{3}{2}\ket{\psi_1}=\begin{bmatrix}
      \cfrac{3}{2\sqrt{3}}\\0\\\cfrac{3i}{2\sqrt{3}}
    \end{bmatrix}.
  \end{align*}
  \item The projections operators onto each state $\ket{\psi_0}$ and $|\ket{\psi_1}$ are:
  \begin{align*}
    \rho_0&=\ket{\psi_0}\bra{\psi_0}=\begin{bmatrix}
      \cfrac{1}{\sqrt{2}}\\\cfrac{i}{2}\\\cfrac{1}{2}
    \end{bmatrix}\cdot\begin{bmatrix}
      \cfrac{1}{\sqrt{2}}&\cfrac{-i}{2}&\cfrac{1}{2}
    \end{bmatrix}=\begin{bmatrix}
      \cfrac{1}{2}&\cfrac{-i}{2\sqrt{2}}&\cfrac{1}{2\sqrt{2}}\\
      \cfrac{i}{2\sqrt{2}}&\cfrac{1}{4}&\cfrac{i}{4}\\
      \cfrac{1}{2\sqrt{2}}&\cfrac{-i}{4}&\cfrac{1}{4}
    \end{bmatrix}\\
    \rho'_1&=\ket{\psi'_1}\bra{\psi'_1}=\begin{bmatrix}
      \cfrac{3}{2\sqrt{3}}\\0\\\cfrac{3i}{2\sqrt{3}}
    \end{bmatrix}\cdot \begin{bmatrix}
      \cfrac{3}{2\sqrt{3}}&0&\cfrac{-3i}{2\sqrt{3}}
    \end{bmatrix}=\begin{bmatrix}
      \cfrac{9}{12}&0&\cfrac{-9i}{12}\\
      0&0&0\\
      \cfrac{9i}{12}&0&\cfrac{9}{12}
    \end{bmatrix}.
  \end{align*}
  Notice we have used $\ket{\psi_1'}$ instead of $\ket{\psi_1}$. At first glance, both projectors look Hermitian. We can confirm it mathematically:
  \begin{align*}
    \rho^\dagger_0&=\left(\begin{bmatrix}
      \cfrac{1}{2}&\cfrac{-i}{2\sqrt{2}}&\cfrac{1}{2\sqrt{2}}\\
      \cfrac{i}{2\sqrt{2}}&\cfrac{1}{4}&\cfrac{i}{4}\\
      \cfrac{1}{2\sqrt{2}}&\cfrac{-i}{4}&\cfrac{1}{4}
    \end{bmatrix}^*\right)^T=\left(\begin{bmatrix}
      \cfrac{1}{2}&\cfrac{i}{2\sqrt{2}}&\cfrac{1}{2\sqrt{2}}\\
      \cfrac{-i}{2\sqrt{2}}&\cfrac{1}{4}&\cfrac{-i}{4}\\
      \cfrac{1}{2\sqrt{2}}&\cfrac{i}{4}&\cfrac{1}{4}
    \end{bmatrix}\right)^T=\begin{bmatrix}
      \cfrac{1}{2}&\cfrac{-i}{2\sqrt{2}}&\cfrac{1}{2\sqrt{2}}\\
      \cfrac{i}{2\sqrt{2}}&\cfrac{1}{4}&\cfrac{i}{4}\\
      \cfrac{1}{2\sqrt{2}}&\cfrac{-i}{4}&\cfrac{1}{4}
    \end{bmatrix}=\rho_0\\
    \rho'^\dagger_1&=\left(\begin{bmatrix}
      \cfrac{9}{12}&0&\cfrac{-9i}{12}\\
      0&0&0\\
      \cfrac{9i}{12}&0&\cfrac{9}{12}
    \end{bmatrix}\right)^T=\left(\begin{bmatrix}
      \cfrac{9}{12}&0&\cfrac{9i}{12}\\
      0&0&0\\
      \cfrac{-9i}{12}&0&\cfrac{9}{12}
    \end{bmatrix}\right)^T=\begin{bmatrix}
      \cfrac{9}{12}&0&\cfrac{-9i}{12}\\
      0&0&0\\
      \cfrac{9i}{12}&0&\cfrac{9}{12}
    \end{bmatrix}=\rho'_1.
  \end{align*}
\end{enumerate}
%
\subsection*{Problem 6}
We can use Taylor expansion to bring fown the matrix:
\begin{align}
  e^{A}=\sum_{n=0}^\infty\frac{A^n}{n!}=\mathds{1}+A+\frac{A^2}{2!}+\cdots+\frac{A^n}{n!}+\cdots
  \label{eq:taylorexpansionexponentialmatrix}
\end{align}

Then, the matrix $\sigma_x$ is elevated to an increasing power. It is then important to know how it behaves:
\begin{align*}
  \sigma_x=\begin{bmatrix}
    0&1\\1&0
  \end{bmatrix},\quad\sigma_x^2=\begin{bmatrix}
    1&0\\0&1
  \end{bmatrix}=I,\quad\sigma_x^2=\begin{bmatrix}
    0&1\\1&0
  \end{bmatrix}=\sigma_x,\quad \sigma_x^4=\begin{bmatrix}
    1&0\\0&1
  \end{bmatrix}=\sigma_x^2=I,\quad\cdots
\end{align*}
We conclude the following:
\begin{align*}
  \sigma^n_x=\begin{cases}
    \sigma_x,&\text{$n$ odd}\\
    I_{2\times2},&\text{$n$ even}
  \end{cases}.
\end{align*}
The other part of the term, $i\alpha$, is merely a constant we can take out of the matrix. Now, using the Taylor expression with the results above:
\begin{align}
  e^{i\alpha\sigma_x}&=I_{2\times2}+(i\alpha)\sigma_x+\frac{(i\alpha)^2}{2!}\sigma_x^2+\frac{(i\alpha)^3}{3!}\sigma_x^3+\frac{(i\alpha)^4}{4!}\sigma_x^4\notag\\
  &=I_{2\times2}+i\alpha\sigma_x-\frac{\alpha^2}{2!}I_{2\times2}-i\frac{\alpha^3}{3!}\sigma_x+\frac{\alpha^4}{4!}I_{2\times2}\notag\\
  &=I_{2\times2}\left[1-\frac{\alpha^2}{2!}+\frac{\alpha^4}{4!}-\cdots\right]+i\sigma_x\left[\alpha-\frac{\alpha^3}{3!}+\cdots\right]\notag\\
  e^{i\alpha\sigma_x}&\stackrel{(a)}{=}I_{2\times2}\cos\alpha+i\sigma_x\sin\alpha.
\end{align}
In $(a)$ we have used the very well-known series expansion of $\cos\alpha$ and $\sin\alpha$.
%
\subsection*{Problem 7}
The matrix to use is:
\begin{align*}
  \sigma_y=\begin{bmatrix}
    0&-i\\i&0
  \end{bmatrix}
\end{align*}
Taking the first fourth powers of $\sigma_y$:
\begin{align*}
\sigma_y=\begin{bmatrix}
  0&-i\\i&0
\end{bmatrix},\quad\sigma_y^2=\begin{bmatrix}
  1&0\\0&1
\end{bmatrix}=I,\quad\sigma_y^3=\begin{bmatrix}
  0&-i\\i&0
\end{bmatrix}=\sigma_y,\quad\sigma_y^4=\begin{bmatrix}
  1&0\\0&1
\end{bmatrix}=I
\end{align*}
Therefore,
\begin{align*}
  \sigma_y^n=\begin{cases}
    \sigma_y,&\text{$n$ odd}\\
    I_{2\times2},&\text{$n$ even}
  \end{cases}
\end{align*}
Performing the same expansion as before:
\begin{align}
  e^{i\alpha\sigma_y}&=I_{2\times2}+(i\alpha)\sigma_y+\frac{(i\alpha)^2}{2!}\sigma_y^2+\frac{(i\alpha)^3}{3!}\sigma_y^3+\frac{(i\alpha)^4}{4!}\sigma_y^4\notag\\
  &=I_{2\times2}+i\alpha\sigma_y-\frac{\alpha^2}{2!}I_{2\times2}-i\frac{\alpha^3}{3!}\sigma_y+\frac{\alpha^4}{4!}I_{2\times2}\notag\\
  &=I_{2\times2}\left[1-\frac{\alpha^2}{2!}+\frac{\alpha^4}{4!}-\cdots\right]+i\sigma_y\left[\alpha-\frac{\alpha^3}{3!}+\cdots\right]\notag\\
  e^{i\alpha\sigma_y}&=I_{2\times2}\cos\alpha+i\sigma_y\sin\alpha.
\end{align}

Now, we consider the general case where $\sigma_u=\lambda\sigma_x+\mu\sigma_y$:
\begin{align*}
  e^{i\alpha\sigma_u}=e^{i\alpha(\lambda\sigma_x+\mu\sigma_y)}=\sum_{n=1}^\infty\frac{(i\alpha)^n(\lambda\sigma_x+i\alpha\mu\sigma_y)^n}{n!}.
\end{align*}
We will verify if $\sigma_x$ and $\sigma_y$ commute in order to simplify the above expresion:
\begin{align*}
  [\sigma_x,\sigma_y]=\begin{bmatrix}
    0&1\\1&0
  \end{bmatrix}\begin{bmatrix}
    0&-i\\i&0
  \end{bmatrix}-\begin{bmatrix}
    0&-i\\i&0
  \end{bmatrix}\begin{bmatrix}
    0&1\\1&0
  \end{bmatrix}=\begin{bmatrix}
    i&0\\0&-i
  \end{bmatrix}-\begin{bmatrix}
    -i&0\\0&i
  \end{bmatrix}=\begin{bmatrix}
    2i&0\\0&2i
  \end{bmatrix}\neq\begin{bmatrix}
    0&0\\0&0
  \end{bmatrix}.
\end{align*}
They dont commute, because of a minus sign. However, we now know that $\sigma_x\sigma_y+\sigma_y\sigma_x=0$. We have to 
develop the $(\lambda\sigma_x+\mu\sigma_y)^n$ to derive something. First, we compute the first four terms:
{\small
\begin{align*}
    \sigma_u^1=(\lambda\sigma_x+\mu\sigma_y)=&\sigma_u\\
    \sigma_u^2=(\overbrace{\lambda^2+\mu^2}^{1})I_{2\times2}+\lambda\mu(\overbrace{\sigma_x\sigma_y+\sigma_y\sigma_x}^{0})=&I_{2\times2}\\
    \sigma_u^3=&\sigma_u^2\sigma_u=I_{2\times2}\sigma_u=\sigma_u\\
    \sigma_u^4=&\sigma_u^2\sigma_u^2=I_{2\times2}\\
    \vdots&
\end{align*}}
We have then,
\begin{align*}
  \sigma_u^n=\begin{cases}
    \sigma_u,&\text{$n$ odd}\\
    I_{2\times2},&\text{$n$ even}
  \end{cases}.
\end{align*}
Therefore,
\begin{align*}
  e^{i\alpha\sigma_u}&=I_{2\times2}+(i\alpha)\sigma_u+\frac{(i\alpha)^2}{2!}\sigma_u^2+\frac{(i\alpha)^3}{3!}\sigma_u^3+\frac{(i\alpha)^4}{4!}\sigma_u^4\\
  &=I_{2\times2}+i\alpha\sigma_u-\frac{\alpha^2}{2!}I_{2\times2}-i\frac{\alpha^3}{3!}\sigma_u+\frac{\alpha^4}{4!}I_{2\times2}\\
  &=I_{2\times2}\left[1-\frac{\alpha^2}{2!}+\frac{\alpha^4}{4!}-\cdots\right]+i\sigma_u\left[\alpha-\frac{\alpha^3}{3!}+\cdots\right]\\
  e^{i\alpha\sigma_u}&=I_{2\times2}\cos\alpha+i\sigma_u\sin\alpha.
\end{align*}
Obtaining a similar relation as before:
\begin{align}
  e^{i\alpha\sigma_u}=e^{i\alpha(\lambda\sigma_x+\mu\sigma_y)}=I_{2\times2}\cos\alpha+i\sigma_u\sin\alpha,\quad\sigma_u=\lambda\sigma_x+\mu\sigma_y.
  \label{eq:formula}
\end{align}

The others exponential required can used the formula we have just obtained. 
For $e^{2i\sigma_x}$ and $(e^{i\sigma_x})^2$ we have:
\begin{align*}
  e^{2i\sigma_x}=I_{2\times2}\cos2+i\sigma_x\sin2,\quad\text{versus}\quad
  \begin{array}{rl}
    (e^{i\sigma_x})^2&=(I_{2\times2}\cos1+i\sigma_x\sin1)(I_{2\times2}\cos1+i\sigma_x\sin1)\\
    &=[\cos^2(1)I_{2\times2}-\sin^2(1)\sigma^2_x]+i[2\cos(1)\sin(1)\sigma_x]\\
    &=[\cos^2(1)-\sin^2(1)]I_{2\times2}+i[2\cos(1)\sin(1)]\sigma_x\\
    (e^{i\sigma_x})^2&\stackrel{(a)}{=}[\cos2]I_{2\times2}+i[\sin2]\sigma_x.
  \end{array}
\end{align*}
where in $(a)$ we have used the following trigonometric identities:
\begin{align*}
  \cos2\theta=\cos^2\theta-\sin^2\theta,\quad\text{and}\quad\sin2\theta=2\cos\theta\sin\theta
\end{align*}
We conclude that:
\begin{align*}
  e^{2i\sigma_x}=(e^{i\sigma_x})^2=I_{2\times2}\cos2+i\sigma_x\sin2.
\end{align*}
This is expected, as is the same operator that is being computed.


On the other hand, the next test involve both $\sigma_x$ and $\sigma_y$ and because we know they 
dont commute, the terms $e^{i(\sigma_x+\sigma_y)}$ and $e^{i\sigma_x}e^{i\sigma_y}$ will be different.
First, lets recall that $\lambda^2+\mu^2=1$ and in this case $\lambda=\mu=1$. We need to normalize the $\sigma$ terms:
\begin{align*}
  \sigma'_u=\frac{\sigma_u}{\sqrt{2}}=\frac{1}{\sqrt{2}}\sigma_x+\frac{1}{\sqrt{2}}\sigma_y\Longrightarrow\lambda=\mu=\frac{1}{\sqrt{2}}.
\end{align*}

Then, the first exponential is ($\alpha=1,\lambda=\mu=1\sqrt{2}$):
\begin{align*}
  e^{i(\frac{1}{\sqrt{2}}\sigma_x+\frac{1}{\sqrt{2}}\sigma_y)}=I_{2\times2}\cos1+i\frac{\sigma_x+\sigma_y}{\sqrt{2}}\sin1,
\end{align*}
versus ($\alpha=1/\sqrt{2}$):
\begin{align*}
  \begin{array}{rl}
    e^{i\frac{1}{\sqrt{2}}\sigma_x}e^{i\frac{1}{\sqrt{2}}\sigma_y}&=[I_{2\times2}\cos(1/\sqrt{2})+i\sigma_x\sin(1/\sqrt{2})][I_{2\times2}\cos(1/\sqrt{2})+i\sigma_y\sin(1/\sqrt{2})]\\
    &=\left[\cos^2(1/\sqrt{2})I_{2\times2}-\sin^2(1/\sqrt{2})\sigma_x\sigma_y\right]+i\left[\cos(1/\sqrt{2})\sin(1/\sqrt{2})(\sigma_x+\sigma_y)\right].
  \end{array}
\end{align*}
We see that both are different.

%
\subsection*{Problem 9}
\begin{enumerate}[itemsep=0pt,topsep=0pt,label=\alph*.]
  \item For a Hamiltonian operator $H$ with $H\ket{\psi_n}=E_n\ket{\psi_n}$ and an arbitrary operator $A$ with $A\ket{\varphi_n}=a_n\ket{\varphi_n}$, 
  we have:
  \begin{align}
    \braket{\varphi_n|[A,H]|\varphi_n}&=\braket{\varphi_n|(AH-HA)|\varphi_n}\notag\\
    &=\braket{\varphi_n|AH|\varphi_n}-\braket{\varphi_n|HA|\varphi_n}\notag\\
    &=E_n\braket{\varphi_n|A|\varphi_n}-a_n\braket{\varphi_n|H|\varphi_n}\notag\\
    &=E_na_n\braket{\varphi_n|\varphi_n}-a_nE_n\braket{\varphi_n|\varphi_n}\notag\\
    &=\braket{\varphi_n|\varphi_n}(E_na_n-E_na_n)\notag\\
    \braket{\varphi_n|[A,H]|\varphi_n}&=0,\quad\forall A.\label{eq:commutatormeanvalue}
  \end{align} 
  \item 
  \begin{enumerate}[itemsep=0pt,topsep=0pt]
    \item[$\alpha$.]
    Given that $H$ is 
    \begin{align*}
      H=\frac{1}{2m}P^2+V(X),
    \end{align*}
    where $V(X)$ can be thought as a function of an operator and $P$, by comparison with the standard Hamiltonian, is $P=-i\hbar\partial/\partial x$.

    We will use algebra of commutator, function of an operator and assume that $X$ and $P$ are the position and momentum operators.
    \begin{itemize}[itemsep=0pt,topsep=0pt]
      \item The commutator between the Hamiltonian and momentum is:
      \begin{align*}
        [H,P]&=\left[\frac{P^2}{2m}+V(X),P\right]\\
        &=\left[\frac{P^2}{2m},P\right]+[V(X),P]\\
        &=\frac{P^3}{2m}-\frac{P^3}{2m}+V(X)P-PV(X)\\
        &=V(X)P-PV(X)\\
        [H,P]&=[V(X),P].
      \end{align*}
      The last equation can be further developed if we note that
      \begin{align*}
        [X,P]=i\hbar,\quad[X,i\hbar]=0,\quad[P,i\hbar]=0.
      \end{align*}
      Then, we apply the property $[F(A),B]=[A,B]F'(A)$:
      \begin{align}
        [H,P]=[X,P]\frac{\partial V(X)}{\partial x}=i\hbar\frac{\partial V(X)}{\partial X}.
        \label{eq:commutatorHP}
      \end{align}
      \item The commutator of the Hamiltonian and the position is
      \begin{align*}
        [H,X]&=\left[\frac{P^2}{2m}+V(X),X\right]\\
        &=\left[\frac{P^2}{2m},X\right]+[V(X),X]\\
        [H,X]&\stackrel{(a)}{=}\left[\frac{P^2}{2m},X\right]+0\\
      \end{align*}
      In $(a)$, any function of $X$ commute with itself. We can demonstrate thi by expressing $V(X)$ as a Taylor expansion and computing the commutator with $X$:
      \begin{align*}
        V(X)=\sum_{n=0}^\infty a_nX^n\Longrightarrow\left[\sum_{n=0}^\infty a_nX^n,X\right]=\sum_{n=0}^\infty a_n[X^n,X],
      \end{align*}
      but $[X^n,X]$ is:
      \begin{align*}
        [X^n,X]=X^nX-XX^n=X^{n+1}-X^{n+1}=0\Longrightarrow[V(X),X]=0.
      \end{align*}

      Continuing the problem,
      \begin{align}
        [H,X]&=\frac{1}{2m}[P^2X-XP^2]\notag\\
        &=\frac{1}{2m}[P(PX)-(XP)P]\notag\\
        &\stackrel{(b)}{=}\frac{1}{2m}[P(XP-i\hbar)-(i\hbar+PX)P]\qquad\left([X,P]=i\hbar\right)\notag\\
        &=\frac{1}{2m}[PXP-i\hbar P-i\hbar P-PXP]\notag\\
        &=\frac{1}{2m}[-i2\hbar P]\notag\\
        [H,X]&=-\frac{i\hbar}{m}P.\label{eq:commutatorHX}
      \end{align}
      Where in $(b)$ we have used the canonical commutator relation between position and momentum operators.
      \item The commutator of the Haimltonian and the product of the position and momentum is
      \begin{align}
        [H,XP]&=[H,X]P+X[H,P]\notag\\
        &=\left(-\frac{i\hbar}{m}P\right)P+X\left(i\hbar\frac{\partial V(X)}{\partial x}\right)\notag\\
        [H,XP]&=-\frac{i\hbar P^2}{m}+i\hbar X\frac{\partial V(X)}{\partial X}.\label{eq:commutatorHXP}
      \end{align}  
      We have used the result of previous commutators in the process.
    \end{itemize}
    \item[$\beta$.]  To prove that $\braket{\varphi_n|P|\varphi_n}=0$, we use equation \eqref{eq:commutatormeanvalue} and \eqref{eq:commutatorHX}:
    \begin{align*}
      \braket{\varphi_n|[A,H]|\varphi_n}=0,\quad\text{and}\quad[H,X]=-\frac{i\hbar}{m}P.
    \end{align*}
    if we solve for $P$ in the second equation and perform $\braket{\varphi_n|\cdot|\varphi_n}$ over $P$ we have:
    \begin{align*}
      P&=-\frac{m}{i\hbar}[H,X]\bigr/\braket{\varphi_n|\cdot|\varphi_n}\\
      &=-\frac{m}{i\hbar}\braket{\varphi_n|[H,X]|\varphi_n}\\
      \braket{\varphi_n|P|\varphi_n}&=\frac{m}{i\hbar}\braket{\varphi_n|[X,H]|\varphi_n}.
    \end{align*}
    Since we have derivated \eqref{eq:commutatormeanvalue} for an arbitrary operator, we can do $A=X$ and set directly zero:
    \begin{align}
      \braket{\varphi_n|P|\varphi_n}=0.
    \end{align}
    \item[$\gamma$.] To establish a relation between $\braket{\varphi_n|P^2/2m|\varphi_n}$ and $\braket{\varphi_n|XdV/dX|\varphi_n}$, we will use equation 
    \eqref{eq:commutatormeanvalue}, \eqref{eq:commutatorHX}, \eqref{eq:commutatorHP}, and \eqref{eq:commutatorHXP}.

    We employ equation \eqref{eq:commutatormeanvalue}
    \begin{align*}
      \braket{\varphi_n|[H,A]|\varphi_n}=0,\quad\forall A
    \end{align*}
    to define $A=XP$ and use equation \eqref{eq:commutatorHXP}, which we know is
    \begin{align*}
      [H,XP]=-\frac{i\hbar P^2}{m}+i\hbar X\frac{dV(X)}{dX}.
    \end{align*}
    Then, 
    \begin{align*}
      \braket{\varphi_n|-\frac{i\hbar P^2}{m}+i\hbar X\frac{dV(X)}{dX}|\varphi_n}&=0\\
      -\frac{i\hbar}{m}\braket{\varphi_n|P^2|\varphi_n}+i\hbar\braket{\varphi_n|X\frac{dV(X)}{dX}|\varphi_n}&=\\
      \braket{\varphi_n|X\frac{dV(X)}{dX}|\varphi_n}&=2\braket{\varphi_n|\frac{P^2}{2m}|\varphi_n}=2E_n.
    \end{align*}
    Consequently, both quantities are related by a constant term of $2$.

    For the other relation, we assume that $V_o$ is a constant value and therefore can commute.
    
    


    Establish relation between the mean value of kinetic energy $E_k=\braket{\varphi_n|\dfrac{P^2}{2m}|\varphi_n}$ and $\braket{\varphi_n|X\dfrac{dV}{dX}|\varphi_n}$.
    
    How is related the meand value of the potential energy $\braket{\varphi_n|V(X)|\varphi_n}$ to the mean value of the kinetic energy when $V(X)=V_oX^s,\quad s=2,4,6,\cdots\;V_0>0$ 



  \end{enumerate}
\end{enumerate}

%%
\section*{Part II}
%
\subsection*{II-1}
  The ket is already defined, we need to find the constant $c$ to make it orthonormal: $\braket{\psi|\psi}=1$.
  This is achieves by computing its scalar product and equating it to one. First, we interpret the coefficients as the projections of $\psi$ onto the 
  $\{\ket{u_n}\}$ basis and use it to construct a column matrix:
  \begin{align*}
    \ket{\psi}_{\{\ket{u_n}\}}=\begin{bmatrix}
      \braket{u_1|\psi}\\\braket{u_2|\psi}\\\braket{u_3|\psi}\\\braket{u_4|\psi}
    \end{bmatrix}=c\begin{bmatrix}
      2\\-i\sqrt{3}\\-3e^{i\theta}\\3
    \end{bmatrix}.
  \end{align*}
  We now compute the scalar product as a matrix multiplication:
  \begin{align*}
    \braket{\psi|\psi}=c^2\begin{bmatrix}
      2&i\sqrt{3}&-3e^{-i\theta}&3
    \end{bmatrix}\cdot\begin{bmatrix}
      2\\-i\sqrt{3}\\-3e^{i\theta}\\3
    \end{bmatrix}=c^2(4+3+9+9)=25c^2=1\longrightarrow c=1/5.
  \end{align*}
  Therefore, the ket in matrix form is:
  \begin{align}
    \ket{\psi}_{\{\ket{u_n}\}}=\frac{1}{5}\begin{bmatrix}
      2\\-i\sqrt{3}\\-3e^{i\theta}\\3
    \end{bmatrix}.
  \end{align}
  %
  \subsection*{II-2}
  \begin{enumerate}[itemsep=0pt,topsep=0pt,label=(\alph*)]
    \item For $\ket{u_2}$, we arrange the representation as a column vector:
    \begin{align*}
      \ket{u_2}=\begin{bmatrix}
          0\\1\\0\\0
      \end{bmatrix}
    \end{align*}
    \item For $\bra{u_3}$ we arrange the representation os a row vector:
    \begin{align*}
      \bra{u_3}=\begin{bmatrix}
          0&0&1&0
      \end{bmatrix}
    \end{align*}
    \item The term $\ket{u_2}\bra{u_3}$ is an operator that project the input vector $\ket{\psi}$ onto $\ket{u_3}$ and then assign it to $\ket{u_2}$. I dont really know
    how meaningful is this operation; it seems to be like a cross product. Well, the representation will be the outer product (matrix multiplication) of both elements, which 
    will yields a matrix:
    \begin{align*}
      \ket{u_2}\bra{u_3}=\begin{bmatrix}
          0\\1\\0\\0
      \end{bmatrix}\cdot\begin{bmatrix}
          0&0&1&0
      \end{bmatrix}=\begin{bmatrix}
        0&0&0&0\\
        0&0&1&0\\
        0&0&0&0\\
        0&0&0&0
      \end{bmatrix}.
    \end{align*} 
    This means that given it will only gives you non-zero vectors for collinear vectors of $\ket{u_3}$.
    \item The projector onto $\ket{u_2}$ can be described as $P_{u_2}=\ket{u_2}{\bra{u_2}}$. The matrix representation is then the product of 
    its column vector times the adjoint of the column vector (row vector of complex conjugates elements):
    \begin{align*}
      P_{u_2}=\ket{u_2}\bra{u_2}=\begin{bmatrix}
        0\\1\\0\\0
      \end{bmatrix}\cdot\begin{bmatrix}
        0&1&0&0
      \end{bmatrix}=\begin{bmatrix}
        0&0&0&0\\
        0&1&0&0\\
        0&0&0&0\\
        0&0&0&0
      \end{bmatrix}.
    \end{align*}
    \item This expression of projects the vector $\ket{\psi}$ onto $\ket{u_n}$ and assign it to $\ket{u_m}$. the summ on $i$ project the input to the nth-element of the basis
    $\{\ket{u_1},\ket{u_2},\ket{u_3},\ket{u_4}\}$. Then, the other summ assign it to the mth-element of the same basis.
    With a little of algebra, we can express its matrix representation:
    \begin{align*}
      \sum_{m=1}^{m=4}\sum_{n=1}^{m=4}\ket{u_m}\bra{u_n}=\left[\sum_{m=1}^{m=4}\ket{u_m}\right]\cdot\left[\sum_{n=1}^{m=4}\bra{u_n}\right]
    \end{align*}
    Each state $\ket{u_m}$ provides a one non-zero element at the mth-position of the column vector. However, summing them all produces a vector full of ones.
    The same applies to the bra $\bra{u_n}$ and. Consequently, the multiplication is:
    \begin{align*}
      \left[\sum_{m=1}^{m=4}\ket{u_m}\right]\cdot\left[\sum_{n=1}^{m=4}\bra{u_n}\right]=\begin{bmatrix}
        1\\1\\1\\1
      \end{bmatrix}\cdot\begin{bmatrix}
        1&1&1&1
      \end{bmatrix}=\begin{bmatrix}
        1&1&1&1\\
        1&1&1&1\\
        1&1&1&1\\
        1&1&1&1
      \end{bmatrix}.
    \end{align*}
  \end{enumerate}
%
\subsection*{II-3}
We know the action of the operator $Q$ on each ket of the basis. The result is a different element within the same basis. The basis is orthonormal, meaning that 
they areall linearly independent each other, which means that every row of the $Q$ representation in $\{\ket{u_n}\}$ will only have one non-zero element, at a different 
location that all the others rows.

The problems then reduces to find out the elements of the ith-row with the mth-row of the output vector. Doing this with the four equations and constructing from them 
the operator, we have
\begin{align*}
  Q=\begin{bmatrix}
    0&0&0&-i\\
    0&0&2&0\\
    0&2&0&0\\
    i&0&0&0
  \end{bmatrix},
\end{align*}
which is a Hermitian operator: $Q=Q^\dagger$.


%\nocite{*}
%\bibliographystyle{plain}   % or unsrt, alpha, apalike, etc.
%\bibliography{refs}

\end{document}
