\documentclass[letterpaper,11pt,twoside]{article}
\usepackage{graphicx} % Required for inserting images
\usepackage[table,xcdraw,dvipsnames]{xcolor}
\usepackage{amsmath,amsfonts,amssymb,amsthm}
\usepackage{listings}
\usepackage{lipsum}
\usepackage{hyperref}
\usepackage{enumitem}

\usepackage{tikz}
\usepackage[siunitx, RPvoltages]{circuitikz}
\usetikzlibrary{3d}
\usepackage{comment}
\usepackage{caption,subcaption}
\usepackage{pgfplots}
\pgfplotsset{compat=newest} % or a newer version if available
\usepgfplotslibrary{groupplots}
\usetikzlibrary{pgfplots.groupplots}
\usetikzlibrary{shapes.geometric, arrows}
\tikzstyle{arrow} = [->,>=stealth,shorten >=2pt]
\newcommand{\ket}[1]{|#1\rangle}
\newcommand{\bra}[1]{\langle#1|}
\newcommand{\br}{\bm{r}}
\newcommand{\bR}{\bm{R}}
\newcommand{\bp}{\bm{p}}
\newcommand{\bP}{\bm{P}}
\newcommand{\braket}[1]{\langle#1\rangle}
\newcommand{\F}{\mathscr{F}}
\newcommand{\E}{\mathscr{E}}
\usepackage{dsfont}
\usepackage{cancel}
\usepackage{bm}
\usepackage{fancyhdr}
\usepackage[utf8x]{inputenc}
\usepackage[T1]{fontenc}
\usepackage[margin=0.8in,top=1in,bottom=1in]{geometry}
%%%%%
\begin{filecontents*}{refs.bib}
@book{bornwolf,
  author    = {Born, M. and Wolf, E.},
  title     = {Principles of Optics},
  publisher = {Pergamon Press},
  edition   = {7},
  year      = {1999}
}
@book{hecht,
  author    = {Hecht, E.},
  title     = {Optics},
  publisher = {Addison-Wesley},
  edition   = {5},
  year      = {2016}
}
\end{filecontents*}
%
\newcommand{\institution}{University of Arizona}
\newcommand{\autor}{Nicolás Hernández Alegría}
\newcommand{\course}{OPTI 570 Quantum Mechanics}
\newcommand{\assignment}{Assignment 4}
%
\title{\textbf{\assignment}\\\course\\{\Large\institution}}
\author{\autor}
\date{\today\\Total time: - hours}
%
\renewcommand{\sectionmark}[1]{\markright{#1}}
\fancypagestyle{mainstyle}{
    \fancyhf{} % Clear all header and footer fields
    \fancyfoot[C]{\thepage}
    \fancyhead[LE,RO]{\course} % Section name on odd pages
    \fancyhead[LO,RE]{\assignment}
    % Optional: Thin rules
    \renewcommand{\headrulewidth}{0pt} % Header rule
    \renewcommand{\footrulewidth}{0pt} % No footer rule
}
%
\begin{document}

\pagestyle{mainstyle}
\maketitle
%%

\section*{Problem I}
\begin{enumerate}[itemsep=0pt,topsep=0pt,label=\alph*)]
  \item The sixth postulates of Quantum mechanics states the time evolutions of the state $|ket{\psi}$: 
  \begin{align}
    i\hbar\frac{d}{dt}\ket{\psi(t)}=H(t)\ket{\psi(t)}.
    \label{eq:schrodinger}
  \end{align}
  If we multiplty equation \eqref{eq:schrodinger} by $\bra{\phi(t)}$ and use the product rule of differentiation we get 
  \begin{align*}
    \frac{d}{dt}\braket{\psi(t)|\phi(t)}&=\left[\frac{d}{dt}\bra{\psi(t)}\right]\ket{\phi(t)}+\bra{\psi(t)}\left[\frac{d}{dt}\ket{\phi(t)}\right].
  \end{align*}
  On the one hand, from Schrodinger equation we have 
  \begin{align*}
    \frac{d}{dt}\ket{\phi(t)}=\frac{1}{i\hbar}H(t)\ket{\phi(t)}.
  \end{align*}
  On the other hand, if we compute the adjoint of the Schrodinger equation for $\ket{\psi(t)}$:
  \begin{align*}
    \frac{d}{dt}\bra{\psi(t)}=-\frac{1}{i\hbar}\bra{\psi(t)}H(t).\qquad(H^\dagger(t)=H(t))
  \end{align*}
  Replacing both results into the product rule yields:
  \begin{align}
    \frac{d}{dt}\braket{\psi(t)|\phi(t)}&=\left[-\frac{1}{i\hbar}\bra{\psi(t)}H(t)\right]\ket{\phi(t)}+\bra{\psi(t)}\left[\frac{1}{i\hbar}H(t)\ket{\phi(t)}\right]\notag\\
    &=-\frac{1}{i\hbar}\braket{\psi(t)|H(t)|\phi(t)}+\frac{1}{i\hbar}\braket{\psi(t)|H(t)|\phi(t)}\label{eq:interproblem1}\\
    \frac{d}{dt}\braket{\psi(t)|\phi(t)}&=0.\notag
  \end{align}
  %%
  The evolution of a state $\ket{\psi(t_0)}$ to $\ket{\psi(t)}$ follows a linear fashion. Therefore, the transformation can be represented by a linear evolution operator 
  $U(t,t_0)$ so that 
  \begin{align}
    \ket{\psi(t)}=U(t,t_0)\ket{\psi(t_0)}=U(t,t_1)U(t_1,t_0)\ket{\psi(t_0)}.
  \end{align}
  The time erivative is zero means that the argument must be a constant:
  \begin{align*}
    \frac{d}{dt}\braket{\varphi(t)|\phi(t)}=0\Longrightarrow \braket{\varphi(t)|\phi(t)}=\braket{\varphi(t_0)|\phi(t_0)}=\text{cte}.
  \end{align*}
  We put the definition of the evolution operator
  \begin{align*}
    \braket{\varphi(t_0)U^\dagger(t,t_0)U(t,t_0)|\phi(t_0)}=\braket{\varphi(t_0)|\phi(t_0)}
  \end{align*}
  By comparison, we have that 
  \begin{align*}
    U^\dagger(t,t_0)U(t,t_0)=\mathds{1},
  \end{align*}
  which is the definition of an unitary operator.
  %%
  \item If $\ket{\psi(t_0)}=\ket{\phi_n^i}$, then the state is fully characterized by a single eigenvector of the Hamiltonian. This corresponds to its expansion
  \begin{align*}
    \ket{\psi(t)}=c_n(t)\ket{\phi_n^i},\quad\text{with}\quad c_n(t)=\braket{\phi_n^i|\psi(t)}=1.
  \end{align*} 
    We use the coeffficient in the Schrodinger equation:
    \begin{align*}
      i\hbar\frac{d}{dt}\braket{\phi_n^i|\psi(t)}=\braket{\phi_n^i|H|\psi(t)}=E_n\braket{\phi_n^i|\psi(t)}.
    \end{align*}
    The first and last equation, construct a first-order differential equation in the coefficient $c_n(t)$:
    \begin{align*}
      i\hbar\frac{d}{dt}c_n(t)=E_nc_n(t)\longrightarrow c_n(t)=c_n(t_0)e^{-E_n(t-t_0)/\hbar},
    \end{align*}
    where $c_n(t_0)$ is the value for $\psi(t_0)$, in this case, $\psi(t_0)=\ket{psi_n^i}$ and $c_n(t_0)=1$. Therefore,
    \begin{align*}
      \ket{\psi(t)}=1e^{-iE_n(t-t_0)/\hbar}\ket{\psi(t_0)}.
    \end{align*}
  \item In the general case, where the state vector has more than one coefficient of projection, we can do the same with the inclusion of the summation for each coefficient, including 
  also the degree of degeneracy. We can think of the previous result as just one of the several terms, but all share the same structure.
  We give a small derivation,
  \begin{align*}
    \sum_{n,i}i\hbar\frac{d}{dt}c^i_n(t)=\sum_{n,i}E_nc^i_n(t)\longrightarrow\sum_{n,i}c_{n,i}(t)=\sum_{n,i}c_{n,i}(t_0)e^{-E_n(t-t_0)/\hbar},
  \end{align*} 
  The ecuations above are for each eigenvector, which then construct the state vector:
  \begin{align*}
    \ket{\psi(t)}=\sum_{n,i}c_n^i(t_0)e^{-iE_n(t-t_0)/\hbar}\ket{\phi_n^i}.
  \end{align*} 
  To be a stationary state, there only have to be a global factor common to all the eigenvectors. However, if the state vector is represented by several eigenvectors,
  each one will have its own phase, that represent the relative phase factor. How behaves this phase factor depends on the argument of the esponential term:
  $-iE_n(t-t_0)/\hbar$. The only variable we can study is $E_n$ that corresponds to the eigenvalue. In order to remain the phase equal for all the eigenvectors, 
  we need that $E_n=\text{cte}$, that is, we live in a single eigensubspace of the Hamiltonian, which may have several eigenvectors but all share the same eigenvalue and therefore 
  the same phase factor. 
\end{enumerate}
%%
\section*{Problem II}
\begin{enumerate}[itemsep=0pt,topsep=0pt,label=\alph*.]
  \item We need to normalized the wavefunction in the range $x\in(-\infty,\infty)$ so that its norm is unitary.
  \begin{align*}
    \int_{-\infty}^\infty dx\;|\psi(x)|^2&=\int_{-\infty}^\infty dx\;\left|N\frac{e^{ip_0x/\hbar}}{\sqrt{x^2+a^2}}\right|^2=N^2\int_{-\infty}^\infty dx\;\frac{1}{x^2+a^2}=N^2\frac{\pi}{a}=1\longrightarrow N=\sqrt{\frac{a}{\pi}}.
  \end{align*}
  The integral was computed with the change of variable $x=a\tan\theta$. The wave function is then
  \begin{align*}
    \psi(x)=\sqrt{\frac{a}{\pi}}\frac{e^{ip_0x/\hbar}}{\sqrt{x^2+a^2}}.
  \end{align*}
  \item To find the probability in the range given, we integrate $|\psi(x)|^2$ in the interval. We also notice that the integrand will be even, so we only integrate one part and multiply it by two:
  \begin{align*}
    \sqrt{\frac{a}{\pi}}\int_{-a/\sqrt{3}}^{a/\sqrt{3}}\frac{1}{x^2+a^2}=\sqrt{\frac{2a}{\pi}}\int_{0}^{a/\sqrt{3}}\frac{1}{x^2+a^2}=\frac{\sqrt{\pi}}{3\sqrt{a}}.
  \end{align*}
  \item To get the mean value of the momentum, we will compute the following:
  \begin{align*}
    \braket{P}_\psi=\braket{\psi|P|\psi}=\braket{\psi|\mathds{1}P|\psi}=\int dx\;\braket{\psi|x}\braket{x|P|\psi}=\int dx\;\psi^*(x)[-i\hbar\partial_x\psi(x)].
  \end{align*}
  The time derivative is:
  \begin{align*}
    \partial_x\psi(x)=\sqrt{\frac{a}{\pi}}\left[\frac{ip_o}{\hbar}(x^2+a^2)^{-1/2}-x(x^2+a^2)^{-3/2}\right]e^{ip_0x/\hbar}.
  \end{align*}
  The multiplication $\psi^(x)\cdot\partial_x\psi(x)$ is:
  \begin{align*}
    \psi^*(x)\cdot\partial_x\psi(x)&=\sqrt{\frac{a}{\pi}}(x^2+a^2)^{-1/2}e^{-ip_0x/\hbar}\cdot\sqrt{\frac{a}{\pi}}\left[\frac{ip_o}{\hbar}(x^2+a^2)^{-1/2}-x(x^2+a^2)^{-3/2}\right]e^{ip_0x/\hbar}\\
    &=\frac{a}{\pi}(x^2+a^2)^{-1/2}\left[\frac{ip_o}{\hbar}(x^2+a^2)^{-1/2}-x(x^2+a^2)^{-3/2}\right]\\
    \psi^*(x)\cdot\partial_x\psi(x)&=\frac{a}{\pi}\left[\frac{ip_o}{\hbar}(x^2+a^2)^{-1/2}-x(x^2+a^2)^{-2}\right].
  \end{align*}
  We proceed to integrate it:
  \begin{align*}
    \braket{P}_\psi&=\frac{-i\hbar a}{\pi}\left[\frac{ip_o}{\hbar}\int_{-\infty}^\infty dx\;(x^2+a^2)^{-1/2}-\int_{-\infty}^\infty dx\;x(x^2+a^2)^{-2}\right]\\
    &=\frac{-i\hbar a}{\pi}\left[\frac{ip_o}{\hbar}\frac{\pi}{|a|}-0\right]\\
    &=\frac{-i\hbar a}{\pi}\frac{ip_o}{\hbar}\frac{\pi}{|a|}\\
    \braket{P}_\psi&=p_o.
  \end{align*}
  
\end{enumerate}
%%
\section*{Problem III}
\begin{enumerate}[itemsep=0pt,topsep=0pt,label=\alph*.]
  \item In this case, the wave function is expanded in a discrete set $\{\psi_n\}$ of length four. Each of these eigenvectors has a $E_n$
  asociated. We are asked of the probability of getting an energy less than $E_{\approx2.4}$, which is the same than asking for a probability 
  less or equal than $E_2$.
  \begin{align*}
    P(E\leq E_2)=\frac{1}{\braket{\psi|\psi}}\sum_{n=1}^2\braket{\psi|P_n|\psi}=\frac{1}{\braket{\psi|\psi}}\left[\braket{\psi|P_1|\psi}+\braket{\psi|P_2|\psi}\right]=\frac{|a_1|^2+|a_2|^2}{|a_1|^2+|a_2|^2+|a_3|^2+|a_4|^2}.
  \end{align*}
  \item On the one hand, the mean value of the enegy is:
  \begin{align*}
    \braket{E}_\psi=\frac{\braket{\psi|H|\psi}}{\braket{\psi|\psi}}=\frac{\sum_n\braket{\psi|\varphi_n}\braket{\varphi_n|H|\psi}}{\braket{\psi|\psi}}=\frac{\sum_n|a_n|^2E_n}{\braket{\psi|\psi}}=\frac{|a_1|^2E_1+|a_2|^2E_2+|a_3|^2E_3+|a_4|^4E_4}{|a_1|^2+|a_2|^2+|a_3|^2+|a_4|^2}.
  \end{align*}
  On the other hand, the RMS deviation of the energy is:
  \begin{align*}
    \Delta E=\sqrt{\braket{E^2}-\braket{E}^2}.
  \end{align*}
  The term $\braket{E^2}_\psi$ must be recalculated following the previous procedure:
  \begin{align*}
    \braket{E^2}_\psi=\frac{\braket{\psi|H^2|\psi}}{\braket{\psi|\psi}}=\frac{\sum_n|a_n|^2E^2_n}{\braket{\psi|\psi}}=\frac{|a_1|^2E^2_1+|a_2|^2E^2_2+|a_3|^2E^2_3+|a_4|^4E^2_4}{|a_1|^2+|a_2|^2+|a_3|^2+|a_4|^2}.
  \end{align*}
  Therefore, we have using the summation notation:
  \begin{align*}
    \Delta E=\sqrt{\frac{\sum_n|a_n|^2E^2_n}{\braket{\psi|\psi}}-\left(\frac{\sum_n|a_n|^2E_n}{\braket{\psi|\psi}}\right)^2}.
  \end{align*}
  \item The Hamiltonian of this system does not depend on time so that the system in conservative. We already know the expansion of the state vector at $t=0$. Therefore, we only need to ass the phase temporal 
  factor to each of them to construct the state vector at time $t$:
  \begin{align*}
    \ket{\psi(t)}=\sum_{n=1}^4a_ne^{-iE_n(t-t_0)/\hbar}\ket{\varphi_n}.
  \end{align*}
  Because the system is projected in more than one eigenvector of the Hamiltonian, it is not a stationary state. This implies that the the state vector 
  will be different over time due to the relative phase factors, and therefore we expect to $\braket{E}$ and $\Delta E$ to change.
  \item When the measurement is performed, the results corresponds to the eigenvalue $E_4$ ($n=4$). This means that the eigenvector obtained in the state of the system
  is $\ket{\varphi_4}$, with the corresponding normalization factor:
  \begin{align*}
    \ket{\psi(t)}\stackrel{(E_4)}{\Longrightarrow}\frac{P_4\ket{\varphi}}{\sqrt{\braket{\psi|P_4|\psi}}}=\frac{a_4}{|a_4|}e^{-iE_4(t-t_0)/\hbar}\ket{\varphi_4}.
  \end{align*}
  Now, we have the state $\ket{\varphi_4}$ just after the measurement. However, because the Hamiltonian is time-independent, it will only evolve a global phase factor that will 
  not affect the physical meanining of the state. In consequence, after the measurement we will always get $\ket{\varphi_4}$.
\end{enumerate}
%%
\section*{Problem IV}
We have the followings eigenpairs for each observator;
\begin{align*}
  H:\qquad&\{(\hbar\omega_0,\ket{u_1}),(2\hbar\omega_0,\ket{u_2}),(2\hbar\omega_0,\ket{u_3})\}.\\
  A:\qquad&\{(a,\ket{u_1}),(a,\frac{\ket{u_2}+\ket{u_3}}{\sqrt{2}}),(-a,\frac{\ket{u_2}-\ket{u_3}}{\sqrt{2}})\}.\\
  B:\qquad&\{(b,\frac{\ket{u_1}+\ket{u_2}}{\sqrt{2}}),(-b,\frac{\ket{u_1}-\ket{u_2}}{\sqrt{2}}),(b,\ket{u_3})\}.
\end{align*}
We give a small derivation of how the eigenvector of $A$ were obtained. The first case is trivial, but for the second row we have:
\begin{align*}
  A\ket{u_2}=a\ket{u_3}.
\end{align*}
We can see that it doesnt map onto $\ket{u_2}$, but on other vector. We then think of the eigenvector as a combination of them:
\begin{align*}
  \ket{v}=\alpha\ket{u_2}+\beta\ket{u_3}.
\end{align*}
Then,
\begin{align*}
  A\ket{v}=\lambda\ket{v}\longrightarrow a(\beta\ket{u_2}+\alpha\ket{u_3})=\lambda(\alpha\ket{u_2}+\beta\ket{u_3}).
\end{align*}
Equation vectors we find:
\begin{align*}
  a\beta=\lambda\alpha\land a\alpha=\lambda\beta\Longrightarrow\lambda^2=\alpha^2.
\end{align*}
For $\lambda=+a$, $\beta=\alpha$ and $\ket{v_+}\propto\ket{u_2}+\ket{u_3}$. For $\lambda=-a$, $\beta=-\alpha$ and $\ket{v_-}\propto\ket{u_2}-\ket{u_3}$. After normalize each 
results we have the eigenket we have listed.
\begin{enumerate}[itemsep=0pt,topsep=0pt,label=\alph*.]
  \item The reuslts can only be eigenvalues of the Hamiltonian by the third postulate. Each one has a given probability, therefore:
  \begin{align*}
    &\hbar\omega_0,\quad\text{with}\quad P(\hbar\omega_0)=\braket{\psi|P_{\hbar\omega_0}|\psi}=\frac{1}{2}\\
    &2\hbar\omega_0,\quad\text{with}\quad P(2\hbar\omega_0)=\braket{\psi|P_{2\hbar\omega_0}|\psi}=\frac{1}{4}+\frac{1}{4}=\frac{1}{2}.
  \end{align*}
  The mean value is:
  \begin{align*}
    \braket{H}_\psi=\braket{\psi|H|\psi}=\sum_{n=1}^2\sum_{i=1}^{g_n}\braket{\psi|H|u_n^i}\braket{u_n^i|\psi}=\sum_{n=1}^2E_nP(E_n)=(\hbar\omega_0)\frac{1}{2}+(2\hbar\omega_0)\frac{1}{2}=\frac{3\hbar\omega_0}{2}.
  \end{align*}
  On the other hand, the term $\braket{H^2}_\psi$ is:
  \begin{align*}
    \braket{H^2}_\psi=\sum_{n=1}^2E_n^2P(E_n)=(\hbar\omega_0)^2\frac{1}{2}+(2\hbar\omega_0)^2\frac{1}{2}=\frac{5\hbar^2\omega_0^2}{2}.
  \end{align*}
  Then,
  \begin{align*}
    \Delta H=\sqrt{\braket{H^2}-\braket{H}^2}=\sqrt{\frac{5\hbar^2\omega_0^2}{2}-\left(\frac{3\hbar\omega_0}{2}\right)^2}=\frac{\hbar\omega_0}{2}.
  \end{align*}
  \item sagasg
  \item asgasg
  \item sgsagag
  \item asgasg
\end{enumerate}
%%
\section*{Problem V}
\begin{enumerate}[itemsep=0pt,topsep=0pt,label=\alph*.]
  \item assagag
  \item sagasg
  \item asgasg
\end{enumerate}



\section*{Optional}
\subsection*{Part 2.}
\begin{enumerate}[itemsep=0pt,topsep=0pt,label=\alph*)]
  \item We can show it by using properties of Hermitian operation:
  \begin{align*}
      G^\dagger=(iF)^\dagger=-iF^\dagger=-iF=-G\Longrightarrow G^\dagger=-G.
  \end{align*}
  \item It can be prooved by looking the eigenvalues of the operator:
  \begin{align*}
    G\ket{\psi}=g\ket{\psi}\qquad\text{and}\quad(G\ket{\psi})^\dagger&=(g\ket{\psi})^\dagger\\
    \bra{\psi}G^\dagger&=g^*\bra{\psi}\\
    -\bra{\psi}G&=g^*\bra{\psi}\\
    \bra{\psi}G&=-g^*\bra{\psi}.
  \end{align*}
  Therefore, if under an adjoint operator the eigenvalue changes from $g$ to $-g^*$, it means that it is a pure imaginary term. If it would have consisted of also a real part,
  it will not be possible to extract the minus sign from $g$. The same can be done for the expectation value:
  \begin{align*}
    \braket{\psi|G|\psi}^\dagger=\braket{\psi|G^\dagger|\psi}=-\braket{\psi|G|\psi}.
  \end{align*}
  The minus sign reveals the same nature on this number.
  \item The commutator already has a property for the adjoint, we will use it and also the inversion of the arguments with the minus sign included:
  \begin{align*}
    ([A,B])^\dagger=[B^\dagger,A^\dagger]=-[A^\dagger,B^\dagger]=-([A,B])^\dagger.
  \end{align*}
  The other follows the same idea:
  \begin{align*}
    \braket{\psi|[A,B]|\psi}^\dagger=\braket{\psi|[A,B]^\dagger|\psi}=-\braket{\psi|[A,B]|\psi}.
  \end{align*}
  We have used the previous results of the commutator directly.
\end{enumerate}

%\nocite{*}
%\bibliographystyle{plain}   % or unsrt, alpha, apalike, etc.
%\bibliography{refs}

\end{document}
