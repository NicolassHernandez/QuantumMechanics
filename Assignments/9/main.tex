\documentclass[letterpaper,11pt,twoside]{article}
\usepackage{graphicx} % Required for inserting images
\usepackage[table,xcdraw,dvipsnames]{xcolor}
\usepackage{amsmath,amsfonts,amssymb,amsthm}
\usepackage{listings}
\usepackage{lipsum}
\usepackage{hyperref}
\usepackage{mathrsfs}
\usepackage{tikz}
\usepackage{mathtools}
\usetikzlibrary{arrows.meta}
\usepackage{enumitem}

\usepackage{tikz}
\usepackage[siunitx, RPvoltages]{circuitikz}
\usetikzlibrary{3d}
\usepackage{comment}
\usepackage{caption,subcaption}
\usepackage{pgfplots}
\pgfplotsset{compat=newest} % or a newer version if available
\usepgfplotslibrary{groupplots}
\usetikzlibrary{pgfplots.groupplots}
\usetikzlibrary{shapes.geometric, arrows}
\tikzstyle{arrow} = [->,>=stealth,shorten >=2pt]
\newcommand{\ket}[1]{|#1\rangle}
\newcommand{\bra}[1]{\langle#1|}
\newcommand{\br}{\bm{r}}
\newcommand{\bR}{\bm{R}}
\newcommand{\bp}{\bm{p}}
\newcommand{\bP}{\bm{P}}
\newcommand{\braket}[1]{\langle#1\rangle}
\newcommand{\F}{\mathscr{F}}
\newcommand{\E}{\mathscr{E}}
\newcommand{\re}[1]{\text{Re}\left(#1\right)}
\newcommand{\im}[1]{\text{Im}\left(#1\right)}
\usepackage{dsfont}
\usepackage{cancel}
\usepackage{bm}
\usepackage{fancyhdr}
\usepackage[utf8x]{inputenc}
\usepackage[T1]{fontenc}
\usepackage[margin=0.8in,top=1in,bottom=1in]{geometry}
\newcommand{\hn}{\bm{\hat{n}}}
\newcommand{\hr}{\bm{\hat{r}}}
\newcommand{\hx}{\bm{\hat{x}}}
\newcommand{\hy}{\bm{\hat{y}}}
\newcommand{\hz}{\bm{\hat{z}}}
%%%%%
\begin{filecontents*}{refs.bib}
@book{bornwolf,
  author    = {Born, M. and Wolf, E.},
  title     = {Principles of Optics},
  publisher = {Pergamon Press},
  edition   = {7},
  year      = {1999}
}
@book{hecht,
  author    = {Hecht, E.},
  title     = {Optics},
  publisher = {Addison-Wesley},
  edition   = {5},
  year      = {2016}
}
\end{filecontents*}
%
\newcommand{\institution}{University of Arizona}
\newcommand{\autor}{Nicolás Hernández Alegría}
\newcommand{\course}{OPTI 570 Quantum Mechanics}
\newcommand{\assignment}{Assignment 9}
%
\title{\textbf{\assignment}\\\course\\{\Large\institution}}
\author{\autor}
\date{\today\\Total time: 7 hours}
%
\renewcommand{\sectionmark}[1]{\markright{#1}}
\fancypagestyle{mainstyle}{
    \fancyhf{} % Clear all header and footer fields
    \fancyfoot[C]{\thepage}
    \fancyhead[LE,RO]{\course} % Section name on odd pages
    \fancyhead[LO,RE]{\assignment}
    % Optional: Thin rules
    \renewcommand{\headrulewidth}{0pt} % Header rule
    \renewcommand{\footrulewidth}{0pt} % No footer rule
}
%
\begin{document}

\pagestyle{mainstyle}
\maketitle
%%
\section*{Problem I}
\begin{enumerate}[itemsep=0pt,topsep=0pt,label=\alph*)]
  \item The general expression for the transition probability is:
  \begin{align*}
    P_{\ket{+}\to\ket{-}}(t)=\left|\frac{\Omega_0}{\Omega}\right|^2\sin^2\frac{\Omega t}{2},\quad \Omega=\sqrt{\Omega_0^2+\Delta^2}.
  \end{align*}
  By evaluating the different detuning given, we have:
  \begin{align*}
    \Delta=0:&\qquad \Omega=\Omega_0\Longrightarrow P_{\ket{+}\to\ket{-}}(t)=\sin^2\frac{\Omega_0t}{2}\\
    \Delta=|\Omega_0|:&\qquad \Omega=\sqrt{2}\Omega_0\Longrightarrow P_{\ket{+}\to\ket{-}}(t)=\frac{1}{2}\sin^2\frac{\sqrt{2}\Omega_0t}{2}\\
    \Delta=2|\Omega_0|:&\qquad \Omega=\sqrt{5}\Omega_0\Longrightarrow P_{\ket{+}\to\ket{-}}(t)=\frac{1}{5}\sin^2\frac{\sqrt{5}\Omega_0t}{2}
  \end{align*}
  For visualization, we will set $\Omega_0=1$.
  \begin{figure}[h!]
    \centering
    \begin{circuitikz}[xscale=1,yscale=2,scale=.6]
      \def\Omo{1}
      \draw[arrow](0,0)--({6*pi/\Omo+1},0)node[below]{$t$};
      \draw[arrow](0,0)--(0,2)node[right]{$f(t)$};
      \draw[very thick,OliveGreen,domain=0:{6*pi/\Omo},samples=500] plot(\x,{ ( sin( deg(\Omo*\x/2) ) )^2 });
      \draw[very thick,Black,domain=0:{6*pi/\Omo},samples=500] plot(\x,{ (1/2)*( sin( deg(sqrt(2)*\Omo*\x/2) ) )^2 });
      \draw[very thick,NavyBlue,domain=0:{6*pi/\Omo},samples=500] plot(\x,{ (1/5)*( sin( deg(sqrt(5)*\Omo*\x/2) ) )^2 });
      \foreach \x in {1,...,6}{\draw({\x*pi/\Omo},.1)--({\x*pi/\Omo},-.1)node[below]{$\frac{\x\pi}{|\Omega_0|}$}; }
      \foreach \y in {1,1/2,1/5}{\draw(.1,\y)--(-.1,\y)node[left]{\small$\y$};}
      \begin{scope}[shift={(21,1)}] % move legend to top-right
          \draw[very thick,OliveGreen](0,0)--(0.8,0)node[right,black]{$\Delta=0$};
          \draw[very thick,Black](0,-0.5)--(0.8,-0.5)node[right,black]{$\Delta=|\Omega_0|$};
          \draw[very thick,NavyBlue](0,-1)--(0.8,-1)node[right,black]{$\Delta=2|\Omega_0|$};
      \end{scope}
    \end{circuitikz}
  \end{figure}
  \item We have the following.
  \begin{figure}[h!]
    \centering
    \includegraphics[width=.9\columnwidth]{1b.jpg}
    \caption{Problem I, part b)}
  \end{figure}
  \item The probability is the projection of the Bloch vector $\br(t)$ onto the measurement direction.
  For the cases given and that $\Delta=|\Omega_0|$ and $\beta=0$, we have that:
  \begin{align*}
    P_{\ket{x+}}(t)=\frac{3-\cos(\Omega t)}{4},\quad\text{and}\quad P_{\ket{y+}}(t)=\frac{1-\frac{1}{\sqrt{2}}\sin(\Omega t)}{2}.
  \end{align*}
  The y are plotted in the following:
  \begin{figure}[h!]
    \centering
    \begin{circuitikz}[xscale=1,yscale=2,scale=.6]
      \def\Omo{1}
      \draw[arrow](0,0)--({2*pi/\Omo+2},0)node[below]{$t$};
      \draw[arrow](0,0)--(0,2)node[right]{$f(t)$};
      \draw[very thick,OliveGreen,domain=0:{2*pi/\Omo},samples=500] plot(\x,{ ( 3-cos( deg(\Omo*\x) ) )/4 });
      \draw[very thick,NavyBlue,domain=0:{2*pi/\Omo},samples=500] plot(\x,{ ( 1-(1/sqrt(2))*sin( deg(\Omo*\x) ) )/2 });
      \foreach \x in {1,2}{\draw({\x*pi/\Omo},.1)--({\x*pi/\Omo},-.1)node[below]{$\frac{\x\pi}{\Omega}$}; }
      \foreach \y in {1,1/2}{\draw(.1,\y)--(-.1,\y)node[left]{\small$\y$};}
      \draw(0,.1)--(0,-.1)node[below]{$0$};
      \begin{scope}[shift={(8,1)}] % move legend to top-right
          \draw[very thick,OliveGreen](0,0)--(0.8,0)node[right,black]{$P_{\ket{x+}}(t)$};
          \draw[very thick,NavyBlue](0,-0.5)--(0.8,-0.5)node[right,black]{$P_{\ket{y+}}(t)$};
      \end{scope}
    \end{circuitikz}
  \end{figure}
\end{enumerate}


%%
\section*{Problem II}
\begin{enumerate}[itemsep=0pt,topsep=0pt,label=\alph*)]
  \item The unit-vector in cartesian coordinates expressed in terms of the spherical quantities is:
  \begin{align*}
    \hr=\sin\theta\cos\phi\hx+\sin\theta\sin\phi\hy+\cos\theta\hz.
  \end{align*}
  We now try to use the table given in the Field guide to substitute these coefficients and express $\hr$ in terms of the Spherical harmonics.
  The z-direction is the easiest as it only has one quantitiy involved. We now that $Y_1^0$ has cosine of that angle, so we can use it to say that:
  \begin{align*}
    Y_1^0=\sqrt{\frac{3}{4\pi}}\cos\theta\longrightarrow\cos\theta=\sqrt{\frac{4\pi}{3}}Y_1^0.
  \end{align*}
  For the x-direction, we have the term $\sin\theta\cos\phi$ meaning we need to combine some spherical to have this product form.
  Using $Y_1^{\pm1}$ we can create a cosine by considering both sign and collect the expoential:
  \begin{align*}
    Y_1^{-1}-Y_1^{1}=\sqrt{\frac{3}{8\pi}}\sin\theta\left[e^{-i\phi}+e^{i\phi}\right]=\sqrt{\frac{3}{2\pi}}\sin\theta\cos\phi\longrightarrow\sin\theta\cos\phi=\sqrt{\frac{2\pi}{3}}(Y_1^{-1}-Y_1^1).
  \end{align*}
  Similarly, we play with $Y_1^{\pm1}$ to get the $\sin\phi$ term:
  \begin{align*}
    Y_1^{-1}+Y_1^{-1}=\sqrt{\frac{3}{8\pi}}\sin\theta\left[-e^{-i\phi}+e^{i\phi}\right]=2i\sqrt{\frac{3}{8\pi}}\sin\theta\sin\phi\longrightarrow\sin\theta\sin\phi=-i\sqrt{\frac{2\pi}{3}}(Y_1^{-1}+Y_1^1).
  \end{align*}
  Therefore, we finally have 
  \begin{align*}
    \hr=\sqrt{\frac{2\pi}{3}}(Y_1^{-1}-Y_1^1)\hx-i\sqrt{\frac{2\pi}{3}}(Y_1^{-1}+Y_1^1)\hy+\sqrt{\frac{4\pi}{3}}Y_1^0\hz.
  \end{align*}
  \item We have already substituted $x$ for $\sin\theta\cos\phi$ and in terms of the spherical harmonics. The only quantity we need to compute is the $r$, which is obtained by squaring the compoents.
  \begin{align*}
    r^2&=x^2+y^2+z^2=\left[\sqrt{\frac{2\pi}{3}}(Y_1^{-1}-Y_1^1)\right]^2+\left[-i\sqrt{\frac{2\pi}{3}}(Y_1^{-1}+Y_1^1)\right]^2+\left[\sqrt{\frac{4\pi}{3}}Y_1^0\right]^2\\
    &=\frac{2\pi}{3}\left[(Y_1^{-1})^2-2Y_1^{-1}Y_1^1+(Y_1^1)^2+(Y_1^{-1})^2+2Y_1^{-1}Y_1^1+(Y_1^1)^2\right]+\frac{4\pi}{3}(Y_1^0)^2\\
    r^2&=\frac{4\pi}{3}\left[(Y_1^{-1})^2+(Y_1^1)^2+(Y_1^0)^2\right].
  \end{align*} 
  Then, we have 
  \begin{align*}
    F(x,y,z)=\frac{(Y_1^{-1}-Y_1^1-iY_1^{-1}-iY_1^1+\sqrt{2}Y_1^0)}{\sqrt{2}[(Y_1^{-1})^2+(Y_1^1)^2+(Y_1^0)^2]}.
  \end{align*}
\end{enumerate}

%%
\section*{Problem III}
The action of a B-field gives a precession about it with angular frequency of 
\begin{align*}
  \omega=-\gamma B_0.
\end{align*}
The angular displacement in the duration of the pulse is:
\begin{align*}
  \alpha=\omega_1\tau.
\end{align*}
These two quantities must be considerd for each B-field.

First, we start at $\ket{+}_z$. The first pulse is along $+y$ for a duration of $\tau_y$, which produces a rotation of the polar angle of 
\begin{align*}
  \theta=-\gamma B_0\tau_y.
\end{align*}
The vector still lives in the $x-z$ plane. Nesxt, the second pulse is along $+z$ for a time of $\tau_z$. 
This is a rotation about the $+z$ axis, or a rotatin in the azimuth angle of the spherical coordinates $\phi$.
The rotation is 
\begin{align*}
  \phi=-\gamma B_0\tau_z.  
\end{align*}
Solving for each equation for the pulse duration yields:
\begin{align*}
  \tau_y=-\frac{\theta}{\gamma B_0},\quad\text{and}\quad\tau_z=-\frac{\phi}{\gamma B_0}.
\end{align*}


%%
\section*{Problem IV}
\begin{enumerate}[itemsep=0pt,topsep=0pt,label=\alph*)]
  \item Writing the Hailtonian yields:
  \begin{align*}
    H=pc\mathds{1}+\frac{c^4}{2pc}\begin{bmatrix}
      m_1^2&0\\0&m_2^2
    \end{bmatrix}.
  \end{align*}
  \item We can write it in that form if we compare the elements:
  \begin{align*}
    H_1&=pc+\frac{c^4}{2pc}m_1^2E_m+\epsilon\\
    H_2&=pc+\frac{c^4}{2pc}m_2^2=E_m-\epsilon.
  \end{align*}
  Then,
  \begin{align*}
    H_1+H_2&=2E_m=[2pc+\frac{c^4}{2pc}(m_1^2+m_1^2)]\longrightarrow E_m=pc+\frac{c^4}{4pc}(m_1^2+m_2^2)\\
    H_1-H_2&=2\epsilon=\frac{c^4}{2pc}(m_1^2-m_2^2)\longrightarrow\epsilon=\frac{c^4}{4pc}(m_1^2-m_2^2)=-\frac{c^4}{4pc}\delta_m^2.
  \end{align*}
  Thus, our Hamiltonian takes the following form:
  \begin{align*}
    H=\left[pc+\frac{c^4}{4pc}(m_1^2+m_2^2)\right]\mathds{1}-\frac{c^4}{4pc}\delta_m^2\sigma_z.
  \end{align*}
  \item The transformation matrix $M$ is:
  \begin{align*}
    M=\begin{bmatrix}
      \sin\beta&\cos\beta\\-\cos\beta&\sin\beta
    \end{bmatrix}.
  \end{align*}
  Then, to transform the hamiltonian we do:
  \begin{align*}
    H_{\{\ket{v_e},\ket{v_\mu}\}}&=MH_{\{\ket{v_1},\ket{v_2}\}}M^\dagger\\
    &=\epsilon\begin{bmatrix}
      \sin\beta&\cos\beta\\-\cos\beta&\sin\beta
    \end{bmatrix}\begin{bmatrix}
      1&0\\0&-1
    \end{bmatrix}\begin{bmatrix}
      \sin\beta&-\cos\beta\\\cos\beta&\sin\beta
    \end{bmatrix}\\
    &=\epsilon\begin{bmatrix}
      \sin\beta&\cos\beta\\-\cos\beta&\sin\beta
    \end{bmatrix}\begin{bmatrix}
      \sin\beta&-\cos\beta\\-\cos\beta&-\sin\beta
    \end{bmatrix}\\
    H_{\{\ket{v_e},\ket{v_\mu}\}}&=-\frac{c^4}{4pc}\delta_m^2\begin{bmatrix}
      \cos2\beta&-2\sin2\beta\\
      -2\sin2\beta&-\cos2\beta
    \end{bmatrix}.
  \end{align*}
  \item By comparing the above result with the form required, we can conclude by looking the element $11$ and $21$ that:
  \begin{align*}
    \frac{\hbar}{2}\Delta&=-\frac{c^4}{4pc}\delta_m^2\cos2\beta\longrightarrow\Delta=-\frac{c^4}{2\hbar pc}\delta_m^2\cos2\beta\\
    \frac{\hbar}{2}\Omega_0&=\frac{c^4}{4pc}\delta_m^2\sin2\beta\longrightarrow\Omega_0=\frac{c^4}{2\hbar pc}\delta_m^2\sin2\beta.
  \end{align*}
  We can also compute $\Omega$:
  \begin{align*}
    \Omega=\sqrt{\Omega_0^2+\Delta^2}=\frac{c^4}{2\hbar pc}\delta_m^2.
  \end{align*}
  \item Using the numerical values of the variables, by evaluating we have that:
  \begin{align*}
    \Omega_0&=\frac{2.5\times10^{-3}(eV)^2}{2(6.6\times10^{-16}\;eV\cdot s)(10^{10}\;eV)}\sin160^\circ=64.776\;s^{-1}\\
    \Delta&=-\frac{2.5\times10^{-3}\;(eV)^2}{2(6.6\times10^{-16}\;eV\cdot s)(10^{10}\;eV)}\cos160^\circ=177.972\;s^{-1}.
  \end{align*}
  Then,
  \begin{align*}
    \Omega=\sqrt{\Omega_0^2+\Delta^2}=189.394\;s^{-1}.
  \end{align*}
  \item The probability asked is the transition probability, given we are initially in $\ket{\nu_\mu}$. This probability is:
  \begin{align*}
    P_{\mu\to e}(t)=|\braket{\nu_e|e^{-iHt\hbar}|\nu_\mu}|^2=\frac{|\Omega_0|^2}{\Omega^2}\sin^2\frac{\Omega t}{2}=\sin^2(2\beta)\sin^2\left(\frac{c^4\delta_m^2}{4\hbar pc}t\right).
  \end{align*}
  \item We need to do the change of variable $t=d/c$ for the above probability:
  \begin{align*}
    P_{\mu\to e}(d)=\sin^2(2\beta)\sin^2\left(\frac{c^3\delta_m^2}{4\hbar pc}d\right).
  \end{align*}
  Note that now we have $c^3$.
  \item To maximize the probability, we need the sin squared function of distance to be 1. It have the same period as $\sin(\cdot)$ which is ieasier to manipulate, so:
  \begin{align*}
    \sin\frac{c^3\delta_m^2}{4\hbar pc}d&=1\bigr/\sin^{-1}(\cdot)\\
    \frac{c^3\delta_m^2}{4\hbar pc}d&=\frac{\pi(2n-1)}{2},\quad n\in\mathbb{Z}.
  \end{align*}
  The shortest distance is obtained by picking the first zero, $n=0$,
  \begin{align*}
    \frac{c^3\delta_m^2}{4\hbar pc}d&=\frac{\pi}{2}\\
    d&=\frac{2\pi\hbar pc}{c^3\delta_m^2}=\frac{2\pi(6.6\times10^{-16}\;eV)(10^{10}\;eV)(3\times10^8\;m/s)}{2.5\times10^{-3}\;(eV)^2}=4.976\times10^3\;km.
  \end{align*}
  \item Evaluating the probability with $d=300\;km$ yields:
  \begin{align*}
    P_{\mu\to e}(300\times10^3)=\sin^2(160^\circ)\sin^2\left(\frac{2.5\times10^{-3}\;(eV)^2}{4(6.6\times10^{-16}\;eV)(10^{10}\;eV)(3\times10^8\;m/s)}300\times10^3\right)=0.0011.
  \end{align*}
  \item The density operator in the $\{\ket{\nu_e},\ket{\nu_\mu}\}$ is:
  \begin{align*}
    \rho_{\{\ket{\nu_e},\ket{\nu_\mu}\}}=\frac{1}{4}\ket{\nu_e}\bra{\nu_e}+\frac{3}{4}\ket{\nu_\mu}\bra{\nu_\mu}=\begin{bmatrix}
      \frac{1}{4}&0\\0&\frac{3}{4}
    \end{bmatrix}.
  \end{align*}
  In the $\{\ket{\nu_1},\ket{\nu_2}\}$, we use the transformation matrix used in prevous part to convert this operator:
  \begin{align*}
    \rho_{\{\ket{\nu_1},\ket{\nu_2}\}}&=M^\dagger\rho_{\{\ket{\nu_e},\ket{\nu_\mu}\}}M\\
    &=\begin{bmatrix}
      \sin\beta&-\cos\beta\\
      \cos\beta&\sin\beta
    \end{bmatrix}\begin{bmatrix}
      \frac{1}{4}&0\\0&\frac{3}{4}
    \end{bmatrix}\begin{bmatrix}
      \sin\beta&\cos\beta\\-\cos\beta&\sin\beta
    \end{bmatrix}\\
    &=\begin{bmatrix}
      \sin\beta&-\cos\beta\\
      \cos\beta&\sin\beta
    \end{bmatrix}
    \begin{bmatrix}
      \frac{1}{4}\sin\beta&\frac{1}{4}\cos\beta\\
      -\frac{3}{4}\cos\beta&\frac{3}{4}\sin\beta
    \end{bmatrix}\\
    \rho_{\{\ket{\nu_1},\ket{\nu_2}\}}&=\begin{bmatrix}
      \frac{1}{2}+\frac{1}{4}\cos2\beta&-\frac{1}{4}\sin2\beta\\
      -\frac{1}{4}\sin2\beta&\frac{1}{2}-\frac{1}{4}\cos2\beta
    \end{bmatrix}.
  \end{align*}
  We see that in both representation, the trace is unitary.
\end{enumerate}



%\nocite{*}
%\bibliographystyle{plain}   % or unsrt, alpha, apalike, etc.
%\bibliography{refs}

\end{document}
