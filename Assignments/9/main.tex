\documentclass[letterpaper,11pt,twoside]{article}
\usepackage{graphicx} % Required for inserting images
\usepackage[table,xcdraw,dvipsnames]{xcolor}
\usepackage{amsmath,amsfonts,amssymb,amsthm}
\usepackage{listings}
\usepackage{lipsum}
\usepackage{hyperref}
\usepackage{mathrsfs}

\usepackage{enumitem}

\usepackage{tikz}
\usepackage[siunitx, RPvoltages]{circuitikz}
\usetikzlibrary{3d}
\usepackage{comment}
\usepackage{caption,subcaption}
\usepackage{pgfplots}
\pgfplotsset{compat=newest} % or a newer version if available
\usepgfplotslibrary{groupplots}
\usetikzlibrary{pgfplots.groupplots}
\usetikzlibrary{shapes.geometric, arrows}
\tikzstyle{arrow} = [->,>=stealth,shorten >=2pt]
\newcommand{\ket}[1]{|#1\rangle}
\newcommand{\bra}[1]{\langle#1|}
\newcommand{\br}{\bm{r}}
\newcommand{\bR}{\bm{R}}
\newcommand{\bp}{\bm{p}}
\newcommand{\bP}{\bm{P}}
\newcommand{\braket}[1]{\langle#1\rangle}
\newcommand{\F}{\mathscr{F}}
\newcommand{\E}{\mathscr{E}}
\newcommand{\re}[1]{\text{Re}\left(#1\right)}
\newcommand{\im}[1]{\text{Im}\left(#1\right)}
\usepackage{dsfont}
\usepackage{cancel}
\usepackage{bm}
\usepackage{fancyhdr}
\usepackage[utf8x]{inputenc}
\usepackage[T1]{fontenc}
\usepackage[margin=0.8in,top=1in,bottom=1in]{geometry}
%%%%%
\begin{filecontents*}{refs.bib}
@book{bornwolf,
  author    = {Born, M. and Wolf, E.},
  title     = {Principles of Optics},
  publisher = {Pergamon Press},
  edition   = {7},
  year      = {1999}
}
@book{hecht,
  author    = {Hecht, E.},
  title     = {Optics},
  publisher = {Addison-Wesley},
  edition   = {5},
  year      = {2016}
}
\end{filecontents*}
%
\newcommand{\institution}{University of Arizona}
\newcommand{\autor}{Nicolás Hernández Alegría}
\newcommand{\course}{OPTI 570 Quantum Mechanics}
\newcommand{\assignment}{Assignment 9}
%
\title{\textbf{\assignment}\\\course\\{\Large\institution}}
\author{\autor}
\date{\today\\Total time: 8 hours}
%
\renewcommand{\sectionmark}[1]{\markright{#1}}
\fancypagestyle{mainstyle}{
    \fancyhf{} % Clear all header and footer fields
    \fancyfoot[C]{\thepage}
    \fancyhead[LE,RO]{\course} % Section name on odd pages
    \fancyhead[LO,RE]{\assignment}
    % Optional: Thin rules
    \renewcommand{\headrulewidth}{0pt} % Header rule
    \renewcommand{\footrulewidth}{0pt} % No footer rule
}
%
\begin{document}

\pagestyle{mainstyle}
\maketitle
%%
\section*{Problem I}
\begin{enumerate}[itemsep=0pt,topsep=0pt,label=\alph*)]
  \item The general expression for the transition probability is:
  \begin{align*}
    P_{\ket{+}\to\ket{-}}(t)=\left|\frac{\Omega_0}{\Omega}\right|^2\sin^2\frac{\Omega t}{2},\quad \Omega=\sqrt{\Omega_0^2+\Delta^2}.
  \end{align*}
  By evaluating the different detuning given, we have:
  \begin{align*}
    \Delta=0:&\qquad \Omega=\Omega_0\Longrightarrow P_{\ket{+}\to\ket{-}}(t)=\sin^2\frac{\Omega_0t}{2}\\
    \Delta=|\Omega_0|:&\qquad \Omega=\sqrt{2}\Omega_0\Longrightarrow P_{\ket{+}\to\ket{-}}(t)=\frac{1}{2}\sin^2\frac{\sqrt{2}\Omega_0t}{2}\\
    \Delta=2|\Omega_0|:&\qquad \Omega=\sqrt{5}\Omega_0\Longrightarrow P_{\ket{+}\to\ket{-}}(t)=\frac{1}{5}\sin^2\frac{\sqrt{5}\Omega_0t}{2}
  \end{align*}
  For visualization, we will set $\Omega_0=1$.
  \begin{figure}[h!]
    \centering
    \begin{circuitikz}[xscale=1,yscale=2,scale=.6]
      \def\Omo{1}
      \draw[arrow](0,0)--({6*pi/\Omo+1},0)node[below]{$t$};
      \draw[arrow](0,0)--(0,2)node[right]{$f(t)$};
      \draw[very thick,OliveGreen,domain=0:{6*pi/\Omo},samples=500] plot(\x,{ ( sin( deg(\Omo*\x/2) ) )^2 });
      \draw[very thick,Black,domain=0:{6*pi/\Omo},samples=500] plot(\x,{ (1/2)*( sin( deg(sqrt(2)*\Omo*\x/2) ) )^2 });
      \draw[very thick,NavyBlue,domain=0:{6*pi/\Omo},samples=500] plot(\x,{ (1/5)*( sin( deg(sqrt(5)*\Omo*\x/2) ) )^2 });
      \foreach \x in {1,...,6}{\draw({\x*pi/\Omo},.1)--({\x*pi/\Omo},-.1)node[below]{$\frac{\x\pi}{|\Omega_0|}$}; }
      \foreach \y in {1,1/2,1/5}{\draw(.1,\y)--(-.1,\y)node[left]{\small$\y$};}
    \end{circuitikz}
  \end{figure}
\end{enumerate}


%%
\section*{Problem II}


%%
\section*{Problem III}


%%
\section*{Problem IV}




%\nocite{*}
%\bibliographystyle{plain}   % or unsrt, alpha, apalike, etc.
%\bibliography{refs}

\end{document}
