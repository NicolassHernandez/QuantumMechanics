\documentclass[letterpaper,11pt,twoside]{article}
\usepackage{graphicx} % Required for inserting images
\usepackage[table,xcdraw,dvipsnames]{xcolor}
\usepackage{amsmath,amsfonts,amssymb,amsthm}
\usepackage{listings}
\usepackage{lipsum}
\usepackage{hyperref}
\usepackage{mathrsfs}
\usepackage{tikz}
\usepackage{mathtools}
\usetikzlibrary{arrows.meta}
\usepackage{enumitem}

\usepackage{tikz}
\usepackage[siunitx, RPvoltages]{circuitikz}
\usetikzlibrary{3d}
\usepackage{comment}
\usepackage{caption,subcaption}
\usepackage{pgfplots}
\pgfplotsset{compat=newest} % or a newer version if available
\usepgfplotslibrary{groupplots}
\usetikzlibrary{pgfplots.groupplots}
\usetikzlibrary{shapes.geometric, arrows}
\tikzstyle{arrow} = [->,>=stealth,shorten >=2pt]
\newcommand{\ket}[1]{|#1\rangle}
\newcommand{\bra}[1]{\langle#1|}
\newcommand{\br}{\bm{r}}
\newcommand{\bR}{\bm{R}}
\newcommand{\bp}{\bm{p}}
\newcommand{\bP}{\bm{P}}
\newcommand{\braket}[1]{\langle#1\rangle}
\newcommand{\F}{\mathscr{F}}
\newcommand{\E}{\mathscr{E}}
\newcommand{\re}[1]{\text{Re}\left(#1\right)}
\newcommand{\im}[1]{\text{Im}\left(#1\right)}
\usepackage{dsfont}
\usepackage{cancel}
\usepackage{bm}
\usepackage{fancyhdr}
\usepackage[utf8x]{inputenc}
\usepackage[T1]{fontenc}
\usepackage[margin=0.8in,top=1in,bottom=1in]{geometry}
\newcommand{\hn}{\bm{\hat{n}}}
\newcommand{\hr}{\bm{\hat{r}}}
\newcommand{\hx}{\bm{\hat{x}}}
\newcommand{\hy}{\bm{\hat{y}}}
\newcommand{\hz}{\bm{\hat{z}}}
%%%%%
\begin{filecontents*}{refs.bib}
@book{bornwolf,
  author    = {Born, M. and Wolf, E.},
  title     = {Principles of Optics},
  publisher = {Pergamon Press},
  edition   = {7},
  year      = {1999}
}
@book{hecht,
  author    = {Hecht, E.},
  title     = {Optics},
  publisher = {Addison-Wesley},
  edition   = {5},
  year      = {2016}
}
\end{filecontents*}
%
\newcommand{\institution}{University of Arizona}
\newcommand{\autor}{Nicolás Hernández Alegría}
\newcommand{\course}{OPTI 570 Quantum Mechanics}
\newcommand{\assignment}{Assignment 10}
%
\title{\textbf{\assignment}\\\course\\{\Large\institution}}
\author{\autor}
\date{\today\\Total time: 7 hours}
%
\renewcommand{\sectionmark}[1]{\markright{#1}}
\fancypagestyle{mainstyle}{
    \fancyhf{} % Clear all header and footer fields
    \fancyfoot[C]{\thepage}
    \fancyhead[LE,RO]{\course} % Section name on odd pages
    \fancyhead[LO,RE]{\assignment}
    % Optional: Thin rules
    \renewcommand{\headrulewidth}{0pt} % Header rule
    \renewcommand{\footrulewidth}{0pt} % No footer rule
}
%
\begin{document}

\pagestyle{mainstyle}
\maketitle
%%
\section*{Problem I}
\begin{enumerate}[itemsep=0pt,topsep=0pt,label=\alph*)]
  \item The probabilities for x and y directions are:
  \begin{align*}
    P_x=|\braket{\phi_x|\phi_\mu}|^2=\cos^2\mu,\quad\text{and}\quad P_y=|\braket{\phi_y|\phi_\mu}|^2=\sin^2\mu.
  \end{align*}
  For circular polarization we do the same:
  \begin{align*}
    P_{\sigma_+}&=|\braket{\sigma_+|\phi_\mu}|^2=\left|-\frac{1}{\sqrt{2}}\left[(\bra{\phi_x}-i\bra{\phi_y})(\cos\mu\ket{\phi_x}+\sin\mu\ket{\phi_y})\right]\right|^2=\frac{1}{2}(\sin^2\mu+\cos^2\mu)=\frac{1}{2}\\
    P_{\sigma_-}&=1-P_{\sigma_+}=\frac{1}{2}.
  \end{align*}
  \item The state producted by the source is:
  \begin{align*}
    \ket{\psi}=\frac{1}{\sqrt{2}}(\ket{\phi_x^L}\ket{\phi_y^R}-\ket{\phi_x^L}\ket{\phi_y^R}).
  \end{align*}
  The joint measurement is performed through the following projector which is the tensorproduct of the measurement for each photon:
  \begin{align*}
    P_{jk}=P_j^L\otimes P_k^R=\ket{\phi_j^L}\ket{\phi_k^R}\bra{\phi_j^L}\bra{\phi_k^R},\quad j,k\in\{x,y\}.
  \end{align*}
  We are let $j,k$ to vary in either direction so that we will have four types of measurements:
  \begin{align*}
    P_{xy}&=\braket{\psi|P_{xy}|\psi}=|\braket{\phi_x^L\phi_y^R|\psi}|^2=\frac{1}{2}\\
    P_{yx}&=\braket{\psi|P_{yx}|\psi}=|\braket{\phi_y^L\phi_x^R|\psi}|^2=\frac{1}{2}\\
    P_{xx}&=\braket{\psi|P_{xx}|\psi}=|\braket{\phi_x^L\phi_x^R|\psi}|^2=0\\
    P_{yy}&=\braket{\psi|P_{yy}|\psi}=|\braket{\phi_y^L\phi_y^R|\psi}|^2=0
  \end{align*}
  Only the non-zero probabilities are possible state the system may be left after the measurement, which are then:
  \begin{align*}
    \ket{\psi}\stackrel{P_{xy}}{\longrightarrow}\frac{P_{xy}\ket{\psi}}{\sqrt{\braket{\psi|P_{xy}|\psi}}}=\ket{\phi_x^L\phi_y^R},\quad\text{and}\quad
    \ket{\psi}\stackrel{P_{yx}}{\longrightarrow}\frac{P_{yx}\ket{\psi}}{\sqrt{\braket{\psi|P_{yx}|\psi}}}=\ket{\phi_y^L\phi_x^R}.
  \end{align*}
  On the other hand, measurement of a single photon is kind of a marginal measurement. The probabilities cn be constructed from the above probabilities:
  \begin{align*}
    P_x^L&=P_{xy}+P_{xx}=\frac{1}{2}+0=\frac{1}{2}\\
    P_x^R&=P_{xx}+P_{yx}=0+\frac{1}{2}=\frac{1}{2}\\
    P_y^L&=P_{yx}+P_{yy}=\frac{1}{2}+0=\frac{1}{2}\\
    P_y^R&=P_{xy}+P_{yy}=\frac{1}{2}+0=\frac{1}{2}
  \end{align*} 
  \item We first change of basis, by passing from $\phi_{x,y}$ to $\phi_{+,-}$. The transformation matrix is:
  \begin{align*}
    M=\frac{-1}{\sqrt{2}}\begin{bmatrix}
      1&i\\1&-i
    \end{bmatrix}\longrightarrow M^\dagger=\frac{-1}{\sqrt{2}}\begin{bmatrix}
      1&1\\-i&i
    \end{bmatrix}
  \end{align*}.
  By extracting the equation from $M^\dagger$ ans subsitute them in $\phi_{x,y}$ we find the following equivalent state:
  \begin{align*}
    \ket{\psi}\propto\frac{1}{\sqrt{2}}(\ket{\phi_+^L\phi_-^R}-\ket{\phi_-^L\phi_+^R}).
  \end{align*}
  The measurements are analogous to the previous part, but now in the circular polarizaed basis, with the extacly same measurement projector, but 
  with $j,k$ changed:
  \begin{align*}
    P_{jk}=P_j^L\otimes P_k^R=\ket{\phi_j^L}\ket{\phi_k^R}\bra{\phi_j^L}\bra{\phi_k^R},\quad j,k\in\{-,+\}.
  \end{align*}
  Therefore,
  \begin{align*}
    P_{++}&=|\braket{\phi_+^L\phi_+^R|\psi}|^2=0\\
    P_{+-}&=|\braket{\phi_+^L\phi_-^R|\psi}|^2=\frac{1}{2}\\
    P_{-+}&=|\braket{\phi_-^L\phi_+^R|\psi}|^2=\frac{1}{2}\\
    P_{--}&=|\braket{\phi_-^L\phi_-^R|\psi}|^2=0.
  \end{align*}
  We see the polarization for either basis is exactly the same, but the states after the measurement will be different.
\end{enumerate}
%%
\section*{Problem II}
In the deuterium atom we have that $I=1$ and $S=1/2$.
\begin{enumerate}[itemsep=0pt,topsep=0pt,label=\alph*)]
  \item In the state $1s$, we have $n=1$ and $l=0$. The quantum number J is:
  \begin{align*}
    J&\in\{|L-S|,|L-S|+1,\cdots,L+S-1,L+S\}=\left\{\frac{1}{2}\right\}.
  \end{align*}
  For this unique value, we have
  \begin{align*}
    J=\frac{1}{2}:\quad F\in\{|J-I|,|J-I|+1,\cdots,J+I-1,J+I\}=\left\{\frac{1}{2},\frac{3}{2}\right\}.
  \end{align*}
  \item For the state $2p$, we have $n=2$ and $l=1$. The quantum number J is:
  \begin{align*}
    J\in\left\{|1-\frac{1}{2}|,1+\frac{1}{2}\right\}=\left\{\frac{1}{2},\frac{3}{2}\right\}.
  \end{align*}
  For each J, we have F:
  \begin{align*}
    J=\frac{1}{2}:&\quad F\in\{|\frac{1}{2}-1|,\frac{1}{2}+1\}=\left\{\frac{1}{2},\frac{3}{2}\right\}\\
    J=\frac{3}{2}:&\quad F\in\{|\frac{3}{2}-1|,|\frac{3}{2}-1|+1,\frac{3}{2}+1\}=\left\{\frac{1}{2},\frac{3}{2},\frac{5}{2}\right\}.
  \end{align*}
\end{enumerate}

%%
\section*{Problem III}
We are asked to compute the matrix element of $A$, but the initial and final states are in the TAM basis, which is not suitable due to the action of $A$ only on $\ket{L,M_L}$; in the TAM basis those are encoded.
We need therefore to convert from TAM basis to TP basis each $\ket{\phi}$ and $\ket{\phi'}$.
We need to do the following for eacth state:
\begin{align*}
  \ket{F,M_F}\stackrel{CG}{\longrightarrow}\ket{J,M_J}\ket{I,M_I}\stackrel{CG}{\longrightarrow}\ket{L,M_L}\ket{S,M_S}\ket{I,M_I}.
\end{align*}



We do that for each state then,
\begin{enumerate}[itemsep=0pt,topsep=0pt,label=\alph*)]
  \item We first have that 
  \begin{align*}
    5^2S_{1/2}\Longrightarrow n=5,\;S=\frac{1}{2},\;L=0,\;J=1/2.
  \end{align*}
  We also ommit those quantities as are the same during all the computations. 
  We then need to express $\ket{F=1,M_F=1}$, and we select $J_1=I=3/2,\;J_2=J=1/2$ for convention. To decouple this ket in $J$ and $I$, we use the 
  coefficient from the table to say that:
  \begin{align*}
    3/2\times1/2\rfloor\quad\ket{1,1}_F=&\sqrt{\frac{3}{4}}\ket{1/2,-1/2}_J\ket{3/2,3/2}_I-\sqrt{\frac{1}{4}}\ket{1/2,1/2}_J\ket{3/2,1/2}_I.
  \end{align*}
  If we look at the following decomposition in the table, with $J_1=S=1/2,J_2=L=0$, we wont get anything. However, we can use the conservation of 
  angular momentum $M_J=M_S+M_L$ to select the states that corresponds. For $L=0$ we only have $M_L=0$. For $S=1/2$, we have $M_S=\pm1/2$ and only the negative one satisfies the 
  conservation of AM. Therefore,
  \begin{align*}
    M_J=M_S+M_L\rfloor\quad\ket{1/2,-1/2}_J=\ket{0,0}_L\ket{1/2,-1/2}_S.
  \end{align*}
  We do the same for the remaining ket:
  \begin{align*}
    M_J=M_S+M_L\rfloor\quad\ket{1/2,1/2}_J=\ket{0,0}_L\ket{1/2,1/2}_S.
  \end{align*}
  Thus, putting all together:
  \begin{align*}
    \ket{\phi}=\ket{1,1}_F=&\textcolor{blue}{\sqrt{\frac{3}{4}}\ket{0,0}_L\ket{1/2,-1/2}_S\ket{3/2,3/2}_I}-\textcolor{red}{\sqrt{\frac{1}{4}}\ket{0,0}_L\ket{1/2,1/2}_S\ket{3/2,1/2}_I}.
  \end{align*}
  We do this process for $\ket{\phi'}$, which is completely analogous to this one.
  \begin{align*}
    3/2\times1/2\rfloor\quad\ket{1,1}_{F'}=\sqrt{\frac{3}{4}}\ket{1/2,-1/2}_{J'}\ket{3/2,3/2}_{I'}-\sqrt{\frac{1}{4}}\ket{1/2,1/2}_{J'}\ket{3/2,1/2}_{I'}
  \end{align*}
  In this case we have $J_1=L=1$ and $J_2=S=1/2$, so that we use the $1\times1/2$ table.
  \begin{align*}
    1\times1/2\rfloor\quad\ket{1/2,-1/2}_{J'}=&\sqrt{\frac{1}{3}}\ket{1,0}_{L'}\ket{1/2,-1/2}_{S'}-\sqrt{\frac{2}{3}}\ket{1,-1}_{L'}\ket{1/2,1/2}_{S'}\\
    \ket{1/2,1/2}_{J'}=&\sqrt{\frac{2}{3}}\ket{1,1}_{L'}\ket{1/2,-1/2}_{S'}-\sqrt{\frac{1}{3}}\ket{1,0}_{L'}\ket{1/2,1/2}_{S'}
  \end{align*}
  Putting all together:
  \begin{align*}
    \ket{\phi'}=\ket{1,1}_{F'}=&\sqrt{\frac{3}{4}}\left[\sqrt{\frac{1}{3}}\ket{1,0}_{L'}\ket{1/2,-1/2}_{S'}-\sqrt{\frac{2}{3}}\ket{1,-1}_{L'}\ket{1/2,1/2}_{S'}\right]\ket{3/2,3/2}_{I'}\\
    &-\sqrt{\frac{1}{4}}\left[\sqrt{\frac{2}{3}}\ket{1,1}_{L'}\ket{1/2,-1/2}_{S'}-\sqrt{\frac{1}{3}}\ket{1,0}_{L'}\ket{1/2,1/2}_{S'}\right]\ket{3/2,1/2}_{I'}\\
    =&\textcolor{blue}{\sqrt{\frac{1}{4}}\ket{1,0}_{L'}\ket{1/2,-1/2}_{S'}\ket{3/2,3/2}_{I'}}-\sqrt{\frac{1}{2}}\ket{1,-1}_{L'}\ket{1/2,1/2}_{S'}\ket{3/2,3/2}_{I'}\\
    &-\sqrt{\frac{1}{6}}\ket{1,1}_{L'}\ket{1/2,-1/2}_{S'}\ket{3/2,1/2}_{I'}+\textcolor{red}{\sqrt{\frac{1}{12}}\ket{1,0}_{L'}\ket{1/2,1/2}_{S'}\ket{3/2,1/2}_{I'}}.
  \end{align*}
  Finally, we compute the matrix element considering that only $L'-L=\pm1$ and $M_{L}'-M_L=0$ for z polarization. Also, recall that $A$ only acts on the $L$-basis:
  \begin{align*}
    T=|\braket{\phi'|A|\phi}|^2=\left|\left[\sqrt{\frac{1}{4}}\sqrt{\frac{3}{4}}-\sqrt{\frac{1}{4}}\sqrt{\frac{1}{12}}\right]A_{1000}\right|^2=\frac{1}{12}|A_{1000}|^2.
  \end{align*}
  \item For $\ket{\phi}$, $n=5,\;S=1/2,\;L=0,\;J=1/2$:
  \begin{align*}
    3/2\times1/2\rfloor\quad\ket{1,0}_F=\sqrt{\frac{1}{2}}\ket{1/2,-1/2}_J\ket{3/2,1/2}_I-\sqrt{\frac{1}{2}}\ket{1/2,1/2}_J\ket{3/2,-1/2}_I.
  \end{align*}
  \begin{align*}
    M_J=M_L+M_S\rfloor\quad\ket{1/2,-1/2}_J&=\ket{0,0}_L\ket{1/2,-1/2}_S\\
    M_J=M_L+M_S\rfloor\quad\ket{1/2,1/2}_J&=\ket{0,0}_L\ket{1/2,1/2}_S
  \end{align*}
  \begin{align*}
    \ket{\phi}=\ket{1,0}_F=\textcolor{blue}{\sqrt{\frac{1}{2}}\ket{0,0}_L\ket{1/2,-1/2}_S\ket{3/2,1/2}_I}-\textcolor{red}{\sqrt{\frac{1}{2}}\ket{0,0}_L\ket{1/2,1/2}_S\ket{3/2,-1/2}_I}.
  \end{align*}
  For $\ket{\phi'}$, $n=5,\;S=1/2,\;L=1,\;J=1/2$:
  \begin{align*}
    3/2\times1/2\rfloor\quad\ket{1,0}_{F'}=\sqrt{\frac{1}{2}}\ket{1/2,-1/2}_{J'}\ket{3/2,1/2}_{I'}-\sqrt{\frac{1}{2}}\ket{1/2,1/2}_{J'}\ket{3/2,-1/2}_{I'}.
  \end{align*}
  \begin{align*}
    1\times1/2\rfloor\quad\ket{1/2,-1/2}_{J'}&=\sqrt{\frac{1}{3}}\ket{1,0}_{L'}\ket{1/2,-1/2}_{S'}-\sqrt{\frac{2}{3}}\ket{1,-1}_{L'}\ket{1/2,1/2}_{S'}\\
    1\times1/2\rfloor\quad\ket{1/2,1/2}_{J'}&=\sqrt{\frac{2}{3}}\ket{1,1}_{L'}\ket{1/2,-1/2}_{S'}-\sqrt{\frac{1}{3}}\ket{1,0}_{L'}\ket{1/2,1/2}_{S'}
  \end{align*}
  \begin{align*}
    \ket{\phi'}=\ket{1,0}_{F'}=&\sqrt{\frac{1}{2}}\left[\sqrt{\frac{1}{3}}\ket{1,0}_{L'}\ket{1/2,-1/2}_{S'}-\sqrt{\frac{2}{3}}\ket{1,-1}_{L'}\ket{1/2,1/2}_{S'}\right]\ket{3/2,1/2}_{I'}\\
    &-\sqrt{\frac{1}{2}}\left[\sqrt{\frac{2}{3}}\ket{1,1}_{L'}\ket{1/2,-1/2}_{S'}-\sqrt{\frac{1}{3}}\ket{1,0}_{L'}\ket{1/2,1/2}_{S'}\right]\ket{3/2,-1/2}_{I'}\\
    &=\textcolor{blue}{\sqrt{\frac{1}{6}}\ket{1,0}_{L'}\ket{1/2,-1/2}_{S'}\ket{3/2,1/2}_{I'}}-\sqrt{\frac{1}{3}}\ket{1,-1}_{L'}\ket{1/2,1/2}_{S'}\ket{3/2,1/2}_{I'}\\
    &-\sqrt{\frac{1}{3}}\ket{1,1}_{L'}\ket{1/2,-1/2}_{S'}\ket{3/2,-1/2}_{I'}+\textcolor{red}{\sqrt{\frac{1}{6}}\ket{1,0}_{L'}\ket{1/2,1/2}_{S'}\ket{3/2,-1/2}_{I'}}.
  \end{align*}
  \begin{align*}
    T=|\braket{\phi'|A|\phi}|^2=\left|\left[\sqrt{\frac{1}{6}}\sqrt{\frac{1}{2}}-\sqrt{\frac{1}{6}}\sqrt{\frac{1}{2}}\right]A_{1000}\right|^2=0.
  \end{align*}
  \item For $\ket{\phi}$, $n=5,\;S=1/2,\;L=0,\;J=1/2$:
  \begin{align*}
    \ket{\phi}=\ket{1,1}_F=&\textcolor{blue}{\sqrt{\frac{3}{4}}\ket{0,0}_L\ket{1/2,-1/2}_S\ket{3/2,3/2}_I}-\textcolor{red}{\sqrt{\frac{1}{4}}\ket{0,0}_L\ket{1/2,1/2}_S\ket{3/2,1/2}_I}.
  \end{align*}
  For $\ket{\phi'}$, $n=5,\;S=1/2,\;L=1,\;J=1/2$:
  \begin{align*}
    3/2\times1/2\rfloor\quad\ket{2,1}_{F'}=\sqrt{\frac{1}{4}}\ket{1/2,-1/2}_{J'}\ket{3/2,3/2}_{I'}+\sqrt{\frac{3}{4}}\ket{1/2,1/2}_{J'}\ket{3/2,1/2}_{I'}.
  \end{align*}
  \begin{align*}
    1\times1/2\rfloor\quad\ket{1/2,-1/2}_{J'}=\sqrt{\frac{1}{3}}\ket{1,0}_{L'}\ket{1/2,-1/2}_{S'}-\sqrt{\frac{2}{3}}\ket{1,-1}_{L'}\ket{1/2,1/2}_{S'}\\
    1\times1/2\rfloor\quad\ket{1/2,1/2}_{J'}=\sqrt{\frac{2}{3}}\ket{1,1}_{L'}\ket{1/2,-1/2}_{S'}-\sqrt{\frac{1}{3}}\ket{1,0}_{L'}\ket{1/2,1/2}_{S'}.
  \end{align*}
  \begin{align*}
    \ket{\phi'}=\ket{2,1}_{F'}=&\sqrt{\frac{1}{4}}\left[\sqrt{\frac{1}{3}}\ket{1,0}_{L'}\ket{1/2,-1/2}_{S'}-\sqrt{\frac{2}{3}}\ket{1,-1}_{L'}\ket{1/2,1/2}_{S'}\right]\ket{3/2,3/2}_{I'}\\
    &+\sqrt{\frac{3}{4}}\left[\sqrt{\frac{2}{3}}\ket{1,1}_{L'}\ket{1/2,-1/2}_{S'}-\sqrt{\frac{1}{3}}\ket{1,0}_{L'}\ket{1/2,1/2}_{S'}\right]\ket{3/2,1/2}_{I'}\\
    =&\textcolor{blue}{\sqrt{\frac{1}{12}}\ket{1,0}_{L'}\ket{1/2,-1/2}_{S'}\ket{3/2,3/2}_{I'}}-\sqrt{\frac{1}{6}}\ket{1,-1}_{L'}\ket{1/2,1/2}_{S'}\ket{3/2,3/2}_{I'}\\
    &+\sqrt{\frac{1}{2}}\ket{1,1}_{L'}\ket{1/2,-1/2}_{S'}\ket{3/2,1/2}_{I'}-\textcolor{red}{\sqrt{\frac{1}{4}}\ket{1,0}_{L'}\ket{1/2,1/2}_{S'}\ket{3/2,1/2}_{I'}}.
  \end{align*}
  \begin{align*}
    T=|\braket{\phi'|A|\phi}|^2=\left|\left[\sqrt{\frac{1}{12}}\sqrt{\frac{3}{4}}-\sqrt{\frac{1}{4}}\sqrt{\frac{1}{4}}\right]A_{1000}\right|^2=\frac{1}{4}|A_{1000}|^2.
  \end{align*}
  \item For $\ket{\phi}$, $n=5,\;S=1/2,\;L=0,\;J=1/2$:
  \begin{align*}
    \ket{1,0}_F=\textcolor{blue}{\sqrt{\frac{1}{2}}\ket{0,0}_L\ket{1/2,-1/2}_S\ket{3/2,1/2}_I}-\textcolor{red}{\sqrt{\frac{1}{2}}\ket{0,0}_L\ket{1/2,1/2}_S\ket{3/2,-1/2}_I}.
  \end{align*}
  For $\ket{\phi'}$, $n=5,\;S=1/2,\;L=1,\;J=1/2$:
  \begin{align*}
    3/2\times1/2\rfloor\quad\ket{2,0}_{F'}=\sqrt{\frac{1}{2}}\ket{1/2,-1/2}_{J'}\ket{3/2,1/2}_{I'}+\sqrt{\frac{1}{2}}\ket{1/2,1/2}_{J'}\ket{3/2,-1/2}_{I'}.
  \end{align*}
  \begin{align*}
    1\times1/2\rfloor\quad\ket{1/2,-1/2}_{J'}&=\sqrt{\frac{1}{3}}\ket{1,0}_{L'}\ket{1/2,-1/2}_{S'}-\sqrt{\frac{2}{3}}\ket{1,-1}_{L'}\ket{1/2,1/2}_{S'}\\
    1\times1/2\rfloor\quad\ket{1/2,1/2}_{J'}&=\sqrt{\frac{2}{3}}\ket{1,1}_{L'}\ket{1/2,-1/2}_{S'}-\sqrt{\frac{1}{3}}\ket{1,0}_{L'}\ket{1/2,1/2}_{S'}.
  \end{align*}
  \begin{align*}
    \ket{\phi'}=\ket{2,0}_{F'}=&\sqrt{\frac{1}{2}}\left[\sqrt{\frac{1}{3}}\ket{1,0}_{L'}\ket{1/2,-1/2}_{S'}-\sqrt{\frac{2}{3}}\ket{1,-1}_{L'}\ket{1/2,1/2}_{S'}\right]\ket{3/2,1/2}_{I'}\\
    &+\sqrt{\frac{1}{2}}\left[\sqrt{\frac{2}{3}}\ket{1,1}_{L'}\ket{1/2,-1/2}_{S'}-\sqrt{\frac{1}{3}}\ket{1,0}_{L'}\ket{1/2,1/2}_{S'}\right]\ket{3/2,-1/2}_{I'}\\
    =&\textcolor{blue}{\sqrt{\frac{1}{6}}\ket{1,0}_{L'}\ket{1/2,-1/2}_{S'}\ket{3/2,1/2}_{I'}}-\sqrt{\frac{1}{3}}\ket{1,-1}_{L'}\ket{1/2,1/2}_{S'}\ket{3/2,1/2}_{I'}\\
    &+\sqrt{\frac{1}{3}}\ket{1,1}_{L'}\ket{1/2,-1/2}_{S'}\ket{3/2,-1/2}_{I'}-\textcolor{red}{\sqrt{\frac{1}{6}}\ket{1,0}_{L'}\ket{1/2,1/2}_{S'}\ket{3/2,-1/2}_{I'}}.
  \end{align*}
  \begin{align*}
    T=|\braket{\phi'|A|\phi}|^2=\left|\left[\sqrt{\frac{1}{2}}\sqrt{\frac{1}{6}}+\sqrt{\frac{1}{2}}\sqrt{\frac{1}{6}}\right]A_{1000}\right|^2=\frac{1}{3}|A_{1000}|^2.
  \end{align*}
\end{enumerate}




%\nocite{*}
%\bibliographystyle{plain}   % or unsrt, alpha, apalike, etc.
%\bibliography{refs}

\end{document}
