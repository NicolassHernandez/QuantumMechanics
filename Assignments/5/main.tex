\documentclass[letterpaper,11pt,twoside]{article}
\usepackage{graphicx} % Required for inserting images
\usepackage[table,xcdraw,dvipsnames]{xcolor}
\usepackage{amsmath,amsfonts,amssymb,amsthm}
\usepackage{listings}
\usepackage{lipsum}
\usepackage{hyperref}
\usepackage{enumitem}

\usepackage{tikz}
\usepackage[siunitx, RPvoltages]{circuitikz}
\usetikzlibrary{3d}
\usepackage{comment}
\usepackage{caption,subcaption}
\usepackage{pgfplots}
\pgfplotsset{compat=newest} % or a newer version if available
\usepgfplotslibrary{groupplots}
\usetikzlibrary{pgfplots.groupplots}
\usetikzlibrary{shapes.geometric, arrows}
\tikzstyle{arrow} = [->,>=stealth,shorten >=2pt]
\newcommand{\ket}[1]{|#1\rangle}
\newcommand{\bra}[1]{\langle#1|}
\newcommand{\br}{\bm{r}}
\newcommand{\bR}{\bm{R}}
\newcommand{\bp}{\bm{p}}
\newcommand{\bP}{\bm{P}}
\newcommand{\braket}[1]{\langle#1\rangle}
\newcommand{\F}{\mathscr{F}}
\newcommand{\E}{\mathscr{E}}
\usepackage{dsfont}
\usepackage{cancel}
\usepackage{bm}
\usepackage{fancyhdr}
\usepackage[utf8x]{inputenc}
\usepackage[T1]{fontenc}
\usepackage[margin=0.8in,top=1in,bottom=1in]{geometry}
%%%%%
\begin{filecontents*}{refs.bib}
@book{bornwolf,
  author    = {Born, M. and Wolf, E.},
  title     = {Principles of Optics},
  publisher = {Pergamon Press},
  edition   = {7},
  year      = {1999}
}
@book{hecht,
  author    = {Hecht, E.},
  title     = {Optics},
  publisher = {Addison-Wesley},
  edition   = {5},
  year      = {2016}
}
\end{filecontents*}
%
\newcommand{\institution}{University of Arizona}
\newcommand{\autor}{Nicolás Hernández Alegría}
\newcommand{\course}{OPTI 570 Quantum Mechanics}
\newcommand{\assignment}{Assignment 5}
%
\title{\textbf{\assignment}\\\course\\{\Large\institution}}
\author{\autor}
\date{\today\\Total time: $\infty$ hours}
%
\renewcommand{\sectionmark}[1]{\markright{#1}}
\fancypagestyle{mainstyle}{
    \fancyhf{} % Clear all header and footer fields
    \fancyfoot[C]{\thepage}
    \fancyhead[LE,RO]{\course} % Section name on odd pages
    \fancyhead[LO,RE]{\assignment}
    % Optional: Thin rules
    \renewcommand{\headrulewidth}{0pt} % Header rule
    \renewcommand{\footrulewidth}{0pt} % No footer rule
}
%
\begin{document}

\pagestyle{mainstyle}
\maketitle
%%
\section*{Problem I}
%
\subsection*{Part 1.}
We define the effective state in the second frame $\ket{\psi_E(t)}=\mathbb{F}(t)\ket{\psi(t)}$, where $\mathbb{F}(t)$ is some unitary time-dependend operator.
Substituting $\ket{\psi(t)}=\mathbb{F}^\dagger(t)\ket{\psi_E(t)}$ into the Schrodinger equation yields:
\begin{align*}
  i\hbar\partial_t\left[\mathbb{F}^\dagger(t)\ket{\psi_E(t)}\right]&=H(t)\left[\mathbb{F}^\dagger(t)\ket{\psi_E(t)}\right]\\
  i\hbar\left[\partial_t\mathbb{F}^\dagger(t)\ket{\psi_E(t)}+\mathbb{F}^\dagger(t)\partial_t\ket{\psi_E(t)}\right]&=H(t)\mathbb{F}^\dagger(t)\ket{\psi_E(t)}\\
  i\hbar\mathbb{F}^\dagger(t)\partial_t\ket{\psi_E(t)}&=\left[H(t)\mathbb{F}^\dagger(t)-i\hbar\partial_t\mathbb{F}^\dagger(t)\right]\ket{\psi_E(t)}\biggr/\mathbb{F}(t)\\
  i\hbar\partial_t\ket{\psi_E(t)}&=\left[\mathbb{F}(t)H(t)\mathbb{F}^\dagger(t)-i\hbar\mathbb{F}(t)\partial_t\mathbb{F}^\dagger(t)\right]\ket{\psi_E(t)}\\
  i\hbar\partial_t\ket{\psi_E(t)}&=H_E(t)\ket{\psi_E(t)},
\end{align*}
where $H_E(t)$ is the effective Hamiltonian:
\begin{align*}
  H_E(t)=\mathbb{F}(t)H(t)\mathbb{F}^\dagger(t)-i\hbar\mathbb{F}(t)\partial_t\mathbb{F}^\dagger(t).
\end{align*}
%
\subsection*{Part 2.}
We know that 
\begin{align*}
  \ket{\psi_I(t)}=\mathbb{U}_0^\dagger(t,t_0)\ket{\psi_S(t)},\quad\text{with}\quad\mathbb{U}_0(t,t_0)=e^{-i(t-t_0)H_0/\hbar}.
\end{align*}
Then,
{
\begin{align*}
  i\hbar\partial_t\left[\mathbb{U}_0^\dagger\ket{\psi_S(t)}\right]&=i\hbar\partial_t\mathbb{U}^\dagger\ket{\psi_S(t)}+i\hbar\mathbb{U}_0^\dagger\partial_t\ket{\psi_S(t)}\\
  &=i\hbar\partial_t\mathbb{U}_0^\dagger\ket{\psi_S(t)}+\mathbb{U}_0^\dagger H_S(t)\ket{\psi_S}\\
  &=\left[i\hbar(\partial_t\mathbb{U}_0^\dagger)\mathbb{U}_0+\mathbb{U}_0^\dagger H_S(t)\mathbb{U}_0\right]\ket{\psi_I(t)}\\
  &=\left[-\mathbb{U}_0^\dagger H_0\mathbb{U}_0+\mathbb{U}_0^\dagger(H_0+W(t))\mathbb{U}_0\right]\ket{\psi_I(t)}\quad(i\hbar\partial_t\mathbb{U}^\dagger_0=-\mathbb{U}_0^\dagger H_0)\\
  i\hbar\partial_t\ket{\psi_I(t)}&=\left[\mathbb{U}_0(t,t_0)^\dagger W(t)\mathbb{U}_0(t,t_0)\right]\ket{\psi_I(t)}\\
  i\hbar\partial_t\ket{\psi_I(t)}&=H_E(t)\ket{\psi_I(t)},
\end{align*}}
where $H_E(t)$ is the effective Hamiltonian:
\begin{align*}
  H_E(t)=\mathbb{U}_0(t,t_0)^\dagger W(t)\mathbb{U}_0(t,t_0).
\end{align*}
%%
\section*{Problem II}
\begin{enumerate}
  \item The probability for energies greater than $2\hbar\omega$ is then
  \begin{align*}
    P(E>2\hbar\omega)=\sum_{n\geq2}|c_n|^2,\quad c_n=\braket{n|\psi(t)}.
  \end{align*}
  If $P=0$, then all $c_n=0,\;n\geq2$. Only $c_0$ and $c_1$ may be non-zero.
  \item The normalization condition means that 
  \begin{align*}
    \sum_{n<2}|c_n|^2=1\Longrightarrow |c_0|^2+|c_1|^2=1.
  \end{align*}
  The mean value of the energy is
  \begin{align*}
    \braket{H}=\braket{\psi|H|\psi}=|c_0|^2E_0+|c_1|^2E_1=\frac{1}{2}\hbar\omega|c_0|^2+\frac{3}{2}\hbar\omega|c_1|^2.
  \end{align*}
  If $\braket{H}=\hbar\omega$, we have a system of equation composed of the normalization and mean value expression:
  \begin{align*}
    \left.
    \begin{array}{l}
      \displaystyle\frac{1}{2}\hbar\omega|c_0|^2+\frac{3}{2}\hbar\omega|c_1|^2=\hbar\omega\\
      \displaystyle|c_0|^2+|c_1|^2=1
    \end{array}\right\rfloor\longrightarrow
    |c_0|^2=|c_1|^2=\frac{1}{2}.
  \end{align*}
  \item First, we develop the mean value of $X$:
  \begin{align*}
    \braket{X}&=\braket{\psi|X|\psi}=\frac{1}{2}(\bra{0}+e^{-i\theta_1}\bra{1})X(\ket{0}+e^{i\theta_1}\ket{1})=\frac{1}{2}\left[\braket{0|X|0}+e^{i\theta_1}\braket{0|X|1}+e^{-i\theta_1}\braket{1|X|0}+\braket{1|X|1}\right].
  \end{align*}
  The last result is due to the result we have obtained in the previous incise. Now, we use the matrix element of $X$ of the harmonic oscillator:
  \begin{align*}
    X=\sqrt{\frac{\hbar}{2m\omega}}(a+a^\dagger),\quad\text{where}\quad a\ket{n}=\sqrt{n}\ket{n-1},\;\;a^\dagger\ket{n}=\sqrt{n+1}\ket{n+1}.
  \end{align*}
  We compute the terms separately,
  \begin{align*}
    \braket{0|X|0}&=\sqrt{\frac{\hbar}{2m\omega}}\braket{0|(a+a^\dagger)|0}=\sqrt{\frac{\hbar}{2m\omega}}\left[\braket{0|a|0}+\braket{0|a^\dagger|0}\right]=\sqrt{\frac{\hbar}{2m\omega}}\left[\braket{0|0}+\braket{0|1}\right]=0,\\
    \braket{1|X|1}&=\sqrt{\frac{\hbar}{2m\omega}}\left[\braket{1|a|1}+\braket{1|a^\dagger|1}\right]=\sqrt{\frac{\hbar}{2m\omega}}\left[\sqrt{1}\braket{1|0}+\sqrt{2}\braket{1|2}\right]=0,\\
    \braket{0|X|1}&=\sqrt{\frac{\hbar}{2m\omega}}\left[\braket{0|a|1}+\braket{0|a^\dagger|1}\right]=\sqrt{\frac{\hbar}{2m\omega}}\left[\sqrt{1}\braket{0|0}+\sqrt{2}\braket{0|2}\right]=\sqrt{\frac{\hbar}{2m\omega}},\\
    \braket{1|X|0}&=\sqrt{\frac{\hbar}{2m\omega}}\left[\braket{1|a|0}+\braket{1|a^\dagger|0}\right]=\sqrt{\frac{\hbar}{2m\omega}}\left[\braket{1|0}+\sqrt{1}\braket{1|1}\right]=\sqrt{\frac{\hbar}{2m\omega}}.
  \end{align*}
  We put these results in $\braket{X}$:
  \begin{align*}
    \braket{X}=\sqrt{\frac{\hbar}{2m\omega}}\frac{e^{i\theta_1}+e^{i\theta_1}}{2}=\sqrt{\frac{\hbar}{2m\omega}}\cos\theta_1=\frac{1}{2}\sqrt{\frac{\hbar}{m\omega}}.
  \end{align*}
  The last relation means that 
  \begin{align*}
    \cos\theta_1=\frac{\sqrt{2}}{2}\longrightarrow \theta_1=\pm\frac{\pi}{4}\quad(\text{inside one period}).
  \end{align*}
  \item The time evolution is:
  \begin{align*}
    \ket{\psi(t)}=\sum_{n=0}^1c_ne^{-iE_nt/\hbar}\ket{n}=\frac{1}{\sqrt{2}}\left(e^{-i\omega t/2}\ket{0}+e^{i\theta_1}e^{-i3\omega t/2}\ket{1}\right)
  \end{align*}
  We can factor out the common phase that translates to global phase factor so that we have
  \begin{align*}
    \ket{\psi(t)}\propto\frac{1}{\sqrt{2}}\left(\ket{0}+e^{i(\theta_1-\omega t)\ket{1}}\right)\longrightarrow\theta_1(t)=\theta_1-\omega t.
  \end{align*}
  We use out previous result of $\braket{X}$ and replace $\theta_1$ by $\theta_1(t)$:
  \begin{align*}
    \braket{X}(t)=\sqrt{\frac{\hbar}{2m\omega}}\cos(\omega t-\theta_1).
  \end{align*}
  The argument of the cosine is reversedas the one in part c) due to the restriction of $\cos\theta_1=1/\sqrt{2}$.


\end{enumerate}


%%
\section*{Problem III}


%%
\section*{Problem IV}

%\nocite{*}
%\bibliographystyle{plain}   % or unsrt, alpha, apalike, etc.
%\bibliography{refs}

\end{document}
