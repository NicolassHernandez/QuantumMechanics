\documentclass[letterpaper,11pt,twoside]{article}
\usepackage{graphicx} % Required for inserting images
\usepackage[table,xcdraw,dvipsnames]{xcolor}
\usepackage{amsmath,amsfonts,amssymb,amsthm}
\usepackage{listings}
\usepackage{lipsum}
\usepackage{hyperref}

\usepackage{tikz}
\usepackage[siunitx, RPvoltages]{circuitikz}
\usetikzlibrary{3d}
\usepackage{comment}
\usepackage{caption,subcaption}
\usepackage{pgfplots}
\pgfplotsset{compat=newest} % or a newer version if available
\usepgfplotslibrary{groupplots}
\usetikzlibrary{pgfplots.groupplots}
\usetikzlibrary{shapes.geometric, arrows}
\tikzstyle{arrow} = [->,>=stealth,shorten >=2pt]

\usepackage{cancel}
\usepackage{bm}
\usepackage{fancyhdr}
\usepackage[utf8x]{inputenc}
\usepackage[T1]{fontenc}
\usepackage[margin=0.8in,top=1in,bottom=1in]{geometry}
%%%%%
\begin{filecontents*}{refs.bib}
@book{bornwolf,
  author    = {Born, M. and Wolf, E.},
  title     = {Principles of Optics},
  publisher = {Pergamon Press},
  edition   = {7},
  year      = {1999}
}
@book{hecht,
  author    = {Hecht, E.},
  title     = {Optics},
  publisher = {Addison-Wesley},
  edition   = {5},
  year      = {2016}
}
\end{filecontents*}
%
\newcommand{\institution}{University of Arizona}
\newcommand{\autor}{Nicolás Hernández Alegría}
\newcommand{\course}{OPTI 502 Quantum Mechanics}
\newcommand{\assignment}{Assignment 1}
%
\title{\textbf{\assignment}\\\course\\{\Large\institution}}
\author{\autor}
\date{\today\\Total time: 6 hours}
%
\renewcommand{\sectionmark}[1]{\markright{#1}}
\fancypagestyle{mainstyle}{
    \fancyhf{} % Clear all header and footer fields
    \fancyfoot[C]{\thepage}
    \fancyhead[LE,RO]{\course} % Section name on odd pages
    \fancyhead[LO,RE]{\rightmark}
    % Optional: Thin rules
    \renewcommand{\headrulewidth}{0pt} % Header rule
    \renewcommand{\footrulewidth}{0pt} % No footer rule
}
%
\begin{document}

\pagestyle{mainstyle}
\maketitle
%%
\section{Exercise 1}% 9.30
The general second degree real polynomial is of the form:
\begin{align}
  P(x)=Ax^2+Bx+C.
\end{align}
The coefficients are related each other if a binomial is desired to construct:
\begin{align*}
  (Dx+E)^2=D^2x^2+2DEx+E^2.
\end{align*}

The argument must be completed by summing 


\section{Exercise 2}
In the equation above, the exponential has no units as well as the term $\sqrt{2\pi}$. 

\section{Exercise 3}


\section{Exercise 4}

\section{Exercise 5}

\section{Exercise 6 (done)}
Eulers formula, in a general manner, states that:
\begin{align}
  e^{a\pm i\theta}=e^{a}(\cos\theta\pm i\sin\theta).
\end{align}
The sine and cosine functions can be defined in terms of the exponential function by setting $a=0$ and using the taylor expansion.
Without prove the equivalences are the following:
\begin{align*}
  \sin\theta=\frac{e^{i\theta}-e^{-i\theta}}{2i},\quad\text{and}\quad\cos\theta=\frac{e^{i\theta}+e^{-i\theta}}{2}.
\end{align*}
For the first identity, replacing the above definitions yields:
\begin{align*}
  \sin(2\theta)&=2\sin(\theta)\cos(\theta)\\
  \frac{e^{i2\theta}-e^{-i2\theta}}{2i}&=\cancel{2}\left(\frac{e^{i\theta}-e^{-i\theta}}{2i}\right)\left(\frac{e^{i\theta}+e^{-i\theta}}{\cancel{2}}\right)\\
  &=\frac{e^{i2\theta}+\cancel{e^0}-\cancel{e^{0}}-e^{-i2\theta}}{2i}\\
  \frac{e^{i2\theta}-e^{-i2\theta}}{2i}&=\frac{e^{i2\theta}-e^{-i2\theta}}{2i}.
\end{align*}
Similarly, for the second identity:
\begin{align*}
  \cos^2(\theta)&=\frac{1}{2}[1+\cos(2\theta)]\\
  \left(\frac{e^{i\theta}+e^{-i\theta}}{2}\right)^2&=\frac{1}{2}\left[1+\frac{e^{i2\theta}+e^{-i2\theta}}{2}\right]\\
  \frac{e^{i2\theta}+2e^{0}+e^{-i2\theta}}{4}&=\frac{1}{2}\left[\frac{2+e^{i2\theta}+e^{-i2\theta}}{2}\right]\\
  \frac{e^{i2\theta}+2+e^{-i2\theta}}{4}&=\frac{e^{i2\theta}+2+e^{-i2\theta}}{4}.
\end{align*}
%%
\section{Exercise 7}

\section{Exercise 8}

\section{Exercise 9 (Done)}
Given the $\bm{M}$ matrix, the eigenvalue problem is the following:
\begin{align}
  \bm{M}\bm{v}=\lambda\bm{v},
\end{align}
where $\lambda$ are the eigenvalues to be obtained and $\bm{v}$ are the eigenvectors associated.
In order to get the eigenvalues, the following equation must be solved:
\begin{align}
  |\bm{M}-\lambda\bm{I}|=0,
\end{align}
where $\bm{I}$ is the identity matrix. Substituting $\bm{M}$ and $\bm{I}$,
\begin{align*}
  \left|\begin{pmatrix}
    3&0&0\\0&4&0\\0&0&2
  \end{pmatrix}-\begin{pmatrix}
    \lambda&0&0\\0&\lambda&0\\0&0&\lambda
  \end{pmatrix}\right|=\begin{vmatrix}
    3-\lambda&0&0\\0&4-\lambda&0\\0&0&2-\lambda
  \end{vmatrix}=0.
\end{align*}
The above results is easy to solve, as the determinant of a diagonal matrix is the product of its diagonal elements:
\begin{align}
  (3-\lambda)(4-\lambda)(2-\lambda)=0\Longrightarrow\lambda\in\{2,3,4\}.
\end{align}
Because the polinomial is expressed a a productory of its roots, the eigenvalues are the roots of the equation.
%%
\section{Exercise 10 (Done)}
Given the $\bm{M}$ matrix, the eigenvalue problem is similar to previous exercise. In this case, we have:
\begin{align*}
  \left|\bm{M}-\lambda\bm{I}\right|=
  \left|\begin{pmatrix}
    0&-i\\i&0
  \end{pmatrix}-\begin{pmatrix}
    \lambda&0\\0&\lambda
  \end{pmatrix}\right|=\left|\begin{pmatrix}
    -\lambda&-i\\i&-\lambda
  \end{pmatrix}\right|=\lambda^2-1=0\Longrightarrow\lambda\in\{-1,1\}\in\mathbb{R}.
\end{align*}
The eigenvalues are real numbers, as expected from a Hermitian matrix: $\bm{M}=\bm{M}^\dagger$.
The eigenvectors are obtained replacing each $\lambda$ in the rearranged eigenvalue problem.
\begin{itemize}
  \item For $\lambda_1=-1$, we have
  \begin{align*}
    (\bm{M}-\lambda_1\bm{I})\bm{v}_1&=\bm{0}\\
    (\bm{M}+\bm{I})\bm{v}_1&=\\
    \left(\begin{pmatrix}
      0&-i\\i&0
    \end{pmatrix}+\begin{pmatrix}
      1&0\\0&1
    \end{pmatrix}\right)\bm{v}_1&=\\
    \begin{pmatrix}
      1&-i\\i&1
    \end{pmatrix}\begin{pmatrix}
    v_1^{(1)}\\v_1^{(2)}
    \end{pmatrix}&=\\
    \begin{pmatrix}
      1v_1^{(1)}-iv_1^{(2)}\\
      iv_1^{(1)}+1v_1^{(2)}
    \end{pmatrix}&=\begin{pmatrix}
      0\\0
    \end{pmatrix}
  \end{align*}
  Solving each row provides the same answer, namely, $v_1^{(2)}=-iv_1^{(1)}$. That implies that replacing this results gives:
  \begin{align}
    \bm{v}_1=\begin{pmatrix}
    v_1^{(1)}\\v_1^{(2)}
    \end{pmatrix}=\begin{pmatrix}
      v_1^{(1)}\\-iv_1^{(1)}
    \end{pmatrix}=v_1^{(1)}\begin{pmatrix}
      1\\-i
    \end{pmatrix}\Longrightarrow\bm{v}_1=\begin{pmatrix}
      1\\-i
    \end{pmatrix}.
  \end{align}
  %
  \item For $\lambda_2=1$, we have
  \begin{align*}
    (\bm{M}-\lambda_2\bm{I})\bm{v}_2&=\bm{0}\\
    (\bm{M}-\bm{I})\bm{v}_2&=\\
    \left(\begin{pmatrix}
      0&-i\\i&0
    \end{pmatrix}-\begin{pmatrix}
      1&0\\0&1
    \end{pmatrix}\right)\bm{v}_2&=\\
    \begin{pmatrix}
      -1&-i\\i&-1
    \end{pmatrix}\begin{pmatrix}
    v_2^{(1)}\\v_2^{(2)}
    \end{pmatrix}&=\\
    \begin{pmatrix}
      -1v_2^{(1)}-iv_2^{(2)}\\
      iv_2^{(1)}-1v_2^{(2)}
    \end{pmatrix}&=\begin{pmatrix}
      0\\0
    \end{pmatrix}
  \end{align*}
  Solving each row yields $v_2^{(2)}=iv_2^{(1)}$. In this case, the eigenvector is:
  \begin{align}
    \bm{v}_2=\begin{pmatrix}
    v_2^{(1)}\\v_2^{(2)}
    \end{pmatrix}=\begin{pmatrix}
      v_2^{(1)}\\iv_2^{(1)}
    \end{pmatrix}=v_2^{(2)}\begin{pmatrix}
      1\\i
    \end{pmatrix}\Longrightarrow\bm{v}_2=\begin{pmatrix}
      1\\i
    \end{pmatrix}.
  \end{align}
\end{itemize}
To give unitary normalized eigenvectors, we compute the square root of their inner product with themselves:
\begin{align*}
  [\bm{v}_1^\dagger\cdot\bm{v}_1]^{1/2}=\left[\begin{pmatrix}
    1&i
  \end{pmatrix}\cdot\begin{pmatrix}
    1\\-i
  \end{pmatrix}\right]^{1/2}=\sqrt{2},\quad\text{and}\quad
  [\bm{v}_2^\dagger\cdot\bm{v}_2]^{1/2}=\left[\begin{pmatrix}
    1&-i
  \end{pmatrix}\cdot\begin{pmatrix}
    1\\i
  \end{pmatrix}\right]^{1/2}=\sqrt{2}.
\end{align*}
We thus use each normalization factor to divide the respective eigenvector, so that we finally have:
\begin{align*}
  \bm{v}\in\{\bm{v}_1,\bm{v}_2\}=\left\{\frac{1}{\sqrt{2}}\begin{pmatrix}
  1\\-i\end{pmatrix},\frac{1}{\sqrt{2}}\begin{pmatrix}
  1\\i\end{pmatrix}\right\}.
\end{align*}

\section{Exercise 11 (Done)} 
The norm of the given complex matrix is obtained as the square root of the inner product of the complex conjugate of the matrix times its
original form, as follows
\begin{align*}
  \|\bm{v}\|=[\bm{v}^\dagger\cdot\bm{v}]^{1/2}=\left[
    \begin{pmatrix}
      -3&-4i
    \end{pmatrix}\cdot\begin{pmatrix}
      -3\\4i
    \end{pmatrix}
  \right]^{1/2}=[(-3)(-3)-(4i)(4i)]^{1/2}=[9+16]^{1/2}=\sqrt{25}=5.
\end{align*}



\section{Exercise 12}

\section{Exercise 13}

\section{Exercise 14 (Done)}
The sketch of the function $f(t)=\sin^2(\frac{\Omega t}{2})$ is shown in Figure \ref{fig:plotex14}.
\begin{figure}[htbp]
  \centering
  \begin{circuitikz}[scale=.8,xscale=0.5]
    \def\Om{1}
    \draw[arrow](0,0)--({8*pi/\Om+1},0)node[below]{$t$};
    \draw[arrow](0,0)--(0,2)node[right]{$f(t)$};
    \draw[very thick,NavyBlue,domain=0:{8*pi/\Om},samples=100] plot(\x,{ (sin(deg(\Om*\x/2)) )^2 });
    \draw[very thick,dashed,NavyBlue,domain=-{pi/\Om}:0,samples=10] plot(\x,{ (sin(deg(\Om*\x/2)) )^2 });
    \draw(0,.1)--(0,-.1)node[below]{0}({2*pi/\Om},.1)--({2*pi/\Om},-.1)node[below]{$\frac{2\pi}{\Omega}$}
    ({4*pi/\Om},.1)--({4*pi/\Om},-.1)node[below]{$\frac{4\pi}{\Omega}$}
    ({6*pi/\Om},.1)--({6*pi/\Om},-.1)node[below]{$\frac{6\pi}{\Omega}$}(.1,1)--(-.1,1)node[left]{$1$};
  \end{circuitikz}
  \caption{Plot of the function $f(t)=\sin^2(\frac{\Omega t}{2})$.}
  \label{fig:plotex14}
\end{figure}

The behavior is oscilatory as a simple sinusoidal function. The period of $f(t)$ is  
\begin{align*}
  \frac{2\pi}{T}t=\frac{\Omega}{2}t\longrightarrow T=\frac{4\pi}{\Omega}.
\end{align*}
The ceros of the function are the same of its unsquared version $\sin(\frac{\Omega t}{2})$, which are obtained through
\begin{align*}
  \sin\left(\frac{\Omega t}{2}\right)&=0/\sin^{-1}(\cdot)\\
  \frac{\Omega t_k}{2}&=k\pi,\quad k\in\mathbb{Z}\\
  t_k&=\frac{2k\pi}{\Omega},\quad k\in\mathbb{Z}.
\end{align*} 
This can also be obtained by using trigonometric identities.
With respect to the output values, the function oscilates in the range $f(t)\in[0,1]$. This clapped
range is due to the square $(\cdot)^2$ that mirrors the sinusoidal where it is negative to a positive value.
In fact, 
\begin{align*}
  -1<\sin(\cdot)<1\Longrightarrow \begin{array}{rl}
    (-1)^2<&\sin^2(\cdot)<1^2/(\cdot)^2\\
    0<&\sin^2(\cdot)<1
  \end{array}.
\end{align*}

%%
\section{Exercise 15}

%%
\section{Exercise 16}

%%
\section{Exercise 17 (Done)}
Let be $\omega\in\mathbb{C}$, obtaining the cube-root of $1$ is the same as solving the equation $\omega^3=1$.
If $\omega$ is expressed in polar form, we have
\begin{align*}
  \omega^2\equiv\rho^2 e^{i3\theta}\equiv\rho^2(\cos3\theta+i\sin3\theta)=1.
\end{align*}
The radius $\rho$ is easy to acquire if the second equality with the last one are compared. Therefore,
\begin{align*}
  \rho^3e^{i3\theta}=1e^{i0}\Longrightarrow\rho=1.
\end{align*}
In doing the same with the third and last equations we have the following relations
\begin{align*}
  \rho^3\cos3\theta=1\quad\text{and}\quad\rho^3\sin3\theta=0.
\end{align*}
From the sine equation, we can formulate the $\theta$ values:
\begin{align*}
  \sin3\theta=0/\sin^{-1}(\cdot)\Longrightarrow \theta_k=\frac{k\pi}{3},\quad k\in\{0,1,2\}.
\end{align*}
The $k$ has been restricted to the first three roots desired.
Using $\rho$ and $\theta$ obtained allows to express the three cube-roots of $1$ as
\begin{align}
  \begin{array}{rl}
    \omega_0&=1\left(\cos\frac{2\cdot0\cdot\pi}{3}+i\sin\frac{2\cdot0\cdot\pi}{3}\right)=1+i0\\
    \omega_1&=1\left(\cos\frac{2\cdot1\cdot\pi}{3}+i\sin\frac{2\cdot1\cdot\pi}{3}\right)=-\frac{1}{2}+i\frac{\sqrt{3}}{2}\\
    \omega_2&=1\left(\cos\frac{2\cdot2\cdot\pi}{3}+i\sin\frac{2\cdot2\cdot\pi}{3}\right)=-\frac{1}{2}-i\frac{\sqrt{3}}{2}\\
  \end{array}
\end{align}

The principal three cube-roots obtained above are ploted in the complex plane as shown in figure \ref{fig:ex17}.
\begin{figure}[htbp]
    \centering
    \begin{circuitikz}[scale=1]
    \draw[arrow](-2,0)--(2,0)node[below]{$\text{Re}(z)$};
    \draw[arrow](0,-2)--(0,2)node[right]{$\text{Im}(z)$};
    \draw[arrow,very thick,NavyBlue](0:0)--(0:1)node[above right]{$\omega_0$};
    \draw[arrow,very thick,NavyBlue](0:0)--(120:1)node[above left]{$\omega_1$};
    \draw[arrow,very thick,NavyBlue](0:0)--(240:1)node[below left]{$\omega_2$};
    \draw[black](0,0)+(0:1)arc[start angle=0,end angle=360,radius=1](1,1)node[above right]{$\rho=1$}; 
    \end{circuitikz}
    \caption{First three cube-roots of $1$ in the complex plane.}
    \label{fig:ex17}
\end{figure}





\nocite{*}
\bibliographystyle{plain}   % or unsrt, alpha, apalike, etc.
\bibliography{refs}

\end{document}
