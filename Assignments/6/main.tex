\documentclass[letterpaper,11pt,twoside]{article}
\usepackage{graphicx} % Required for inserting images
\usepackage[table,xcdraw,dvipsnames]{xcolor}
\usepackage{amsmath,amsfonts,amssymb,amsthm}
\usepackage{listings}
\usepackage{lipsum}
\usepackage{hyperref}
\usepackage{enumitem}

\usepackage{tikz}
\usepackage[siunitx, RPvoltages]{circuitikz}
\usetikzlibrary{3d}
\usepackage{comment}
\usepackage{caption,subcaption}
\usepackage{pgfplots}
\pgfplotsset{compat=newest} % or a newer version if available
\usepgfplotslibrary{groupplots}
\usetikzlibrary{pgfplots.groupplots}
\usetikzlibrary{shapes.geometric, arrows}
\tikzstyle{arrow} = [->,>=stealth,shorten >=2pt]
\newcommand{\ket}[1]{|#1\rangle}
\newcommand{\bra}[1]{\langle#1|}
\newcommand{\br}{\bm{r}}
\newcommand{\bR}{\bm{R}}
\newcommand{\bp}{\bm{p}}
\newcommand{\bP}{\bm{P}}
\newcommand{\braket}[1]{\langle#1\rangle}
\newcommand{\F}{\mathscr{F}}
\newcommand{\E}{\mathscr{E}}
\newcommand{\re}[1]{\text{Re}\left(#1\right)}
\newcommand{\im}[1]{\text{Im}\left(#1\right)}
\usepackage{dsfont}
\usepackage{cancel}
\usepackage{bm}
\usepackage{fancyhdr}
\usepackage[utf8x]{inputenc}
\usepackage[T1]{fontenc}
\usepackage[margin=0.8in,top=1in,bottom=1in]{geometry}
%%%%%
\begin{filecontents*}{refs.bib}
@book{bornwolf,
  author    = {Born, M. and Wolf, E.},
  title     = {Principles of Optics},
  publisher = {Pergamon Press},
  edition   = {7},
  year      = {1999}
}
@book{hecht,
  author    = {Hecht, E.},
  title     = {Optics},
  publisher = {Addison-Wesley},
  edition   = {5},
  year      = {2016}
}
\end{filecontents*}
%
\newcommand{\institution}{University of Arizona}
\newcommand{\autor}{Nicolás Hernández Alegría}
\newcommand{\course}{OPTI 570 Quantum Mechanics}
\newcommand{\assignment}{Assignment 6}
%
\title{\textbf{\assignment}\\\course\\{\Large\institution}}
\author{\autor}
\date{\today\\Total time: 15 hours}
%
\renewcommand{\sectionmark}[1]{\markright{#1}}
\fancypagestyle{mainstyle}{
    \fancyhf{} % Clear all header and footer fields
    \fancyfoot[C]{\thepage}
    \fancyhead[LE,RO]{\course} % Section name on odd pages
    \fancyhead[LO,RE]{\assignment}
    % Optional: Thin rules
    \renewcommand{\headrulewidth}{0pt} % Header rule
    \renewcommand{\footrulewidth}{0pt} % No footer rule
}
%
\begin{document}

\pagestyle{mainstyle}
\maketitle
%%
\section*{Problem I}
\begin{enumerate}[itemsep=0pt,topsep=0pt,label=\alph*)]
  \item On the one hand, the action of $\tilde{a}(t)$ is 
  \begin{align*}
    \tilde{a}(t)\ket{\varphi_n}&=U^\dagger(t,0)aU(t,0)\ket{\varphi_n}=U^\dagger(t,0)ae^{-i\left(N+\frac{1}{2}\right)\omega t}\ket{\varphi_n}=e^{-i\left(n+\frac{1}{2}\right)\omega t}U^\dagger(t,0)a\ket{\varphi_n}\\
    &=\sqrt{n}e^{-i\left(n+\frac{1}{2}\right)\omega t}U^\dagger(t,0)\ket{\varphi_{n-1}}=\sqrt{n}e^{-i\left(n+\frac{1}{2}\right)\omega t}e^{i\left(n-\frac{1}{2}\right)\omega t}\ket{\varphi_{n-1}}=\sqrt{n}e^{-i\omega t}\ket{\varphi_{n-1}}.
  \end{align*}
  Therefore,
  \begin{align*}
    \tilde{a}(t)\ket{\varphi_n}=\sqrt{n}e^{-i\omega t}\ket{\varphi_{n-1}}=e^{-i\omega t}a\ket{\varphi_n}\longrightarrow \tilde{a}(t)=e^{-i\omega t}a.
  \end{align*}
  On the other hand, the action of $\tilde{a}^\dagger(t)$ is 
  \begin{align*}
    \tilde{a}^\dagger(t)\ket{\varphi_n}&=U^\dagger(t,0)a^\dagger U(t,0)\ket{\varphi_n}=e^{-i\left(n+\frac{1}{2}\right)\omega t}U^\dagger(t,0)a^\dagger\ket{\varphi_n}=\sqrt{n+1}e^{-i\left(n+\frac{1}{2}\right)\omega t}U^\dagger(t,0)\ket{\varphi_{n+1}}\\
    &=\sqrt{n+1}e^{-i\left(n+\frac{1}{2}\right)\omega t}e^{i\left(n+\frac{3}{2}\right)\omega t}\ket{\varphi_{n+1}}.
  \end{align*}
  Consequently,
  \begin{align*}
    \tilde{a}^\dagger(t)\ket{\varphi_n}=\sqrt{n+1}e^{i\omega t}\ket{\varphi_{n+1}}=e^{i\omega t}a^\dagger\ket{\varphi_n}\longrightarrow \tilde{a}^\dagger(t)=e^{i\omega t}a^\dagger.
  \end{align*}
  \item We can compute the operators if we stimulate them with a ket $\ket{\varphi_n}$. We make use of the $a,a^\dagger$ expression for $\tilde{X}$ and $\tilde{P}$.
  \begin{align*}
    \tilde{X}(t)\ket{\varphi_n}=\frac{\sigma}{\sqrt{2}}\left[U^\dagger a^\dagger U+U^\dagger aU\right]\ket{\varphi_n}.
  \end{align*}
  But, we have already computed these operations in the previous part, so we will use it here:
  \begin{align*}
    \tilde{X}(t)\ket{\varphi_n}&=\frac{\sigma}{\sqrt{2}}\left[e^{i\omega t}a^\dagger+e^{-i\omega t}a\right]\ket{\varphi_n}\Longrightarrow \tilde{X}(t)=\frac{\sigma}{\sqrt{2}}\left[e^{i\omega t}a^\dagger+e^{-i\omega t}a\right].
  \end{align*}
  We can further simplified if we develop the complex exponential, obtaining:
  \begin{align*}
    \tilde{X}(t)=X\cos\omega t+\frac{P}{m\omega}\sin\omega t.
  \end{align*}
  In the same manner, we have for $\tilde{P}$:
  \begin{align*}
    \tilde{P}(t)\ket{\varphi_n}=\frac{i\hbar}{\sqrt{2}\sigma}\left[U^\dagger a^\dagger U-U^\dagger aU\right]\ket{\varphi_n}=\frac{i\hbar}{\sqrt{2}\sigma}\left[e^{i\omega t}a^\dagger-e^{-i\omega t}a\right]\ket{\varphi_n}
  \end{align*}
  Therefore,
  \begin{align*}
    \tilde{P}(t)=\frac{i\hbar}{\sqrt{2}\sigma}\left[e^{i\omega t}a^\dagger-e^{-i\omega t}a\right].
  \end{align*}
  And, 
  \begin{align*}
    \tilde{P}(t)=P\cos\omega t-m\omega X\sin\omega t.
  \end{align*}
  They oscillate in time with the operators of position an momentum acting on kets, the coefficients of each trigonometric function resemble to the classical motion of HO.
  \item To show that $U^\dagger\ket{x}$ is an eigenvector of $P$ with a given eigenvalue, we must satisfy the eigenequation of $P$. Using the $\tilde{P}$ found in the previous part, we have:
  \begin{align*}
    PU^\dagger(\frac{\pi}{2\omega},0)\ket{x}&=U^\dagger UPU^\dagger\ket{x}=U^\dagger \tilde{P}^\dagger(\pi/2\omega)\ket{x}=U^\dagger[P\cos\pi/2+m\omega X\sin\pi/2]\ket{x}\\
    PU^\dagger(\frac{\pi}{2\omega},0)\ket{x}&=U^\dagger[m\omega X\ket{x}]=m\omega xU^\dagger(\frac{\pi}{2\omega},0)\ket{x}.
  \end{align*}
  In consequence, $U^\dagger(\pi/2\omega,0)\ket{x}$ is an eigenvector of $P$ with eigenvalue $m\omega x$.
  Similarly, we have that $U^\dagger(\pi/2\omega,0)\ket{p}$ is an eigenvector of $X$ with eigenvalue $-p/m\omega$:
  \begin{align*}
    XU^\dagger(\frac{\pi}{2\omega},0)\ket{p}=-\frac{p}{m\omega}U^\dagger(\frac{\pi}{2\omega})\ket{p}.
  \end{align*}
  \item It turnt out that when $t_q=q\pi/2\omega$ evolves, the operators $\tilde{X}$ and $\tilde{P}$ also evolve in a way that in some cases we have purely the action of $X$ 
  while in others the action of $P$. Assuming that $\psi(x,0)$ is normalized, all the other evolutions will remains normalized as the evolutino operator is unitary.
  Then,
  \begin{align*}
    \psi(x,t_q)=\braket{x|U(t_q)|\psi(0)},\quad t_q=\frac{q\pi}{2\omega}.
  \end{align*}
  For the $tilde{P}(t_1)$ we have:
  \begin{align*}
    \tilde{P}(t_1)=U(t_1)PU^\dagger(t_1)=m\omega X.
  \end{align*}
  Multiplying to the left by $\bra{x}$ and to the right by $U(t_1)$ yields:
  \begin{align*}
    (\bra{x}U(t_1))P=m\omega x\bra{x}U(t_1)=c\bra{p=m\omega x}.
  \end{align*}
  Then,
  \begin{align*}
    \psi(x,t_1)=\braket{x|U(t_1)|\psi(0)}=C\braket{p=m\omega x|\psi(0)}=c\tilde{\psi}(m\omega x),
  \end{align*}
  where $\tilde{\psi}$ is te Fourier transform evaluated at $p=m\omega x$. To find $c$, we impose normalization:
  \begin{align*}
    |c|^2\int_{-\infty}^\infty|\tilde{\psi}(m\omega x)|^2dx=\frac{|c|^2}{m\omega}\int_{-\infty}^\infty|\tilde{\psi(p)}|^2\;dp=1\longrightarrow|c|=\sqrt{m\omega}.
  \end{align*}
  The phase of $c$ can be found by taking the global factor of the evolution operator:
  \begin{align*}
    U(t_1)=e^{iHt/\hbar}=e^{-i\omega t/2}e^{-ia^\dagger a t}\longrightarrow \angle c|_{t=t_1}=e^{-i\pi/4}.
  \end{align*}
  So that:
  \begin{align*}
    \psi(x,t_1)=e^{-i\pi/4}\sqrt{m\omega}\tilde{\psi}(m\omega x).
  \end{align*}
  When we evolve to $t_2$, there is another $90^\circ$ that performs a Fourier transform over $\psi(t_1)$. By property of this transformation, two successive applications return
  the original function with the coordinates inverted:
  \begin{align*}
    \psi(x,t_1)=\psi(-x,0).
  \end{align*}
  Then, for $t_3$ we obtain $\psi(t_1)$ with the negative obtained above:
  \begin{align*}
    \psi(x,t_3)=\sqrt{m\omega}\tilde{\psi}(-m\omega x).
  \end{align*}
  And another $t_4$ will return the original function:
  \begin{align*}
    \psi(x,t_4)=\psi(-(-x),0)=\psi(x,0).
  \end{align*}
  This can be done indefinitely.
  \item We select $\varphi_n(x)$ the energy eigenstate of the HO. We saw that 
  \begin{align*}
    \psi(x,\frac{\pi}{2\omega})=e^{-i\pi/4}\sqrt{m\omega}\tilde{\psi}(m\omega x).
  \end{align*}
  A stationary state only evolves in its global phase factor, so:
  \begin{align*}
    \psi(x,\frac{\pi}{2\omega})=e^{-iE_n(\pi/2\omega)/\hbar}\varphi_n(x)=e^{-i(n+1/2)\pi/2}\varphi_n(x)=e^{-i\pi/4}e^{-in\pi/2}\varphi_n(x).
  \end{align*}
  Equating this result with the previous one:
  \begin{align*}
    e^{-i\pi/4}e^{-in\pi/2}\varphi_n(x)&=e^{-i\pi/4}\sqrt{m\omega}\tilde{\psi}(m\omega x)\\
    \varphi_n(x)&=i^n\sqrt{m\omega}\tilde{\varphi}_n(m\omega x).
  \end{align*}
  \item 
  \begin{enumerate}[itemsep=0pt,topsep=0pt,label=\roman*)]
    \item This corresponds to a plane wave. For $t_1=\pi/2\omega$ we have its Fourier transform, which is a delta function shifted proportional to $t_1$. Then, 
    we recover the plane wave with an additinoal exponential term, and so on.
    \item Im not sure about this one, I know its Fourier transform but its difficult to describe it. It has a singularity proportional to $\rho$.
    \item This is a rect function, so at $t_1$ we obtain a sinc function and so on.
    \item This is a Gaussian function, so its Fourier transform remains Gaussian.
  \end{enumerate}
\end{enumerate}


\section*{Problem II}
\begin{enumerate}[itemsep=0pt,topsep=0pt,label=\alph*)]
  \item We know that $\alpha_0=\ket{\psi(0)}$ is a coherent state, so $a\ket{\alpha}=\alpha\ket{\alpha}$. Then, replcaing the definitions of $X$ and $P$ in terms of $a,a^\dagger$ and doing algebra we find:
  \begin{align*}
    \braket{\alpha_0|P|\alpha_0}&=\frac{i\hbar}{\sqrt{2}\sigma}\braket{\alpha_0|(a^\dagger-a)|\alpha_0}=\frac{i\hbar}{\sqrt{2}\sigma}\left[\braket{\alpha_0|a^\dagger|\alpha_0}-\braket{\alpha_0|a|\alpha_0}\right]=\frac{i\hbar}{\sqrt{2}\sigma}\left[\alpha_0^*-\alpha_0\right]=-\frac{i2\hbar}{\sqrt{2}\sigma}\im{\alpha_0}.
  \end{align*} 
  In the same manner,
  \begin{align*}
    \braket{\alpha_0|X|\alpha_0}&=\frac{\sigma}{\sqrt{2}}\braket{\alpha_0|(a^\dagger+a)|\alpha_0}=\frac{\sigma}{\sqrt{2}}\left[\braket{\alpha_0|a^\dagger|\alpha_0}+\braket{\alpha_0a|\alpha_0}\right]=\frac{\sigma}{\sqrt{2}}[\alpha_0^*+\alpha_0]=\frac{2\sigma}{\sqrt{2}}\re{\alpha_0}.
  \end{align*}
  Then,
  \begin{align*}
    \alpha_0=\frac{1}{\sqrt{2}}\left[\frac{\braket{X}(0)}{\sigma}+i\frac{\sigma\braket{P}(0)}{\hbar}\right].
  \end{align*}
  These mean values are the ones we have computed.
  \item Due to the momentum kick, we know that 
  \begin{align*}
    \alpha_0=\frac{ip_o\sigma}{\sqrt{2}\hbar}.
  \end{align*}
  Then, doing the same as in the previous part, we have:
  \begin{align*}
    \braket{\alpha_0|H|\alpha_0}&=\hbar\omega\braket{\alpha_0|(a^\dagger a+1/2)|\alpha_0}=\hbar\omega\left[|\alpha_0|^2+\frac{1}{2}\right]=\frac{p_0^2\sigma^2\omega}{2\hbar}+\frac{\hbar\omega}{2}.
  \end{align*}
  But, the oscillator length is given by:
  \begin{align*}
    \sigma=\sqrt{\frac{\hbar}{m\omega}},
  \end{align*}
  so
  \begin{align*}
    \braket{\alpha_0|H|\alpha_0}=\frac{p_0^2}{2m}+\frac{\hbar\omega}{2}.
  \end{align*}
  \item The eigenvalue is $\alpha(t)$:
  \begin{align*}
    \alpha(t)=\frac{ip_0\sigma}{\sqrt{2}\hbar}e^{-i\omega t}.
  \end{align*}
  This is because the state vector remains a coherent state over time, and under that condition the eigenequation of $a$ gives you $\alpha_0$ when acting on such a vector.
  \item We use the expresion of $a$ in terms of $X$ and $P$:
  \begin{align*}
    [a,X]=\left[\frac{1}{\sqrt{2}}\left\{\frac{X}{\sigma}+i\frac{\sigma}{\hbar}P\right\},X\right]=i\frac{\sigma}{\sqrt{2}\hbar}[P,X]=i\frac{\sigma}{\sqrt{2}\hbar}(-i\hbar)=\frac{\sigma}{\sqrt{2}}.
  \end{align*}
  For the other commutator, we can use the commutator of a function property:
  \begin{align*}
    [a,T(p_0)]=[a,e^{ip_0X/\hbar}]=[a,X]\frac{d(e^{ip_oX/\hbar})}{dX}=\frac{\sigma}{\sqrt{2}}\frac{ip_0}{\hbar}e^{ip_0X/\hbar}=\alpha_0T(p_0).
  \end{align*}
  Using the definition of the commutator and the result above:
  \begin{align*}
    aT(p_0)-T(p_0)a=\alpha_0T(p_0)\longrightarrow aT(p_0)=(a+\alpha_0)T(p_0).
  \end{align*}
  \item Yes, it remains a coherent state. The only adding is a global phase factor. The eigenvalue at $t=t_1$ is $\alpha(t_1)=\alpha_0e^{-i\omega t_1}$.
  \begin{align*}
    a(T(p_0)\ket{\psi(t_1)})=T(p_0)(a+\alpha_0)\ket{\psi(t_1)}=(\alpha(t_1)+\alpha_0)T(p_0)\ket{\psi(t_1)}.
  \end{align*}
  so that 
  \begin{align*}
    \alpha_2=\alpha(t_1)+\alpha_0=\alpha_0(e^{-i\omega t_1}+1).
  \end{align*}
  \item We do the same as before, but with the new $\alpha$.
  \begin{align*}
    \braket{\alpha_2|H|\alpha_2}&=\hbar\omega\left[|\alpha_0(1+e^{-i\omega t_1})|^2+1/2\right]=\hbar\omega|\alpha_0|^2(2+2\cos\omega t_1)+\frac{\hbar\omega}{2}=\hbar\omega\frac{2p_0^2\sigma^2}{2\hbar^2}(1+\cos\omega t_1)+\frac{\hbar\omega}{2}\\
    \braket{\alpha_2|H|\alpha_2}&=\frac{p_0^2}{m}(1+\cos\omega t_1)+\frac{\hbar\omega}{2}.
  \end{align*}
  \item This means that the mean value $\braket{H}$ before and after the second kick must be equal:
  \begin{align*}
    \braket{H}_{\text{before}}=\frac{p_0^2}{2m}+\frac{\hbar\omega}{2},\quad\text{and}\quad\braket{H}_{\text{after}}=\frac{p^2_0}{m}(1+\cos\omega t_1)+\frac{\hbar\omega}{2}.
  \end{align*}
  We set both to be equal and solve for $t_1$:
  \begin{align*}
    \frac{p^2_0}{m}(1+\cos\omega t_1)+\frac{\hbar\omega}{2}&=\frac{p_0^2}{2m}+\frac{\hbar\omega}{2}\\
    \cos\omega t_1&=-\frac{1}{2}\\
    t_1&\in\left\{\frac{2\pi}{3\omega},\frac{4\pi}{3\omega}\right\}.
  \end{align*}
  \item Each time evolution given by $t_1=2\pi/3\omega$ rotates a third of the full cycle, so with three we complete a full period. We know that a kick adds $\alpha$, while a time evolution 
  rotates along the phase-space. The sequence is
  \begin{align*}
    Q=T(p_0)U(t_1)T(p_0)U(t_1)T(p_0).
  \end{align*}
\end{enumerate}


%5
\section*{Problem III}
We consider the system as a tensor product of each dimension.
\begin{enumerate}[itemsep=0pt,topsep=0pt,label=\alph*)]
  \item The changes in position can be represented in the following way:
  \begin{align*}
    \ket{\alpha_x}=T(m\delta_v)\ket{\varphi_0},\quad\ket{\alpha_y}=\ket{\varphi_0},\quad\ket{\alpha_z}=S(1\;mm)\ket{\varphi_0}.
  \end{align*}
  At $t=0^+$, we have
  \begin{align*}
    \alpha_x(0)=\frac{1}{\sqrt{2}}\frac{i\sigma m\delta_V}{\hbar},\quad\alpha_y(0)=0,\quad\alpha_z(0)=\frac{1}{\sqrt{2}}\frac{1\;mm}{\sigma}.
  \end{align*}
  The maximum displacements in $z$ is $A_z=1\;mm$, whereas for $x$ we know that $p_x(0)=m\omega\delta_v$ so:
  \begin{align*}
    A_x(0)=\frac{\delta_v}{m\omega}=\frac{\delta_v}{100}.
  \end{align*}
  \item In this case, $A_x=A_z$ and 
  \begin{align*}
    \frac{\delta_v}{\omega}=A_z=1\;mm\longrightarrow\delta_v=(1\;mm)\omega=100\;mm/s=0.1\;m/s.
  \end{align*}
  \item The probability follows a Poisson distribution. First, we have that 
  \begin{align*}
    \alpha_x=\frac{i}{\sqrt{2}}\sqrt{\frac{m}{\hbar\omega}}\delta_v,\quad\alpha_z=\frac{1\;mm\sigma}{\sqrt{2}}.
  \end{align*}
  The square of them corresponds to the Poisson mean and we will take the floor of that value as the most likelikood:
  \begin{align*}
    |\alpha_x|^2=\frac{1}{2}\frac{m\delta_v^2}{\hbar\omega}=\frac{1}{2}\frac{(1.5\times10^{-25})(0.1)^2}{(1.054571817\times10^{-34})(100)}=71118.91.
  \end{align*}
  Therefore,
  \begin{align*}
    \ket{n_x=71118},\quad\ket{n_z=71118},\quad\ket{n_y=0}.
  \end{align*}
  \item They both are zero:
  \begin{align*}
    \braket{X}_{\text{after}}=\braket{P}_{\text{after}}=0.
  \end{align*}

\end{enumerate}


%%
\section*{Problem IV}

\subsection*{Part 1.}
\begin{enumerate}[itemsep=0pt,topsep=0pt,label=\alph*)]
  \item We can treat the problem as a harmonic oscillator. The potential makes a force of the form
  \begin{align*}
    F_r=-\partial_rV(r)=-\frac{GM_Em}{R_E^3}r=-m\omega^3r,\quad\omega=\sqrt{\frac{GM_E}{R_E^3}}.
  \end{align*}
  Using the information provided, we can obtain the period as:
  \begin{align*}
    T=\frac{2\pi}{\omega}=2\pi\sqrt{\frac{R_E^3}{GM_E}}=5085.24\;s\equiv84.75\;min.
  \end{align*}
  For the case of the velocity, we know that because we are ignoring other forces the system must conserves the energy and it will be distributed in potential and kinetic energy.
  So,
  \begin{align*}
    -\frac{3}{2}\frac{GM_Em}{R_E}=\frac{1}{2}mv_c^2-\frac{3}{2}\frac{GM_Em}{R_E}\longrightarrow v_c=\sqrt{\frac{GM_E}{R_E}}=7907.67\;m/s\equiv28.467\;km/h.
  \end{align*}
\end{enumerate}

\subsection*{Part 2.}
\begin{enumerate}[itemsep=0pt,topsep=0pt,label=\alph*),start=2]
  \item For the potential $r<R$, we can take Gauss law to give a structure and then froce throught the whole space to be something continuous. However,
  I will compare it with the form of the graviational force to deduce that:
  \begin{align*}
    V_C(r)=-\frac{3}{2}\frac{k_ee^2}{R}+\frac{k_ee^2}{2R^3}r^2
  \end{align*}
  Now, the force can be derived similarly and equated to the clasical harmonic oscillator:
  \begin{align*}
    F=-\partial_rV_C(r)=-\frac{k_ee^2}{R^3}r=-m_p\omega^2r\longrightarrow\omega=\sqrt{\frac{k_ee^2}{m_pR^3}}=2.945\times10^{15}\;rad/s.
  \end{align*}
  \item The oscillator length $\sigma$ is:
  \begin{align*}
    \sigma=\sqrt{\frac{\hbar}{m_p\omega}}=4.588\times10^{-12}\;m.
  \end{align*}
  \item The step of the energy is $\Delta E=\hbar\omega$, and this value is also $hc/\lambda$ so equating them and solving for $\lambda$:
  \begin{align*}
    \lambda=\frac{hc}{\hbar\omega}=\frac{2\pi c}{\omega}=640\;nm.
  \end{align*}
  \item Using the general potential we have derived evaluated at $r=R$ yields 
  \begin{align*}
    V_C(R)=-\frac{k_ee^2}{R}.
  \end{align*}
  At that position, there is no kinetic energy as it is at rest. Equating it with the total energy of an arbitrary position provides:
  \begin{align*}
    \frac{1}{2}m_pv^2+V_C(r)=-\frac{k_ee^2}{R}\longrightarrow v=\sqrt{\frac{k_ee^2}{m_pR^3}(R^2-r^2)}.
  \end{align*}
  At the origin, we have $r=0$ so 
  \begin{align*}
    v=\sqrt{\frac{k_ee^2}{m_pR}}=7.4\times10^4\;m/s\equiv2.66\times10^5\;km/h.
  \end{align*}
  It is about $10$ times faster that under gravitational potential.
\end{enumerate}




%\nocite{*}
%\bibliographystyle{plain}   % or unsrt, alpha, apalike, etc.
%\bibliography{refs}

\end{document}
