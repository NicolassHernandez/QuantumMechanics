\documentclass[letterpaper,11pt,twoside]{article}
\usepackage{graphicx} % Required for inserting images
\usepackage[table,xcdraw,dvipsnames]{xcolor}
\usepackage{amsmath,amsfonts,amssymb,amsthm}
\usepackage{listings}
\usepackage{lipsum}
\usepackage{hyperref}
\usepackage{enumitem}

\usepackage{tikz}
\usepackage[siunitx, RPvoltages]{circuitikz}
\usetikzlibrary{3d}
\usepackage{comment}
\usepackage{caption,subcaption}
\usepackage{pgfplots}
\pgfplotsset{compat=newest} % or a newer version if available
\usepgfplotslibrary{groupplots}
\usetikzlibrary{pgfplots.groupplots}
\usetikzlibrary{shapes.geometric, arrows}
\tikzstyle{arrow} = [->,>=stealth,shorten >=2pt]
\newcommand{\ket}[1]{|#1\rangle}
\newcommand{\bra}[1]{\langle#1|}
\newcommand{\br}{\bm{r}}
\newcommand{\bR}{\bm{R}}
\newcommand{\bp}{\bm{p}}
\newcommand{\bP}{\bm{P}}
\newcommand{\braket}[1]{\langle#1\rangle}
\newcommand{\F}{\mathscr{F}}
\newcommand{\E}{\mathscr{E}}
\usepackage{dsfont}
\usepackage{cancel}
\usepackage{bm}
\usepackage{fancyhdr}
\usepackage[utf8x]{inputenc}
\usepackage[T1]{fontenc}
\usepackage[margin=0.8in,top=1in,bottom=1in]{geometry}
%%%%%
\begin{filecontents*}{refs.bib}
@book{bornwolf,
  author    = {Born, M. and Wolf, E.},
  title     = {Principles of Optics},
  publisher = {Pergamon Press},
  edition   = {7},
  year      = {1999}
}
@book{hecht,
  author    = {Hecht, E.},
  title     = {Optics},
  publisher = {Addison-Wesley},
  edition   = {5},
  year      = {2016}
}
\end{filecontents*}
%
\newcommand{\institution}{University of Arizona}
\newcommand{\autor}{Nicolás Hernández Alegría}
\newcommand{\course}{OPTI 570 Quantum Mechanics}
\newcommand{\assignment}{Assignment 6}
%
\title{\textbf{\assignment}\\\course\\{\Large\institution}}
\author{\autor}
\date{\today\\Total time: 5 hours}
%
\renewcommand{\sectionmark}[1]{\markright{#1}}
\fancypagestyle{mainstyle}{
    \fancyhf{} % Clear all header and footer fields
    \fancyfoot[C]{\thepage}
    \fancyhead[LE,RO]{\course} % Section name on odd pages
    \fancyhead[LO,RE]{\assignment}
    % Optional: Thin rules
    \renewcommand{\headrulewidth}{0pt} % Header rule
    \renewcommand{\footrulewidth}{0pt} % No footer rule
}
%
\begin{document}

\pagestyle{mainstyle}
\maketitle
%%
\section*{Problem I}
\begin{enumerate}[itemsep=0pt,topsep=0pt,label=\alph*)]
  \item On the one hand, the action of $\tilde{a}(t)$ is 
  \begin{align*}
    \tilde{a}(t)\ket{\varphi_n}&=U^\dagger(t,0)aU(t,0)\ket{\varphi_n}=U^\dagger(t,0)ae^{-i\left(N+\frac{1}{2}\right)\omega t}\ket{\varphi_n}=e^{-i\left(n+\frac{1}{2}\right)\omega t}U^\dagger(t,0)a\ket{\varphi_n}\\
    &=\sqrt{n}e^{-i\left(n+\frac{1}{2}\right)\omega t}U^\dagger(t,0)\ket{\varphi_{n-1}}=\sqrt{n}e^{-i\left(n+\frac{1}{2}\right)\omega t}e^{i\left(n-\frac{1}{2}\right)\omega t}\ket{\varphi_{n-1}}=\sqrt{n}e^{-i\omega t}\ket{\varphi_{n-1}}.
  \end{align*}
  Therefore,
  \begin{align*}
    \tilde{a}(t)\ket{\varphi_n}=\sqrt{n}e^{-i\omega t}\ket{\varphi_{n-1}}=e^{-i\omega t}a\ket{\varphi_n}.
  \end{align*}
  On the other hand, the action of $\tilde{a}^\dagger(t)$ is 
  \begin{align*}
    \tilde{a}^\dagger(t)\ket{\varphi_n}&=U^\dagger(t,0)a^\dagger U(t,0)\ket{\varphi_n}=e^{-i\left(n+\frac{1}{2}\right)\omega t}U^\dagger(t,0)a^\dagger\ket{\varphi_n}=\sqrt{n+1}e^{-i\left(n+\frac{1}{2}\right)\omega t}U^\dagger(t,0)\ket{\varphi_{n+1}}\\
    &=\sqrt{n+1}e^{-i\left(n+\frac{1}{2}\right)\omega t}e^{i\left(n+\frac{3}{2}\right)\omega t}\ket{\varphi_{n+1}}.
  \end{align*}
  Consequently,
  \begin{align*}
    \tilde{a}^\dagger(t)\ket{\varphi_n}=\sqrt{n+1}e^{i\omega t}\ket{\varphi_{n+1}}=e^{i\omega t}a^\dagger\ket{\varphi_n}.
  \end{align*}
  \item We can compute the operators if we stimulate them with a ket $\ket{\varphi_n}$. We make use of the $a,a^\dagger$ expression for $\tilde{X}$ and $\tilde{P}$.
  \begin{align*}
    \tilde{X}(t)\ket{\varphi_n}=\frac{1}{\sqrt{2}}\left[U^\dagger a^\dagger U+U^\dagger aU\right]\ket{\varphi_n}.
  \end{align*}
  But, we have already computed these operations in the previous part, so we will use it here:
  \begin{align*}
    \tilde{X}(t)\ket{\varphi_n}&=\frac{1}{\sqrt{2}}\left[e^{i\omega t}a^\dagger+e^{-i\omega t}a\right]\ket{\varphi_n}\Longrightarrow \tilde{X}(t)=\frac{1}{\sqrt{2}}\left[e^{i\omega t}a^\dagger+e^{-i\omega t}a\right].
  \end{align*}
  In the same manner, we have for $\tilde{P}$:
  \begin{align*}
    \tilde{P}(t)\ket{\varphi_n}=\frac{1}{\sqrt{2}}\left[U^\dagger a^\dagger U-U^\dagger aU\right]\ket{\varphi_n}=\frac{1}{\sqrt{2}}\left[e^{i\omega t}a^\dagger-e^{-i\omega t}a\right]\ket{\varphi_n}
  \end{align*}
  Therefore,
  \begin{align*}
    \tilde{P}(t)=\frac{1}{\sqrt{2}}\left[e^{i\omega t}a^\dagger-e^{-i\omega t}a\right].
  \end{align*}
  They are like trigonometric functions cosine and sine, with the different that the operator $a,a^\dagger$ is in the middle.
  \item To show that a ket is an eigenvector of an operator, it will have to satisfy its eigenequation with a constant representing the eigenvalue: $P\ket{\psi}=\lambda\ket{\psi}$.
  \item asfa
  \item asfafasf
  \item asfsa
\end{enumerate}


\section*{Problem II}
\begin{enumerate}[itemsep=0pt,topsep=0pt,label=\alph*)]
  \item asfaf
  \item asfasf
  \item asfafasf
  \item asfa
  \item sgag
  \item asgag
  \item asgagasg
  \item asg
\end{enumerate}


%5
\section*{Problem III}
\begin{enumerate}[itemsep=0pt,topsep=0pt,label=\alph*)]
  \item asfaf
  \item asfasf
  \item asfafasf
  \item asfa
\end{enumerate}


%%
\section*{Problem IV}

\subsection*{Part 1.}
\begin{enumerate}[itemsep=0pt,topsep=0pt,label=\alph*)]
  \item 
\end{enumerate}

\subsection*{Part 2.}
\begin{enumerate}[itemsep=0pt,topsep=0pt,label=\alph*),start=2]
  \item asfaf
  \item asfasf
  \item asfafasf
  \item asfa
\end{enumerate}




%\nocite{*}
%\bibliographystyle{plain}   % or unsrt, alpha, apalike, etc.
%\bibliography{refs}

\end{document}
