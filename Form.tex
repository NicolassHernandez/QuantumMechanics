\section*{Formula sheet}
%%
\subsection{Useful formulas}
\begin{table}[h!]
    \centering
    \resizebox{\columnwidth}{!}{%
    \begin{tabular}{lL|lL}
        Closure relation (discrete)&\displaystyle\sum_k\sum_{i=1}^{g_k}\ket{v_k^i}\bra{v_k^i}=\mathds{1}&Closure relation (continuous)&\displaystyle\int_{\beta}d\beta\;\ket{\omega_\beta}\bra{\omega_\beta}=\mathds{1}\\
        Glauber Formula&e^{A+B}=e^{A}e^{B}e^{-\frac{1}{2}[A,B]}&Generalized uncertainty relation&\Delta A\Delta B\geq\frac{1}{2}|\braket{[A,B]}|\\
        Function of an operator&\displaystyle F(A)=\sum_{n=0}^\infty f_n(A-a)^n&&\Delta Q=\sqrt{\braket{Q^2}-\braket{Q}^2}\\
        Eigenequation of $F(A)$&F(A)\ket{\psi}=F(\lambda)\ket{\psi}&&\\
        Transformation $\{u\}\rightarrow\{v\}$&\mathbb{M}_{jk}=\braket{u_j|v_k}&$\ket{\psi}_{\{u\}}=\mathbb{M}\ket{\psi}_{\{v\}}$&\ket{\psi}_{\{v\}}=\mathbb{M}^\dagger\ket{\psi}_{\{u\}}\\
        &&$A_{\{u\}}=\mathbb{M}A_{\{v\}}\mathbb{M}^\dagger$&A_{\{v\}}=\mathbb{M}^\dagger A_{\{u\}}\mathbb{M}
    \end{tabular}}
\end{table}
%%
\subsection{Basis}
\begin{table}[h!]
    \centering
    \resizebox{.7\columnwidth}{!}{%
    \begin{tabular}{L|L|L}
        \text{Quantity}&\text{Discrete basis (sum over $j,k$)}&\text{Continuous basis (integrate over $\beta,\beta'$)}\\
        \hline
        \mathds{1}&=\displaystyle\sum\ket{v_k}\bra{v_k}&=\displaystyle\int d\beta\;\ket{\omega_\beta}\bra{\omega_\beta}\\
        \ket{\psi}=\mathds{1}\ket{\psi}&=\displaystyle\sum\ket{v_k}\braket{v_k|\psi}&=\displaystyle\int d\beta\;\ket{\omega_\beta}\braket{\omega_\beta|\psi}\\
        \bra{\varphi}=\bra{\varphi}\mathds{1}&=\displaystyle\sum\braket{\varphi|v_k}\bra{v_k}&=\displaystyle\int d\beta\;\braket{\varphi|\omega_\beta}\bra{\omega_\beta}\\
        A=\mathds{1}A\mathds{1}&=\displaystyle\sum\sum\ket{v_j}\braket{v_j|A|v_k}\bra{v_k}&=\displaystyle\iint d\beta\;d\beta'\;\ket{\omega_\beta}\braket{\omega_\beta|A|\omega_{\beta'}}\bra{\omega_{\beta'}}
    \end{tabular}}
\end{table}
\begin{table}[h!]
    \centering
    \resizebox{.6\columnwidth}{!}{%
    \begin{tabular}{L|L|L}
        \text{Quantity}&\text{$X$ representation}&\text{$P_x$ representation}\\
        \hline
        X&x&i\hbar\;\partial/\partial p\\
        P_x&-i\hbar\;\partial/\partial x&p\\
        \ket{x'}&\braket{x|x'}=\delta(x-x')&\braket{p|x'}=\frac{1}{\sqrt{2\pi\hbar}}\exp(-ix'p/\hbar)\\
        \ket{p'}&\braket{x|p'}=\frac{1}{\sqrt{2\pi\hbar}}\exp(ixp'/\hbar)&\braket{p|p'}=\delta(p-p')\\
        \ket{\psi}&\braket{x|\psi}=\psi(x)&\braket{p|\psi}=\tilde{\psi}(p)
    \end{tabular}}
\end{table}
\begin{table}[h!]
    \centering
    \begin{tabular}{l|l}
        \multicolumn{2}{c}{Fourier transforms for 3D wavefunctions}\\
        \hline
        $\tilde{\psi}(\bp)=\fourier{\psi(\bR)}=\left(\frac{1}{2\pi\hbar}\right)^{3/2}\int_{-\infty}^\infty d^3\bR\;e^{-i\bR\cdot\bp/\hbar}\psi(\bR)$&$\psi(\bR)=\ifourier{\tilde{\psi}(\bp)}=\left(\frac{1}{2\pi\hbar}\right)^{3/2}\int_{-\infty}^\infty d^3\bp\;e^{i\bR\cdot\bp/\hbar}\tilde{\psi}(\bp)$\\
        $\fourier{\psi^{(n)}(x)}=\left(\frac{ip}{\hbar}\right)^n\tilde{\psi}(p)$&$\tilde{\psi}^{(n)}(p)=\fourier{\left(-\frac{ix}{\hbar}\right)^n\psi(x)}$\\
        $\tilde{\psi}(p-p_0)=\fourier{e^{ip_0x/\hbar}\psi(x)}$&$e^{-ipx_0/\hbar}\tilde{\psi}(p)=\fourier{\psi(x-x_0)}$\\
        $\fourier{\psi(cx)}=\tilde{\psi}(p/c)/|c|$&$\int_{-\infty}^\infty dx\;\varphi^*(x)\psi(x)=\int_{-\infty}^\infty dp\;\tilde{\varphi}^*(p)\tilde{\psi}(p)$\\
        $\psi(x)$ real $[\tilde{\psi}(p)]^*=\tilde{\psi}(-p)$&$\psi(x)$ imaginary $[\tilde{\psi}(p)]^*=-\tilde{\psi}(-p)$\\
        $\Delta x\Delta p\geq\hbar$&$$
    \end{tabular}
\end{table}
\begin{comment}
\end{comment}
%%
\subsection*{Commutators}
\begin{table}[h!]
    \centering
    \resizebox{\columnwidth}{!}{%
    \begin{tabular}{l|l}
        $[A,B]=-[B,A]$&$[A+B,C+D]=[A,C]+[A,D]+[B,C]+[B,D]$\\
        $[A,B]^\dagger=[B^\dagger,A^\dagger]$&$[A,BC]=[A,B]C+B[A,C]$\\
        $[F(A),A]=0$&$[A,B]=0\Longrightarrow[F(A),B]=[A,F(B)]=[F(A),F(B)]=0$\\
        $[A,[B,C]]+[B,[C,A]]+[C,[A,B]]=0$&$[A,[A,B]]=[B,[A,B]]=0\Longrightarrow[A,F(B)]=[A,B]\cfrac{dF(B)}{dB}$\\
        $e^{A}e^{B}=e^{A+B}e^{\frac{1}{2}[A,B]}\;([A,[A,B]]=[B,[A,B]]=0)$&$[A,[A,B]]=[B,[A,B]]=0\Longrightarrow[F(A),B]=[A,B]\cfrac{dF(A)}{dA}$\\
        $[X,P]=i\hbar$&$[H,X]=-\frac{i\hbar}{m}P$\\
        $[H,P]=i\hbar\frac{dV(X)}{dX}$&$\braket{\varphi_n|[A,H]|\varphi_n}=0,\;\forall A$
    \end{tabular}}
\end{table} 

\subsection*{Key points}
\begin{itemize}[itemsep=0pt,topsep=0pt]
    \item When a matrix has a block form, we can compute the eigenvalues in each submatrix.
    \item The orthonormalized eigenvectors of a matrix create the basis by concatenating them.
    \item If the matrix is diagonal, the exponential acts directly onto the elements.
    \item Evolution opterator is defined as $U=e^{-iHt/\hbar}$, and then it evolves the stae by matrix multiplication $U\ket{\psi}$.
    \item Eigenvectors that are related in the eigenequation must be considerd to the constuction of the new basis.
    \item You can reduce the dimension of an operator to its eigensubspace when only acting inside it.
    \item To know the action of an operator you can stimulate it by applying $\ket{\phi}$ or $\bra{\psi}$.
    \item In the operation $\ket{u_i}\bra{u_j}$, the element will be located at $ij$ in the matrix.
\end{itemize}